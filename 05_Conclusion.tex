\chapter*{Conclusion}
% pour faire apparaitre l'introduction dans le sommaire
\addcontentsline{toc}{chapter}{Conclusion}
% Pour que l'entete soit correcte car chapter* ne redefinit pas l'entete.
\markboth{CONCLUSION}{}

\begin{quote}[\textit{Les Nuits d'une Demoiselle} (1963), paroles de Guy Breton]{Colette Renard (interprète)}
Que c'est bon d'être demoiselle \\
Car le soir, dans mon petit lit \\
Quand l'étoile Vénus étincelle \\
Quand doucement tombe la nuit \\  
Je me fais sucer la friandise \\
Je me fais caresser le gardon \\
Je me fais empeser la chemise \\
Je me fais picorer le bonbon \\
Je me fais frotter la péninsule \\
Je me fais béliner le joyau \\
Je me fais remplir le vestibule \\
Je me fais ramoner l'abricot {[...]} \\
Et vous me demanderez peut-être \\
Ce que je fais le jour durant \\
Oh! cela tient en peu de lettres \\
Le jour... je baise, tout simplement.\footnote{Nous remercions l'algorithme de Spotify de nous avoir fait découvrir cette chanson pendant notre travail de thèse via la fonction écoute recommandée.}
\end{quote}

% Rappel de la problématique
Nous avions commencé cet ouvrage avec une question \enquote{simple}: est-il possible pour un ordinateur d'apprendre à reconnaître une isotopie, en particulier celle de la sexualité, en latin ? L'isotopie étant simplement définie par une répétition de sèmes, elle peut prendre un nombre de formes illimitées: vocabulaire à \enquote{sens premier}, métaphores, métonymies, etc. L'ensemble du panel des figures de styles, jusqu'aux cacemphatons de Cicéron (\enquote{\textit{cum nobis}}, \enquote{\textit{illam dicam}}), peut être mis au service d'une langue figurée. En ajoutant à cela la présomption d'isotopie, même une phrase comme \enquote{Sous son regard fatigué, il branchait sa clef usb.} peut être perçue comme sexuelle. Quand il s'agit de parler de sexe, tant de sèmes sont activables (par exemples, ceux liés au mouvement, au contact, à certaines formes, etc.), que même \enquote{caresser le gardon} devient obscène. La question n'est pas seulement \enquote{peut-on faire faire de la lexicographie à la machine} mais elle est aussi \enquote{peut-on lui faire faire vraiment de la stylistique ?}

Cette question est d'autant plus intéressant que le domaine linguistique et culturel auquel elle s'applique est celui du latin. Du point de vue culturel, contrairement à la détection d'isotopies sexuelles en langue française du XX\textsuperscript{e} ou du XXI\textsuperscript{e} siècle, il ne nous est pas possible de poser la question à l'auteur, ou d'utiliser d'autres témoins vivants pour évaluer la pertinence d'une interprétation. Et la langue latine reste encore sous-dotée, sous étudiée, par le domaine du traitement automatique des langues: son étude est généralement réservée aux quelques spécialistes en linguistique latine bons dans le champ numérique. Si l'on regarde les travaux de lemmatisation produits ces dernières années, toutes sont issus de latinistes restés dans leur champ ou transfuges disciplinaires: Enrique Manjavacas, créateur de Pie, a une licence de Philologie classique, les équipes du LASLA avec Dominique Longrée mènent leur travail en collaboration avec quelques TAListes réceptifs à ces problématiques, les membres de l'ERC LiLa sont tous ou presque latinistes de formation, etc. C'est d'autant plus dommage que la langue latine, avec ses deux mille ans d'histoire, et donc d'évolutions, offrent un \enquote{terrain de jeu} remarquable pour évaluer de nouveaux outils: elle possède une morphologie extrêmement riche, des périodes de variations graphiques, des phénomènes dialectaux et scripturaux depuis les papyri et graffitis jusqu'aux chartes médiévales... Comme le dit F.~Rastier, \enquote{[la philologie fait partie des ]disciplines injustement oubliées, du moins dans le domaine des Traitements automatiques du langage\footcite{rastier2005enjeux}}. Quelle montagne de travail à gravir pour qui voudrait se laisser tenter par la combinaison de ces deux mondes...

Enfin, en dehors de la stylistique, cette question est intéressante à travers la combinaison du thème qu'elle traite (la sexualité) et des méthodes qu'elle se propose d'utiliser (numériques). Ces dernières années, le problème des biais dans les \enquote{algorithmes d'intelligence artificielle} ont produit un grand nombre de débats: \enquote{est-il normal que ``métier d'infirmier'' soit lié à ``femme'' dans Google} est une question relativement souvent posée dans les médias\footnote{Et les biais politiques, les effets de bulle et le classement des contenus affichés, problème lié à ces biais, les ont rejoint depuis les problèmes des élections américaines de 2016.}. On pose des questions éthiques à la machine et les modèles de langue, qui sont faits pour représenter le réel qu'on leur a fourni, sont soumis à des objectifs extra-linguistiques. Là où cette \enquote{manie} de sur-apprendre est une faiblesse pour leur utilisation moderne, cette capacité de la machine à capturer les clichés est une force pour l'étude de civilisations passées. Bien sûr, il ne faut pas utiliser la machine seule, car les textes ne sont pas les seules sources à notre disposition et qu'ils sont bien évidemment des fenêtres réduites sur ces civilisations. Avec les informations externes aux textes, nous sommes capables de dépasser le \enquote{problème du mouton noir\footnote{Nous avons entendu ce termes de nombreuses fois, mais jamais nous n'avons réussi à trouver de manière certaine sa source. Nous proposons tout de même celle-ci, à propos des \textit{embeddings} de Google. \textcite{daume_blacksheep_2016}.}}: le \textit{black sheep problem} correspond au phénomème qui pousse tout algorithme dont la représentation du monde est issue de textes uniquement a estimé que le mouton sera noir, car on ne précise moins souvent qu'il est blanc. Or, à propos de la sexualité romaine, la compréhension des clichés nous intéresse autant que les exceptions. L'approche algorithmique offre dans ce cadre une nouvelle méthode heuristique pour l'études des civilisations passées, qui permet d'appuyer, de relativiser ou de faire émerger de manière computationnelle des hypothèses qui pré-existaient ou non à l'avénement du numérique dans les sciences humaines.

\paragraph{Corpus, exempliers, lettres classiques}

Il est intéressant de voir comment, chez F.~Rastier, B.~Pincemin et ou D.~Mayaffre, la question du corpus est \enquote{centrale\footcite[p.~59]{mayaffre2008occurrence}}, et comment elle est valorisée. Au contraire, nous avons vu que l'histoire des corpus latins -- et de grec quand la référence était nécessaire -- avait très peu été documentée et étudiée. Nous avons rassemblé une nouvelle documentation -- qui pourra être complétée -- inédite sur le sujet, de la fondation des premiers corpus grecs aux corpus latins des années 1990 pour les premiers projets, jusqu'à 2021 pour les plus récents avec l'inclusion de l'\textit{Internet Archive}. Les informations commencent bien évidemment à se perdre, et, peu à peu, les responsables des premiers projets approchent d'une fin de carrière bien méritée. \textit{Quid} de leur expérience, en dehors des articles bien évidemment élogieux pour obtenir un nouveau financement ? Nous avons découvert à cette occasion une bibliographie insoupçonnée et difficile d'accès, \enquote{d'origine} qui nous permet d'accéder à cette histoire. Par exemple, \enquote{Bits, Bytes \& Biblical Studies} de John. J.~Hughes\footcite{hughes_bits_1987} n'est présent qu'en un seul exemplaire en France, d'après le Sudoc ou le CCfr, ou en trente-sept exemplaires en Europe d'après le WorldCat, et pourtant, il est la seule source connue à donner une image aussi précise sur l'ensemble des projets informatiques dans le domaine gréco-latin ayant cours dans les années 1980.

Nous avons aussi découvert les révolutions numériques qui ont suivi le milieu des années 1990 et les promesses des deux dernières décades, à travers, entre autres, la reconnaissance de caractère. Là où les chercheurs de \textit{Perseus} ou du TLG disaient vouloir éviter l'OCR car le nombre d'erreurs était trop important sur de l'imprimé du XX\textsuperscript{e} ou de la fin du XIX\textsuperscript{e}, nous sommes en train de voir émerger des modèles de reconnaissance de textes permettant le traitement d'éditions des XVI\textsuperscript{e} et XVII\textsuperscript{e} mais aussi de manuscrits médiévaux. Cette révolution permettra peut-être d'en accompagner une autre, qui serait celle d'un basculement vers les éditions nativement numériques et la mise à disposition de textes par les éditeurs scientifiques. La très grande majorité des corpus étudiés sont aujourd'hui constitués de textes sans apparat critique, ce qui limite la philologie numérique à une étude d'un texte parmi d'autres, et empêche de s'intéresser à une étude méthodique et globale des phénomènes de variations. Nous avons vu que c'est l'ambition de projets comme la \textit{Digital Latin Library}, mais sommes en peine de trouver des exemples. Cette histoire nous a donc permis de comprendre l'état des corpus en latin à notre disposition, dans la mesure où nous nous sommes concentrés sur une histoire très anglo-saxonne, le côté italien étant intégré à l'histoire des corpus, mais l'histoire des corpus en Italie n'étant pas abordée.

À partir de cette analyse de la production des corpus sur cinquante ans, nous avons pu entamer une réflexion sur les critères techniques qui doivent définir un corpus, qui n'est pas un concept désincarné mais bien une réalisation d'ingénierie en plus d'une collection de documents, toutes deux poussées par une connaissance du domaine. L'ambition de cette thèse n'était pas de créer un corpus nouveau -- nous n'aurions pas eu le temps -- mais bien de rassembler, là où c'était possible, un nouvel ensemble de textes à même de constituer une base de recherche pour la philologie et l'histoire latine, computationnelles ou non. Les critères retenus sont simples: le texte doit être encodé en XML TEI, afin de garantir son interopérabilitié, en accès libre, \textit{open source} et donc amendable, et enfin actionnable. L'interopérabilité et l'utilisation d'un standard permet de pérenniser l'information et d'en assurer une forme de documentation minimale: un document TEI, d'autant plus un document simple comme ceux que nous avons, est assuré d'être compréhensible et maniable grâce aux connaissances d'un groupe, ici celui des humanités numériques, et le partage de compétence qui s'y fait. L'accès libre comme l'\textit{open source} assurent non seulement la correctabilité des sources (erreur d'OCR, fausse attribution, texte manquant) mais aussi la reproductibilité des expériences. Cette dernière amènent non seulement vers un niveau de confiance plus grand, mais permet aussi de comparer des approches en s'assurant de l'utilisation des mêmes données. Enfin, nous avons vu que la partie \textit{machine actionable} avait un impact important pour nos corpus. Contrairement à une grande partie du TAL, nos corpus ne sont pas pensés comme des corpus pour linguistes, mais pour des lectures humaines. L'injection et la reconnaissance de l'importance des systèmes de structures logiques permet d'extraire des documents indépendants ou semi-indépendants des fichiers. Nous avons montré qu'ignorer le rôle des (S)ATU mène à des résultats différents: s'il s'agit de s'appuyer sur un traitement numérique du texte pour une question de philologie ou d'histoire, il faut pouvoir s'assurer de la qualité de l'analyse. Encore une fois, qui estimerait que le dernier mot d'une épigramme est co-occurrent du premier mot de la suivante dans un commentaire littéraire ou une analyse linguistique ?

Les critères techniques nous ont permis de faire émerger un corpus principal, notre meta-corpus, constitué en majeure partie de corpus pré-existants voire encore en activité. À travers \textit{Perseus}\footcite{perseus_latinLit}, le CSEL d'\textit{Open Greek and Latin}\footcite{csel_latinlit} et la \textit{DigilibLT}\footcite{digiliblt}, une couverture de l'espace chronologique de -250 à +800 est possible. Elle n'est pas parfaite, nous avons du ajouter des textes de la période classique comme de la période tardo-antique, notamment les \textit{Priapées} ou l'\textit{Anthologie Latine}. Le corpus final est constitué de près de vingt millions de mots et il demande encore du travail de conversion vers une structuration \textit{machine actionable} pour des textes de \textit{Perseus} et de la \textit{DigilibLT}: le corpus possible, en utilisant uniquement les oeuvres à disposition, reste donc immense.

Le meta-corpus compilé, nous nous sommes tourné vers la thématique qui constitue notre terrain d'essai, à savoir la sexualité dans la littérature latine classique et tardive. L'histoire de la sexualité est un domaine très récent, ou en tout cas complètement renouvelé depuis l'ouvrage fondamental de Michel Foucault. Nous avons montré qu'il existe plusieurs vagues de recherche sur le sujet dans le domaine historique, mais qu'au contraire, l'étude de son lexique est un peu plus ancienne, avec quelques glossaires massifs au XIX\textsuperscript{e} siècle. Cette lexicographie s'est retrouvée intégralement reprise en une dizaine d'années, à travers deux thèses et une monographie: Enrico Montero Cartelle a fini son doctorat sur le sujet en 1973\footcite{montero_cartelle_aspectos_1973}, Amy Richlin en 1978\footcite{richlin_sexual_1978}, et James N.~Adams a conclu ce mouvement par sa monographie, qui fait source pour notre recherche, le \textit{Latin Sexual Vocabulary}\footcite{adams}. Nous avons montré qu'Adams s'intègre dans une école mancunienne de linguistique, et son oeuvre sur le sujet dépasse la \enquote{simple} monographie citée. Il s'avère d'ailleurs que le vocabulaire compilé par Adams n'est pas exhaustif -- et c'est normal, l'oeuvre est déjà massive -- et fait preuve de manques systématiques: les adjectifs liés à la moralité sexuelle (\textit{lascivus}, \textit{mollis}, \textit{impudicus}, etc.), le vocabulaire de l'abstinence et la virginité, le domaine médical et la période tardive chrétienne sont soit sous-traités soit totalement ignorés.

À travers notre lecture d'Adams, première pierre fondatrice et fondamentale d'un projet qui peut grandir, nous avons établi, à partir des mêmes principes établis pour notre corpus, une collection d'échantillons. Nous proposons de nommer ce type de collections \enquote{exemplier numérique} en cela qu'il a une vocation doublement pédagogique. Il a en effet deux publics différents: d'une part, les (apprentis) chercheurs, enseignants et amateurs du latin, qui peuvent puiser dans une lecture et recherche de cette base de données des exemples pour mieux comprendre un texte ou un phénomène social; d'autre part, la machine, public nouveau des documents numériques, vouée à apprendre à partir de données consignées par des spécialistes afin de produire de nouveaux savoirs. Nous avons présenté les limites de notre approche d'Adams, où une lecture parfois trop linéaire à poser un problème dans notre classement sous-thématique (par exemple, \textit{violent} n'inclut pas \textit{arme}) et pose les bases de la nécessité d'un \textit{thesaurus} avant d'aller plus loin. Cependant, l'exemplier produit contient plus de deux mille cinq cent échantillons, qui permettent désormais d'entraîner une machine à reconnaître des isotopies autant qu'elle peut faire base de données. Nous présentons d'ailleurs les prémices d'une interface d'exploration de ces extraits collectés, avec les fonctionnalités qui semblent à notre sens nécessaire pour assurer, en plus d'une exploitation statistique pour les plus hardis numériquement, une simple visualisation et contextualisation des données.

\paragraph{Lemmatisation du latin}
%% Comment je suis arrivé grâce à la technique à faire apprendre à la machine de la stylistique.

\enquote{On doit compléter et sans doute dépasser la question distributionnelle du texte par une conception morphosémantique qui tienne compte des inégalités qualitatives entre formes. 100 Formes sémantiques et textualité}

\paragraph{La détection d'isotopie}
% Apports de la thèse

\paragraph{Apports}

\paragraph{Et après ?}

\enquote{Un corpus est un regroupement structuré de textes intégraux, documentés, éventuellement enrichis par des étiquetages, et rassemblés: (i) de manière théorique réflexive en tenant compte des discours et des genres, et (ii) de manière pratique en vue d’une gamme d’applications\footcite{rastier2005enjeux}}
%% Défricher et construire une méthodo, un système d'enquête
%% Généricisation de l'objet
%% Reproductibilité et réutilisabilité
%% Manière de transmettre les outils
%% Comment on peut en servir ?

% Ouverture

