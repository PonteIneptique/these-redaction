\chapter*{Remerciements}
% pour faire apparaitre l'introduction dans le sommaire
\addcontentsline{toc}{chapter}{Remerciements}
% Pour que l'entete soit correcte car chapter* ne redefinit pas l'entete.
\markboth{REMERCIEMENTS}{}

La rédaction de la conclusion de ce doctorat et la soutenance à venir marquent une étape presque finale dans la formation scolaire qui a été la mienne. Ces années ont été parsemées de rencontres décisives, de soutiens importants, et il semble logique de profiter de ces quelques pages pour remercier les personnes qui m'ont mené ou permis d'être là à rédiger ces derniers mots de mon doctorat. 

Je tiens d'abord à remercier vivement les membres de mon jury, qui, toutes et tous, ont accepté de participer à cette étape d'évaluation dans des délais particulièrement courts. Dans ce contexte de thèse soumise à deux mois de son rendu, je tiens à remercier les deux prérapporteurs de cette thèse, Laurent Romary et Elena Pierazzo, qui ont accepté de venir lire quelques latinitudes dans un environnement plus large d'humanités numériques. Je remercie aussi pour leur participation au jury Bruno Bureau, Marie-Claire Beaulieu et Helma Dik. Je ne peux qu'être un peu déçu de ne pas pouvoir vivre cette soutenance en leur présence et discuter de la thèse dans les couloirs, et j'espère avoir l'occasion de rediscuter de ces travaux après la soutenance.

Il y a huit ans, j'ai été présenté à deux professeurs lyonnais, étudiant parisien et ingénieur londonien que j'étais. Ces deux enseignants, Bruno Bureau et Christian Nicolas, m'ont non seulement accueilli, mais ils ont aussi accepté de me suivre: Bruno Bureau en devenant le président de mon jury de thèse, Christian Nicolas en dirigeant celle-ci. Je remercie ici vivement Christian Nicolas, car je ne peux imaginer un meilleur encadrement: ses commentaires, ses corrections, ses conseils -- même en dehors de la thèse -- ont toujours été précieux. Je garde un souvenir joyeux des quelques moments de discussion en \enquote{présentiel} qu'ont permis mes années lyonnaises et les replis du COVID. 

Je remercie mon employeur, l'École nationale des Chartes, et en particulier sa directrice, Mme Bubenicek, qui m'a permis de faire ma thèse en me recrutant il y a cinq ans. Ce recrutement, avec le temps de recherche qui allait avec, m'a permis de mener à bien mes recherches. Je remercie mes collègues qui m'ont toujours exprimé un soutien dans mon projet doctoral, en particulier le service de la scolarité avec qui je partage le troisième étage depuis mon arrivée, et mes collègues de la recherche et du corps enseignant. Je pense en particulier à Vincent Jolivet et Julien Pilla qui m'ont toujours soutenu à l'École, y compris quand il s'agissait de faire passer Pyrrha de SQLite à PostgreSQL. Il m'est difficile de nommer l'ensemble des collègues sans risquer d'en oublier un dans l'urgence d'une veille de dépôt, j'espère qu'ils et elles se reconnaîtront. Je remercie aussi les partenaires de l'École sur quelques projets pour leur compréhension quant à ma moins grande disponibilité pour des projets annexes.

Je pense bien évidemment au LASLA, sans qui cette thèse n'aurait pas été la réussite qu'elle est, et que j'ai pu contacter grâce au soutien de mon directeur. Je remercie le professeur Dominique Longrée et 
Margherita Fantoli, désormais \textit{assistant professor} en humanités numériques qui ont été mes points de contact du laboratoire et qui ont su me faire confiance dans le partage de leurs données.

J'adresse une pensée à tous les collègues du monde de l'\textit{open source}. Je remercie en particulier Mike Kestemont et Enrique Manjavacas pour \textit{Pie}, leur réactivité et leur ouverture, Lisa Cerrato et Alison Babeu pour leurs réponses, même des années après avoir quitté la sphère Perseus, sur des questions de catalogage ou de conversion Capitains du corpus latin. Je pense bien évidemment aux collègues de DTS et du monde de l'épigraphie numérique: Hugh Cayless, Jonathan Robbie, Ian W. Scott au titre de DTS et Tom Eliott pour son aide sur l'incursion dans l'histoire de l'épigraphie numérique. 

Je remercie mes étudiants et mes étudiantes qui toutes ces années m'ont donné des raisons de rester dans le domaine universitaire. En cinq ans de doctorat, j'ai eu le plaisir de voir des dizaines de mémoires ou de projets en programmation ou données d'une compétence rare, j'ai accompagné autant que j'ai pu vu une centaine de diplômés et de diplômées qui ont désormais de brillantes carrières, et j'ai déjà eu la joie d'écrire des articles avec une de mes anciennes étudiantes, Alix Chagué, que je remercie pour la confiance qu'elle m'a donnée. Je pense à d'autres étudiantes et étudiants, comme Edward Gray qui m'a promis de boire un verre une fois la thèse rendue. Je pense aux étudiantes qui ont converti Tite Live et aux promotions TNAH et HN de 2020 qui ont converti des textes du DigilibLT et qui ont donc, indirectement, participé à ma thèse. Je remercie enfin les promotions de M1 et de M2 actuelles. En particulier, puisque je les vois tous les mercredis depuis septembre, je remercie la promotion de M2 d'avoir rendu ces cours de python et de git extrêmement agréables.

Après mes étudiantes et mes étudiants, il me semble important de remercier les enseignants sans qui, je pense, je ne serais pas en train de rédiger ces derniers mots. J'ai une pensée toute particulière pour Mme Christine Batut-Hourquebie, qui m'a donné un véritable amour de la langue latine et de la grammaire et qui m'a accompagné jusque dans ma préparation du bac ou de ma licence 1 (ou des deux, ma mémoire commence à faire défaut), lorsque perdu dans l'apprentissage du grec elle me donna quelques cours particuliers pour me débloquer. J'ai une pensée pour ma professeur d'allemand, Mme Bourciez-Gros, qui m'a donné un amour des langues étrangères et m'a probablement donné ma rigueur grammaticale dans l'apprentissage de ces dernières. Je pense aussi à Mme Delamarre, qui m'a fait découvrir Jules Supervielle, mais surtout qui m'a soutenu dans mon départ vers des horizons moins \enquote{scientifiques}, je pense aux autres professeurs, que je ne peux tous et toutes nommer, qui sont déterminants dans un parcours.

Je remercie Aurélien Berra, pour tant de choses: la découverte des humanités numériques dans le séminaire ESHN, son soutien dans ma candidature au King's College en 2013, le réseau de DHers dans lequel il m'a permis de m'intégrer, son invitation à évaluer mon premier mémoire en lettres classiques, son amitié sincère et les quelques verres bus depuis le premier séminaire.

Je remercie mes collègues du King's, en particulier Stuart Dunn qui m'a fait l'honneur d'être le directeur de mon mémoire et qui m'a aussi remis sur la voie de la diplomation alors que j'hésitais à ne poursuivre mon métier de \textit{research engineer} sans redevenir étudiant. Je remercie mes collègues de Perseus, de Perseids et d'Open Greek and Latin, Matt Munson et Bridget Almas, qui ont fait des années d'ingénieur des années de formation. Bridget a toujours su m'écouter râler, Matt m'a donné une clef pour enfin débloquer ma peur de la page blanche dans ma thèse. Je remercie Marie-Claire Beaulieu qui encore il y a quelques mois m'a fait confiance pour la suite de \textit{Perseids}.

Je remercie le cercle numérique de jeunes chercheurs que j'ai découvert à l'École nationale des chartes: Simon Gabay, Marie Puren, Florian Cafiero, Jean-Baptiste Camps, Nicolas Perreaux, Ariane Pinche, Vincent Jolivet. Je remercie tout particulièrement Simon Gabay et Marie Puren pour leurs relectures exhaustives, exaspérées peut-être parfois, qui permettent à cette thèse d'être un peu plus lisible qu'elle allait l'être. Je remercie l'ensemble de ce cercle pour les conseils en matière de carrière ou de recherche, et leur soutien.

Je remercie enfin mes amis d'enfance, car ils m'ont permis de m'échapper parfois de la recherche, ou tout simplement m'ont récupéré ici et là quand je revenais d'un pays ou d'une ville éloignée et que les transports faisaient défaut. Je remercie mes amis de l'université, qui ont toujours été des soutiens sans faille, en particulier Anthony Glaise à qui je dois d'avoir mieux compris le grec et qui a bien voulu me suivre dans cette folle idée d'annoter linguistiquement des textes latins.

Je remercie ma famille, qui m'a accompagné depuis tant d'années. Je pense à ma nourrice Josette et à son fils Vincent, qui, alors que j'arrivais fraîchement étudiant à Lille me trouva un emploi de serveur pour financer mon année. Je pense à mes tantes et oncles qui m'ont accompagné au musée ou dans des librairies ou tout simplement par des encouragements. Je pense à mes cousins, en particulier à Maureen et Thomas qui m'ont hébergé dans la banlieue d'Épinay alors que je devenais étudiant parisien. Je pense à mes grands-parents, en particulier ma grand-mère qui ne savait pas le grec et qui me fit réviser mes déclinaisons grâce à une ardoise velleda et qui nous a quittés à un mois du dépôt. Je pense à mon grand-père, qui fut mon colocataire pendant quelque temps alors que j'entrais à l'EHESS puis aux Chartes. Je pense à mes beaux-parents aussi qui m'ont accueilli alors que je n'étais pas lyonnais et qui nous ont soutenus dans nos pérégrinations doctorales.

Je pense à mes parents, Dominique et Patricia, qui m'ont toujours poussé à faire ce que j'aimais. C'est ainsi que j'allie finalement le latin et l'informatique. Je pense à ma soeur, qui, avec mes parents, a toujours su être là, en particulier quand il fallait garder notre fils dans les dernières années de thèse. Mes derniers remerciements sont réservés à ma femme et à mon fils. À ma femme, car elle a été un roc, qu'elle m'a permis de me relancer dans la recherche, qu'elle m'a permis de rencontrer mon directeur de thèse, et qu'elle m'a donné confiance en ma capacité d'arriver là où je suis. À mon fils, car il me donnait des perspectives, et qu'il m'a accordé, quelques fois, des nuits sans réveils.