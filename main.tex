\documentclass[12pt]{book}

\usepackage[a4paper]{geometry}
\usepackage[utf8]{inputenc}
\usepackage[LGR, T1]{fontenc}

% Biblio ?
\usepackage[backend=biber, sorting=nyt, style=enc,block=ragged]{biblatex}

\usepackage[ greek.polutoniko,french]{babel}
\usepackage{graphicx}
\usepackage{booktabs}
\usepackage{multirow}
\usepackage{caption}
\usepackage{subcaption}
\usepackage{glossaries}
\usepackage{dirtytalk}
\usepackage{adjustbox}
\usepackage{listings}
\usepackage[onehalfspacing]{setspace}
\usepackage{microtype}
\usepackage{newtxtext}
\usepackage{verbatim} %For comments
\usepackage{afterpage}
%\usepackage{showframe}
\usepackage{titlesec}
\usepackage{tabularx}


% permet de faire une table des matieres par chapitre
\usepackage[french]{minitoc}

% ajoute (entre autre) la bibliographie dans la table des matieres 
\usepackage[nottoc]{tocbibind}
\usepackage{minted}
\usepackage{csquotes}
\usepackage{rotating}
\usepackage[hidelinks]{hyperref}
\usepackage{color}

\definecolor{gray}{rgb}{0.4,0.4,0.4}
\definecolor{darkblue}{rgb}{0.0,0.0,0.6}
\definecolor{cyan}{rgb}{0.0,0.6,0.6}

\lstset{
  basicstyle=\ttfamily,
  showstringspaces=false,
  commentstyle=\color{gray}\upshape,
  columns=fullflexible,
%  frame=single,
  breaklines=true,
  postbreak=\mbox{\textcolor{red}{$\hookrightarrow$}\space}
}

\lstdefinelanguage{XML}
{
  morestring=[b]",
  morestring=[s]{>}{<},
  morecomment=[s]{<?}{?>},
  stringstyle=\color{black},
  identifierstyle=\color{darkblue},
  keywordstyle=\color{cyan},
  morekeywords={xmlns,version,type}% list your attributes here
}

\addbibresource{bibliography/1-digital-classics.bib}
\addbibresource{bibliography/1-introduction-dl.bib}
\addbibresource{bibliography/1-lemmatisation.bib}
\addbibresource{bibliography/TAL.bib}
\addbibresource{bibliography/sources.bib}
\addbibresource{bibliography/Corpora.bib}
\addbibresource{bibliography/commentaires.bib}
\addbibresource{bibliography/4-classification.bib}
\addbibresource{bibliography/annexes.bib}

\newcommand{\starbreak}{%
    \begin{center}%
    $\ast$~$\ast$~$\ast$%
    \end{center}%
}
%\title{Dire la sexualité  : "une étude lexicographique par apprentissage profond"}
\title{Détection d'isotopies par apprentissage profond: l'exemple de la sexualité en latin classique et tardif}
\author{Thibault Clérice}
\date{\today}

\makeglossaries

\newcommand{\newpara}{%
    \vskip 0.5cm %
}

\onehalfspacing



\let\oldquote\quote
\let\endoldquote\endquote
\renewenvironment{quote}[2][]
  {\if\relax\detokenize{#1}\relax
     \def\quoteauthor{#2}%
   \else
     \def\quoteauthor{#2~---~#1}%
   \fi
   \oldquote}{\par\nobreak\smallskip\hfill(\quoteauthor)%
   \endoldquote\addvspace{\bigskipamount}}
   
\newcommand{\sectionbreak}{\clearpage}
\setcounter{secnumdepth}{4}

\titleformat{\paragraph}
{\normalfont\normalsize\bfseries}{\theparagraph}{1em}{}
\titlespacing*{\paragraph}{0pt}{3.25ex plus 1ex minus .2ex}{1.5ex plus .2ex}

\begin{document}

% le titre
\maketitle

% preparation des minitocs
\dominitoc

% table des matieres generale
\tableofcontents


\newglossaryentry{micro-average}{
    name="Micro-average",
    description={Moyenne des résultats par classe pondérés par leur décompte, \textit{eg.} pour un score de 0.9 sur 100 entrées et 0.1 sur 1000 entrées, le micro-average sera de
        \[Micro=\frac{0.9 * 100 + 0.1 * 1000 }{ 1000+100 } = 0.17\]
    }
}

\newglossaryentry{macro-average}{
    name="Macro-average",
    description={Moyenne des résultats par classe sans pondération, \textit{eg.} \[Macro=\frac{0.9 + 0.1}{2} = 0.5\]
    }
}

\newglossaryentry{ecart-type}{
    name="Écart-type",
    description={To Do}
}

\newglossaryentry{precision}{
    name="Précision",
    description={Pourcentage de classification correcte pour une classe, suivant la formule         \[Precision = \frac{VP}{VP-FP}\]
    }
}

% Institutions
\newacronym{ehess}{EHES}{École des Hautes Études en Sciences Sociales}
\newacronym{enc}{ENC}{École Nationale des Chartes}
\newacronym{ciham}{CIHAM}{Centre Interuniversitaire d'Histoire et d'Archéologie Médiévales}
\newacronym{neh}{NEH}{National Endowment for the Humanities}
\newacronym{nsf}{NSF}{National Science Foundation}
\newacronym{chs}{CHS}{Center for Hellenic Studies}
\newacronym{iso}{ISO}{International Standard Organisation}
\newacronym{ucla}{UCLA}{University of California Los Angeles}
\newacronym{arcep}{ARCEP}{Autorité de régulation des communications électroniques}
\newacronym{adho}{ADHO}{Alliance of Digital Humanities Organizations}
\newacronym{LC}{LC}{Library of Congress}
\newacronym{VIAF}{VIAF}{Virtual International Authority File}
% Projets et corpus

\newacronym{ldlt}{LDLT}{Library of Digital Latin Texts}
\newacronym{csel}{CSEL}{Corpus scriptorum ecclesiasticorum latinorum}
\newacronym{ogl}{OGL}{Open Greek and Latin}
\newacronym{phi}{PHI}{Packard Humanites Institute}
\newacronym{hccp}{HCCP}{Harvard Classics Computer Project}
\newacronym{pl}{PL}{Patrologia Latina}
\newacronym{pld}{PLD}{Patrologia Latina Database}
\newacronym{bpm}{BPM}{Bulletin de Philosophie Médiévale}
\newacronym{llt}{LLT}{Library of Latin Texts}

% Technologies
\newacronym{ocr}{OCR}{Optical Character Recognition}
\newacronym{cms}{CMS}{Content Management System}
\newacronym{tei}{TEI}{Text Encoding Initiative}
\newacronym{sgml}{SGML}{Standard Generalized Markup Language}
\newacronym{gui}{GUI}{Graphical User Interface}
\newacronym{os}{OS}{Operating System}
\newacronym{csv}{CSV}{Comma Separated Values}
\newacronym{ohco}{OHCO}{Ordered Hierarchy of Content Object}
\newacronym{CQL}{CQL}{Corpus Query Language}

% inclusion des chapitres
%- Simon Gabay
- Bridget Almas
- Jean-Baptiste Camps
- Marie Puren et Florien Cafiero
- Christine Battut Hourquebie
- Anthony Glaise
- Matt Munson
- Dominique et Patricia
- Ariane Pinche
- Antonin Clérice

\printglossary[type=\acronymtype,nonumberlist]

% inclusion des chapitres
\chapter*{Introduction}
% pour faire apparaitre l'introduction dans le sommaire
\addcontentsline{toc}{chapter}{Introduction}
% Pour que l'entete soit correcte car chapter* ne redefinit pas l'entete.
\markboth{Introduction}{}
\label{intro}

\begin{quote}[\enquote{Le Poète Martial}]{G.~Boissier}
    \enquote{Martial n’est pas du nombre des poètes dont on entretienne volontiers les écoliers : ses ouvrages, si pleins d’esprit et d’agrément, contiennent des obscénités dégoûtantes, et l’on n’ose rien dire aux jeunes gens des jolies choses qui s’y trouvent, de peur de leur donner l’envie de lire le reste.\footcite{boissier_poete_1900}}
\end{quote}

\begin{quote}[\enquote{Réalisme et poésie chez Martial}]{É.~Wolf}
    \enquote{On ne saurait donc trop féliciter le jury de l'agrégation externe de lettres classiques d'avoir pour la première fois, sauf erreur, dans l'histoire de ce concours, mis ses \textit{Épigrammes} au programme, en choisissant deux livres parmi ceux qui sont les moins susceptibles d'effaroucher.\footcite{wolff_realisme_1997}}
\end{quote}

Que se passe-t-il hors des classes de latin pour que les oeuvres de certains auteurs obtienne une telle réputation ? En effet, qui n'a pas étudié le latin en licence ou en master, ou n'a pas étudié l'histoire romaine sera rapidement interloqué, peut-être même choqué, par la liberté d'expression autour de ce que nous appelons la sexualité dans les textes latins. Pire encore, cet écart entre nos deux sociétés se fait encore plus sentir quand celui qui ne connaît pas découvre -- avec effroi ? -- qu'il était tout à fait accepté d'avoir des peintures suggestives dans les chambres à coucher, que des lampes figuraient sans honte des scènes explicites dont le rôle reproductif n'est pas vraiment avéré, et qu'un sexe en érection fait de pierre, de bois ou de fer pouvait trôner devant la porte d'une maison, accompagné de clochettes qui rythmaient probablement au gré des vents les rues romaines, avec pour seul but connu d'éloigner le mauvais oeil\footcite{parker_bells_2018}. Si l'on ne connaît pas, du côté littéraire, la poésie complète de Catulle, les \textit{Priapées} ou les \textit{Épigrammes} de Martial\footnote{Dont Dominique Noguez a donné une brillante traduction littéraire en plus d'une traduction littérale à même de donner un aperçu de la richesse d'un tel auteur.}, on n'a probablement pas découvert cette partie des textes latins et de la culture romaine. Et pourtant, même Caton, connu pour sa rigueur extrême, aurait conseillé aux jeunes hommes d'aller au bordel, dans les limites du raisonnable: cela évitait de s'approcher des femmes mariées\footnote{Horace, \textit{Satires}, 1, 2, 31--35: \enquote{lorsqu'un hideux désir a gonflé leurs veines, c'est qu'il convient aux jeunes de descendre [au bordel], plutôt que de broyer les épouses des autres}, \textcite[p.~30]{puccini_delbey_vie_2010}.}.

La question du statut de la sexualité romaine se pose donc, et se renouvelle même depuis cinquante ans. Mais elle n'est pas juste affaire d'obscénités: si Martial est le plus grand utilisateur de \textit{pedico} (enculer), \textit{irrumo} (violer la bouche) ou \textit{futuo} (baiser), le reste de la littérature latine n'est pas sans parler de sexe. L'obscénité est chez Martial au service d'un style, tout comme des tournures figurées le sont dans d'autres genres, plus ou moins \enquote{littéraires}, plus ou moins \enquote{techniques}. Dans la poésie latine, Catulle comme Tibulle parle de sexualité, mais n'empruntent pas nécessairement les mêmes chemins que celui de l'épigramme satirique. Quand Tibulle écrit en \textit{Élégie}, I, 10, 53--54 \enquote{Mais alors les guerres de Vénus s'enflamment, et la femme déplore ses cheveux arrachés et les portes qui ont été forcées}\footnote{\enquote{\textit{Sed Veneris tune bella calent; scissosque capillos Femina perfractas conqueriturque fores.}}}, il ne fait aucun doute que les guerres n'en sont pas vraiment, ni les portes d'ailleurs, et que les cheveux sont probablement plus tirés qu'arrachés. La métaphore filée permet ici de parler de sexe sans qu'aucune vulgarité lexicale n'apparaissent, sans réduire la portée érotique de la scène. Mais la littérature latine n'est pas qu'une affaire de poètes, qu'ils soient vulgaires ou non, d'orateurs et d'historiens, qui sont généralement les trois grands genres d'auteurs lus et connus: elle est aussi la langue de scientifiques (médecins, astrologues, personnes intéressées dans le fonctionnement du règne animal), de l'exégèse biblique, de grammairiens, etc. Or, ces genres à la marge de l'enseignement classique sont aussi l'espace de création lexicales, d'emprunts, d'usages de figure de style afin d'éviter de \enquote{dire les choses}.

Si l'on cherche ainsi à entrer dans l'histoire de la sexualité romaine à travers le prisme de l'écrit et donc du vocabulaire latin, \enquote{dire les choses} n'est donc pas qu'affaire de lexique mais aussi de style. Il n'est alors pas question de lister uniquement les termes obscènes et non-obscènes qui, sans contexte, dans un dictionnaire, feraient référence sans aucun doute à la sexualité. Il s'agit de recenser l'ensemble des formulations qui, dans un texte, montre ou dénote une activité sexuelle, et seulement avec cet partie du \enquote{dire}, il est possible pour la recherche de mieux comprendre la notion romaine de cette \enquote{sexualité}, notion qui elle-même n'existe pas à Rome. James Noel Adams\footnote{Dont nous avons appris le décès dans les derniers mois de rédaction de cette recherche.}, lexicographe et spécialiste du latin vulgaire, a marqué définitivement ce domaine en publiant en 1982 un ouvrage, \textit{The Latin Sexual Vocabulary}\footcite{adams}, où il s'efforce justement d'embrasser l'expression de la sexualité dans toutes ses variations, et d'en établir une catalogue raisonné.

\paragraph{Le latiniste et l'ordinateur}

Mais J.~N.~Adams a produit ce travail à un âge où l'ordinateur, probable medium par lequel ce texte est lu, n'était pas encore un outil indispensable à la recherche. En 1982, ni le Packhard Humanities Institute et son corpus latin, ni le \textit{Perseus Projet}, ni même pour les périodes tardive la \textit{Library of Latin Texts} de Brepols et la \textit{Patrologia Latina Database} n'existaient. Le \textit{Thesaurus Languae Latinae} (TLL), dictionnaire unilingue constitué depuis plus de cent ans à Munich, régnait alors presque seul comme forme de base de données: pour chaque entrée du dictionnaire, une liste de sens puis de références pouvait nourrir suffisamment une recherche en prenant en compte les publications sur le sujet. En 2022, les études latines s'appuient bien sûr toujours sur le TLL et des éditions scientifiques papiers, mais l'apparition de corpus numériques massifs, d'outils de recherche appropriés, de base de données bibliographiques ont permis d'atteindre bien plus facilement une masse d'information auparavant difficilement accessible.

Les années 1980 sont le début d'une massification de l'accès à l'informatique, d'abord dans les lieux de vie commune (bibliothèques, travail, écoles) puis dans les foyers, avec une translation Amérique du Nord -- Reste du Nord Économique (dont Europe de l'Ouest) -- Sud Économique. L'arrivée du \textit{web} et sa propre massification dans les années 2000 ouvrent, dans le monde de la recherche, la porte à d'autres changements épistémologiques. Dans le milieu des lettres classiques et des sciences humaines en général, l'un de ces changements est indéniablement celui de l'explosion des méthodes dites des \enquote{\textit{Humanities Computing}}, des \enquote{\textit{Digital Humanities}} puis des \enquote{Humanités Numériques}. Sans chercher nécessairement à redéfinir ces dernières sur de nombreuses pages et prétendre à une définition universelle, nous proposerons donc la nôtre. Au même titre que la paléographie ou l'ecdotique, les humanités numériques sont pour nous des \enquote{sciences auxiliaires des sciences humaines}. C'est-à-dire qu'il s'agit d'un domaine qui connaît sa propre recherche, ses propres débats et ses propres ambitions, mais qui a vocation à s'appliquer \textit{in fine} à des sujets de sciences humaines et à être mis en pratique. 

Aborder l'expression écrite de la sexualité latine sans avoir recours aux humanités numériques et aux derniers développements informatiques n'aurait pas eu de sens, car il n'aurait pas été possible, probablement dépasser le travail d'Adams et de ses prédécesseurs. Mais tout comme Adams n'a pas fait totalement table rase du passé\footnote{Il reste assez décrié pour son attitude envers ses prédécesseurs et ses pairs. \textit{Cf.} \textcite{richlin_sexual_1978}.}, il nous faut bâtir ce travail sur celui d'Adams. Notre recherche se construit autour d'une simple question: est-il possible, à partir du travail pré-existant d'Adams, de venir compléter celui-ci avec des méthodes computationnelles ? Si détecter les moments où la littérature latine parle de sexualité est une option ouverte par les méthodes quantitatives, alors une nouvelle voix s'ouvre aussi pour notre connaissance du domaine latin en dehors même de la sexualité. Le latin étant un corpus fermé, auquel de nombreuses questions (lexicographiques comme anthropologiques) se heurtent face à son ampleur, produire une méthode autorisant les chercheurs à construire des relevés qui dépassent la recherche d'occurrences de forme est un pas vers la constitution de bases de données secondaires -- nous parlerons d'\enquote{exempliers numériques} -- permettant elles-mêmes de faire avancer le savoir.
% Définition Adams et séxualités

\paragraph{L'isotopie et le dépassement de l'occurrence}

Dans son article de 1985, François Rastier définit le concept d'isotopie comme \enquote{la récurrence d'un même trait sémantique\footcite{rastier_isotopie_1985}}. Le sème est chez Rastier un \enquote{élément de signification {[...]} commun à différent mots {[}\textit{sème générique}{]}\footcite{pincemin1999semantique}}. Une approche par le principe de l'isotopie, et non du vocabulaire, est primordiale pour aborder le travail d'Adams, car elle permet de dépasser la limite du glossaire en accordant aux sèmes une place d'envergure dans le traitement d'une thématique particulière, ici celle de la sexualité. Par exemple, quand Tibulle parle de \enquote{guerres de Vénus}, c'est l'ajout des sèmes /confrontation/, /corps à corps/, /deux camps/, /soumission/ même peut-être et /amour/ (de déesse) qui mènent à une image claire pour le lecteur, celle de deux corps nus qui cherchent à \enquote{soumettre} ou à \enquote{vaincre} l'autre dans un désir érotique\footnote{La métaphore guerrière pour l'amour (\enquote{\textit{Love is war}}) est fréquente et produit bons nombres d'activations sémantiques. \textcite{lakoff_metaphors_2003}}. La notion d'isotopie pousse à dépasser le simple relevé de formes pour devenir un relevés d'extraits et de contextes, ceux sans qui il n'est pas possible de faire sens.

Les notions d'extraits et de contextes sont elles-mêmes développées par F.~Rastier et son \enquote{école}. Où se situe la limite d'un contexte ? Pour Bénédicte Pincemin, un \enquote{extrait} est opposé à \enquote{une totalité et [à une] unité contextualisante\footcite{pincemin1999semantique}}, mais c'est bien à partir d'extraits que se fait la recherche de co-occurrence, et pour \enquote{forcer le trait} comme le fait Damon Mayaffre\footcite{mayaffre2008occurrence}, tout est problablement un extrait d'autre chose, tant les combinaisons sont possibles et les niveaux variés: mot, mots, phrases, paragraphes, section, livre, volume, série, genre, période, etc. Évidemment, B.~Pincemin ne s'oppose pas à l'utilisation de l'extrait, mais elle en définit les limites à prendre en compte: pour qu'un extrait \enquote{fasse sens}, il faut qu'il soit, bonant malant, contextualisant. Il faut donc, si l'on souhaite relever des isotopies, s'assurer de ne pas oublier la notion de répétition de sèmes. Mais F.~Rastier pousse la notion de contexte beaucoup plus loin, en insistant sur l'importance de la \enquote{présomption d'isotopie\footcite{rastier_isotopie_1985}}, responsable de l'actualisation des sèmes. Cette présomption signifie que le lecteur s'attend à trouver des thématiques, des répétitions thématiques, en fonction du texte qu'il lit. Pour lui, le \enquote{langage neutre, purement dénotatif} n'existe pas. Les textes relèvent donc de \enquote{pratiques sociales}, partagent des traits génériques voire thématiques, et font partie d'un \enquote{discours} qui le régit. Dans ce contexte, il est non seulement important de relever les unités textuelles qui, ensemble, permettent de déceler l'isotopie, mais d'aider à \enquote{présumer} son existence en fournissant les éléments suffisants à l'identification des traits extralinguistiques.  L'isotopie de la sexualité, ou isotopie sexuelle chez certains auteurs\footcite{leon_semes_1976}, a d'ailleurs la chance de s'exprimer dans un nombre de registres incroyables, auxquels on a parfois donné des noms: quand on parle de texte pornographie ou de texte érotique, on distingue une intensité de ce qui est montré et un degré de figuratif avant tout.

Certaines expressions et certaines isotopies ne peuvent être comprises qu'à travers le prisme du sociolecte et du genre littéraire. Prenons deux exemples issus des oeuvres de Cicéron. En \textit{Fam.} 9.22.2, l'orateur propose d'éviter les succession des mots \enquote{\textit{cum nos}} ou encore \enquote{\textit{illam dicam}} car elles provoquent des associations de syllabes qui produisent des \enquote{obscénités}, ici \textit{cunnus} (chatte) et \textit{landicam} (clitoris)\footnote{C'est ce que les grammairiens appellent un \textit{kakemphaton}. \textit{Cf.} \textcite{nicolas2007gros}}. On comprend d'autant plus ce qu'il ne faut pas lire ou entendre d'une part parce que Cicéron dit que c'est vulgaire (\enquote{\textit{Nam obscenus est. [...] potuit obscenius?}}: \enquote{En effet, c'est obscène [...] Puisse-t-il exister plus obscène ?}) et d'autre part parce que la notion d'obscénité chez Cicéron en particulier nous fait attendre tout mot, quelque soit son registre, qui dépasse le sien retenu d'orateur. Si \textit{cunnus} est vulgaire par exemple, \textit{landica}, dans les attestations que nous avons, est généralement réservé aux textes médicaux\footnote{Caelius Aurelianus, \textit{Gynaeciorum}, 1.13, 2.1, 2.112; Muscio, \textit{Gynaecia}, 8; une attestation dans les \textit{Priapées}, 79. Peut-être que le terme était plus vulgaire à l'époque de Cicéron et qu'il a perdu sa force au fil des siècles. \textit{Cf.} \enquote{con} en français.}. Le deuxième exemple chez Cicéron est celui de \textit{lascivus} dans la qualification de Pompée, un ennemi politique venant de perdre un jugement (\textit{Att.}, 2.3): \enquote{\textit{Epicratem suspicor, ut scribis, lascivum fuisse}} (\enquote{Je suspecte qu'Épicrate (Pompée) avait l'air, comme tu l'écris, \textit{lascivus}}. \textit{Lascivus} est un mot généralement utilisés par les poètes, pour parler de jeunes enfants ou esclaves (\textit{pueri}), et de jeunes femmes dans des contextes de jeux, en particulier sexuels. Plus tard, il intervient chez les auteurs chrétiens presque uniquement pour critiquer une attitude érotique. \textit{Lascivus} n'est utilisé en prose dans les genres historiques et dans la bouche d'orateurs que pour traduire une féminité décriée\footnote{Chez Ausone, pour Othon (3, 24); dans les \textit{Histoires Augustes}, à propos d'Hadrien (14, 11).}. Ici, le sème /sexuel/ de \textit{lascivus}, celui /grec/ du surnom /Epicrate/, celui encore de /pouvoir/ dans l'étymologie de cette appellation grecque -- particulièrement dans un contexte de défaite juridique --, le caractère inédit de l'usage du terme \textit{lascivus} chez Cicéron et ses pairs, la situation (que vient faire une \textit{lascivité} dans une histoire de sortie de procés ?), et enfin l'importance portée à la description de ses habits dans la phrase qui suit (souliers et bandelettes blanches) participent au renforcement du mot et posent même un doute sur sa \enquote{vulgarité} chez Cicéron (d'autant plus qu'il s'agit ici d'une lettre, et non d'un discours: peut-être qu'un relâchement est possible ?). On pourrait presque traduire, en trahissant un peu, en poussant la vulgarité, \enquote{qu'Épicrate avait l'air de s'être fait baiser}.
% Définition Isotopie
% Au moment de la définition d'isotopie, parler d'isotopie de la sexualité ou d'isotopie sexuelle

Nous ne sommes pas les premiers à se poser la question d'un traitement automatique de la détection isotopique. F.~Rastier lui-même note l'importance des corpus, leur utilité pour le linguiste et surtout les traitements qu'ils permettent déjà dans les années 1990\footcite{rastier_semantique_1996}. Les premières approches pour la détection d'isotopie ou leur collection a été celle de l'analyse lexicale, cherchant des co-occurrences particulières dans de vastes corpus: c'est par exemple ce que fait Damon Mayaffre dans sa recherche sur les discours présidentiels\footcite{mayaffre2008occurrence}. Dans une autre mesure, les travaux sur les discours des premiers ministres chez Sjöblom et Leblanc reprennent cette ambition en comparant plusieurs approches quantitatives des textes. Mais ces deux exemples ont une méthode particulière: c'est à partir de termes et de relevés de termes, d'occurrences et de co-occurrences qu'ils vont essayer de trouver les isotopies qui forment les textes, voire qui les distinguent d'un auteur ou d'une période à l'autre. Or, Rastier établit une nécessité de \enquote{mettre  au  point  un  système  d’aide à l’analyse  sémantique  qui dépasse les méthodes fondées sur les co-occurrences de mots clé, et qui permette de sélectionner les sous-corpus pertinents en fonction des tâches à accomplir\footcite[p.~31]{rastier_semantique_1996}}. S'il semble penser à la détection de multiples isotopies dans ce contexte, nous reprenons dans notre travail la notion de présomption d'isotopie: nous chercherons à savoir si un extrait de texte porte une isotopie sexuelle plutôt que d'essayer de voir si, parmi les isotopies présentes, l'une d'entre elles est sexuelle.

\paragraph{Vers une méthode reproductible}

\enquote{Pour le linguiste, le corpus est un outil de travail essentiel\footcite{pincemin1999semantique}}, mais il est devenu important dans bien d'autres domaines. Si un véritable intérêt pour

% Corpus: revoir les trucs sur l'isotopie, y a de bonnes citations.

% Généricisation de l'objet
% Reproductibilité et réutilisabilité
% Manière de transmettre les outils
% Comment on peut en servir ?

% Dans mon introduction générale, un engagement à la fois méthodologique et expérimentale fort: je travaille pour bâtir un modèle, à réfléchir à partir d'outils existants, ma thèse ne doit pas être lu comme une thèse sur la sexualité mais comme une approche stylistique guidée par le TAL ? Peut-être différent des prédateurs sexuels etc. mais en fait détection de l'implicite.
% BTW, limites floues de la zone de compréhension ?
% Défricher et construire une méthodo, un système d'enquête

\paragraph{Structure de l'étude}

Après cinq années de recherche, nous présentons donc un travail qui se veut pluridisciplinaire, résolument ré-exploitable, pédagogique (si nous avons réussi), et parfois indécent - par (malin) plaisir mais aussi par nécessité. Il a été important pour nous d'inscrire notre démarche \enquote{d'humanités numériques} dans une histoire qui est propre à celles-ci, en produisant quelques fois un rassemblement inédit d'une documentation sur l'histoire de sa collaboration, et parfois de ses confrontations, avec les études latines. En qualité d'ancien ingénieur ayant participé à plusieurs projets, et en qualité d'ami ou de collègues d'éditeurs, nous avons tenté de redonner leur place, aux côtés des chefs de projet, à ces \enquote{petites mains} (quand nous les avons retrouvées et dans la lignée de travaux comme ceux de Julianne Nyhan\footcite{nyhan2017uncovering}) dont l'activité fut fondamentale pour la réussite de ces initiatives. Bien trop de personnes sont ignorées dans l'histoire de ces corpus: bien que nous ayons été employé par Perseus, il nous aura fallu rédiger une thèse pour enfin découvrir qu'Elli Mylonas a été non seulement une membre fondatrice du projet, mais qu'elle porte probablement le plus de responsabilité dans le succès de la pérennisation des données d'origine. Les données, comme les logiciels, sont des produits scientifiques qui aujourd'hui sont indissociables de la recherche.
% Introduction

Nous traiterons notre sujet en quatre grandes chapitres. Le premier chapitre présente le corpus, le second chapitre, le plus court, sert d'avantage d'appui pédagogique pour comprendre les systèmes d'apprentissage machine à l'ère de l'apprentissage profond, le troisième s'intérèsse à la question de la lemmatisation et à l'annotation linguistique du latin, et le quatrième enfin présente notre expérience finale, à savoir la détection automatique d'isotopie.

Dans notre premier chapitre, nous présentons deux corpus, dans leur production comme dans leur caractéristiques: le premier est celui des sources latines numériques que nous pouvons exploiter (le meta-corpus), le second est celui des extraits latins récupérés à partir du travail de James Noel Adams, le \textit{Latin Vocabulary of Sexuality}\footcite{adams}. Dans ce cadre, nous commençons par aborder la question des corpus sous l'angle de leur histoire, particulièrement riche depuis les années 1980, tant du point de vue des changements techniques que des acteurs ou des orientations thématiques. Ensuite, nous développons la question de la production du meta-corpus. Constitué de sources inédites numériquement et issues de projets historiques, il pose la question des besoins opérationnels pour mener à bien une recherche quantitative mais aussi des formats dans l'objectif d'une pérennisation de notre travail. Enfin, nous traitons la question spécifique de l'isotopie sexuelle en latin, sous le prisme de la constitution d'une compilation d'extraits présentant cette particularité sémantique. Nous réintroduisons à cette occasion la question de l'histoire de la sexualité mais aussi de sa lexicographie en latin.

Notre second chapitre porte sur le vocabulaire et les méthodes du traitement automatique des langues. Il est beaucoup plus court que l'ensemble des autres chapitres, et a pour vocation de permettre de comprendre les deux suivants. À la limite de l'annexe, il nous a semblé nécessaire de le laisser dans le corps de texte, afin de s'assurer que les lecteurs néophytes en la matière puissent comprendre la suite, quitte à y revenir. Il revient d'abord sur des considérations générales (qu'est-ce qu'un texte pour un algorithme ?) et sur les méthodes d'entraînement et d'évaluation en traitement automatique des langues. Le reste du chapitre se concentre à présenter des types de réseaux et en essayant de mettre à disposition de chacun une compréhension minimale de ses spécificités.

Notre troisième chapitre est entièrement dédié à la lemmatisation du latin. Nous profitons de la question de la lemmatisation, sous-entendue numérique, pour la recontextualiser dans une histoire qui dépasse le cadre de l'ordinateur: celle des concordanciers et des index. Nous étudions ensuite la riche histoire et les révolutions qu'ont connues les outils de lemmatisation en fonction de l'évolution des puissances de calcul. Les dernières méthodes reposant sur un entraînement supervisé, elles requièrent des corpus: nous présentons donc à cet effets les différents corpus. Nous mettons en place en suite une expérience, visant à la fois à faire émerger des configurations de réseaux neuronaux les plus efficaces pour notre recherche et pour le latin, mais aussi à évaluer les limites de tels outils. Nous présentons enfin le modèle de lemmatisation final, que nous appelons LASLA+ en remerciements au corpus fourni par le laboratoire liégois.

Notre quatrième chapitre aborde enfin la question des modèles de détection d'isotopie. Nous établissons d'abord un large panorama des méthodes qui peuvent se rapprocher de notre sujet et des tentatives d'incursions du traitement statistique dans le domaine des études antiques, afin d'en retirer des éléments à expérimenter en termes de technologies ou d'utilisations des données.

% Comment orienter la lecture du jury via l'exposé et l'introduction: de quoi parle la thèse ?

% 


\chapter{Constitution du corpus}

\section{Une histoire des corpus latins numériques}

Le travail sur la langue latine nécessite \textit{de facto} des corpus, et \textit{a priori} en nécessite des numériques s'il s'agit d'une approche computationnelle. Si la tradition papier des corpus académiques des Teubner ou des Belles Lettres s'inscrira bientôt dans leur troisième centenaire\footnote{Si l'éditeur Teubner semble s'attaquer dès les années 1810 à l'impression d'ouvrages philologiques, la \textit{Bibliotheca Scriptorum Graecorum et Latinorum Teubneriana} ne voit le jour \enquote{qu'en} 1849. Elle prédate les deux autres collections généralistes majeurs, la collection \textit{Oxford Classical Texts} et les \textit{Belles Lettres}. \cite{andre_cent-cinquante_1974}}, l'histoire des corpus littéraires numériques n'a fêté que très récemment son cinquantenaire avec les prémices du \textit{Thesaurus Linguae Graecae}.

Aussi, nous proposons de revenir sur les cinquante dernières années de numérisation et de mise à disposition des textes latins, principalement des textes littéraires. Nous proposons un découpage en trois périodes de cette révolution numérique des corpus: la première concerne l'apparition des disquettes et CD de corpus qui émaille les décennies 1960 à 1990; la seconde (1995-2005) concerne l'apparition en ligne de ces premiers corpus, mais aussi une autre forme de révolution, celle des corpus non académiques; la troisième (2005-aujourd'hui) concerne enfin l'expansion du numérique comme version fondamentale des corpus et l'apparition de \enquote{méga corpus}.

Il faudra cependant commencer cette introduction au chapitre par un avertissement: la documentation disponible sur la publication des corpus numériques est presque inexistante, souvent de seconde main, à travers de rares témoignages ou d'encore plus rares citations et ne permet souvent pas de retrouver avec toute l'exactitude souhaitée la première date de publication de tel ou tel ouvrage. Jusqu'à aujourd'hui, la citation des corpus numériques n'est pas entrée dans les usages, tout comme la citation des oeuvres quand on en fait le commentaire: rares sont les chercheurs qui spécifient en bibliographie l'édition précise qu'ils ont utilisée quand ils mentionnent Virgile ou Martial. Aussi, nous nous excusons d'avance  si des informations présentées ici sont inexactes, si des corpus oubliés le sont, et nous invitons grandement notre champ à capturer rapidement cette histoire, l'archiver avant qu'il ne soit trop tard: si les décades 70 et 80 ne sont pas très loin, elles semblent bien floues sur le plan de l'histoire des corpus\footnote{Il nous semble propice, pour qui voudra, de s'intéresser à une histoire orale des projets fondateurs, à une recherche en archives pour chacun de ces projets, avant qu'il ne soit trop tard.}.

\subsection{Les \enquote{incunables} du numérique}

Nous nous intéressons ici à la naissance des corpus numériques littéraires, ayant pour vocation d'être lus ou utilisés pour des recherches dès leur conception numérique. À cette fin, nous excluons les travaux d'annotation linguistique et de construction de concordanciers de R. Busa ou du LASLA  car ils avaient des ambitions plus spécifiques et ne proposaient pas comme but premier de pouvoir lire le texte\footnote{Mais nous en parlerons plus tard, \textit{cf.} \ref{lemmatisation:concordanciers}}. Or, il est difficile comme nous le disions plus haut de savoir à quand remontent les premières productions de corpus.

% La documentation d’époque et de première main est très pauvre sur ces outils (date, recherche en cours), on les retrouve principalement dans des reviews

\subsubsection{Années 60, années 80: premiers corpus, premiers CD-ROMs}

Le corpus littéraire le plus ancien dont il est fait mention est celui financé par le \textit{National Endowment for the Humanities} (NEH) et cité par Theodore F. Brunner dans son article rétrospectif de 1993 centré sur la recherche états-unienne \textit{Classics and the Computer}\footcite{brunner_classics_1993}. En 1968, Nathan Greenberg et John. J. Bateman obtiennent un financement de la NEH de 19.800\$ \footcite{noauthor_neh_2018} complété par 40.000\$ de financeurs secondaires, dont IBM\footnote{D'après \cite{brunner_classics_1993}: le \textit{Digital Computer Laboratory} de l'université d'Illinois, the \textit{Kiewit Computation Center} du Dartmouth College, la \textit{National Science Foundation}, la fondation Ford et l'entreprise IBM donc.}. Avec ce dernier, ils organisent une école d'été titrée \textit{Summer Institute in Computer Applications to Classical Studies}\footnote{L'équivalent de 59~800\$ au 31 janvier 1968 est de 479~745\$ en août 2021 d'après le calculateur d'inflation du \textit{Bureau of Labor Statistics}, \cite{noauthor_cpi_nodate})}. Cet événement donne naissance à un corpus d'une vingtaine d'oeuvres grecques et latines plus ou moins complètes: on y trouve à côté des classiques homériques des morceaux d'oeuvres, parfois inattendus, dont la découpe est particulière, tel le poème 64 de Catulle qui est édité seul, trois oeuvres de l'\textit{Appendix Vergiliana}, les livres I, IV, IX et XII de l'\textit{Énéide}. En dehors d'un rapport, ce corpus ne semble pas avoir eu une vie particulièrement riche, ni de nom d'ailleurs: il est pris en charge par l'\textit{American Philological Association}, est dupliqué à la demande par des institutions, mais très vite se voit couper de tous fonds supplémentaires, là où, comme le note 20 ans plus tard Brunner, le fond des monographies n'est pas touché\footcite{brunner_classics_1993}.

Au début des années 70, nous trouvons la trace d'un seul autre corpus, celui du \textit{Thesaurus Linguae Grecae} (TLG) dont les prémices remontent à 1971\footcite{brunner_classics_1993} et dont la naissance est actée en 1973\footnote{Parmi les articles cités par T. F. Brunner lui-même sur la fondation du TLG, au moins un est indisponible en France: \cite{hugues_homer_1987}}. Si son nom est dérivé d'un projet humaniste du 16\textsuperscript{e} siècle et est en écho à celui du dictionnaire \textit{Thesaurus Linguae Latinae} (TLL), il ne s'agit que d'une simple inscription dans une tradition des grands travaux humanistes: à contrario des deux derniers, ce projet se veut dès les premières conférences un corpus de texte et non un thésaurus, un dictionnaire fortement enrichi. Les premières versions du corpus voient rapidement le jour pour atteindre 61 millions de mots en 1988\footcite{brunner_overcoming_1988}, via une externalisation de la copie \enquote{manuelle} des volumes en Corée du Sud (1972-1980) puis aux Philippines\footcite[p. 111]{helgerson_cd-rom_1988}. Ce corpus pose une difficulté de taille, à savoir son alphabet: en 1972, seuls les caractères ASCII\footnote{\textit{American Standard Code for Information Interchange}} existent informatiquement, ils sont au nombre de 128 et couvrent les nombres, les caractères latins hors diacritiques et les signes de ponctuation. Il faut trouver une solution pour les caractères grecs, et c'est un certain David W. Packard qui trouve une méthode pour encoder ces derniers, le BetaCode. Cette méthode de transcription, dont il reste des traces encore aujourd'hui dans des fichiers de Perseus, propose l'encodage des diacritiques via les signes de ponctuation: ainsi, la parenthèse \texttt{)} remplace l'esprit doux, tandis que le pipe \texttt{|} représente le iota souscrit, par exemple, \textgreek{αναλαβόντες δὲ καθ᾽ ἕκαστον} donne \texttt{analabo/ntes de\ kaq`e(/kaston.}


Présent à la réunion de fondation du projet TLG et résolveur du problème d'encodage, David W. Packard n'est pas seulement important pour ce dernier: il fonde le \acrfull{phi} (\acrshort{phi}) en 1987\footcite{helgerson_cd-rom_1988}, institut ayant pour visée de produire des corpus, dont un équivalent latin du TLG\footnote{Le corpus latin n'est qu'un des multiples corpus du PHI, même si l'on utilise souvent PHI uniquement pour se référer au corpus latin.}. Docteur en langues anciennes depuis 1967 et spécialiste des tablettes en linéaire A, il est nommé en octobre 1968 comme membre du \textit{Special Committee for Computer Problems} aux côtés de N. A. Greenberg, Stephen Waite, William H. Willis et Robert Dyer, qui en est le président. La première version CD-ROM apparait en 1991 (PHI\#5), les versions précédentes n'ont pas laissé beaucoup de traces et il existe un certain flou autour de la chronologie: un article de 1991 de J. Raben mentionne qu'il est \enquote{en cours de direction par David W. Packard}\footcite{raben_humanities_1991}, un article de S. Hockey parle d'environ 8 millions de mots en 1994\footcite{hockey_electronic_1994}. Il semble qu'un premier CD de textes latins, en particulier de la \textit{Bible}, soit publié rapidement, dès 1987 dans le contexte d'un projet annexe du \textit{Center for Computer Analysis of Texts} (CCAT)\footcite{groves_tovs_1990, cornell_greek_1989}.

Le fossé temporel qui sépare les deux \enquote{premiers} projets américains encore présents s'explique d'une part par le coût que représentent ces projets, d'autre part par le manque d'équipement informatique au début des années 1970. Il faut attendre l'avènement du \textit{micro-computer} (micro-ordinateur en français, terme tombé en désuétude pour ordinateur tout simplement ou bien même PC) et de l'Apple II par exemple pour voir une montée de l'équipement informatique. Plus encore, le marché de l'informatique se popularise avec l'arrivée des interfaces utilisateurs graphiques (\acrshort{gui}), notamment à travers l'Apple Lisa (1983)\footcite{noauthor_history_2021} ou l'Apple Macintosh (1984) ou encore leur équivalent DOS et Windows déployés par IBM. Et au-delà même du micro-ordinateur, c'est le standard CD-ROM qui apparait et permet de partager des données beaucoup plus importantes en 1984\footnote{Le standard est créé plus tôt, mais ne s'applique d'abord pas aux données.}. Certains historiens parlent d'une montée en puissance, dans le secondaire comme dans le milieu académique, de l'usage des ordinateurs en classe\footcite{simkin_introduction_1989, latousek_fifty_2001}. Mais le changement s'opère sur toute la population américaine: d'après un rapport de 1999\footcite{kominski1999access}, on voit doubler  entre 1984 et 1989 le nombre de foyers américains ayant un ordinateur, une augmentation de 64\% de l'usage de l'ordinateur à l'école pour les 3-17 ans, de 41.5\% pour les 18+ à l'école\footnote{Par déduction, il devrait s'agir principalement du milieu universitaire.} et de 33\% au travail (\textit{cf.} Table \ref{tab:computer-ownership}).

\begin{table}[ht]
\centering
\begin{tabular}{l|rrr}
                                               & 1984 & 1989 & 1993 \\ \hline  \hline
Foyer avec un ordinateur                       & 7.9  & 14.4 & 22.8 \\ \hline
3-17 ans ayant accès à un ordinateur à l'école & 28.0 & 46.0 & 60.6 \\
18+ ans ayant accès à un ordinateur à l'école  & 30.8 & 43.6 & 53.8 \\
18+ ans ayant accès à un ordinateur au travail & 24.6 & 36.8 & 45.8 \\ \hline
\end{tabular}
\caption{Niveau d'accès et d'usage en \% des ordinateurs aux États-Unis sur la décennie 1984-1993, d'après l'\textit{U.S. Census Bureau, Current Population Survey, October 1984, 1989, 1993} repris par \cite{kominski1999access}}
\label{tab:computer-ownership}
\end{table}

Il faut comprendre à quel point l'histoire des projets américains est intimement liée aux grandes entreprises du domaine de l'informatique. Si elles sont souvent présentes en financement suite à des demandes, comme IBM sur le co-financement NEH de 1968-69, ou Apple comme nous le verrons pour Perseus, elles sont aussi présentes à travers les réseaux sociaux de la côte ouest. En effet, qu'il s'agisse de PHI ou de TLG, des enfants de grands patrons sont à la source du financement des projets: ainsi, David W. Packard (\acrshort{ucla}) est le fils du co-fondateur de Hewlett-Packard et utilise cette ressource pour financer le PHI; de son côté, Marianne McDonald finance le TLG alors qu'elle n'est qu'étudiante en licence grâce à son père, patron de la \textit{Zenith corpustion}, entreprise méconnue en France, mais importante pour les États-Unis puisqu'elle y commercialise alors télévisions et bouquets de chaînes. Le financement de ces entreprises (1 million de dollars offerts par M. McDonald, en 1972, soit environ 6,656 millions de dollars d'août 2021) est constant et semble \enquote{inévitable} pour ces projets jusqu'à la fin des années 80.

\subsubsection{La lente apparition du projet Perseus}

Dans les années 80, en parallèle du développement de PHI et du TLG, un autre futur mastodonte du corpus en lettres classiques commence à se formaliser: Perseus. Mais l'histoire de Perseus ne commence pas comme l'histoire d'un concurrent à PHI et au TLG, mais bien comme un ajout à ces derniers. 

% Contexte de la création de Perseus: le project HCCP
En effet, en 1982, Gregory R. Crane, alors doctorant à Harvard, ainsi que Neel Smith, Kenneth Morrell et Elli Mylonas cherchent à améliorer l'écosystème pour l'étude des langues anciennes sur plateforme informatique. À cette période-ci, il faut comprendre que le TLG n'est disponible que sur sa propre plateforme matérielle et logicielle, à savoir l'Ibycus, développé spécifiquement par D. W. Packard et financé par HP. Or, il s'agit aussi de la période de \enquote{standardisation} de la programmation, notamment à travers le développement d'Unix et de ses clones (dont GNU). Dans un article rétrospectif sur l'histoire du champ, G. R. Crane\footcite{schreibman_classics_2004} parle du développement du moteur de recherche pour le TLG permettant de faire usage des données du TLG. En effet, dès 1994, alors qu'E. Mylonas présente le projet Perseus\footcite{mylonas_perseus_1993}, elle intègre l'histoire de Perseus dans son rapport au TLG: l'équipe historique de Perseus s'intéresse d'abord à produire des ajouts pour le TLG, dont un \enquote{puissant moteur de recherche plein-texte}\footnote{\textit{\enquote{... spawned at Harvard a software project which developed a powerful full-text retrieval system.}}}. Dans son ouvrage massif \textit{Bits, Bytes and Biblical Studies} de 1986\footcite[p. 598]{hughes_bits_1987}, J. J. Hughes parle du \acrfull{hccp} (\acrshort{hccp}) qui cherche alors à développer pour UNIX et en particulier pour Mac un nouveau système complet autour de l'édition, de l'entrée de données et de la recherche plein texte. À cette époque, Perseus ou l'HCCP sont financés tour à tour par IBM, Apple (y compris à travers une stratégie globale d'adoption de la firme à la pomme par Harvard) et Xerox du côté des entreprises.

% HCCP et Morpheus
La fin des années 1980 montre encore l'intérêt d'abord de l'équipe Perseus pour l'amélioration de l'environnement de travail - en grec ancien uniquement pour le moment. Le TLG et le PHI-CCAT proposent depuis quelques années alors un outil pour la lemmatisation et l'annotation morphologique du grec ancien, appelé MORPH et développé encore une fois par David. W. Packard en assembleur puis dans son propre langage de programmation, l'IBYX\footcite[p.554-555]{hughes_bits_1987}. L'équipe de Crane propose donc d'abord d'améliorer MORPH et développe Morpheus, qui gère désormais les accents et les dialectes\footcite{mylonas_perseus_1993} et propose une formalisation par règle de la langue grecque. L'ensemble se repose sur un dictionnaire central, l'\textit{Intermediate Liddell-Scott Lexicon}, ce qui permet donc aux utilisateurs d'avoir un référentiel de lemmes consultable et navigable.

% De l'HCCP à Perseus: compléter le TLG
Et c'est à travers l'ensemble de ce travail autour de l'infrastructure logicielle que l'HCCP finit par devenir le \textit{Perseus project}. Commencé en 1990, le projet ne vise pas à concurrencer PHI et TLG. G. R. Crane et son équipe affirment dès le départ cette absence de concurrence: \enquote{\textit{The Perseus Project, with its broad range of materials, was designed to complement the textual focus of the TLG}}\footcite[p. 134]{mylonas_perseus_1993}. Il va donc chercher à compléter ce dernier en apportant de nouvelles informations, comme - pour la première fois - des traductions des textes classiques et des ressources graphiques. Le premier Perseus vise ainsi à accompagner d'images les corpus textuels disponibles jusqu'ici - on parle alors de 10~000 images à obtenir entre 1990 et 1993 - compilées avec les textes sur \enquote{\textit{compact disks and video disks}}.

% Textes et traductions
Si l'information textuelle en langue originale n'est pas avancée comme étant au centre du projet Perseus, l'équipe promet tout de même d'amasser 100 MB de données d'ici la fin du projet. Le corpus original se veut centré autour du Ve siècle avant notre ère avec des incursions vers d'autres classiques et accompagné de traductions, anciennes, modernisées et modernes fournies par des partenaires\footnote{Les premiers auteurs mentionnés sont \enquote{Eschyle, Sophocle, Hérodote, Pindare, {[...]} Pausanias, Pseudo-Appolodore, les vies grecques de Plutarque, {[...]} Homère, Aristophane, les orateurs attiques, Thucydide, la poésie élégiaque et lyrique, Platon et un peu d'Aristote {[sic]}. Des morceaux intéressants de Diodore de Sicile et Strabon} seront ajoutés plus tard. \cite{mylonas_perseus_1993}}. Cette sélection, plus restreinte que celle du TLG, vise alors les étudiants et non les chercheurs: il s'agit d'accompagner les hellénistes en formation et les non-spécialistes - comme les historiens - dans la lecture des textes en proposant des versions numériques alignées avec leur traduction \footnote{\enquote{\textit{The choice to include translations is to allow students and other scholars who are not fluent readers of Greek to work {[...]} and to broaden the circumstasnces in which Perseus will be consulted.}}, \cite[p. 136]{mylonas_perseus_1993}}. La création des données textuelles est alors faite par copie au clavier, les technologies d'OCR étant trop génératrices d'erreurs à l'époque\footnote{L'équipe a testé l'OCR au début des années 90 et estime alors que le temps de correction n'est pas plus intéressant qu'une copie manuelle.}.

Les années 80 sont des années particulièrement riches technologiquement, nous l'avons vu, et en particulier en termes de standardisation de l'écosystème informatique: partager information et code entre entreprises et consultants, entre chercheurs ou entre projets devient une problématique importante. Et ces années-là voient apparaitre un nouveau langage, le SGML\footnote{\textit{Standard Generalized Markup Language}}, un langage à balise destiné à structurer l'information textuelle plus facilement et adopté par l'\acrfull{iso} (\acrshort{iso}) en 1986. Un an plus tard, 32 chercheurs en sciences humaines et sociales se rencontrent au Vassar College de Poughkeepsie, dans l'état de New York , et posent des principes d'interopérabilités, qu'ils nomment alors les \textit{Poughkeepsie Principles}\footcite{vanhoutte_introduction_2004}. Ces principes\footcite{noauthor_design_1988}, au nombre de 9, définissent les lignes directrices pour la fondation des \textit{Text Encoding Guidelines} et commencent ainsi par celui d'obtenir un \textit{standard} pour l'échange de données dans le contexte des recherches en sciences humaines. Cet objectif est au centre de ce qui devient, en 1990, la \textit{Text Encoding Initiative} et sa première version des \textit{guidelines} qui visent à encadrer la manière d'encoder l'information textuelle et ses métadonnées. La \textit{Text Encoding Initiative} vise alors à \enquote{fournir des \textit{guidelines} explicites qui définissent un format textuel approprié au partage de données et à leur analyse; le format doit être indépendant du point de vue matériel\footnote{Comme nous l'avons vu, la richesse matérielle à l'époque fait qu'il existe encore de grandes possibilités de conflits entre différentes manière de gérer des données à cause de l'implémentation physique du principe informatique.} et de celui du logiciel, rigoureux dans sa définition des objets textuels, facile à utiliser, et compatible avec les standards existants. On attend du SGML de fournir une base adéquate pour ces \textit{guidelines}}\footnote{\enquote{\textit{The primary goal of the Text Encoding Initiative is to provide explicit guidelines which define a text format suitable for data interchange and data analysis; the format should be hardware and software independent, rigorous in its definition of textual objects, easy to use, and compatible with existing standards. The \acrlong{sgml} (\acrshort{sgml}) is expected to provide an adequate basis for the guidelines. }}, \cite{noauthor_design_1988}}.

Il est alors compréhensible, devant cette révolution de l'encodage du texte, de voir le projet Perseus adopter SGML dès sa conceptualisation\footcite[p. 138]{mylonas_perseus_1993}, bien qu'aucun de ses membres fondateurs n'ait participé à la réunion de Poughkeepsie\footnote{Il est intéressant de voir que l'article publié en 93 ne parle pas de TEI directement, mais bien de SGML, tout au plus est renvoyée en notes et bibliographie une mention du travail de Lou Burnard sur la TEI, au même titre que de l'utilisation de la technologie SGML par le département de la défense, \textit{cf.} \cite[notes 8 et 9, p.~155]{mylonas_perseus_1993}}. Ils adoptent en effet ce standard dès le départ comme format d'archivage en estimant que seul un format d'archivage standardisé permettra de survivre aux évolutions technologiques et en particulier de survivre au logiciel utilisé à l'époque, à savoir \textit{Hypercard} sur Mac: plus de vingt ans plus tard, les corpus originaux de Perseus sont toujours disponibles, on ne peut que confirmer cette intuition. Mais cette opportunité prise, il reste aussi à l'équipe de traduire en SGML les pratiques de mise en page et d'édition du domaine de l'antiquité, à savoir ses modes de citation en structures logiques ou éditoriales (chapitre, section, vers, pages de \textit{Stephanus} pour Platon), afin de ne pas rompre avec cette tradition philologique: le passage de l'imprimé au numérique permet ainsi de traduire les informations fournies par la mise en page en métadonnées sur le texte. Ainsi, en dehors de ces informations éditoriales, une annotation supplémentaire dans le SGML de la métrique, des \enquote{types de discours dans la prose historique et rhétorique}, les noms des intervenants dans les pièces est considérée dès la conception du projet\footcite[p. 137]{mylonas_perseus_1993}.

Bien que les données textuelles soient ultimement celles qui nous intéressent pour notre travail, ignorer la partie non textuelle du projet Perseus à sa fondation ne permettrait pas de comprendre en quoi ce projet ne se pose - au départ - pas comme un concurrent au TLG. Pour les ressources sur l'archéologie, Perseus souhaite en effet se constituer comme une \textit{bibliothèque}, avec une couverture dont la sélection est le résultat d'un \enquote{\textit{opportunisme guidé}}\footcite[p. 145]{mylonas_perseus_1993}. L'objectif est de rassembler, pour la première fois sous une forme numérique, un outillage pédagogique et de recherche permettant d'aborder une grande variété d'objets et de thèmes pour la Grèce ancienne: cela comprend photographies, dessins, plans, mais aussi descriptions ou textes d'introduction thématique traitant de la sculpture ancienne par exemple. Selon les fondateurs du projet\footcite[p. 143]{mylonas_perseus_1993}, par manque d'expertise entre autres et de concurrents numériques prédatant ce projet, l'équipe de Perseus va chercher à rassembler les principes de trois modèles:
\begin{itemize}
    \item ceux d'une archive photographique, avec des descriptions sommaires qui se concentrent sur la description de l'image elle-même;
    \item d'une base de données ou d'un catalogue muséal ou de fouilles, centré sur l'objet et concentré sur la description de propriétés, mais sans projet éditorial;
    \item d'une publication plus enrichie, du type \enquote{catalogue archéologique multi-volumes}  proposant à la fois des volumes de textes et des planches, mais nécessitant une plus grande sélection, et donc omission, d'objets.
\end{itemize}
Le résultat de cette sélection doit offrir une modélisation suffisante pour découvrir, se former, et enseigner. Son implémentation suit encore les principes de SGML pour les contenus textuels et une modélisation complexe des métadonnées permettant formellement un enrichissement par des contributeurs extérieurs à l'avenir\footnote{\enquote{\textit{Perseus cannot possibly foot the costs of assembling the quantities {[of information ...]}; there, we must design a system that will not merely permit but encourage collaboration.}}, \cite[p. 148]{mylonas_perseus_1993}. Nous verrons plus tard que cet objectif deviendra un \textit{leitmotiv} de G. Crane à travers les évolutions de Perseus.}

\subsubsection{Période manquante ? L'apparition de la patristique numérique}

% Introduction du CETEDOC et du CLCLT2
Jusqu'au projet Perseus, l'ensemble des efforts se font sur les périodes classiques, canoniques, celles du \enquote{bon grec} ou du \enquote{bon latin}, des orateurs ou dramaturges, des poètes épiques, celles des oeuvres que l'on étudie pour l'agrégation en France. Les pères de l'Église sont rarement inclus dans les projets, et s'ils le sont, ils sont sous-représentés et n'y apparaissent alors que partiellement. Cette scission, entre période chrétienne et période classique, se retrouve aussi dans le travail des corpus: si on trouve le LASLA à Louvain pour s'occuper de la période classique (jusqu'à la fin du Ier siècle environ), un autre laboratoire se fonde en 1968 sous la direction de Paul Tombeur pour traiter des données \enquote{médiévales}: le Centre de Traitement Électronique des Documents, ou CETEDOC\footcite[p. 70]{gueret_analyse_1977}. Ce centre se concentre pendant vingt ans à la production de données similaires à celles du LASLA, des concordances, des données lemmatisées. En 1984\footcite{iogna-prat_centre_1984}, le centre se compose \enquote{d'un ingénieur informaticien, un analyste informaticien, une secrétaire, une (demi-!) {[sic]} assistance, deux universitaires dont P. Tombeur {[...]} et des vacataires}. Les services que propose le centre incluent alors la reprographie de thèses, la mise à disposition des données collectées et l'accueil de chercheur pour faire traiter des textes \enquote{à la mode} du CETEDOC.

% Du CETEDOC au CLCLT
%  Rappel que Tombeur était à Poughkeepsie
Les années 80 représentent cependant un tournant pour le centre: la question de la mise à disposition de corpus \enquote{médiévaux}~-~il faut entendre ici pères de l'Église et textes médiévaux en général~-~se fait de plus en plus pressante par son absence des corpus principaux, PHI et TLG. En 1981, à Liège, au congrès mené par le LASLA sur \enquote{l'informatique et les sciences humaines}, Paul Tombeur parle alors de publier un \textit{Thesaurus Patrum Latinorum}, englobant les textes chrétiens latins et les textes médiévaux publiés dans les collections \textit{Corpus Christianorum, Series latina} et \textit{Continuatio Mediaevalis}\footcite{tombeur_constitution_1981}. Le directeur du centre est présent à Poughkeepsie en 1987 et signe l'appel, répétant ainsi ces nouvelles ambitions\footcite{burnard_report_1988}. Et de fait, en 1991 sort chez Brepols la \textit{CETEDOC Library of Christian Latin Texts on CD-ROM}, ou CLCLT, une base de données comprenant 21 millions de mots et l'équivalent de 300 volumes imprimés\footcite[p. 90]{bucknall_review_1994}. Si T. Bucknall la compare dès lors avec les bases PHI ou TLG, la situation est légèrement différente: le CLCLT est avant tout une base à interroger plus qu'un corpus à lire, et c'est ainsi qu'il est implémenté.

% Possibilités et limites du CLCLT2: impression de 30 résultats par exemple
La base CLCLT consiste alors en une interface donnant accès à un système de recherche (par forme, par groupe de formes, par forme partielle, par proximité entre formes), mais repose sur un séquençage du texte en \textit{sententiae}, des phrases que les éditeurs ont produites dans leur édition. Choix regrettable si l'on en croit les comptes-rendus de l'époque, tant elle produit des disparités: en effet, \enquote{les uns {[éditeurs]} paraissent préférer des phrases très longues {[tandis que]} les autres s'appliquent à hacher menu le discours}\footcite{gryson_nouvelle_1992}. Et de ce séquençage dépend alors bon nombre de recherches qui ne peuvent inclure les éléments de \textit{sententiae} voisines. La base est cependant munie d'un très grand nombre de métadonnées, de notes critiques sur le texte, sur son authenticité et son attribution par exemple. On peut y lire les textes, bien que l'on ait vu plus confortable: les oeuvres ne comprennent pas d'index, et si l'on veut lire le chapitre 14 d'un long ouvrage, il faudra passer de page en page manuellement. Le logiciel est uniquement disponible sur PC, en particulier sous DOS à l'époque. Les résultats sont imprimables, mais les comptes-rendus divergent: si R. Gryson semble indiquer l'absence de limite lors de l'impression\footcite[p. 421]{gryson_nouvelle_1992}, les autres sources, dont T. Bucknall\footcite[p. 94]{bucknall_review_1994}, semblent s'accorder sur une limite pour le téléchargement ou l'impression à 30 lignes consécutives de texte.

% L'apparition de la PLD
Le vide laissé par PHI et le TLG ont cependant intéressé d'autres éditeurs que Paul Tombeur, puisqu'un concurrent au CLCLT apparait au même moment: la \textit{Patrologia Latina Database}, ou PLD, éditée par l'entreprise Chadwyck-Healey. Basée sur une numérisation de la patrologie de Migne, une somme des textes chrétiens du IIe au bas moyen-âge éditée au XIXe siècle économique\footnote{\enquote{Migne présente sa Patrologie comme une \textit{bibliotheca oeconomica} et {[...]} comme étant du bon, bon marché}, \cite[p. 228]{tombeur_pld_1993}}, elle propose sous une interface remaniée \textit{DynaText} et à partir de fichiers en SGML TEI\footcite{smith_dynatext_1993} de lire ou de chercher à l'intérieur d'un immense corpus sur 5 CD-Roms. C'est à notre connaissance le premier projet commercial en SGML TEI, et le premier très large projet qui utilise cette technologie. Contrairement au CLCLT, la PLD fonctionne sous les OS principaux de l'époque (Mac, \enquote{UNIX avec X-Windows} et Windows\footcite{smith_dynatext_1993}). 221 volumes de la PL sont repris et acceptent une recherche plein texte plus ou moins équivalente à celle du CLCLT. Les métadonnées des premières versions sont par contre particulièrement pauvres: les périodes sont divisées sommairement en deux périodes, \textit{medieval}, avant 1500, et \textit{modern}, après cette date. Elle permet par contre la lecture ciblée de documents, et ne nécessite pas, comme le CLCLT, de faire défiler manuellement les contenus. Enfin, contrairement au CLCLT, elle permet l'export du SGML et contient l'apparat critique des textes qu'elle comporte.

Cette collision littéraire et temporelle conduit les deux bases de données à être comparées et à faire naître des controverses. D'abord, car les deux objets ne font pas le même prix: la PLD est annoncée originellement pour 50~000\$ tandis que le CETEDOC l'est pour 3~800, avec des mises à jour bi annuelles\footcite{bucknall_review_1994}. Le prix de la PLD semble varier beaucoup, y compris suite à la réaction du public: on parle de 70~000\$ quand elle fut annoncée sur bandes magnétiques au début des années 90 et de 45~000\$ en précommande dans l'article de Ron W. Crown\footcite{crown_comparing_2000}, de 27~000£ en 1995 chez R. Gryson\footnote{Avec un taux de change en 1995 d'environ 1.55\$ pour 1£, 41~850\$, d'après \cite{noauthor_british_2021}}, de prix négociés chez certaines petites bibliothèques aussi bas que 5~000\$\footcite[Note 10, p.~108]{crown_comparing_2000} qui la rendent alors hautement compétitive avec le CLCLT. En 1993, une discographie\footcite{pellen_les_1993} nous permet de comparer ces prix: Perseus se vend pour 230\$, le TLG pour 5~860 francs français hors taxe\footnote{Avec un taux de change à 5.66, 1~035\$ d'après \cite{noauthor_france_nodate}}. Avec des prix relativement stables pour les outils cités, un article de Beth Juhl indique un prix de 50\$ par CD en 1995 pour le PHI\footcite{juhl_ex_1995}. À cette époque, l'offre de la PLD est donc plus de dix fois plus chère que toute autre base de données majeure en lettres classiques.

Ensuite, le fond de la controverse dépasse cependant de loin les questions des possibilités des différents outils\footnote{Bien que certaines fonctionnalités de la PLD soient \enquote{discutables}: d'après R. Gryson, les \enquote{titres et sommaires, références scripturaires, appels de notes} font partie du texte dans les résultats, et les recherches en contexte incluent, si le terme est en début d'œuvre ou en fin, le contenu de l'œuvre suivante ou précédente. \cite[p. 148]{gryson_patrologia_1997}} et celles du prix. Le principal reproche fait à la PLD concerne sa source, la patrologie de Migne. Qu'il s'agisse de T. Bucknall ou de R. Gryson, les comptes-rendus sont sévères: la patrologie de Migne n'est pas \enquote{conforme aux exigences de la science moderne}\footcite[p. 147]{gryson_patrologia_1997}: éditions datées du 16e siècle, reprise telles quelles par leur collateur au milieu du 19e siècle, erreurs d'attribution \enquote{inacceptables}, et pire, erreurs d'impressions qui se retrouvent ensuite dans le texte proposé par la PLD, car directement copié, sans vérification, par les équipes de Chadwyck-Healey, occasionnant, en plus de possibles erreurs de copies, une augmentation du nombre de coquilles dans la base. La controverse est clairement lancée par P. Tombeur en 1992 lors de son article introductif au Bulletin de Philosophie Médiévale\footcite{tombeur_informatique_1992} (BPM) qui cherche à donner des perspectives au domaine médiéval dans ses projets numériques, en appelant notamment à ne pas dupliquer les efforts. Après avoir présenté son projet au CETEDOC comme \enquote{ne voulant pas être une simple mise en mémoire des oeuvres {[...]} rassemblées par Migne}\footcite[p. 44]{tombeur_informatique_1992}, il présente en contraste la PLD comme une \enquote{photocopie électro-magnétique de l'oeuvre de l'abbé Migne}\footcite[p. 45]{tombeur_informatique_1992} dont le contenu lui-même est douteux. Un droit de réponse peu avisé de Sir Chadwyck-Healey précise qu'il ne s'agit pas de simples fac-similés (avait-il vraiment compris que P. Tombeur parlait de photographie ?), mais bien de textes recopiés\footcite{chadwick-healey_droit_1993}, ce que re-précise P. Tombeur dans une réponse au droit de réponse\footcite{tombeur_reponse_1993} qu'il fera suivre ensuite d'un article plus large de critique - dans le même volume - de la PLD\footcite{tombeur_pld_1993}. L'affaire semble se clore, dans le BPM en tout cas, en 1994 avec l'ultime réponse d'un membre de la PLD\footcite{jordan_facts_1994}. Si des problèmes techniques sont évoqués, que des annonces publicitaires sont interprétées et réinterprétées\footnote{Par exemple, il ne serait pas sûr que les formes courantes du type \textit{ipse} soient cherchables car constituant des \textit{stop-words}}, le problème revient toujours sur la qualité des données originales. Dans sa comparaison des deux outils\footcite{crown_comparing_2000}, R. W. Crown semble absoudre rapidement les auteurs de la PLD pour recommander l'usage de cette dernière -~à condition que le budget suive~-~car elle ne nécessiterait pas l'usage de sources papiers et formerait un \enquote{véritable "e-Book"}. Si l'on en croit les autres comptes-rendus, et notamment les faiblesses en termes d'attribution des textes et l'historique derrière les éditions, cela semble loin d'être vrai en 1995.

Quoi qu'il en soit, entre 1970 et 1995, on voit apparaître de nombreux projets, dont nous n'avons retenu que les principaux et les survivants, qui cherchent à numériser les corpus, entre autres pour rendre plus rapide le travail des chercheurs. Ces \enquote{incunables}, comme les appelle R. W. Crown\footcite[p.~107]{crown_comparing_2000}, forment alors une évolution considérable, sans véritablement transformer les approches du texte: il s'agit de trouver plus facilement un terme, et en cela, c'est une réussite. Deux exemples d'époques sont souvent cités: John J. Hughes, dans un article de 1986\footcite{hughes_ibycus_1986} cité par L. W. Helgerson\footcite{helgerson_cd-rom_1988}, qui indique qu'une recherche de \textit{\textgreek{διαθεκε}} avait pris 25 minutes sur le TLG pour 1~079 résultats alors même que cette recherche, pour des résultats moins importants, lui avait pris \enquote{la plus grande partie d'une semaine dans les bibliothèques de l'université de Cambridge}; Peter Zahn, en 1992, qui explique comment le CLCLT lui a permis rapidement d'identifier un nouveau fragment d'Augustin en cinq minutes, là où la recherche manuelle lui avait pris cinq jours\footcite[p. 427]{zahn_kirchenvater-texte_1992}.

\subsection{Web, standards et corpus: changement d'échelles, changement de pratiques}

Dès le milieu des années 1990, la connexion à internet commence à rentrer dans les foyers avant de vivre une explosion au début du troisième millénaire, en France comme dans le reste du nord économique. D'après T. Karsenti et G. Clermont, on parle d'un passage de seize à sept cents millions d'internautes dans le monde entre 1995 et 2006\footcite{karsenti_les_2006}. Aux États-Unis, en 1995, les chiffres sont de dix millions de foyers avec un accès internet (et dix-huit millions équipés d'un modem, mais sans connexion\footcite{nw_americans_1995}). Dans les bibliothèques américaines, 25\% d'entre elles fournissent un accès à internet en 1996, mais ce chiffre cache la spécificité du déploiement technique: 96\% des villes de 250~000 à 499~999 habitants et 84\% des villes de plus d'un million d'habitants mettent à disposition des connexions internet dans leurs espaces anciennement réservés au papier\footcite{zumalt_internet_1998}. En France, la pénétration de la technologie est un peu plus lente au démarrage, mais les chiffres de l'\acrfull{arcep} (\acrshort{arcep}) montrent une forte croissance: 1,28 million d'abonnements en 1998, 5,33 millions en 2000 (dont 68~000 xDSL), 12,648 millions en 2005; en 2020, ce chiffre atteint 30,627 millions (\textit{cf.} table \ref{tab:chap1:croissance-abonnements-internets}). Avec cette révolution de l'accès aux contenus \enquote{dématérialisé} vient donc l'ère du corpus sans CD-ROM.

\begin{table}[ht]
\centering
\resizebox{\textwidth}{!}{%
\begin{tabular}{l|r|rrrrr}
\hline
                                   & 1998 & 2000     & 2005     & 2010     & 2015     & 2020     \\ \hline
Bas débit (en millions)                         & 1.28 & 5.26     & 3.75     & 0.48     & 0.09     &          \\
Haut débit                         &      & 0.07     & 8.90     & 20.23    & 22.66    & 15.96    \\
Très haut débit                    &      &          &          & 1.13     & 4.21     & 14.67    \\ \hline
Total (en millions)                              & 1.28 & 5.33     & 12.65    & 21.84    & 26.96    & 30.63    \\ \hline
Croissance en 5 ans (sauf 98-2000) &      & 416.45 \% & 237.27 \% & 172.69 \% & 123.42 \% & 113.62 \% \\
Croissance en 10 ans               &      &         &          & 409.74 \% &          & 140.23 \% \\ \hline
\end{tabular}%
}
\caption[Caption for LOF]{Évolution du nombre d'abonnements internet en France d'après l'ARCEP (hors abonnements mobiles)\footnotemark. La décennie 2000-2010 représente une croissance extrêmement importante, démontrant bien la pénétration de cette technologie dans les habitudes des Français.}
\label{tab:chap1:croissance-abonnements-internets}
\end{table}
\footnotetext{\cite{lautorite_de_regulation_des_communications_electroniques_indicateurs_nodate}}

\subsubsection{Du CD-ROM à internet}

En parallèle de la massification vue plus tôt de l'accès à l'ordinateur (\textit{c.f.} table \ref{tab:computer-ownership}), on assiste à la naissance du \textit{Personal Computer} et de monopoles dans le marché des systèmes d'exploitation\footcite{schlender_whos_1990}. En 1983, IBM et Apple représentent environ 40\% des ventes (aux États-Unis), les 60\% restants contenant toute une myriade d'autres OS. En 1990, les PC IBM (dont DOS) représentent 85\% du marché, Apple est second à 5\%. En 1998, 90,5\% du marché appartient à Windows pour les \acrfull{os} (\acrshort{os}), Mac tient bon à 5\% et Linux grimpe légèrement sur ce marché avec 2,1\%\footcite{miles_windows_1999, reimer_total_2005}. Cette croissance et solidification du marché autour de deux interfaces, proches, permet une simplification de l'apprentissage de l'informatique. Et avec la massification -- au moins dans les bibliothèques -- de l'accès à internet, le passage des médias CD-ROM à des sites internet a -- semble-t-il -- été une conversion évidente. \textit{Perseus} -- avant même sa version 2.0 -- s'y convertit dès 1995, la \textit{Patrologia} en 1996, la \textit{Duke Databank of Documentary Papyri}, sorti en 1982 sur bandes, en 1996 aussi. Le CLCLT du CETEDOC  et PHI restent uniquement sur CD-ROM en 1997 parmi les corpus originaux. Pour le premier, ce passage se fait vers le web avec la version 6 en 2005, soit 9 ans après son concurrent direct. P.~Tombeur prend bien en compte internet dans son rapport au \acrfull{bpm} (\acrshort{bpm}) de 1997, mais ne fait aucun lien avec son oeuvre\footcite{tombeur_informatique_1997}, il faudra attendre une mention en 2004 pour voir se préciser une version en ligne\footcite{tombeur_augustin_2004}, qui sera ensuite renommée \acrfull{llt} (\acrshort{llt}) en 2009.

Dans son article de comparaison entre la PLD et le CLCLT\footcite{crown_comparing_2000}, Ron W. Crown mentionne quant à lui la nouvelle interface de la PLD en ligne, mais aussi les nouveaux choix économiques qui l'accompagnent. Contrairement à Perseus, la PLD fait le choix du site web à abonnement, coûtant entre 400\$ par an pour les universités possédant la PLD en CD-ROM et 3~995\$ par an pour les universités aux plus hauts budgets. L'interface ne diffère pas foncièrement de la version locale et permet la lecture des documents. Des autres incunables, le TLG fait le choix du modèle payant ou semi-payant tandis que le PHI, bien que sortant extrêmement tard, entre 2011 et 2015\footnote{Archive.org donne une archive en 2011 du site des oeuvres latines -- que nous avons tendance à conserver -- tandis que la \textit{review} du RIDE donne une sortie en 2015. \cite{daniel_kozak_classical_2018}}, est mis à disposition gratuitement en ligne, mais avec d'importantes restrictions concernant sa réutilisation et son partage. Ainsi, il faut seize ans, de 1995 à 2011, pour retrouver en ligne l'ensemble des incunables. 

% De Perseus à la Perseus Digital Library
\enquote{La seconde génération de ressources électroniques pour lettres classiques}, d'après le titre du premier index proposé par Maria Pantelia\footcite{pantelia_electronic_1994}, commence entre autres par le passage du CD-ROM au web, et réalisé en premier par Perseus, qui semble avoir été le plus rapide à faire cette transition. Dans un article de 1996\footcite{crane_building_1996}, G. R. Crane parle de cette transition -- presque \enquote{indolore} -- d'une application reposant sur des \textit{Hypercards} à un site web\footnote{On ne parle pas d'application web avant quelques années. Une recherche sur Google N-Gram, avec ses défauts, montre une apparition du terme autour des années 2000-2005, avec la naissance du web 2.0}. Selon lui, si le passage a été autant facile, c'est grâce au choix technologique coûteux du SGML (\enquote{Big costs, huge potential, growing benefits}\footcite[p. 7]{crane_building_1996}) pour encoder les textes dès l'origine du projet. Si les utilisateurs, les visiteurs ou \enquote{clients} (au sens web comme au sens mercantile) ne voient peu ou pas l'avantage de ce choix\footcite[p. 8]{crane_building_1996}, il a de fait permis de transformer facilement l'intégralité des données en HTML en un temps record.

Les années 1995 et suivantes ne sont que l'apparition du web comme média, et tout comme celle du CD-ROM, ne permettent d'abord que de faire émerger des outils assez simples. En 1995, l'interface de Perseus est identique à celle du CD-ROM, et donne accès aux données sans modification de l'interaction humain-machine. Il n'est pas facile de savoir quelle quantité de données visuelles a pu être portée dans ce passage sur le web, d'autant que, pour certaines, il est attesté que des problèmes de droits, cédés uniquement pour les versions CD, ont émergé très rapidement\footcite[p.~3]{crane_building_1996}. Cette interface est mise à jour en 1996 en même temps que la version 2.0 sur CD-ROM pour Mac (il faudra attendre 2000 pour une version CD compatible toutes plateformes)\footcite[p.~109]{rockwell_interface_2020}. D'après Rockwell et ses co-auteurs, le passage au web n'aurait pas influencé les ventes des CD-ROM. Nous pouvons attribuer -- hypothétiquement -- cette absence de fluctuation à quatre facteurs:
\begin{itemize}
    \item la relative habitude des bibliothèques d'acheter les CD-ROM et ouvrages mis à jour;
    \item le coût relativement faible du CD-ROM, comparativement aux autres \enquote{incunables};
    \item l'intérêt pour les ressources audiovisuelles qu'il contient et qui ne se trouvent pas sur le web;
    \item le possible élargissement de la base client via le produit d'appel -- involontaire -- que pouvait représenter le site web.
\end{itemize}
Perseus 2.0 est la dernière version CD-ROM de Perseus, contractuellement obligé envers l'université de Yale de produire cette version\footcite[p.~3]{crane_building_1996}. 

Contrairement au CD-ROM, le site web de Perseus permet une mise à jour en continu, incluant de nouveaux textes et de nouveaux domaines. Cet avantage se voit dès la version 2.0 Web, qui se démarque de la version CD-ROM en incluant des données hors champ des études grecques. En effet, dès 1996, Perseus rentre dans une nouvelle phase, celle de la fondation d'une \textit{digital library}: l'équipe de G. Crane remporte un financement de la \acrshort{neh} pour un projet nommé \enquote{\textit{Digital Library on Ancient Roman Culture}\footnote{\textit{Grant ED-20456-96} avec un financement de 190.000\$ sur 3 ans, soit 334~729\$ octobre 2021.}}. Avec ce financement, qui annonce d'ailleurs l'inclusion de ses résultats sur CD-ROM sans que cela n'arrive jamais à notre connaissance\footnote{La description porte l'information suivante: \enquote{\textit{To support the development of a digital library on ancient Roman culture which will serve students of Latin and ancient Rome, and which will be published both on CD-ROM and via the World Wide Web.}}. \cite{noauthor_neh_nodate}}, Perseus étend son travail vers le latin. En 1996 aussi, l'équipe de Tufts obtient de son \textit{Tufts Provost Office and Arts\&Sciences Research funds} un financement pour inclure des oeuvres de la renaissance anglaise (Marlowe, Shakespeare)\footcite{crane_perseus_1998}. Cette situation financière et cet élargissement placent Perseus comme pièce importante de l'espace internet consacré aux sources anciennes, avec des pics de 75~000 visites en vingt-quatre heures\footcite{crane_digital_1998}. En 1998, via un appel de la NEH et de la \acrfull{nsf} (\acrshort{nsf}), le deuxième financement le plus important de l'histoire de Perseus après celui qui le lança en 1987 lui fournit le budget nécessaire pour naviguer vers une version 3.0: l'équipe de G. R. Crane en collaboration avec Nancy Allen du \textit{Boston Museum of Fine Arts} et Ross Scaife de l'université du Kentucky obtient 2,8 millions de dollars\footnote{4~793~456\$ en dollars octobre 2021} pour un projet qui efface même l'héritage gréco-latin du projet dans son titre: \enquote{\textit{A Digital Library for the Humanities}}\footcite{crane_digital_1998}. Ce financement permet l'émergence d'une troisième version en 2000, uniquement web, et correspondant à la fin de la seconde vague de financements de Perseus (\textit{cf.} figure \ref{fig:chap1:perseus_fundings}).

\begin{figure}
    \centering
    \includegraphics[width=\linewidth]{figures/chap1/part1/PerseusFinancements.png}
    \caption{Financements américains connus de Perseus, en millions de dollars non constants, d'après les pages \textit{Grants} de Perseus, le CV de G. Crane et les archives de la Mellon Foundation et de la NEH. On distingue clairement trois vagues (1: fondation, 2: web, 3: expansion) et l'exportation de Perseus vers l'Allemagne pendant quelques années pour un retour après la fin des financements à partir de 2017.}
    \label{fig:chap1:perseus_fundings}
\end{figure}

Au milieu de la décennie 2000, la fin du financement \textit{A Digital Library for the Humanities} et un ensemble d'autres financements permettent la sortie d'une nouvelle version majeure\footcite{noauthor_gregory_nodate}, la 4.0. La 3.0 ayant subi de nombreuses modifications via les besoins émergents pour l'ensemble des financements de la seconde phase, son code n'est plus maintenable. La 4.0 est une remise à plat complète du projet Perseus avec un passage vers le langage Java pour le fonctionnement du site, le passage au XML \acrshort{tei} P4 pour ses ressources textuelles et la mise à disposition pour la première fois de ces dernières en téléchargement libre. Ce changement structurel du \textit{backend} s'opère en 2005, suivi d'une mise sous licence \textit{Creative Commons} de ses sources en 2006 et de son code en \textit{open-source} en 2007\footcite{rockwell_face_2013}. Cette troisième phase marquée par la 4.0 voit l'expansion de Perseus dans le domaine textuel se confirmer: à partir de 2000, aucun financement ne concerne la partie archéologique ou visuelle de Perseus, tandis que se développent une bibliothèque sur la guerre civile (2003), une extension pour les textes arabes (2006), le traitement des entités nommées (2007) ou de la grammaire grecque par \textit{treebank} (2008), etc. Les corpus ont grossi, toujours dans un objectif de mise à disposition des traductions\footnote{Avec comme seule source notre expérience pour le projet Perseus, des statistiques de 500~000 visites sur le site par semaine nous sont parvenues.}. En 2013, alors que G. R. Crane obtient la ``chaire Humboldt pour les Humanités Numériques'' à Leipzig, la troisième phase s'éteint, et peu de financements sont obtenus du côté américain, malgré une attache conservée à Tufts.
% Phase 2 de Perseus: Le latin et le reste, version 3.0 puis 4.0

Un phénomène nouveau accompagne cette apparition du web dans les foyers et bibliothèques: la naissance de corpus produits par des non-spécialistes, transformation numérique de ce que l'érudition locale et les sociétés savantes produisaient et produise en papier, avec la mise à disposition par l'effort de particuliers hors du monde académique de documents, données ou analyses scientifiques. Trois corpus existants encore en 2021 naissent sur la période 1995-1998: \textit{Curculio}\footcite{hendry_curculio_1995}, \textit{LacusCurtius}\footcite{lomarcan_lacuscurtius_1999} et \textit{The Latin Library}\footcite{carey_latin_1998} (1998). Les trois se concentrent en particulier sur la problématique des textes latins, et pour cause, ni Perseus ni \acrshort{phi} ne fournissent alors ces corpus. Ils sont rejoints par des projets francophones dans la décennie 2000, qui correspond à l'explosion de l'accès au web en France et en Europe: \textit{Remacle.org} arrive en 2003\footcite{philippe_remacle_site_2008}, \textit{Latin, Grec, Juxta} de Gérard Gréco en 2006\footcite{gerard_greco_latin_2006}. La particularité des sites francophones et de leurs fondateurs tient en leur carrière professionnelle: tous deux sont professeurs de lettres classiques de formation. Quelle que soit la situation professionnelle de ces créateurs de contenu, ils partagent tous la particularité de réaliser ces projets sans financement propre, en dehors des cadres universitaires, avec parfois une exhaustivité particulièrement importante, comme pour \textit{The Latin Library}, et avec une véritable focalisation sur le latin.

\subsubsection{Les projets nés sur le web}

La fin des années 90 et la première décade des années 2000 voient aussi l'émergence de projets nouveaux, cherchant à installer durablement dans le paysage numérique les lettres classiques. Si nous en parlons peu, car nous ne les utiliserons pas dans notre recherche, les premiers à se développer sur le web sont de loin les projets épigraphiques, dont plus d'une vingtaine est déjà répertoriée par Tom Elliott pour le compte de la société américaine d'épigraphie latine et grecque en juillet 1998\footcite{elliott_links_1998}. Mais les projets littéraires s intéressent aussi à la toile et y naissent.

Le premier mouvement de ces projets nés sur le web est celui de projets qui resteront au niveau du HTML: des sites pour lire des textes, donner accès à ces derniers avant tout. En 1996, par exemple, naît la \textit{Bibliotheca Augustana} d'Ulrich Harsch\footcite{harsch_bibliotheca_nodate}, dont il nous est malheureusement impossible de retrouver le contenu original. En 1998, c'est au tour d'\textit{Itinera Electronica} d'apparaître construite autour de deux axes: un ensemble de cours (sur quatre niveaux: acquisition, maîtrise, transmission et approfondissement) et de ressources textuelles dont la mise en ligne ne semble remonter qu'à 2002, d'après les journaux du site\footcite{meurant_itinera_nodate}. La \textit{Roman Law Library} sort en 2001: il est produit par des historiens du droit, hors du domaine des lettres classiques, fruit d'une collaboration internationale, et cherche à couvrir ``depuis les premiers textes de l'époque royale jusqu'aux compilations de la période byzantine''\footcite{lassard_roman_2001}. Les corpus naissent peu à peu aux États-Unis et en Europe, en latin comme pour les autres langues, tant que le catalogue est complexe à construire\footnote{D'autant que le passage du temps fait disparaître ces corpus et le peu de références faites dans les ouvrages ou articles scientifiques n'aident pas à en conserver la trace.}. Les outils de développement web sans apprentissage du code font leur apparition, les compétences intègrent les institutions peu à peu. En 1995 sort \textit{Vermeer Frontpage}, renommé \textit{Microsoft FrontPage} en 1996, qui permet le développement de site web sans compétences avancées en programmation, via une interface graphique. Dès 1997, on voit apparaître l'émergence de guides\footcite{la1997guide} à destination des non-spécialistes des communautés éducatives. Un projet de bibliothèque numérique de ressources slaves fait clairement mention de l'usage de \textit{FrontPage} dans son élaboration\footcite{deyrup1998character} et celle de son corpus, tandis que d'autres, tel le projet \textit{Journeys in Time 1809-1822} à l'université de Macquarie (Australie), rejettent son usage pour la production d'un code ``plus propre''\footcite[p.~41]{10.3316/informit.752609435027594}. Pour la plupart des premiers projets cités, ils survivent -- c'est ainsi qu'on les connait aujourd'hui -- et se sont enrichis, mais ne sont jamais sortis du contexte des sites web statiques, précompilés en HTML.

Or, la fin 90 et surtout le début 2000 voient des innovations majeures dans le monde du développement web et de la gestion de corpus électroniques. D'abord, les bases de données SQL, et notamment les serveurs MySQL, et le langage PHP voient le jour et dominent rapidement le monde du développement amateur tout en se faisant sa place dans le monde du développement professionnel\footnote{Malgré nos recherches, nous n'avons pas trouvé d'autres sources sur ce sujet que celles que nous citons. Et pourtant, le début des années 2000 voit l'émergence de sites à tutoriel comme celui du \textit{Site du zéro}, la réduction des prix pour l'hébergement de sites, la naissance (et la mort) des salons de discussions pour l'entraide qui favorisent clairement la formation en autodidacte d'une nouvelle génération de développeurs. C'est en tout cas notre expérience de ces années-ci.}. Associé aux serveurs Apache\footcite{smith_lamp_nodate}, facile à mettre en place pour les hébergeurs comme Free en France et d'autres, il devient facilement possible de déployer des sites dynamiques à bases de données relationnelles avec une formation rapide à la programmation. Des outils de publication (\acrshort{cms}, \acrlong{cms}) faciles à installer voient le jour et accompagnent ce mouvement technologique\footcite{purer_php_nodate}. Parmi ces applications plus complexes, on notera l'apparition entre autres de \textit{Musisque deoque}\footcite{gelderblom_musisque_2008}. Dans son compte-rendu, Gelderblom indique qu'il s'agirait du premier corpus -- il parle d'archive -- à intégrer les variantes et l'apparat critique\footnote{``\textit{The important innovation of MQDQ is that it is the first large-scale archive to include [apparatus] for a growing number of texts, and that it also provides effective search tools for them}'', \cite[p.233]{gelderblom_musisque_2008}}. D'autres projets similaires se développent, avec au centre de ceux-ci le développement de bases complexes, avec des technologies avancées comparativement à du pur HTML, comme le CGL\footcite{garcea_corpus_2010}. Ce dernier montre par ailleurs l'inconvénient de cette nouvelle couche de complexité: si \textit{MQDQ} est encore en ligne aujourd'hui, le \textit{Corpus Grammaticorum Latinorum} a complètement disparu -- bien qu'une nouvelle version soit prévue; l'hébergement de simples fichiers HTML et celui d'applications complètes ne posent pas les mêmes défis.

Ensuite, les années 2000 sont aussi celles de l'adoption par les \textit{guidelines} TEI du XML, d'abord avec la TEI P4 en 2001 puis avec la TEI P5 en 2007. La technologie prend de plus en plus de place dans plusieurs champ académiques, la liste des participants au meeting de 2003 montre par exemple cette belle diversité de domaines\footnote{\url{https://tei-c.org/Vault/MembersMeetings/2003-info/mm22.html}}. En 2007, une étape supplémentaire est passée: la portée de la réunion annuelle de la TEI change de forme, passant du nom \textit{annual members meeting} à celui d'\textit{annual conference}, et on ne publie plus la liste des participants à cette réunion\footcite{noauthor_members_nodate}. S'il n'a pas fallu ces changements organisationnels pour que la grammaire TEI soit une promesse attirante, ils en sont autant d'indice que les structures qui adoptent ou tenter d'adopter cette technologie. Dès 2001, T. Nellhaus dresse le portait d'un outil pouvant révolutionner les libraires dans leur mise à disposition de corpus grâce à la standardisation qu'elle implique: pour l'auteur, sa souplesse et sa capacité de représenter des faits précis représentent une opportunité, bien qu'il ne soit pas sans défauts\footcite{nellhaus_xml_2001}. Dès 2002, l'\acrfull{ehess} (\acrshort{ehess}) via le laboratoire en médiévistique  \acrfull{ciham} (\acrshort{ciham}) à Lyon adopte la TEI pour l'édition de sermons et d'autres projets à travers les figures tutélaires de Marjorie Burghart et Nicole Dufournaud\footcite{burghart_edition_2011}. Dès 2002 aussi, l'\acrfull{enc} (\acrshort{enc}) adopte via sa cellule numérique le standard\footcite{poupeau_les_2006}. Chacune de ces institutions évoque la même raison: la TEI, à travers son encodage fin de phénomènes divers (linguistiques, historiques, littéraires, etc.), est un langage pivot permettant de nombreuses sorties et interprétations dont le HTML de lecture -- reproduisant presque les limites de l'imprimé augmenté des liens hypertextes -- n'est qu'une vue. C'est la même raison qui permet à G. Crane de crier victoire quelques années plus tôt au sujet du passage de Perseus au web.

Dans le monde des lettres latines antiques, peu de projets adoptent dans un premier temps cette technologie, en dehors de Perseus qui avait parié dessus dès la fin des années 80.  D'une part, il existe le projet \textit{Hyperdonat} qui est à notre connaissance la première et seule édition scientifique d'une oeuvre latine littéraire classique ou tardive raisonnablement longue\footnote{Il existe quelques extraits ici et là, ou quelques oeuvres courtes comme le texte de Calpurnius dont nous parlons plus bas.} à utiliser le média web et des sources TEI\footcite{bureau2008hyperdonat}. L'usage de cette technologie est d'ailleurs justifié par Bruno Bureau comme le seul moyen d'éditer la base de données que représentent les commentaires de Donat, loin de l'édition d'un texte linéaire que serait celui d'un Victor Hugo par exemple\footcite[La comparaison est la nôtre.]{chaire_de_recherche_sur_les_ecritures_numeriques_exemple_2018}. D'autres tentatives existent cependant: mais même des initiatives aussi prometteuses que la \acrfull{ldlt} (\acrshort{ldlt}), cherchant à simplifier et promouvoir l'édition critique de textes en TEI, n'arrive à proposer qu'une seule édition de texte (Calpurnius) après des années de mise en place. Comme le note d'ailleurs Samuel J. Huskey à propos de son projet, en 2019, ``les vraies éditions critiques sur internet sont encore rares''\footnote{``\textit{truly critical editions on the internet are still rare}'', \cite{huskey_digital_2019}}. De l'autre côté du spectre des projets en TEI, hors des objectifs d'éditions critiques, se trouve aussi le projet italien de la  \textit{Latin Digital Library of Late Antiquity} (DigilibLT)\footnote{\textit{Biblioteca digitale di testi latini tardoantichi}, d'où le LT.} dont l'objectif est de produire un nouveau corpus de textes tardifs, absents de Perseus, sans pour autant en proposer de nouveaux établissements de texte. À partir d'éditions imprimées globalement plus récentes que celles de Perseus, plutôt issues de la période 1950-2000, elle propose une collection cataloguée de textes sur la période du deuxième au huitième siècle, incluant des textes impossibles à trouver par ailleurs sous format numérique, au premier titre desquels on trouve les traités de médecine et de gynécologie. Le \textit{DigilibLT}\footcite{lana_metodologie_2012} prend par ailleurs le même chemin que celui du projet \textit{Perseus} en affirmant l'importance du caractère \textit{open access} et libre de son projet, dont le corpus est téléchargeable dès sa fondation\footnote{Dans son article, Maurizio Lana titre une de ses parties ``\textit{Accesso aperto, licenze Creative Commons, software libero}'' (fr. Accès ouvert, licence Creative Commons, logiciel libre). \cite{lana_metodologie_2012}}. Mais le monde de la TEI latine classique et tardive s'arrête là, du moins pour les oeuvres ``littéraires'' (on inclut les traités de médecine): l'épigraphie a -- elle -- bien adopté l'usage de la TEI et de sa variante Epidoc\footcite{elliott2007epidoc} pour la publication de ces corpus de textes\footcite{bodard2007inscriptions,cayless2010epigraphy}.

\subsubsection{OCR et corpus de masse}

En 2011 et 2012, David Bamman, avec G. Crane\footcite{Bamman:2011:MHW:1998076.1998078} puis avec David Smith,\footcite{bamman_extracting_2012} s'intéresse pour la première fois à un corpus en friche: celui des campagnes de numérisation, majoritairement privées dans son cas précis, et du résultat de la reconnaissance optique de caractères (\acrshort{ocr}) issue de ces documents. Il dénombre, en 2012, 27~014 textes catalogués comme latins dans l'index du projet \textit{Internet Archive}, comprenant tout autant les classiques latins que les thèses et ``commentaires de la philosophie d'Hegel'' produits en Allemagne, mais écrits en latin au dix-neuvième siècle\footcite{bamman_extracting_2012}. Puis, en triant les données, y compris manuellement\footcite{bamman_dbammanlatintexts_2018}, il obtient une liste de plus de onze mille ouvrages comprenant environ 1,38 milliard de mots\footnote{Ce chiffre est à prendre avec une certaine précaution: la méthode de calcul des mots n'est pas précisée et la définition de ce qu'est un mots ici non plus.}. À partir de cette récupération de données, il met à disposition les premiers \textit{embeddings} massifs de l'histoire du latin approché par méthode computationnelle. Cependant, ses données sont problématiques: d'une part, la méthode de vérification manuelle du catalogue n'est pas expliquée; d'autre part, le résultat final contient des documents dont l'\acrshort{ocr} est absolument inexploitable (``\texttt{tkei: SiiimiemfiBgMiffem mvemsfimUttrUffHk rejcijps}'' étant un des exemples de mauvaise texte qu'il cite lui-même).

\begin{figure}
    \centering
    \includegraphics[width=.8\linewidth]{figures/ocrSycophant.png}
    \caption{Récapitulatif de la chaîne de traitement appliquée sur Archive.org}
    \label{fig:chap1:workflow-sycophant}
\end{figure}

La littérature classique commence à être bien couverte: il reste quelques textes impossibles à trouver édités numériquement, par exemple les \textit{Declamatio Minores} de Quintilien ou les auteurs fragmentaires, en dehors des dépôts protégés comme ceux du PHI, mais cela reste marginal. Du côté de la littérature tardive, entre DigilibLT, les \textit{Patrologia Latina} et le CSEL, la couverture tend à être meilleure, même si de nombreux textes ne sont pas édités ou ne sont tout simplement pas encore tombés clairement dans le domaine public\footnote{La question du droit d'auteur sur les éditions de textes anciens est complexe, spécifique à chaque territoire, y compris en Europe, et pose de vrais problèmes d'interprétations, dont certains producteurs de corpus se plaignent, comme Philipp Roelli dans \cite{roelli2014corpus}. Sur le plan national, de nombreuses publications existent, comme \cite{combalbert_lediteur_2015, demonet_confiscation_2018}. Sur le plan international, et celui des textes latins \cite{fischer2017digital, dillen_digital_2016}. Sur des textes plus récents, \cite{dusollier_international_2019}}. Au contraire, la littérature latine médiévale et le néo-latin sont particulièrement absents des corpus en ligne, et leur mise à disposition tarde. Dans ce contexte, le recours à ces dépôts massifs, pour élaborer de nouveaux outils ou offrir de nouvelles approches.

L'OCR ayant progressé depuis 2013, et la méthode n'étant pas totalement claire dans l'approche de D. Bamman, nous nous sommes donc intéressés à l'évolution du dépôt \textit{Internet Archive}. Pour faire cela, nous avons traité les données de cette bibliothèque numérique en cinq étapes (reproduites en figure \ref{fig:chap1:workflow-sycophant}). Après avoir obtenu une liste d'oeuvres classées comme étant en latin par l'\textit{Internet Archive}, nous avons téléchargé l'ensemble des oeuvres cataloguées. Une fois téléchargées, on prend au hasard un quart des lignes de chacune des oeuvres et on applique un classificateur de langue\footcite{salcianu2018compact}. Ce classificateur indique des statistiques pour plusieurs langues s'il a des doutes sur cette classification: on retient alors qu'une ligne est classée comme latine si elle a un score supérieur à 60\% en latin\footnote{Ce seuil a été fixé pour prendre en compte l'existence d'oeuvres bilingues.}. Après ce traitement, on applique un outil \textit{ad hoc}, OCR~Sycophant\footcite{Clerice_OCR_Sycophant_2021}, qui classe les lignes en termes de qualité: lisible ou non lisible. OCR~Sycophant a été développé autour de modèles de classification classiques basés sur des n-grams, et a été entraîné à partir d'un dataset de phrases du corpus Archive.org sélectionnées aléatoirement et annotées à la main: ont été classées comme ``sales'' les données qui n'étaient pas en latin (\textit{eg.} ``\texttt{" Mr Bryce's test, on account of the difficulty of pro-}''), qui étaient illisibles (eg. ``\texttt{"7 „ 7- f Ak. —2 vi rt*- ('wbrf-}'') ou difficilement lisibles (\textit{eg.} ``\texttt{(Hciucuto qucfo quob fwutuinlto}''), ou encore qui correspondaient à des lignes considérées comme trop courtes comme du bruit (\textit{eg.} ``\texttt{5}''). Ce classement disponible, on obtient un pourcentage de lignes estimées propres et on compte les mots de chaque texte (un mot étant considéré simplement comme un élément séparé par un espace).

Ce résultat donne des chiffres absolument prometteurs, tout en demandant une forme de patience. On compte dans les documents disponibles presque 5~000 oeuvres avec une qualité d'OCR à plus de 90\%(\textit{cf.} table \ref{tab:chap1:latin-OCR}) représentant 635 millions de mots. Si l'on ajoute à ce corpus les textes avec plus de 80\% de qualité OCR, on atteint les 1,451 milliard de mots et un peu plus de 17~000 ouvrages. Bien sûr, il existe probablement des doublons dans ces oeuvres. Mais un vrai foisonnement de données existe dans ces dépôts: avec ces deux catégories de textes propres, on atteint près de dix fois plus de mots que les oeuvres contenues dans \textit{Corpus Corporum}, et en ne prenant en compte que la bibliothèque \textit{Internet Archive}.

\begin{table}[ht]
\centering
\begin{tabular}{l|rrr}
\toprule
                               & Nombre de volumes & \% du total & Nombre de mots \\ \midrule
Qualité \textgreater 90\% OCR  & 4 946              & 23.73       & 635 201 534    \\
Qualité \textgreater 80\% OCR  & 12 169             & 58.38       & 816 236 079      \\
Qualité \textgreater 60 \% OCR & 3 709              & 17.79       & 182 697 928      \\
Reste                          & 19                 & 0.09        & N/A           \\ \bottomrule
\end{tabular}
\caption{Statistiques sur les ouvrages latins disponibles sur Archive.org au début juillet 2021}
\label{tab:chap1:latin-OCR}
\end{table}

La question de l'OCR et de sa qualité (surtout pour les oeuvres imprimées avant la fin du dix-neuvième siècle), deviennent donc prépondérantes dans le contexte de l'acquisition de textes fac-similaires (et donc non édités). Le problème de la reconnaissance de texte pour les oeuvres de l'époque moderne, avec ses S longs et ses restes d'abréviations, des incunables est peu à peu réglé par la mise à disposition massive de données d'entraînements et de modèles adaptés. Dans ce cadre, le travail de Simon Gabay autour des imprimés du 17e siècle est absolument majeur et fondateur\footcite{simon_gabay_2020_3826894}: son usage a permis de générer de très bons modèles OCR\footcite{gabay:hal-02577236} et, à partir de quelques données latines\footcite{Clerice_CREMMA_16_18_Prints_2021}, permet de produire des données propres de textes latins imprimés avant le 19e siècle, de l'\textit{Utopie} de Thomas More à des oeuvres en zoologie du 18e siècle en passant par l'\textit{Historia de duobus amantibus Euralio et Lucretia}, avec des taux de reconnaissance à plus de 96\%. 

Mais un autre enjeu pour la mise à disposition de texte arrive aussi à travers la Reconnaissance d'Écriture Manuscrite (REM, ou plus communément l'anglais HTR pour \textit{Handwritten Text Recognition}). La transcription automatique des documents de la pratique et des manuscrits littéraires, quel que soit leur genre, apportera une autre masse de données pour l'étude du latin sur le très long terme. Le développement de cette technologie, sa popularisation par le biais de Transkribus\footcite{kahle2017transkribus}, sa mise à disposition \textit{open source} par Ben Kiessling via Kraken\footcite{kiessling2019kraken} puis eScriptorium\footcite{kiessling_escripto}, ont permis l'émergence de modèles partagés extrêmement performants. Les travaux d'Ariane Pinche sur les manuscrits en ancien français\footcite{Pinche_CREMMA_Medieval_an_2021} avec une reconnaissance des caractères supérieur à 95\% ou les travaux de Dominique Stutzmann\footcite{hazem2020books} ont montré que cette approche était prometteuse et pouvait potentiellement passer à l'échelle. Il est probable que l'étude quantitative du latin soit largement redéfinie par la mise à disposition de ces corpus nouveaux sur le moyen terme, en s'attaquant de front aux dépôts institutionnels nationaux ou régionaux, comme la Bibliothèque nationale de France et son dépôt Gallica. Dans ce cadre, des projets comme le Gallicorpora, dont font d'ailleurs partie S. Gabay et A. Pinche, montreront rapidement ce à quoi l'on peut s'attendre sur le court terme, avec le développement d'une chaîne de traitement pour la production de documents fac-similaires encodés finement en TEI.

Si l'approche évoquée précédemment est celle d'une approche massive, bruitée et sans réelle intervention humaine, une approche qualitative des corpus en friche est aussi possible. La révolution de la qualité des données obtenues via OCR a permis aussi de développer de nouveaux projets, comme le projet VELUM dirigé entre autres par Bruno Bon pour la mise en place d'un corpus médiéval latin de texte OCRisés\footcite{bon2019challenges}. Du côté des périodes classiques et tardives, il faut alors se tourner vers l'initiative \acrfull{ogl} (\acrshort{ogl}) menée par G. Crane depuis l'université de Leipzig puis de Tufts.

Idée née entre 2008 et 2009, le projet \acrshort{ogl} est le fruit d'un besoin ressenti par des enseignants et chercheurs en lettres classiques associés au \acrfull{chs} (\acrshort{chs}) d'Harvard\footnote{Les locaux de ce dernier sont par ailleurs complètement distincts de ceux d'Harvard, au point d'être dans deux États et deux villes différentes: Cambridge, Massachusetts et Washington DC}, Neels Smith et Christopher Blackwell\footcite{muellner2019free}. Ce projet a pour objectif dès le départ de produire un corpus pour le grec qui soit intégralement \textit{open source}, gratuit et accessible, et qu'il soit standardisé afin de pouvoir s'assurer de la collaboration et de la réutilisation par des partenaires divers. Il passe d'abord par une tentative de partenariat avec le TLG d'ouvrir leur collection, qui restera lettre morte. Ce refus se traduit en 2010-2011 par une première tentative de compilation d'un nouveau jeu de données, mais il faut attendre 2015-2016 pour que le projet prenne forme à part entière.

Neel Smith est en effet un proche ami de Gregory Crane, ils ont partagé les bancs de Harvard, ont travaillé sur les premières moutures de Perseus en équipe et ont partagé le même directeur de thèse, Gregory Nagy, directeur depuis 2000 du CHS. Or, en 2013, Gregory Crane obtient donc la \textit{Digital Humanities Chair} à l'université de Leipzig, où il a pour objectif de relancer le projet Perseus. Dans ce cadre, ses premières productions sont simples: il faut agrandir le corpus gréco-latin, en particulier sur le premier millénaire de notre ère, et ajouter un grand nombre de traductions. Les pères de l'Église et l'ensemble de la littérature tardive restent inaccessibles à ce moment précis en accès libre. Pour le latin, l'OCR a fait les progrès qui permettent à l'équipe de Perseus de mettre en place de nouveaux corpus, dont les deux premiers sont le \acrfull{csel} (\acrshort{csel})\footcite{noauthor_csel_nodate} et la \acrfull{pl} (\acrshort{pl}) de Migne. Si la \acrshort{pl} a été introduite précédemment et existait dans des versions concurrentes, le CSEL est quant à lui un corpus d'éditions critiques des pères latins de l'Église. Née en 1864, cette initiative autrichienne de 1864 toujours en activité à travers l'université de Salzburg\footcite{noauthor_history_nodate} a publié depuis sa fondation plus d'une centaine de volumes comprenant potentiellement plusieurs oeuvres, comme le volume 10 constitué des oeuvres complètes de Sidulius (IXe siècle), et dont une partie est tombée, avec son apparat critique, dans le domaine public.

Au niveau technique, ces corpus sont transcrits automatiquement, puis corrigés et structurés en XML TEI par des entreprises spécialisées, dont l'entreprise française Jouve et sa succursale malgache. Leur XML est ensuite adapté aux attentes de Perseus par des équipes internes puis mis à disposition sur Github. C'est ce même fonctionnement qui est repris lors de la mise en place de l'alliance entre le CHS, l'université de Mount Alison et les équipes de G. Crane. Dans l'article de L. Muellner, on apprend que l'OCR est réalisée par les équipes de Mount Alison, sous l'égide de Bruce Robertson qui entretient des modèles pour l'OCR grecque. Puis, les données sont envoyées à l'entreprise plurinationale \textit{Digital Divide Data} (DDD; Cambodge, Kenya, Indonésie) pour leur mise en Epidoc, le format choisi par Perseus. On y apprend que le coût budgété de numérisation et d'encodage par DDD est de 50~000\$ pour quatre millions de mots, soit bien moins que les premiers projets des années 70 et 80. Enfin, la mise en conformité et la vérification des données sont assurées par des stagiaires, étudiants de la licence au doctorat, hébergés d'abord uniquement par le CHS puis par l'université de Virginie\footcite{robertson2019optical}.

La production de l'ensemble de ces corpus a permis ensuite à Perseus de se tourner vers une nouvelle version, en partie financée par le CHS, Perseus 5\footnote{\url{https://scaife.perseus.org/}}, dont la mise à jour est automatiquement liée à l'évolution des corpus, contrairement à la version 4, et qui a été l'objet d'une refonte totale de l'infrastructure de Perseus. Cette version a abandonné -- pour le moment -- les données graphiques (archéologiques, histoire de l'art, etc.) pour ne s'intéresser qu'aux données textuelles.

Enfin, avec l'apparition de tous ces projets se pose la question de l'éclatement des corpus sur internet. Entre les données de Perseus, de DigilibLT, d'autres projets comme le \textit{Corpus Grammaticorum Latinorum} de Jussieu (CGL), l'accès à un corpus latin unifié devient problématique. C'est un problème d'autant plus important que ces sites ne partagent pas une architecture commune qui pourrait alors permettre de centraliser les recherches sur de multiples corpus. En 2011, Philipp Roelli, un éditeur de texte aussi intéressé par la linguistique de corpus, met en route le projet \textit{Corpus Corporum} à l'université de Zurich\footcite{roelli2014corpus}. Ce projet est très peu financé en dehors d'une aide de la chaire de latin et d'une partie des fonds de la COST-Action IS1005 dont l'objectif principal était la formalisation d'un réseau de recherche autour du latin médiéval\footnote{\textit{Corpus Corporum} est donc plus une externalité positive de cette dernière que l'un de ses objectifs.}. P. Roelli le définit comme une ``meta-collection'': \textit{Corpus Corporum} ne produit pas de numérisation ou d'édition numérique, il centralise, en harmonisant, les données d'une dizaine (en 2014) puis d'une trentaine d'autres projets (en 2021), accumulant ainsi en un seul endroit presque 164 millions de tokens à la fin 2021, s'étalant du latin classique au néo-latin. Bien que techniquement ``rustique'' du côté client\footnote{L'usage des \textit{iframes} a presque complètement disparu du web, hormis sur \textit{Corpus Corporum}.}, le projet a l'avantage d'être rapide, facile d'usage -- à l'image de \textit{The Latin Library} -- et de permettre l'accès à des corpus perdus comme les CGL, indisponibles sur leur site d'origine depuis quelques années.

Après l'avènement du micro-ordinateur et du web, les corpus latins ont ainsi évolué pour atteindre aujourd'hui une ouverture sans commune mesure. Pour les plus grands classiques, il est possible d'en trouver des éditions voire des traductions -- en anglais majoritairement -- assez facilement en plein texte. Et quand cela n'est pas possible, il peut toujours être fait recours aux dépôts institutionnels ou privés tels qu'Archive.org ou HathiTrust aux États-Unis afin d'obtenir la numérisation d'un de ces ouvrages. Cette révolution, sur presque cinquante ans, est celle de l'accès aux textes latins classiques et tardifs dans leur intégralité -- ou presque -- et de manière gratuite, et permet de produire de nouveaux questionnements, de nouvelles approches.

% Méta-corpus
%% Corpus Corporum

% Côté éditorial
% \subsubsection{Le renouveau \textit{Open Greek And Latin} et l’apport de l’OCR de masse}
% Approche Github 
% Approche API ?

% Côté non-éditorial non universitaire et l'hors-classique un peu aussi
% \subsubsection{Les corpus en jachères}
%Archive.org et institutions patrimoniales qui OCRisent
% Réfléxion autour de l'OCR de Archive.org, statistiques obtenues quand on a fait des fouilles

\section{Constituer un corpus de recherche}


\begin{quote}[\enquote{The \textit{Corpus Corporum}, a new open open Latin text repository and tool}]{Philipp Roelli}
    \textit{The idea was born from my linguistic research to create an open and non-commercial Latin text meta-collection.} \\
    \enquote{L'idée est née de mes recherches linguistiques: créer une méta-collection ouverte et non-commerciale de textes latins.}
\end{quote}

Avec l'histoire des corpus, il est clair que le nombre de corpus disponibles, ouverts, fermés ou à réutilisation limitée est assez important pour produire une recherche plein texte ou la compilation d'extrait. Nous commenterons ici le travail de constitution du corpus final, qui atteint près de vingt millions de mots, dans ces choix éthiques et techniques. Nous discuterons de l'importance de l'existence de corpus annotés finement et du risque que représentent les corpus plein textes. Enfin, nous discuterons de la compilation du corpus à partir de l'oeuvre scientifique de J. N. Adams, des limites de cette dernière mais aussi des méthodes employées pour compiler cette collection.

\subsection{Constitution d’un corpus général de sources littéraires latines} 

\begin{quote}{\cite{camps_ou_2018}}
La distinction entre humanités « numériques » et « computationnelles » est dans l’air. Au‑delà d’un pur choix terminologique distinctif ou d’un retour aux \textit{humanities computing} du XXe siècle, la revendication d’une dimension computationnelle rend compte d’un basculement, à mon sens éminemment souhaitable, d’une perspective tournée vers la diffusion et la publication électronique, à un accent mis sur les données et leur exploitation pour la création de nouveaux savoirs scientifiques.
\end{quote}

Le constat de Jean-Baptiste Camps est juste: depuis les années 2000 et 2010 en particuliers, la technicisation de l'analyse des documents et des sources, à travers la stylométrie par exemple, et le besoin d'une reconnaissance à part de cette technicisation a donné lieu à de nouvelles sous-communautés des humanités numériques, avec leurs réseaux parallèles de conférences. Là où nous différons cependant, c'est sur l'apparente exclusion mutuelle qu'opèrerait ce nouveau paradigme de la recherche numérique: la constitution de corpus, la collection de textes, leur contrôle qualité, voire leur édition sont autant de missions qu'il ne faut pas négliger dans une approche plus ``computationnelle'' des sciences-humaines, sans quoi les humanités computationnelles ne sont plus des humanités mais de l'informatique appliquée.

\subsubsection{Le choix d’un corpus open-source: Traçabilité des textes, textes et reproductibilité}

Qu'est-ce qu'un corpus ? \textit{Le Robert} définit le corpus comme un ``Ensemble fini de textes choisi comme base d'une étude.'' tandis que le \textit{Larousse} prend -- par un heureux hasard -- l'exemple des corpus grecs pour appuyer sa première définition: ``Recueil de documents relatifs à une discipline, réunis en vue de leur conservation : Corpus des inscriptions grecques.'' et rejoint légèrement le \textit{Robert} pour sa seconde (``Ensemble fini d'énoncés écrits ou enregistrés, constitué en vue de leur analyse linguistique.''). Si les deux dictionnaires se rejoignent sur un point, le corpus est une collection de documents, potentiellement de texte, ils évoquent deux finalités différentes. La première, la conservation, sous-entendue la maintenabilité et l'accessibilité en un même endroit, physique ou numérique, d'une ensemble documentaire, n'est mentionnée comme définissante que par le \textit{Larousse}. Dans son article de 2013, Alex H. Poole fait le constat tour à tour que ``les humanités humériques pivotent autour des données''\footnote{``\textit{The digital humanities pivot around data.}''\cite{poole_now_2013}} mais aussi que ces dernières, au format numérique, étaient ``notoirement fragiles, d'une courte espérance de vie, et facile à manipuler sans laisser forcément de traces, rendant la fraude difficile à détecter [..., sachant que] la plupart des données collectées n'étaient ni organisées ni publiées''\footnote{``\textit{Our Cultural Commonwealth} report characterized digital data as “notoriously fragile, short-lived, and easy to manipulate without leaving obvious evidence of fraud”. Worse, much collected data were neither curated nor published whatsoever;'', \cite{poole_now_2013} citant \cite{unsworth2006our}}. Nous partageons ce constat, cette importance de la conservation pour que corpus existe, et établissons ce point comme premier objectif autour de notre corpus. 

La seconde finalité évoquée est celle de l'analyse (``la base d'une étude'', ``en vue d'une analyse linguistique''). Si nous estimons que cette finalité peut être déplacée (le corpus peut être compilé pour qu'une tierce personne s'en empare), elle est bien évidemment centrale dans notre projet. Et elle demande ainsi de définir l'objectif de notre corpus, car celui-ci définira la forme et les informations nécessaires à y retrouver. Nous reviendrons plus tard sur l'impact qu'a cette finalité sur le corpus, dans son annotation et sa documentation(\textit{cf.} \ref{chap1:method-annotation}).

Il faut cependant ajouter à la notion de corpus un autre point: celui de son ouverture, en droit et en accès. A. H. Poole le mentionne partiellement dans la citation avec la question de la ``fraude'' mais la citation de Borgman\footcite{borgman2012conundrum}, reprise par J.-B. Camps\footcite{camps_ou_2018} est à notre sens assez complète. Un corpus doit être ouvert pour
\begin{enumerate}
    \item reproduire ou vérifier la recherche,
    \item rendre les résultats d'une recherche publique disponible pour le public,
    \item rendre la possibilité à d'autres de poser de nouvelles questions aux données
    \item avancer l'état de la recherche et de l'innovation.\footnote{``(1) to reproduce or to verify research, (2) to make results of publicly funded research available to the public, (3) to enable others to ask new questions of extant data, and (4) to advance the state of research and innovation.''\footnote{\cite{borgman2012conundrum} chez \cite{camps_ou_2018}}.}
\end{enumerate}

La question de la reproductibilité, par l'ouverture du corpus et sa documentation, est centrale pour Borgman, pour Poole et pour Jean-Baptiste Camps. Si la traçabilité des sources n'est pas une nouveauté pour les lettres et l'histoire -- la citation de ces dernières est extrêmement codifiée afin d'être compréhensible et exhaustive, la transcription des sources de la pratique souhaitée pour les publications --, la reproductibilité des expériences est définitivement nouvelle. D'abord, car la notion d'expérience en lettres comme en histoire est nouvelle, bien qu'elle ne le soit pas nécessairement partout dans les sciences humaines. Ensuite, car la notion de reproductibilité est tout autant complexe dans le monde des sciences dites dures. Comme le dit J.-B. Camps, ``au fur et à mesure que l’analyse de données prend de l’importance dans la constitution de nouveaux savoirs, le besoin se fait plus criant de vérifier l’intégrité des données, de reproduire les expériences, de vérifier ou infirmer les énoncés qui en découlent.''\footcite{camps_ou_2018}. L'arrivée de ces questionnements scientifiques et la "crise de la reproductibilité" en 2000, que mention J.-B. Camps est suivi peu à peu par une crise en intelligence artificielle\footcite{hutson2018artificial}, traitement automatique des langues\footcite{belz2021systematic} et en histoire\footcite{eijnatten_big_2013}.

Un autre avantage, en partie lié à la reproductibilité, des corpus ouverts est celui de son analyse et en particulier de ses biais. En accumulant des données dont on essaye de tirer des analyses, des conclusions et même simplement des faisceaux d'indices, la possibilité d'introduire, inconsciemment, des biais de corpus et -- à travers eux -- d'établir des conclusions invalides est un danger éminemment connecté aux corpus fermés. Les conséquences peuvent être importantes dans le domaine de l'intelligence artificielle, le \textit{machine learning} ne pouvant qu'apprendre ce qu'on lui montre. L'exemple le plus connu des dernières années est celui de la reconnaissance d'image de Google, qui, en 2019, avait tout simplement catégorisé des personnes afro-américaines comme gorilles\footcite{lohr2018facial, chokshi2019facial}. Si les conséquences pour notre corpus ne pourraient être aussi graves et socialement problématiques, il reste que la question du biais est à prendre en compte. Il ne s'agit pas de promettre l'exhaustivité: le domaine des lettres classiques a depuis longtemps admis la partialité -- dans les deux sens -- de ses sources ainsi que les pertes de nombreuses autres sources. Sur ce sujet, nous reprendrons l'exemple de l'article de I. D. Raji \textit{et al.}\footcite{raji2021ai}:

\begin{quote}{\cite{raji2021ai}}
    Dans le livre d'histoires pour enfants \textit{Sesame Street}, ``Grover and the Everything in the Whole Wide World Museum''[Stiles et Wilcox, 1974], le monstre \textit{Muppet Grover} visite un musée qui prétend présenter "tout ce qui existe dans le monde entier". Des exemples d'objets représentant certaines catégories remplissent chaque pièce. Plusieurs catégories sont arbitraires et subjectives, notamment les salles d'exposition des "choses que l'on trouve sur un mur" et de "la salle des choses qui peuvent vous chatouiller". Certaines sont étrangement spécifiques, comme "La salle des carottes", tandis que d'autres sont inutilement vagues comme "La grande salle". Alors qu'il pense avoir vu tout ce qu'il y a, Grover arrive à une porte intitulée "Tout le reste". Il ouvre la porte et se retrouve dans le monde extérieur.\footnote{\textit{``In the 1974 Sesame Street children’s storybook Grover and the Everything in the Whole Wide World Museum [Stiles and Wilcox, 1974], the Muppet monster Grover visits a museum claiming to showcase “everything in the whole wide world”. Example objects representing certain categories fill each room. Several categories are arbitrary and subjective, including showrooms for “Things You Find On a Wall” and “The Things that Can Tickle You Room”. Some are oddly specific, such as “The Carrot Room”, while others unhelpfully vague like “The Tall Hall”. When he thinks that he has seen all that is there, Grover comes to a door that is labeled “Everything Else”. He opens the door, only to find himself in the outside world.''}}
\end{quote}

Tout comme l'idée d'un musée du monde est absurde, l'absence de biais dans un corpus l'est tout autant. Mais la possibilité de les décrire et des les vérifier à travers un corpus ouvert est primordiale pour la critique des résultats.

Nous ajouterons cependant une dernière possibilité derrière l'ouverture de ces données, particulier à leur dimension numérique: l'\textit{open access} et l'\textit{open source} dans ce contexte permet aussi la croissance et la modification des données sur le long terme, ne figeant pas le corpus en un instant T (bien qu'il soit important de pouvoir revenir à ce dernier pour la reproductibilité). Le corpus de notre recherche doit non seulement survivre à sa publication mais aussi se corriger, s'arranger: il serait probablement présomptueux de le croire exhaustif, sans erreur, et d'autres seront -- nous l'espérons -- intéressés par sa correction ou son extension à d'autres textes, d'autres périodes.



% Rappel des objectifs: à la fois la constitution d'un corpus d'entraînement ET de sources
% Si corpus d'entraînement, besoin de pouvoir accéder à ce corpus, donc corpus ouvert.
% Si corpus de source, besoin de traçabilité

% Donc Choix Open-Source uniquement
% Et choix d'un corpus TEI, en entrée et en sortie

% De la question de traçabilité découle la question  Capitains
% Citabilité, manipulabilité, compatibilité: XML-TEI et Capitains ? (ou dans le .2 ?)
% Le machine actionnable / readable est pas mentionné au final.?

\subsubsection{Méthode d’annotation et de “regroupement”}
\label{chap1:method-annotation}

% Reprendre le travail de Mc Gillivray ici
% Reprendre la méthode de datation
% La question de la datation
% Poser la question de la citabilité et de la section des textes
% Métadonnées de “lecture”: modèles de citation, niveau de citation recommandé (Introduction du concept de SATU ?)

\subsubsection{Corpus: réutilisation, production, contrôle}

% Statistiques sur les corpus Perseus et autres
% La conversion de DigilibLT
% Présentation des corpus Lasciva Roma
%% Priapées
%% Additional-Texts

\subsection{Du corpus au document: qu’est-ce qu’un document pour l’ordinateur ?}

% Environ cinq à dix pages

% L’importance du choix de CapiTainS, rerédaction de l’article précédemment écrit

\subsection{Constitution d’un corpus sur l’expression de la sexualité} % Environ dix pages

\subsubsection{Le choix d’une source: Adams et histoire des tentatives de vocabulaires de la sexualité latine ?} 

% TLL et problème du TLL chez Adams

\subsubsection{Conséquence du choix de Adams}

% Les données épigraphiques: pourquoi non.
%% Difficulté de lemmatisation
%% Présence relativement faible

% Les bornes “chronologiques” du corpus
% Notes sur quelques données absentes

\subsubsection{Corpus résultat: format, métadonnées, limites}

% Format et tags: interprétation


\section{Composition et analyse des corpus employés}

\subsection{Analyse de la diversité du corpus par période, auteur et genre}

% Représentativité
% Les périodes creuses ?

\subsection{Analyse du corpus sexuel final (stats et autres)}
% Représentation et sur-représentation des auteurs
%% L’angle mort de l’étude d’Adams: la période Chrétienne sous-représentée ?
% Une analyse lexicométrique du corpus: termes les plus fréquents “hors” stopword ?
% Création d’un corpus négatif:
%% Spécificité des termes du corpus: nombre de termes commun (lemme comme formes, stop-words inclus)
%% Nombre de textes très communs (% de lemmes communs importants) via une analyse à la Tesserae ou autre ?

%\include{chap2_deep-learning}
%\chapter{Lemmatisation}
\label{sec:lemmatiseurs}

\section{Introduction}
\label{subsec:lemma_intro}

Afin de traiter un texte et d'établir des statistiques sur celui-ci, il est courant de le lemmatiser. La lemmatisation d'un texte, et \textit{a fortiori} d'un mot, correspond à sa transformation en une forme canonique, le \textit{lemme}. Traditionnellement identique à la forme d'entrée d'un dictionnaire, le lemme permet de rassembler sous une même étiquette l'ensemble des formes fléchies et variations graphiques qu'il peut connaître. Pour le latin, il s'agira de rassembler ensemble \textsc{me} et \textsc{mihi} sous une racine commune \textsc{ego}. Cet étiquetage permet de clarifier le signal statistique en éliminant où cela est nécessaire un bruit inhérent aux langues flexionnelles (voire aux langues dont le système graphique est variable, potentiellement influencé localement, comme l'ancien français ou le latin épigraphique). Son utilisation principale a longtemps eu pour but la capacité à trouver dans un texte les occurrences d'un terme puis a permis le développement plus tardivement de la linguistique de corpus\footcite{mellet2002atouts}.

\subsection{Tâches et définitions générales}

La lemmatisation est donc un effort de traduction d'un texte en un ensemble de formes normalisées, de telle façon qu'\textit{un mot} ne doit connaître qu'une seule annotation de lemme. Notre définition\footnote{On retrouve cette définition partagée par de nombreuses ressources, entre autres \textit{Wikipedia} anglais et français, et dans la plupart des articles et cours d'introduction à la lemmatisation récents. Par exemple, chez \textcite{srinidhi_lemmatization_2020}: \quote{\textit{Both in stemming and in lemmatization, we try to reduce a given word to its root word.}}} exclut \textit{ipso facto} les outils d'analyse du vocabulaire tels que \textit{Collatinus} qui propose pour chaque forme l'ensemble de possibilités de lemme qu'il estime compatible avec la forme. Ainsi, là où \textit{est} est identifié comme \textsc{edo} et \textsc{sum} par \textit{Collatinus}, un lemmatiseur fait un choix unique, lié généralement au contexte et à la probabilité afférente d'apparition de chacun des lemmes possibles à cet endroit.

Enfin, dans le cadre de la lemmatisation, on préfèrera l'utilisation du terme "\textit{token}" à celui de "mot". En traitement automatique des langues, le \textit{token} est un élément dans un ensemble. Au niveau texte, on découpe volontiers ce dernier en séquences plus ou moins indépendantes, généralement des phrases pour les tâches qui nous préoccupent: ainsi, dans ``\textit{II mulieres Gallaeque estis. Romanus sum.}'', nous avons deux \textit{tokens}, ``\textit{II mulieres Gallaeque estis.'}' et ``\textit{Romanus sum.}''. Au niveau inférieur, dans ces unités longues indépendantes, le \textit{token} est un élément qui correspond à la fois aux mots (\textit{mulieri}), aux nombres (\textit{II}), aux enclitiques (\textit{-que}), mais aussi aux signes de ponctuation (\textit{.}). Pour le latin, la tokenisation cherchera donc potentiellement à extraire les enclitiques: \mintinline{python}{"Gallaeque"} se découpera ainsi en \mintinline{python}{["Gallae", "-que"]}\footnote{Cette tâche n'est pas aussi facile que pourrait le laisser deviner cet exemple: pour les lemmes en \textit{-o, -onis} de la troisième déclinaison, le choix entre une forme \textit{Observatione} découpée en \mintinline{python}{["Observatio", "-ne"]} ou conservée telle quel pour représenter un ablatif singulier ne peut se faire qu'en contexte.}. Ce travail est d'autant plus important qu'il peut faire une différence notable dans le cadre de l'annotation: ainsi, identifier \mintinline{python}{"P. Naso."} en un \textit{token} de premier niveau et trois \textit{tokens} de deuxième (une phrase, 3 éléments dont une abréviation, \mintinline{python}{["P.", "Naso", "."]}) plutôt qu'en deux de deux chacun (deux phrases de deux éléments chacune, \mintinline{python}{["P", ".", "Naso", "."]}) induira une rupture de syntaxe, un changement de contexte qui compliquerait la lemmatisation de \textit{Naso} par exemple en \textsc{nasus} le nez plutôt qu'en \textit{cognomen} \textsc{Naso}.

En plus de la lemmatisation, on trouve souvent adjointes deux types supplémentaires  relevant de l'annotation automatique (ou tâches) sur ces \textit{tokens}, basées sur des analyses syntaxiques ou morphosyntaxiques. D'une part, on retrouve très souvent l'annotation des \textit{Part-Of-Speech} (POS). Cette tâche est traitée très rapidement - dès les années 1990 - par les annotateurs automatiques tels que TreeTagger\footcite{schmid1994treetagger}. Les catégories de POS, que l'on pourrait traduire par la notion de nature en français, consistent en un ensemble de catégorisations grammaticales (et non sémantiques) dépendantes des distributions syntaxiques, des fonctions syntaxiques, et enfin des morphologies acceptables\footcite{schachter1985parts}. En traitement automatique du langage, qu'il s'agisse par exemple de stylométrie\footcite{Cafieroeaax5489}, de classification des genres de texte\footcite{feldman2009part}, ou enfin d'analyse de sentiments\footcite{wang2015pos}, les POS ont démontré un réel apport comme variable (\textit{feature}) des données d'entrées de modèles de prédiction. D'autre part, on pourra ajouter à l'annotation une information morphologique ou morphosyntaxique. Elle peut être prise comme un tout (féminin singulier) ou comme un ensemble d'informations indépendantes (féminin; singulier). On distinguera l'information purement morphologique, qui veut que \textit{bonum}, à l'accusatif est un masculin neutre, de l'information morphosyntaxique, qui veut que \textit{bonum virum} est un masculin. En latin, on compte 8 de ces catégories (Cas, Nombre, Genre, Degré, Mode, Temps, Voix, Personne), chacune avec ses valeurs particulières, et une catégorie supplémentaire en cas d'absence de l'ensemble de celle-ci, pour des termes sans morphologie, comme "ut". L'usage de ces traits morphologiques comme \textit{features} n'est pas encore très étudié, %
% je crois 
mais on en trouve des exemples dans certaines études sur la reconnaissance d'entités nommées par exemple \footcite[Par exemple]{zirikly2014named}. Leur utilisation comme variable dans notre étude pourrait se révéler utile: en effet, l'intuition voudrait que l'identification des agents (via les cas et la voie des verbes en contexte), de leur genre et enfin les modes verbaux (l'impératif en particulier) puissent apporter des contributions importantes à la classification d'extraits sur le sujet de la sexualité.


\subsection{Une histoire riche, en particulier pour le latin}

La lemmatisation, en latin, possède une histoire particulièrement riche et liée à celle des humanités numériques \footnote{Une majorité des pistes de recherche nous est fournie par la présentation de N. Perreaux, \textit{cf.} \cite{perreaux_lemmatisation_2019}}. Il ne s'agira pas ici de faire une histoire exhaustive de la lemmatisation et de son application au latin, mais plutôt d'évoquer des continuités et des ruptures amenant à l'état que l'on connait aujourd'hui de la matière.

Il faut noter que les études classiques ont fait usage de la lemmatisation avant le passage à son usage informatique, à travers la création des index et en particulier, dans leur forme la plus ancienne, des concordanciers. M. et R. Rouse datent l'apparition de ces derniers dans leur forme plus ou moins actuelle au XIII\up{e} siècle, à travers les trois concordances verbales de la Bible, à savoir, dans l'ordre chronologique\footcite{rouse_concordance_1984}: celle de Saint-Jacques (\textit{circa} 1235), la concordance anglaise (sans datation connue, sans exemplaire survivant) et une troisième concordance, qui ne saurait être rédigée après 1285. Les auteurs de cette histoire de la concordance attribuent par ailleurs l'apparition de cette forme non seulement au besoin d'enseigner et d'étudier les Écritures, mais aussi de rassembler une pratique apparaissant au cours du XII\up{e} siècle de la \textit{concordantia}, qui consistait à adjoindre aux gloses les références d'autres passages de la Bible liés au terme glosé. Quoi qu'il en soit, il apparait dans ces concordances, comme dans les dictionnaires, une vedette (le lemme) avec l'ensemble de ces formes fléchies en contexte. Il faudra attendre le XIV\up{e} siècle, puis le XV\up{e} siècle pour qu'apparaissent deux autres concordances, l'une sur la Septante grecque, l'autre sur l'Ancien Testament hébreu.

\begin{figure}[h]
    \centering
    \includegraphics[width=10cm]{figures/chap3/histoire/concordanciers.png}
    \caption{Accumulation par dates de concordanciers conservés à la Bibliothèque Nationale de France d'après une recherche sur le catalogue général.}
    \label{lemmatisation:concordanciers}
\end{figure}

Et il ne s'agit pas que des concordances où le latin a été le premier à être lemmatisé de manière systématique pour une étude du langage. Dans le cadre du projet de Roberto Busa, à partir de 1949, on assiste à la première lemmatisation enregistrée numériquement (bien que sur des fiches perforées) via une collaboration avec IBM. Ce travail novateur aura pour but la constitution d'un corpus gigantesque de 11 millions de mots pour une publication vers 1980 de 56 volumes. Ce travail titanesque est d'ailleurs généralement vu comme l'un des projets fondateurs des humanités numériques: Roberto Busa est certainement aux humanités numériques ce que Thucydide et Hérodote sont à l'histoire, dans la méthode du premier comme dans le mythe personnel du second. Cette innovation mécanographique est inédite, mais ne connait pas d'impact direct et immédiat: il faut attendre les années 60 et une  croissance à la fois de l'accès à l'outil informatique (sans parler de démocratisation), de la linguistique de corpus et et de la statistique textuelle (entre autres assistée par ordinateur) pour qu'une suite, ou du moins une méthode parallèle, voit le jour. 

La première occurrence universitaire d'un travail systématique de lemmatisation apparait à l'université de Liège avec le travail du LASLA, fondé le 13 septembre 1961\footcite{delatte_laboratoire_1961}. Dans leur article inaugural, les auteurs Louis Delatte et Étienne Évrard traitent de l'importance de la statistique en matière d'études stylistiques, et, partant du constat que les indices à disposition des chercheurs sont \enquote{incomplets, inexacts ou même inexistants}, ils proposent un travail systématique d'annotation des textes latins et leur enregistrement mécanographique. Tout en portant de nombreuses critiques envers un possible manque de rigueur statistique de certains philologues, cet article montre surtout les opportunités, dans la continuité scientifique du domaine gréco-romain, que porterait un tel outil, à savoir l'assurance de détecter des phénomènes non pas subjectivement ou intuitivement, mais à partir de \enquote{comparaisons entre probabilités théoriques et fréquences réelles}. Par ailleurs, l'article annonce la première étape du laboratoire, à savoir l'annotation des œuvres de Sénèque. La perspective des deux fondateurs du laboratoire est révolutionnaire, et la violence à demi-voilée des propos \footnote{\enquote{Dans combien d'allitérations purement fortuites les commentateurs n'ont-ils pas voulu découvrir les intentions les plus subtiles}, \cite[p.~442]{delatte_laboratoire_1961}} rencontre très rapidement des phénomènes de résistance. Dans une perspective d'histoire du domaine de la lemmatisation, Nicolas Perraux\footcite{perreaux_lemmatisation_2019} a fait resurgir une polémique qui va en ce sens entre Pierre Grimal et Louis Delatte. Aux détours de quatre articles (\footcites{grimal_delatte_1964}{delatte_propos_1965}{grimal_index_1966}{delatte_index_1968}) dont deux sont des comptes-rendus, les deux philologues engagent une \enquote{guerre franco-belge}\footcite{verdiere_pierre_1970}, initiée par P. Grimal, tournant autour de trois points principaux:
\begin{itemize}
    \item Sans aucun doute, il y a ici une question de territoire intellectuel. Nous ne pouvons ignorer que P. Grimal tend à se positionner comme spécialiste francophone incontesté de Sénèque\footcite{verdiere_pierre_1970}, et, alors qu'il écrit une critique de l'index de L. Delatte publié en 1962, se prépare aussi à publier le sien. Dans le même contexte, quand L. Delatte fait une critique de la concordance de l'\textit{ad Marciam} de P. Grimal (publiée en 1965), il le fait aussi en défense de la publication de son propre index de 1964.
    \item De manière sûre et certaine, il y a un débat sur la question de la concordance, conservant le contexte, contre l'index, répertoriant uniquement les passages de chaque lemme. Cette question de la forme, en dehors de toute méthode, est par ailleurs ravivée en 1979, quand L. Delatte et deux collègues\footcite{delatte_concordantiae_1979} viennent critiquer la concordance de Sénèque, générée pourtant automatiquement via des fiches mécanographiques, éditée par Roberto Busa.
    \item Enfin, et c'est un problème clair, une résistance complète et totale à la question du numérique de la part de P. Grimal. L. Delatte pose cependant un ensemble de jalons importants pendant les années soixante, non seulement en créant la \textit{Revue de l'Organisation Internationale pour l'Étude des Langues Anciennes par Ordinateur} \footnote{Aujourd'hui nommée \textit{Revue du LASLA}. Autrefois abrégée RELO} mais en fondant sa pratique de la mécanographie sur un pilier simple, celui de tout enregistrer pour pouvoir manipuler, statistiquement ou traditionnellement, des données en séries: \enquote{Le document de travail est, pour nous, le fichier de cartes perforées}\footcite[p~.202]{delatte_index_1968}. Au contraire, P. Grimal se réfugie tour à tour derrière des considérations matérielles (\enquote{D’ailleurs, qui a son ordinateur personnel ?}\footcite[p.~111]{grimal_index_1966}) ou bien une incompréhension des possibilités de l'informatique:
\end{itemize}
    

\begin{quote}
    \blockquote{Renoncer à la machine, ce serait aussi se libérer d'un certain nombre de servitudes, comme l'adoption d'un code qui devient rapidement assez complexe lorsqu'on cherche à condenser un grand nombre de renseignements sur une fiche. La machine ne peut connaître que des catégories bien définies, chacune étant traduite par un symbole. Ces catégories constituent comme un quadrillage à l'intérieur duquel le réel doit entrer de gré ou de force, sans déborder d'une case dans l'autre\footcite[p.~131]{grimal_delatte_1964}}
\end{quote}

Quoi qu'il en soit, si cette polémique, s'étirant sur une dizaine d'années, entre Liège et Paris, nous montre des formes de résistances et des questions quant aux nouvelles formes qui émergent, elle fait aussi ressortir le travail titanesque qui se fait autour de L. Delatte et É. Évrard au LASLA. Il est y fait mention d'une \enquote{machine à lemmatiser} dès 1965\footcite{delatte_programme_1965} qui propose pour une forme donnée l'ensemble des lemmes pouvant s'y rattacher, faisant tourner des rouages algorithmiques autour de quatre sets de données: les formes irrégulières, complètes, ne permettant aucune reconnexion (\textit{e.g.} les formes de \textsc{sum} au présent); les radicaux; les terminaisons avec leurs codes morphologiques; les préfixes verbaux (\textsc{ad-sum} par exemple, afin de réduire la taille des calculs.). Cette base d'analyseur morphologique est la même que pour \textit{Words}\footcite{whitaker_words_1993}, \textit{Morpheus}\footcite{crane_generating_1991}, \textit{LEMLAT}\footcite{bozzi_lemlat_1992} ou encore \textit{Analysis}\footcite{ouvrard_analysis_1992} qui devient ensuite \textit{Collatinus}\footcite{ouvrard_collatinus_1999}. Tous ces outils apparaissent au début des années 90, et deux, \textit{Collatinus} et \textit{Words}, sont issus du travail de passionnés, l'un professeur de latin, Yves Ouvrard, et l'autre colonel de l'armée étatsunienne, William Whitaker\footnote{D'après sa nécrologie (\texttt{https://web.archive.org/web/20180705175913/https://www.findagrave.com/memorial/66889159}), il était en effet membre de la \textit{Defense Advanced Research Projects Agency} (DARPA), mais \textit{a priori} retraité au moment de la création de Words.}.

Cependant, si l'on retient pour lemmatiseur une définition stricte d'annotation d'une forme par un lemme, impliquant un choix en contexte donc, le premier lemmatiseur du latin vient bien plus tard. Et il ne s'agira pas d'un lemmatiseur du latin, mais bien d'un lemmatiseur généraliste: ce que Delatte estimait potentiellement impossible en 1968\footnote{\enquote{Un tel programme, supposant d'ailleurs une analyse conceptuelle, est très difficile sinon impossible à réaliser}, \cite[p.~100]{delatte_index_1968}}), l'augmentation des puissances de calcul et de mémoire vive des ordinateurs a permis de le faire. À partir des années 1990, on voit apparaître des lemmatiseurs prenant en compte le contexte dans le choix des lemmes pour les formes ambivalentes. %(Insertion études par collatinus du nombre de lemme possible par forme ?). 
Pour prendre en compte ce voisinage des mots analysés, on a généralement donné à ces outils des données d'entraînement permettant la reconnaissance de phénomènes lexicaux et on voit alors la naissance de corpus(\textit{cf.} Chapitre Corpus [ToDO]). Parmi ces outils, TreeTagger a, semble-t-il, reçu les faveurs de la communauté des latinistes des périodes classiques et médiévales, bien qu'il ne soit \textit{a priori} pas le plus performant\footcite[Voir]{eger_lexicon-assisted_2015}. Plusieurs hypothèses peuvent être émises quant à cette situation\footnote{Il faudrait, à ce sujet, probablement faire des recherches et des entretiens plus poussés que ne nous le permet notre sujet ici.}. Il se pourrait que les phénomènes suivants aient fortement influencé sa place actuelle dans le domaine: son incorporation via \textit{TXM}\footcite{heiden:halshs-00549779}, sa prise en main relativement tôt due à son ainesse de presque 10 ans sur certains autres taggers, sa facilité d'installation, la constitution de modèles relativement tôt. 


\begin{figure}[h]
    \centering
    \resizebox{\textwidth}{!}{%
    \includegraphics{figures/chap3/histoire/floating-point-operations-per-second.png}%
    }
    \caption{Opération à virgule flottante par seconde (FLOPS) entre processeur traditionnel (CPU) et de carte graphique (GPU). Source: \cite{noauthor_cuda_nodate}}
    \label{lemmatisation:histoire:puissance-gpu}
\end{figure}


Ces lemmatiseurs font face ensuite aux modèles complexes de \textit{deep learning} qui apparaissent au milieu et à la fin des années 2010, grâce à une montée en puissance continue des machines en calcul. En effet, dans un contexte d'amélioration des performances des cartes graphiques et de leurs GPUs (\textit{Graphical Processing Unit}, \textit{cf.} \ref{lemmatisation:histoire:puissance-gpu}), principalement poussée par une consommation en jeux vidéo du grand public\footcite{tanz_how_2017}, les chercheurs ont pu accéder à des machines beaucoup plus puissantes qu'auparavant pour des prix beaucoup plus faibles. En avril 2020, le prix d'une carte graphique professionnelle haut de gamme coûtait encore 7~602€ \footcite{noauthor_pny_2020} contre des prix variant de 1~149 à 1~899€ suivant les marques pour le haut de gamme à destination des particuliers\footcite{noauthor_recherche_2020}\footnote{C'est sur ce modèle que l'ensemble des entraînements de cette thèse a été produit.}, en septembre 2020, ce coût était encore potentiellement divisé par deux chez le même constructeur avec l'arrivée d'une nouvelle génération de cartes graphiques. Cette explosion de puissance et sa démocratisation permettent à de plus en plus de laboratoires de s'équiper, y compris ceux ne relevant pas du domaine des sciences de l'informatique, et aux chercheurs de prototyper de nouveaux modèles, permettant d'obtenir des temps d'entraînement particulièrement réduits\footnote{La très grande majorité des calculs se faisant via opération sur décimaux qui sont plus faciles pour les GPU que les processeurs classiques} et donc d'essayer un plus grand nombre de combinaisons possibles de modèles (en 2017, d'après \ref{lemmatisation:histoire:puissance-gpu}, la différence de puissance est de x4). Parallèlement à cette évolution du matériel, les géants du web investissent dans des librairies de développement telles que \textit{PyTorch} (Facebook) et \textit{Tensorflow} (Google) qui permettent elles aussi de faciliter le prototypage de modèles de prédiction sans avoir à gérer la réimplémentation de modèles mathématiques complexes inhérents à l'apprentissage profond. Ce contexte riche de démocratisation logicielle et matérielle permet à des projets comme Pandora \footcites{kestemont_lemmatization_2017}{de_gussem_integrated_2017} puis Pie\footcite{manjavacas_improving_2019} de naître dans des laboratoires non-spécialistes. 

\section{Les différents types d'outil}

\subsection{Les outils à base de règles (dès 1965)}

Dès 1965 et la création du LASLA donc, L. Delatte et É. Evrard cherchent à automatiser, en partie, la lemmatisation et l'annotation morphologique. Ce système semi-automatique, qui vise à proposer des solutions sans y faire un choix, repose sur l'utilisation de plusieurs bases de données (lexicales, morphologiques, affixales, etc.). Ce système fondé sur des bases de connaissances se retrouvent chez quatre autres outils cités plus haut, à savoir \textit{Words} de W. Whitaker, \textit{Morpheus} de G. R. Crane, \textit{Collatinus} de Y. Ouvrard (puis Philippe Verkerk à partir de 2015) ou encore, parmi les plus récents, \textit{LemLat}. À partir d'une base de lemmes, reliée à des bases de radicaux et de flexions, l'outil tente différentes combinaisons pouvant correspondre à la forme analysée: lorsqu'un radical et une flexion acceptée par ce radical s'accordent, un lemme et ses analyses morphologiques sont donnés (\textit{cf.} \ref{lemmatisation:outils:collatinusAlgorythme} et \ref{lemmatisation:outils:collatinusAlgorythme} pour des exemples basés sur Collatinus). Chacun de ces lemmatiseurs est basé sur un dictionnaire différent, faisant état de traditions philologiques propres à des centres de recherche ou à des systèmes éducatifs. En effet, \textit{Word} utilise l'\textit{Oxford Latin Dictionary}, \textit{Morpheus} utilise le \textit{Lewis \& Short}, le LASLA utilise le \textit{Forcellini}, Y. Ouvrard semble utiliser le \textit{Gaffiot}, \textit{LemLat} utilise principalement le \textit{Georges}. Cette différence pose un problème d'incompatibilité des différents outils. C'est d'ailleurs dans ce cadre que l'ERC LILA s'inscrit en tentant de proposer une réconciliation des différents référentiels de lemmes\footcite{mambrini_harmonizing_2019}. Dans cette catégorie d'outils rentrent aussi les outils à base de dictionnaire de formes comme celui du CLTK\footcite{johnson2014cltk} qui enregistrent pour chaque forme connue les lemmes et les analyses possibles, sans nécessairement chercher l'exhaustivité, et proposent les lemmes correspondant aux formes enregistrées.

\begin{figure}[h]
    \centering
    \resizebox{\textwidth}{!}{%
    \includegraphics{figures/chap3/outils/CollatinusDB.png}%
    }
    \caption{Modèles de données dans Collatinus. Les formes sont générées et analysées à partir des flexions et radicaux, en fonction du modèle. }
    \label{lemmatisation:outils:collatinusDB}
\end{figure}


\begin{figure}[h]
    \centering
    \includegraphics[width=6cm]{figures/chap3/outils/collatinus.png}
    \caption{Algorithme simplifié de Collatinus}
    \label{lemmatisation:outils:collatinusAlgorythme}
\end{figure}

L'avantage principal de ce type d'outils consiste en sa capacité d'ingérer de nouveaux lemmes très facilement. Par exemple, dans Collatinus, \textsc{lascivus} est donné par la ligne "lascīvus=lāscīvus|doctus|||a, um" où "a, um" est fourni dans un but lexicographique, mais n'a aucune utilité pour la génération des formes: seul "\textit{lascivus}" et "\textit{doctus}" (le modèle de déclinaison) suffisent pour cette partie de l'algorithme. Ainsi, pour ajouter un lemme "\textsc{christianus}", au cas où ce lemme tardif n'était pas inscrit dans le \textit{Gaffiot} et donc dans \textit{Collatinus}, il suffirait d'ajouter, scansion omise, "christianus|doctus|||a, um". À partir de ce simple ajout, la forme \textit{christiani} serait nécessairement reconnue. Le désavantage de ces outils réside dans leur incapacité à faire le choix dans les analyses possibles, qu'il s'agisse des analyses morphologiques (\textit{lasciva} est-il un nominatif singulier féminin ou un pluriel neutre ?) ou des choix de lemmes (\textit{ita} est-il un adverbe ou la deuxième personne du singulier présent impératif actif de \textsc{ito}).

\subsection{Les outils sur base statistique (1994 et après)}

Au milieu des années 90, mais surtout au début des années 2000 apparaissent les lemmatiseurs et analyseurs de POS tels que \textit{TreeTagger}\footcite{schmid1994treetagger}, mais aussi \textit{TnT}\footcite{brants_tnt_2000} et \textit{StanfordNLP}\footcite{toutanova_feature-rich_2003}. Leurs modèles sont principalement basés sur des structures à base de probabilités d'occurrences de phénomènes en contexte, en intégrant la POS à l'analyse de lemme: on parle d'ailleurs principalement de POS-Tagger pour ces outils. Pour faire simple, ces analyseurs font usage de dictionnaires avec des fréquences possibles de POS, puis, en fonction d'une séquence de quelques mots (3 ou 4 en général), telle que \enquote{\textit{fortissimi sunt Belgae}} pour \enquote{\textit{Horum omnium fortissimi sunt Belgae}}, ils cherchent à établir l'analyse plus probable, avec des stratégies de décisions et d'éliminations diverses. \textit{TreeTagger} change la donne en 1994 en apportant plusieurs modifications, qui probablement expliquent ses performances sur le latin. D'une part, il constitue un dictionnaire de suffixes en plus du dictionnaire de forme, d'autre part, pour faire face aux \enquote{\textit{ungrammaticalities}}\footcite[p.~2]{schmid1994treetagger} possibles de l'anglais, il attribue une probabilité minimale aux probabilités nulles\footnote{Il n'exclut pas les phénomènes improbables, en corrigeant les probabilités de 0 à 0,00001 par exemple.}. Dans le contexte d'une langue latine à l'ordre des mots non fixe et à la morphologie riche, ces deux facteurs pourraient expliquer une augmentation des scores importants comparés aux modèles précédents.

\begin{figure}[h]
    \centering
    \includegraphics[width=\linewidth]{figures/chap3/outils/treetagger_type.png}
    \caption{Type de fonctionnement des outils de l'époque Treetagger. Une seule représentation des données est prise en compte au moment "d'un seul" calcul. Il s'agit principalement de modèles statistiques basés sur des probabilités d'apparition de phénomènes.}
    \label{lemmatisation:outils:type-treetagger}
\end{figure}

Le problème évident de ces lemmatiseurs reste aujourd'hui leur relation à un dictionnaire de forme et de lemmes qui ne leur permet pas (\textit{cf.} figure \ref{lemmatisation:outils:type-treetagger}), dans la majorité des cas, de prédire un lemme inconnu ou de gérer plus efficacement une forme inconnue en termes de lemmatisation. Les formes sont traitées telles quelles, ce qui pose, même s'ils prennent en compte des suffixes, rapidement des problèmes pour la lemmatisation ou l'annotation  de la POS dans le cadre du latin (avec des marques flexionnelles telles que \textit{-a} communes à la fois aux noms, aux participes, aux adjectifs, etc.).%exemple ? Vraiment difficile de savoir quoi dire ici...

\subsection{Les outils sur base de traduction (Milieu et fin des années 2010 et après)}

En 2016\footcite{kestemont_initial_2016}, Mike Kestemont propose une première version de Pandora\footcite{kestemont_lemmatization_2017}, un tagueur permettant à la fois la lemmatisation, l'annotation de la POS et des traits morphologiques en contexte. Sa particularité est de cibler le latin, à la fois médiéval et classique, en prenant en compte les difficultés inhérentes de cette langue: d'une part, une morphologie très lourde, beaucoup plus lourde que celle de l'anglais souvent pris comme cible par les développeurs de lemmatiseur; d'autre part, une syntaxe particulièrement libre. Ce travail se base alors sur l'état de l'art en traitement automatique des langues, à savoir des modèles d'apprentissage profond (\textit{deep learning}). Le modèle est constitué d'une couche d'\textit{embeddings}, d'un encodeur LSTM et de plusieurs décodeurs fonctionnant soit sur une base LSTM (pour le lemme), soit sur une couche linéaire (pour les tâches morpho-syntaxique).


\begin{figure}[h]
    \centering
    \includegraphics[width=\linewidth]{figures/chap3/outils/Pie.png}
    \caption{Représentation simplifiée de Pie. L'encodeur n'est utilisé que pour les classifications de mots et à l'entraînement comme support d'entraînement supplémentaire. Lettre par lettre, le lemmatiseur traduit la forme en un lemme. Bien que noté \textit{Embeddings}, le module de projection des caractères prend la forme au choix de CNN ou de RNN en plus d'une couche réelle d'\textit{Embeddings} classiques}
    \label{lemmatisation:outils:pie}
\end{figure}

Ce lemmatiseur est perfectionné par E. Manjavacas\footcite{manjavacas_improving_2019} qui propose une architecture de code plus souple permettant entre autres de tester plusieurs configurations. Il apporte des nouveautés telles qu'une meilleure gestion des caractères via des \textit{embeddings} en CNN (plus rapide) et la mise à disposition des modèles GRU pour les encodeurs et décodeurs. Le principe reste le même que pour la traduction automatique: quand pour ce dernier domaine on tente de traduire des mots en anglais à partir d'une phrase en français, \textit{pie} et les lemmatiseurs du même genre essayent de traduire chacune des lettres d'un \textit{token} en lettres de son lemme, une à une. De cette manière, il intègre des règles de lemmatisation telles que \textit{$\textrm{-ae} \rightarrow \textrm{-a}$}. Cela signifie aussi que ce lemmatiseur peut avoir tendance à créer des lemmes inexistants, tout en respectant une certaine forme de logique interne: potentiellement, les lemmes ne sont pas les bons, mais correspondent à ce que le lemmatiseur a estimé être une forme logique. % Faire une étude ici ?

\section{Corpus et méthodes d'évaluation}
\label{subsec:lemma_corpus}

\subsection{Les corpus disponibles}

Pour l'entraînement d'outils de lemmatisation, il existe très peu de corpus latins. On en compte quatre pour la période classique, utilisant deux dictionnaires de référence, mais trois référentiels de POS. Ces quatre corpus sont issus des projets Perseus, Proiel, Perseids et du LASLA. À l'exception de ce dernier, il sont tous à l'origine des corpus de treebank et non de lemmatisation: il semble qu'il n'y ait pas eu, historiquement, d'autres projets que ceux du LASLA pour la lemmatisation manuelle du latin classique.

Le corpus \textit{Perseus}\footnote{Aussi connu sous le nom de \textit{Latin Dependency Treebank}} est décrit dans son article programmatique de 2006\footcite{bamman_design_2006}. L'objectif de ce projet d'annotation ne concerne en aucun cas la lemmatisation: le terme \textit{lemma} n'est présent qu'une seule fois dans l'article initial pour près de 4000 mots et 10 pages de rédaction. Il est réalisé sous la direction de D. Bamman et G. Crane puis de G. Celano et de G. Crane. Dans les années précédant 2020, l'équipe de G. Crane s'est majoritairement focalisée sur le grec, qu'il s'agisse de l'expansion du corpus de textes édités ou de textes treebankés, résultant en une stagnation du corpus latin. D'ailleurs, ce corpus est le plus petit des corpus d'équipes (\textit{LASLA, Proiel, Perseus}): au mois d'avril 2020, celui-ci contenait 76~670 tokens, ponctuations et enclitiques compris, et ne comprenait aucune œuvre complète. Sa base de lemmes est dérivée du Lewis \& Short\footcite{shorts_latin_1958}, les POS ne sont pas éditées en contexte, mais la morphologie l'est.

Le corpus \textit{Harrington} est un corpus issu d'une pratique pédagogique de l'annotation du latin\footcite{noauthor_harrington_nodate} via la plate-forme \textit{Perseids}\footcite{almas_perseids_2016}. Il suit les mêmes recommandations en lemme et en morphologie que le corpus de \textit{Perseus}, mais diffère sur la grammaire de dépendance utilisée. Certaines œuvres de \textit{Perseus} sont réannotées avec cette grammaire, réduisant ainsi la taille cumulée des deux corpus. Contrairement au corpus \textit{Perseus}, il est encore en cours de production par les étudiants de D. Harrington. Au mois d'avril 2020, il contenait 120~029 tokens, enclitiques et signes de ponctuation compris et ne comprenait pas non plus d'œuvre complète. 

Le corpus \textit{PROIEL} est un corpus de projet plurilingue d'étude du \textit{Nouveau Testament} dans les langues indo-européennes anciennes\footcite{haug_creating_2008}. Il suit les mêmes recommandations en morphologie et en lemmatisation que les projets \textit{Perseus} et \textit{Harrington} mais diffère sur ses annotations de POS et sur la grammaire de dépendance utilisée. Corpus toujours en activité, il ne comprend aucune œuvre complète, mais est assurément le corpus le plus important des trois issus du Lewis \& Short. Contrairement aux deux précédents, c'est une collection qui, comme le LASLA, est d'abord un projet fondé par des linguistes et grammairiens avant d'être un produit de chercheurs en littératures ou d'enseignants comme G. Crane et D. Harrington\footnote{Nous parlons ici de fondation, G. Celano étant lui aussi un linguiste.}.

\begin{table}[h]
\centering
\resizebox{\textwidth}{!}{%
\begin{tabular}{l|rrrrrr}
\toprule
        & Tokens             & Ponctuation & Nombre     & Nombre  & Lemmes & Dictionnaire \\ 
        &                    &   comprise  &  d'auteurs &  de textes & uniqes & \\   \midrule
PROIEL  & 225~064            & Non                  & 5                & 6  & 7~246              & Lewis\footnote{Légèrement modifié}\\
Perseus & 79~670             & Oui                  & 12               & 12 & 6~017              & Lewis \\
Harrington & 120~029             & Oui                  & 9               & 12 &  7~675             & Lewis\\
LASLA   & \textbf{1~630~825} & Non                  & \textbf{18}                 &  \textbf{100+}     & 25~135          & Forcellini \\ \bottomrule
\end{tabular}%
}
\caption{Résumé des informations sur les quatre corpus disponibles. Il existe 137 œuvres au sens du LASLA, mais certaines sont des des découpes inhabituelles, nous préférons donc la notation 100+ ici.}
\label{tab:lemmatisation:corpus-entrainement}
\end{table}

Le corpus du LASLA, qui sera retenu de par sa taille et de par sa diversité, est un corpus dont nous avons précédemment parlé et dont la constitution a commencé dans les années 1960\footcites{delatte_laboratoire_1961}{BodsonCodification1966}. Contrairement aux autres, il ne possède aucun texte à partir du 2\up{e} siècle de notre ère, ce qui en fait sa plus grande faiblesse. L'apprentissage machine reposant principalement sur le nombre de données et leur variété, il n'était pas envisageable d'utiliser une autre source que celle produite par les philologues belges. Par ailleurs, des essais en début de thèse nous avaient prouvé l'incapacité des modèles à prédire des résultats très fiables à partir des données de \textit{Perseus} ou de \textit{Proiel}: à cause de la taille beaucoup trop faible de \textit{Perseus}, le modèle ne s'entraînait pas assez, tandis qu'à cause des données peu variées de \textit{PROIEL}, il ne connaissait que la \textit{Vulgate} et quelques mots de Cicéron.

\begin{figure}
    \includegraphics[width=\linewidth]{figures/chap3/corpus/tokens_per_year.png}
    \caption{Tokens accumulés, par année, en fonction des corpus latins bruts accessibles en \textit{open access} (\textit{Capitains}) ou du nombre estimé  par le \textit{Perseus Catalog}.}
    \label{fig:lemmatisation:corpus-entrainement}
\end{figure}

\subsection{Le corpus du LASLA: choix d'étiquetage}

Le corpus du LASLA utilisé présente 1~630~825 tokens dans la version à laquelle nous avons accès. Il est constitué de 25~135 lemmes, 1~008 types d'annotations morphologiques (\textit{par exemple}  \texttt{Ablatif Pluriel} et \texttt{Deuxième personne Pluriel Indicatif Parfait Actif}) pour 28 grandes catégories syntaxiques (nom, verbe, préposition, etc.) divisées là où il est possible de le faire en déclinaisons (nom1, nom2, etc.). On trouve dans le corpus de très rares erreurs d'annotation, principalement due à une information partielle, et celles-ci semblent marginales au regard du nombre de \textit{tokens}. Le LASLA a fait le choix d'un étiquetage pour majeure partie morphosyntaxique avec désambiguïsation en contexte, laissant quelques cas d'ambiguïtés quand le doute était suffisamment fort pour que l'annotateur ou l'annotatrice ne fasse pas choix. Nous revenons donc sur deux choix affectant les résultats d'analyse automatique: d'une part l'annotation du genre, d'autre part l'annotation des verbes à formes composées.

\subsubsection{Le cas du genre}

Le LASLA a fait le choix de réserver \enquote{l'indication du genre pour les adjectifs, numéraux, les adjectifs-pronoms, les formes déclinées du verbe, hormis le gérondif}\footcite[p.~27]{BodsonCodification1966} dont la répartition est décrite en table \ref{table:lasla:genders-par-pos}. Ce choix a pour conséquence de laisser le genre des noms inconnus: on ne pourra distinguer, dans le contexte de notre recherche, les noms par leur genre masculin, neutre ou féminin, l'information faisant défaut. On pourrait imaginer un travail de réannotation de tous les lemmes dont la POS est NOM avec leur genre quand il est fixe. Ce travail serait potentiellement riche d'influence sur les statistiques finales. Par ailleurs, le genre, à l'inverse du cas et du nombre, n'a pas été analysé en contexte (c'est-à-dire syntaxiquement), mais hors contexte (c'est-à-dire morphologiquement), ce qui laisse des ambiguïtés comme \textit{bonum} qui est à la fois masculin et neutre à l'accusatif singulier. Dans ce contexte, le LASLA crée trois genres morphologiques additionnels aux genres classiques: le Commun, le Masculin-Féminin et le Masculin-Neutre (dont la répartition dans le corpus d'entraînement est visible en \ref{table:lasla:genders-par-corpus}). L'explication derrière ce choix, disponible dans l'article de Bodson\footcite{BodsonCodification1966}, réside dans un problème technique de 1966, qui risquait de poser un problème d'export. Malheureusement, ce choix, aujourd'hui pourtant possible à résoudre, crée une forme de dette technique pour plus de 300 000 mots. On remarque cependant, à la marge, par alignement avec les formes possibles sur \textit{Collatinus} \footnote{\textit{Cf.} annexes numériques} qu'une analyse en contexte a été probablement faite à la marge (\textit{cf.} table \ref{table:lasla:genders-alignement}).

\begin{table}[!htb]
    \begin{minipage}[t]{.4\linewidth}
        \centering
        \resizebox{\textwidth}{!}{%
            \begin{tabular}{l|rrr}
            \toprule
                     & PRO    & VER    & ADJ    \\ \midrule
            Com      & 30 004 & 14 841 & 26 384 \\
            Fem      & 37 177 & 16 393 & 34 292 \\
            Masc     & 42 598 & 18 785 & 24 331 \\
            MascFem  & 8 398  & 3 968  & 17 477 \\
            MascNeut & 25 459 & 12 170 & 30 815 \\
            Neut     & 46 479 & 11 151 & 25 381 \\ \bottomrule
            \end{tabular}
        }
        \caption{Répartition des genres par POS}
        \label{table:lasla:genders-par-pos}
    \end{minipage}% \quad
    \hspace{0.19\linewidth} 
    \begin{minipage}[t]{.4\linewidth}
        \centering
        \resizebox{\textwidth}{!}{%
            \begin{tabular}{l|rrr}
            \toprule
                       & Train   & Dev   & Test   \\ \midrule
            Fem        & 77 907   & 986   & 8 971   \\
            Masc       & 76 213   & 925   & 8 576   \\
            Neut       & 73 899   & 993   & 8 119   \\
            Com        & 63 304   & 789   & 7 136   \\
            MascFem    & 26 492   & 322   & 3 030   \\
            MascNeut   & 61 031   & 743   & 6 671   \\
            N/A        & 1 153 885 & 14 788 & 130 465 \\
            \textit{- Dont noms} & 433 117  & 5 634  & 48 840  \\ \bottomrule
            \end{tabular}
        }
        \caption{Répartition des genres par corpus}
        \label{table:lasla:genders-par-corpus}
    \end{minipage} 
\end{table}


\begin{table}[h]
\centering
\begin{tabular}{l|lll}
\toprule
         & MascNeut & MascFem & Com    \\ \midrule
0        & 783      & 653     & 1 478  \\
1        & 219      & 3 863   & 33     \\
2        & \textbf{65 852}   & \textbf{25 155}  & 2 512  \\
3        & 1 591    & 172     & \textbf{67 206} \\ \bottomrule
\end{tabular}
\caption{Nombre de genres possibles par alignement Forme+Cas+Nombre via \textit{PyCollatinus}\footcite[\textit{PyCollatinus} est une traduction en python de \textit{Collatinus, cf. }]{thibault_clerice_2018_1243076}. Les informations qui ne sont pas en gras montrent une différence possible entre une annotation morphologique et morphosyntaxique. Il peut aussi s'agir d'erreurs de \textit{PyCollatinus}.}
\label{table:lasla:genders-alignement}
\end{table}

\subsubsection{Les formes verbales composées et leur annotation}

Un autre choix du LASLA a été d'annoter les participes avec le temps de la forme composée: pour \textit{amatus sum}, \textit{amatus} portera l'information du temps (\texttt{parfait}), du mode (\texttt{indicatif}), de la personne (\texttt{1}) et du genre (\texttt{Masc}) sans porter l'information du cas pourtant présent morphologiquement. Au contraire, \textit{sum} portera la simple annotation de verbe auxiliaire. Cela pose un problème de confusion pour une même forme \textit{amatus} qui peut être annotée comme simple participe parfait passif avec une annotation \texttt{Mode Voix Temps} ajoutée à une annotation \texttt{Genre Nombre Cas}, et une forme amatus (ellipse ou présence de) sum qui elle sera annotée aussi avec \texttt{Mode Voix Temps}, mais sans le triplet \texttt{Genre Nombre Personne}. Dans cette optique, \textit{amatus} peut représenter 6 formes conjuguées hors participes, 7 pour le neutre \textit{amatum} (\textit{cf.} Table \ref{table:amatus_forms}). %
%
% Exemple pour le parfait passif
%
Ainsi, dans la phrase du \textit{De Amicitia} de Cicéron, \say{\textbf{uidetis} in tabella iam ante quanta \textbf{facta} labes primo Gabinia lege biennio post Cassia }, \textit{uidetis} est annoté à la 2\up{e} personne du pluriel indicatif présent actif là où \textit{facta} est annoté à la 3\up{e} du singulier subjonctif parfait passif. Si cette approche est particulièrement intéressante dans un contexte d'analyse morphosyntaxique, elle est d'autant plus difficile à différencier d'un \textit{facta} nominatif pour un lemmatiseur automatique. Quelle différence en effet peut être faite dans la phrase \say{non oculi tacuere tui \textbf{conscriptaque} uino mensa nec in digitis littera nulla fuit} (Ovid. Her. 2.5.17 sqq.) avec le cas précédent ? % D'ailleurs, n'est-ce pas un raté ??

% Fin d'exemple

\newpara

\begin{table}[h]
\centering
\begin{tabular}{@{}lll@{}}
\toprule
Forme & Mode & Temps \\ \midrule
amatus (sum) & Indicatif & Parfait \\
amatus (eram) & Indicatif & Plus-que-parfait \\
amatus (ero) & Indicatif & Futur antérieur \\
amatus (sim) & Subjonctif & Parfait \\
amatus (essem) & Subjonctif & Plus-que-parfait \\
amatus (esse) & Infinitif & Parfait \\
amatum (iri) & Infinitif & Futur \\ \bottomrule
\end{tabular}
\caption{Annotations possibles pour la forme \textit{amatus} dans le LASLA, hors participes}
\label{table:amatus_forms}
\end{table}

Cette multiplicité d'annotation peut rendre difficile le travail de l'annotation automatique, car elle sous-entend une capacité pour le lemmatiseur de reconnaître les formes au nominatif utilisées de manière adjectivale des formes utilisées comme verbe principal ou verbe subordonné. Nous proposons en \ref{subsec:training:lasla-modification} une analyse de modifications pour une simplification du travail du modèle, en vue de la création d'un modèle morphologique et non morphosyntaxique plus performant.

\subsubsection{Création de lemmes, lexicographie, lexicalisation}

La question de la lexicalisation dans le cadre de la lemmatisation est particulièrement importante, en ce qu'elle peut définir ce qui fait lemme. Doit-on pour autant confondre lexicalisation, création de lemme et création lexicographique ? Au sens original du terme \textit{lemme}, la création lexicographique et celles d'un lemme sont confondues, mais le glissement du dictionnaire vers la base de référentiel permet aussi de distinguer les deux et de faire des choix. Par exemple, dans les données du LASLA, on trouvera les lemmes \textsc{Romani}, les Romains, et \textsc{romanus}, romain en adjectif, quoique le \textit{Forcellini}, dictionnaire source utilisé par le LASLA, ne donne que \textsc{Romanus}, et uniquement dans son \textit{Onomasticon}. Cette distinction se retrouve par ailleurs pour l'ensemble des noms de peuples ou d'habitants. Ce résultat est celui du choix de la primauté de la POS comme séparateur des lemmes: à partir du moment où les formes dans le LASLA sont annotées en fonction d'un couple Lemme-POS intangible, la substantivation d'un adjectif donne lieu à la création d'un lemme et donc d'une nouvelle entrée. L'alternative rejetée est celle d'une annotation de la POS en contexte, quelque soit la POS traditionnelle du lemme: on pourrait ainsi avoir un participe passé substantivé en POS NOMcom, mais avec les traits morphologiques du verbe. Ce choix du LASLA n'est pas évident et peut impliquer, dès lors qu'il y a lexicalisation d'adjectifs substantivés, même dans un cas unique ou du moins réduit, une création de lemmes. De fait, doit-on prendre la lexicalisation, c'est-à-dire d'une certaine manière la prise d'autonomie d'une forme pour un nouvel usage, comme phénomène initiateur de création de lemme ? Cette question est difficile, et le débat peut être complexe. Le défi de la diachronie se pose avec l'annotation, à partir du même référentiel, de faire des choix certains et solides. Cela reste un choix et il est clair que chacune des méthodes a ses avantages, et qu'il reste important de quantifier leur impact en termes de créations d'entrées\footnote{Dans le dictionnaire du LASLA, 12,77\% des lemmes adjectifs peuvent correspondre à des participes et des adjectifs, 12,67\% des adverbes correspondent à des ablatifs de NOMcom, 42,96\% des adverbes correspondent à des instrumentaux en -e ou des formes en -iter d'adjectifs et enfin 23,87\% des lemmes de noms communs peuvent correspondre à des adjectifs substantivés.}.

Il ne s'agit pas pour autant de la seule option visant à limiter le nombre de lemmes ou du moins la possibilité de devoir créer des lemmes à occurrence unique. Parmi les autres choix d'annotation, on peut imaginer, dans le cadre où les adverbes dérivés d'adjectifs ou de noms sont des survivances d'un cas instrumental en \textit{-e} long, d'un ablatif singulier pour les substantifs, de ne pas créer de lemme, mais d'annoter, en quasi-uchronie, sur le lemme supposé d'origine ? Serait-il intéressant dans cette situation de répertorier les termes ici sous un cas perdu, étymologique, quand cela est nécessaire, et d'annoter la POS en contexte ? Si ce choix ne semble par exemple pas si facile dans le cas des adjectifs, les substantifs à l'ablatif grammaticalisés du type \textsc{nocte} laissent une part d'interprétation assez forte, la grammaticalisation n'étant pas toujours facilement identifiable\footcite{fruyt_adverbes_2008}, en dehors des cas particuliers de comparatifs qui ne sauraient bien sûr être appliqués à un nom (comme un \textit{noctissime}%\begin{comment}\footnote{Si l'on revient d'ailleurs à \cite{meillet_levolution_1922}, s'agit-il d'une grammaticalisation qui donne lieu à une nouvelle déclinaison au niveau du degré ou une création par analogie dans ce cas précis ?}\end{comment}
ou bien \textit{magis merito} relevé par Claude Brunet\footcite{brunet_merito_2008}). Le risque de cette méthode est clair: dans une volonté d'annotation presque reconstructionniste, on risque de rendre ce travail, lorsqu'il est manuel, encore plus difficile. Dans les différents degrés de création lexicale, il faut donc différencier les problèmes de reconstruction morphologiques (adjectif adverbialisé ou nom adverbialisé à l'instrumental), les problèmes dus à une POS figée par le lemme à la place d'une annotation en contexte (\textsc{romanus} et \textsc{Romani}). Il existe bien sûr des solutions partielles permettant de conserver les données dans l'état, à savoir de créer une base de connaissances claire connectant ensemble ces prises d'autonomie. C'est entre autres une des missions que s'est attribuées le projet de l'ERC LiLa qui court jusqu'en 2023\footcite{passarotti_interlinking_2020} et pour lequel - par exemple - existe une entrée \textit{merito}. Cependant, au moment de la rédaction, le seul type de relation identifié est celle de \textsc{merito} avec une base de lemme (au sens de dérivation) \textsc{mereo}, bien qu'une relation - peut-être trop faible - \textit{isHypoLemma} existe et relie par exemple \textsc{meritus} et \textsc{mereo/mereor}.

\section{Configurations évaluées et processus décisionnel}

\subsection{Impact du choix d'étiquetage des formes passives ou déponentes composées}
\label{subsec:training:lasla-modification}

Le choix d'annoter des formes simples (les participes) par le temps de la forme composée provoque une difficulté d'apprentissage importante. En retirant du lot les formes adjectivales, le \textit{micro-average} des formes simples est de 97,67 là où la même mesure pour les formes composées stagne à 73,30. Par ailleurs, le \textit{macro-average} et l'\textit{écart-type} montre ces disparités (\textit{cf.} Table \ref{table:lasla:formes-simples-formes-composees})\footnote{Ces termes sont redéfinis dans le glossaire: le \textit{micro-average} s'intéresse à la moyenne sur l'ensemble des formes tandis que le \textit{macro-average} se concentre sur la moyenne des scores par catégories.}. La déviation standard des temps simples peut-être majoritairement expliquée par des formes extrêmement rares ou erronées comme l'impératif présent passif (1 occurrence sur le corpus de test, 0 de précision) ce qui appuie l'importance des deux mesures de \textit{micro-} et de \textit{macro-average}. Pour gérer ce problème, on propose de traduire les annotations automatiquement pour ces parfaits: les temps composés utilisant le parfait passif passent en \textsc{Mode-Temps-Voix} à \textsc{Par-Pft-Voix} (où voix correspondra donc à passif, déponent ou semi-déponent). Les modes composés de l'infinitif sont annotés avec le cas, il est donc conservé. Les autres modes passent automatiquement au nominatif et perdent l'annotation de personne. Cette conversion double le nombre de participes futurs, augmente de moitié le nombre de participes parfaits passifs et n'a bien sûr aucune incidence sur les participes présents (\textit{cf.} Table \ref{table:lasla:correction-temps}.) Les résultats (Table \ref{table:lasla:formes-simples-formes-composees}) sont sans appel: l'intégralité des annotations de formes composées (qui correspondent désormais aux participes) connaît un bond de 40 points en \textit{macro-average} et 18 points en \textit{micro-average}. L'écart-type reste fort dans la mesure où certaines classes sont trop rares ou fautives (par exemple, les éléments annotés syntaxiquement comme des participes futurs passifs sont dans leur intégralité des participes parfaits passifs et n'ont pas été touchés par cette modification.). Les classes fautives restantes posent un problème, mais leur poids dans l'entraînement est assez négligeable pour ne pas influer la reconnaissance des participes parfaits passifs, participes futurs actifs et participes présents actifs (Table \ref{table:lasla:main-particips}).

% ToDo: L'annotation de l'auxiliaire ?

\begin{table}[h]
\centering
\begin{tabular}{@{}l|r|lll|lll@{}}
\toprule
                                &         & \multicolumn{3}{l}{Pré-correction} & \multicolumn{3}{l}{Post-correction} \\ \midrule
Précision                       & Support      & Macro    & Écart-Type   & Micro    & Macro     & Écart-Type   & Micro    \\ \midrule
Verbes (hors N/A)               & 39~465        & 65,24   & 39,92        & 93,36    & 90,61     & 22,33        & 96,67   \\
Formes simples                  & 31~254        & 94,17   & 16,79        & 97,80    & 94,17     & 16,77        & 97,86   \\
Formes Composées                & 7~027         & 35,70   & 35,66        & 73,78    & 76,36     & 38,10        & 91,74   \\
- \textit{dont participe}       & 4~946         & 64,30   & 36,10        & 80,87    & 76,36     & 38,10        & 91,74   \\
Formes “adjectivales”           & 1~173         & 90,67   & 15,39        & 92,27    & 93,78     & 05,95        & 94,33   \\ \bottomrule
\end{tabular}
\caption{\Gls{precision} en fonction des catégories de temps sur la base forme composée/simple et les scores de la table \ref{table:lasla:verb-scores}. Les formes autres correspondent au supin, au gérondif, et à l'adjectif verbal, les formes composées contiennent la catégorie participe.}
\label{table:lasla:formes-simples-formes-composees}
\end{table}

% Check participe futur passif ?

\begin{table}[h]
\centering
\begin{tabular}{l|rrr|rrr}
\toprule
 & \multicolumn{3}{c}{Pré-correction} & \multicolumn{3}{c}{Post-correction} \\ 
 & Test & Dev & Train & Test & Dev & Train \\ \midrule
Par-Fut-Act & 214 & 20 & 1726 & 445 & 46 & 3908 \\
Par-Fut-Dep & 14 & 1 & 121 & 30 & 2 & 209 \\
Par-Fut-Pass & 0 & 0 & 0 & 3 & 0 & 54 \\
Par-Fut-SemDep & 1 & 0 & 13 & 3 & 1 & 28 \\
Par-Perf-Act & 1 & 0 & 2 & 1 & 0 & 2 \\
Par-Perf-Dep & 363 & 32 & 3267 & 653 & 65 & 6203 \\
Par-Perf-Pass & 2927 & 309 & 25334 & 4391 & 526 & 38030 \\
Par-Perf-SemDep & 23 & 5 & 217 & 58 & 11 & 537 \\
Par-Pres-Act & 1210 & 137 & 10935 & 1210 & 137 & 10935 \\
Par-Pres-Dep & 188 & 20 & 1493 & 188 & 20 & 1493 \\
Par-Pres-Pass & 0 & 0 & 1 & 0 & 0 & 1 \\
Par-Pres-SemDep & 5 & 5 & 70 & 5 & 5 & 70 \\ \bottomrule
\end{tabular}
\caption{Résultats sur le décompte de participes des conversions automatiques temps composés vers participe. On remarque a posteriori au moins 2 lignes problématiques (Par-Perf-Act) et (Par-Pres-Pass). Le poids de cette erreur sur un macro-average sera important, mais négligeable sur le micro-average}
\label{table:lasla:correction-temps}
\end{table}

\begin{table}[ht]
\centering
\resizebox{\textwidth}{!}{%
\begin{tabular}{l|rrrr|rrrr}
\toprule
 & \multicolumn{4}{c}{Pré-correction}     & \multicolumn{4}{c}{Post-correction} \\ \midrule
              & Précision & Rappel        & F1-Score & Support & Précision     & Rappel        & F1-Score      & Support \\ \midrule
Par-Fut-Act   &   91      &   89          & 90       & 214     & \textbf{97}   & \textbf{99}   & \textbf{98}   & 445     \\
Par-Perf-Pass &   76      &   83          & 79       & 2927    & \textbf{91}   & \textbf{94}   & \textbf{93}   & 4391    \\
Par-Pres-Act  &   94      & \textbf{98}   & 96       & 1210    & \textbf{95}   & 96            & 96            & 1210    \\ \bottomrule
\end{tabular}{}%
}
\caption{Résultats sur les trois formes principales du participe. En dehors d'un avantage de 2 points sur le rappel des participes présents actifs, tous les autres scores connaissent une augmentation notable, malgré une augmentation nette du nombre de données à tester.}
\label{table:lasla:main-particips}
\end{table} 

\subsection{Méthode(s) d'entraînement et résultats préliminaires}

L'entraînement du lemmatiseur Pie nécessite deux choses: il lui faut d'une part un corpus divisé en trois (d'entraînement, d'évaluation et de test), d'autre part, une série d'hyperparamètres concernant l'architecture du modèle et sa méthode d'apprentissage. Les deux ne sont pas indépendants: un corpus extrêmement riche d'une langue à forte variation morphologique ou orthographique demande \textit{a priori} un réseau plus complexe. Dans un premier temps, nous reviendrons sur l'importance de la connaissance du corpus d'entraînement et des potentielles adaptations qu'il faut faire à la marge. Ensuite, nous parlerons des différentes stratégies appliquées pour obtenir le meilleur modèle et proposerons un retour sur ces méthodes.

\subsubsection{Particularités du corpus et prétraitement}

Le corpus du LASLA présente de nombreuses particularités, propres à son statut de corpus annoté manuellement en vue de l'étude de la langue. C'est un corpus que l'on pourrait qualifier d'éditorialisé: des choix de présentation forts sont faits. Parmi ces choix, 
\begin{itemize}
    \item Le corpus ne présente aucune ponctuation syntaxique (coupure de phrase, marque de dialogue): les seuls signes de ponctuation sont réservés aux abréviations. Les phrases sont manuellement coupées au moment de l'annotation par les éditeurs de données. 
    \item Il utilise des parenthèses et des chevrons pour traiter les formes composées du type \textit{scripti <sunt>} et son inverse \textit{<scripti> sunt}.
    \item Il contient des apostrophes marquant des élisions  telles que \textit{venu'}, et des points pour les abréviations telles que \textit{Tib.}. Plus rarement et uniquement pour les fragments, les points indiquent des manques.
    \item Il n'utilise normalement pas de v minuscule ou de j. Leur présence extrêmement rare étant réservée à des oublis de correction. À l'inverse, on ne trouve pas de U majuscule.
    \item Les majuscules ne sont présentes que pour les noms propres et les adjectifs identifiant des peuples. Aucune majuscule n'est présente pour identifier des débuts de phrases.
    \item Le grec est translittéré en betacode, entouré d'un \$ de chaque côté: \enquote{\textit{Graeci \$pa/qh\$ nominant}} pour \textgreek{\textit{πάθη}}.
    \item Certains couples de tokens sont considérés comme relevant d'un seul lemme: \textsc{usu capio}, \textsc{usu verio}, \textsc{res publica}, \textsc{bene dico}, \textsc{bene facio}, etc.
    \item Les lemmes sont désambiguïsés via des indicatifs numériques (de 1 à 5) ou des lettres (A, N) (pour les noms propres ou les peuples).
    \item Les lemmes sont en toute majuscule.
    \item Les nombres romains sont lemmatisés en toutes lettres et ne sont pas normalisés: on trouve ainsi \textit{XIIII} lemmatisé en \textsc{QUATVORDECIM}.
\end{itemize}
Ces particularités demandent un prétraitement important qui a pu varier en fonction des résultats. Pour anticiper ces problèmes, nous avons développé le programme Protogénie qui permet de prétraiter un corpus et surtout de garder en mémoire les éléments composant les corpus \textit{train}, \textit{dev} et \textit{test}. Nous proposons donc une chaîne de prétraitement permettant de normaliser les points d'entrée, y compris pour "simplifier" la tâche d'apprentissage du lemmatiseur. Ainsi, on présente les lemmes en minuscules, sauf pour les noms propres qui conservent une majuscule à l'initiale. Les nombres romains sont traduits en nombres arabes et les lemmes simplifiés en leur équivalent: \textit{XIIII} et son lemme \textsc{QUATVORDECIM} deviennent tous les deux 14. Sur ce point, l'effort de traduction ne relève pas des mêmes mécaniques que pour les autres lemmes: les règles flexionnelles ne sauraient être comparées aux règles de formation mathématique des formes. Or, pour éviter un apprentissage 1 à 1 (forme pour lemme) ou éviter un bruit introduit dans la compréhension du modèle flexionnel, on préfère traiter ces formes différemment, au choix en prétraitant l'information (à force de reconnaissance par règles des nombres romains) ou en conservant comme lemme la forme (\textit{XIII} lemmatisé \textsc{XIIII}). Dans un premier temps, on essaie sur ce point une configuration qui remplace les nombres supérieurs à 3 par 3, conservant ainsi les nombres 1, 2 et 3 et réduisant la taille des \textit{embeddings}. Par ailleurs, ce même outil sert à séparer la morphologie en autant de colonnes nécessaires. Une fois les corpus produits, on applique 3 fonctionnalités supplémentaires: la correction des temps composés (cf. \textit{supra} \ref{subsec:training:lasla-modification}), le recollage des clitiques aux termes qui les précèdent et leur identification dans le lemme afin de faire reposer la responsabilité de cette tâche au lemmatiseur, et enfin, l'ajout aléatoire de majuscules en début de phrases et à certains mots de manière aléatoire afin de déjouer un surapprentissage liant majuscule à l'initiale et noms propres.

\subsubsection{Besoin d'optimisation}

\begin{figure}[ht]
    \centering
    \includegraphics[width=\textwidth]{figures/chap3/entrainement/TrainingDuration96BoxPlot.png}
    \caption{Durée d'entraînement des modèles. Les populations sont les suivantes: (Étape 1, Complète) 50, (Étape 1, Interrompue) 44, (Étape 2, Complète) 11, (Étape 2, Interrompue) 19, (Étape 3, Complète) 11, (Étape 3, Interrompue) 10.}
    \label{fig:lemmatisationTrainingTime}
\end{figure}

Une fois le corpus généré, on peut lancer l'entraînement. Mais \textit{pie} comporte 97 paramètres dont 13 sont des paramètres d'architecture et 18 des paramètres d'optimisation. E. Manjavacas nous avait dans un premier temps fourni des paramètres qui lui semblaient corrects, mais nous avons souhaité vérifier leur optimisation. Dans ce cadre, nous avons utilisé deux méthodes. La première est manuelle et consiste à vérifier quelques paramètres, en particulier ceux hypothétiquement clefs, à savoir les dimensions de réseaux et le nombre de couches d'encodage des caractères ou des contextes qui déterminent en partie la capacité à apprendre du réseau. Cette méthode a eu l'avantage de nous permettre de confirmer l'identification de ces paramètres clefs et de prendre en main ces configurations. Puis, la deuxième évaluation fut réalisée avec un outil d'optimisation des hyperparamètres, \textit{optuna}\footcite{optuna_2019}, en trois phases. D'abord, l'identification des paramètres ayant le plus d'impact sur l'apprentissage: cela aura permis d'identifier sur 94 entraînements qu'avant tout autre paramètre, les \textit{learning rates} et éléments de \textit{patience} (délais avant de réduire le \textit{learning rate} ou d'estimer qu'il n'y a plus d'amélioration possible) et de prévention du surapprentissage (\textit{dropout}, ou perte volontaire d'information) ont un impact majeur sur les résultats\footnote{On peut retrouver dans les annexes numériques les trois configurations: étape 1 config-optim-base-optuna.json, étape 2 config-optim-random-optuna.json et enfin config-optim-hidden-optuna.json.}. L'étape 2 a construit sur les meilleurs paramètres de \textit{learning rate} pour évaluer les paramètres d'entraînement. Enfin, une dernière étude a été faite sur la taille optimale de la couche contexte. Ce travail est assez particulier pour le domaine des lettres, car il implique des délais pour obtenir des résultats d'expérience qui ne sont pas pas familiers à la recherche en sciences humaines: en fonction de la complexité de l'architecture, il faut compter jusqu'à 10h d'entraînement (cf. Figure \ref{fig:lemmatisationTrainingTime}) et donc plusieurs jours voire semaines pour l'évaluation simple des meilleurs paramètres\footnote{Les trois étapes cumulées représentent 17 jours et 8 heures de calcul sans interruption.}.


\subsubsection{Résultats et Variance}

Une fois les paramètres supposément optimaux trouvés, on peut lancer une batterie d'entraînements, \textit{a minima} 5, sur cette configuration et évaluer les résultats. En effet, un entraînement de \textit{machine learning }est particulièrement influençable par des phénomènes aléatoires, liés à l'initialisation des paramètres et poids, à l'ordre d'apparition des mots, etc., et peut représenter une aberration statistique. Pour l'expérience d'optimisation manuelle, on obtient une médiane des écarts-types à 0,10\% et une moyenne à 0,13\%: sur 18 configurations, on peut donc estimer qu'en moyenne, les résultats de tests seront dispersés à 0,13\% du score moyen. Dans le cadre d'un modèle performant en moyenne à 97,5\%, et à titre d'exemple, cela signifie donc que le modèle peut potentiellement avoir des résultats s'étirant de 97,35\% et 97,65\% sur plusieurs entraînements.

\begin{figure}[ht]
    \centering
    \includegraphics[width=0.7\textwidth]{figures/chap3/entrainement/Variance.png}
    \caption{Distribution des écarts-types d'\textit{accuracy} sur la tâche lemme pour 18 configurations de modèles d'entraînement sur 5 entraînements chacun. On estime alors que les modèles peuvent se retrouver, en cas de distribution parfaite, dans une intervalle de résultats d'environ 0,3\%. Pour plus de détails, \textit{cf.} figure \ref{fig:training_variation_per_model} en annexe.}
    \label{fig:training_variation}
\end{figure}

Avec la dernière configuration, le modèle obtient un score de 97,41\% en lemme sur le corpus de test et de 88,45\% sur les formes inconnues (cf. Table \ref{tab:modelFinalLemmatisation}). Les modèles d'annotations de traits morphosyntaxiques fonctionnent tous particulièrement bien, avec la tâche \texttt{cas} en queue qui plafonne à 92,34\%. En agrégeant les résultats, c'est-à-dire en prenant en compte la validité de l'ensemble des annotations automatiques (lemme, traits morphologiques, POS), on obtient les scores de 85,86\% et de 76,68\% sur les formes connues et inconnues. Ces scores sont plus bas, mais ne changent pas les conclusions précédentes: d'abord, car l'ensemble des tâches ne nous intéressent probablement pas (par exemple le degré ne devrait pas apporter beaucoup à notre travail), ensuite, car le modèle reste très performant, 85,86\% étant particulièrement élevé, surtout sur 169~000 tokens de tests. Par ailleurs, avec ses 85,86\%, le modèle n'est pas si loin des scores les plus bas, à savoir le cas (-7 points au total, -2 points sur les formes qui portent ce trait morphologique) et est infiniment plus haut qu'une probabilité de hasard parfait.

\begin{table}[ht]
    \begin{tabular}{l|rrr}
    \toprule
                    & \multicolumn{3}{l}{Accuracy}                           \\
                    & Formes connues & Formes Inconnues &  Formes Concernées \\ \midrule
    Lemme           & 97,41          & 92,92            &                    \\
    POS             & 96,49          & 92,45            &                    \\
    Genre           & 96,28          & 91,49            &   89,98            \\
    Nombre          & 97,02          & 93,85            &   96,44            \\
    Cas             & 92,34          & 87,37            &   87,84            \\
    Degré           & 98,07          & 93,96            &   93,37            \\
    Mode Temps Voix & 98,35          & 90,80            &   94,44            \\
    Personne        & 99,71          & 98,15            &   98,49            \\ \midrule
    Tâches agrégées & 85,86          & 76,68            &                    \\ \bottomrule 
    \end{tabular}
    \caption{Résultats du modèle final. Les formes concernées sont les formes dont le trait morphologique n'est pas absent: par exemple, l'\textit{accuracy} de 87,84\% en cas correspond au taux de succès pour toutes les formes qui sont annotés pour ce trait morphologique, excluant les formes verbales non participiales, les prépositions, etc.}
    \label{tab:modelFinalLemmatisation}
\end{table}

En fin d'expérimentation, un nouvel algorithme chargé de conduire l'apprentissage d'un réseau neuronal (appelé optimiseur), \textit{Ranger}\footcite{wright_new_2019} a attiré notre attention. Accompagné d'un autre système de contrôle du \textit{learning rate} qu'une réduction après rencontre de plateau\footnote{Seule option disponible de \textit{pie} au moment de cette découverte.}, \textit{CosineAnnealing}, il proposerait de meilleurs résultats que les deux grands optimiseurs que sont \textit{Adam} et \textit{SGD} (disponibles par défaut dans \textit{pie})\footnote{Nous utilisons le conditionnel ici, car, au moment de la rédaction, aucune publication vérifiée par les pairs n'était disponible sur le sujet.}. Nous implémentons donc à la fois \textit{Ranger}, \textit{CosineAnnealing} et une surcouche permettant d'imposer un délai sans optimisation du \textit{learning rate} que conseille aussi le créateur de \textit{Ranger} et faisons quelques expériences sur les paramètres\footnote{Les paramètres les plus performants rencontrés sont 0,01 de \textit{learning rate}, 10 de délai, 40 de \textit{T\_0} pour \textit{Ranger}}.Après celles-ci, il s'est avéré que l'usage de ces nouveaux algorithmes permettait non seulement de réduire le temps d'entraînement d'environ 40\%, mais aussi de stabiliser les résultats en réduisant la variance tout en obtenant de meilleurs scores\footcite{clerice_allow_nodate} (\textit{cf.} \ref{fig:lemmatisation:optimiseur:ranger}). Cette modification de notre méthode d'entraînement permet ainsi de tirer le meilleur du réseau tout en réduisant le temps nécessaire à des ré-entraînements en cas de modification du corpus.


\begin{figure}[ht]
    \hspace*{-0.05\linewidth}
    \begin{minipage}[c]{0.55\linewidth}
        \includegraphics[width=1\linewidth]{figures/chap3/entrainement/boxplot_accuracy_ranger.png}
    \end{minipage} \hfill
    \begin{minipage}[c]{0.55\linewidth}
        \includegraphics[width=1\linewidth]{figures/chap3/entrainement/boxplot_time_ranger.png}
    \end{minipage}
    \caption{Comparaison des résultats en fonction des optimiseurs, en temps (heure) et en \textit{accuracy}, sur trois couples: \textit{Adam/RLR} étant le système de base de \textit{pie}, \textit{Adam/CosDelayed} utilisant Adam avec un délai d'utilisation de \textit{CosineAnnealing}, Ranger/CosDelayed reprenant ce système avec \textit{Ranger} à la place d'\textit{Adam}. }
    \label{fig:lemmatisation:optimiseur:ranger}
\end{figure}

\subsubsection{Vers une typologie des erreurs}

À la suite de nos entraînements, le modèle produit est évalué sur un jeu de données de test. Ce dernier n'a pas été vu pendant l'entrainement, et garantit donc autant que possible la qualité des résultats et l'absence de surapprentissage sur le corpus d'entraînement. Suite au test, une table des confusions est mise à disposition par \textit{pie} (ex. \ref{lemma:confusion-table}). Cette table permet de mieux comprendre les différentes erreurs que le modèle est capable de faire. Pour un résultat de 97,41\% sur le corpus de test, il reste 4~395 erreurs. Sur ces erreurs, nous avons des problèmes de lemmes homographes tels que \textit{liber} et \textit{Liber}, mais aussi de pluriels lexicalisés tels que \textit{liberi}. Sur 64 apparitions dans le corpus de test de \textit{liber}, on obtient une précision de 92\% et un rappel de 89\%; pour \textit{liberi} et ses 67 apparitions on retrouve respectivement 91\% et 96\%. \textit{liber} l'adjectif est confondu 6 fois avec \textit{liberi} la substantivation et 1 fois avec \textit{Liber} la divinité, à l'inverse, \textit{liberi} est confondu 2 fois avec \textit{liber} et 1 fois avec \textit{Liber}. Si l'erreur \textit{liberi}/\textit{liber} est excusable - on peut même se demander s'il s'agit d'une erreur - on retrouve par ailleurs de vraies erreurs, avec des créations lexicales telles que \textit{pancoristus} à la place de l'attendu \textit{panchrestus} pour la forme \textit{panchresto}. Ce type d'erreur peut être introduit par un bruit involontaire créé par des formes de lemmes étymologiques qui induisent des variations orthographiques comme on en trouve avec \textit{negligens} plutôt que \textit{neglegens} et \textit{paedico} à la place de \textit{pedico}\footnote{Dans le \textit{Forcellini}, on trouve l'explication de l'orthographe sous la formule \enquote{\textit{Prima syllaba sine diphthongo scribitur vitio serioris aevi}} (la première syllabe est écrite sans diphtongue à cause du vice de la période tardive) en citant les orthographes supposées erronées des \textit{Priapea}, mais celles correctes de Catulle (21.4) et Martial (11.94.6, 11.104.17, 7.67.1, 11.99.2). Seulement, aucune des éditions modernes à notre disposition ne donne cette version en \textit{-ae-}, à savoir les éditions de Lindsay (Oxford University Press, 1929) et de Shackleton (Teubner, 1990) pour Martial, de Lafaye (Budé, 1923), de Hermman (Latomus, 1957, p. 103), de Mynors (Oxford Classical Texts, 1958) voire même la \textit{Teubner} de Mueller (1892) pour Catulle. Une recherche sur le \textit{Eagle Inscriptions Search Engine} donne deux attestations en \textit{paedic-}, une au \textit{-a-} restitué, une dont les sources (EDR et EDH) donnent une autre orthographe que celle fournie par Eagle. Au contraire, sur l'EDCS, 66 résultats (certains sont des doublons) donnent la forme \textit{pedic-}, dont le graffiti pompéien CIL 04, 10693 antérieur donc à Martial. S'il s'agissait d'une vraie diphtongue, cette chute de la prononciation de la diphtongue pourrait avoir eu lieu avant Cicéron, \textit{cf.} \cite{sturtevant_monophthongization_1916}. L'autre hypothèse ici est celle d'une étymologie grecque supposée par des lexicographes qui auraient archaïsé de ce fait la forme du lemme.}. Si ce type d'erreur existe pour le couple \textit{liberi}/\textit{liberi}, on peut par ailleurs, par un croisement avec les données de collatinus, remarquer que:

\begin{enumerate}
    \item 5,54\% des erreurs sont des lemmes verbes remplacés par leur participe parfait passif et donc une forme adjectivale lexicalisée (type \textit{beo}/\textit{beatus})
    \item 5,57\% sont des verbes remplaçant la forme adjectivale, l'inverse de la situation 1.
    \item 1,16\% sont des verbes dont le participe présent rentre en collision avec le lemme attendu (type \textit{negligens}/\textit{negligo}).
    \item 0,96\% sont des erreurs inversées de 3.
    \item 3,59\% sont des noms partageant leur nominatif avec une forme adjectivale (\textit{liberi}/\textit{liber}).
    \item 6,76\% sont des erreurs inversées de 5.
    \item et enfin, 0,50\% des erreurs sont des verbes passifs lemmatisés en actifs, et 0,66\% correspondent à l'inverse.
\end{enumerate}

\begin{table}[]
\begin{tabular}{lrlr}
\toprule
Expected & Total Errors & Predictions & Predicted times \\ \midrule
qui      & 289          & quod        & 111             \\
         &              & quis        & 103             \\
         &              & quam        & 39              \\
         &              & quo         & 24              \\
         &              & qua         & 11              \\
         &              & quiuis      & 1               \\
quis     & 174          & qui         & 156             \\
         &              & quo         & 8               \\
         &              & quam        & 5               \\
         &              & quod        & 3               \\
         &              & qua         & 1               \\
         &              & cuius       & 1               \\
quod     & 108          & qui         & 103             \\
         &              & quis        & 5               \\
multus   & 56           & multum      & 37              \\
         &              & multi       & 19              \\
bonus    & 26           & bonum       & 14              \\
         &              & bene        & 8               \\
         &              & boni        & 3               \\
         &              & Bonus       & 1               \\ \bottomrule
\end{tabular}
\caption{Table des confusions générée par \textit{pie}. Nous retenons pour l'exemple les confusions en lemme les plus fréquentes.}
\label{lemma:confusion-table}
\end{table}

Dans un cadre plus large, sur 4~398 erreurs, 2~942 portent sur des tokens ambigus, connaissants donc plusieurs solutions de lemmatisation, soit 8,08\% de ceux-ci. 451 erreurs portent sur des lemmes inconnus en phase d'entraînement, soit 40\% de ces derniers, mais le faible nombre de ceux-ci ne permet pas d'en tirer de vraies analyses statistiques.

\section{Pie: extensibilité des résultats}

L'entraînement de \textit{pie} sur un corpus test donné nous informe de sa valeur sur le corpus en tant que tel. Se pose la question cependant de la capacité de nos modèles à s'appliquer à un corpus étranger, de quantifier l'impact d'une potentielle spécialisation. Pour étudier cet impact, nous proposons quatre expériences qui permettront d'évaluer l'impact de la taille du corpus d'entraînement, de sa variété en style et en genre, et enfin une étude hors-domaine s'appliquant d'abord à un texte important pour notre corpus, les \textit{Priapea}, puis à des extraits de textes tardifs, le corpus du LASLA s'arrêtant avant la fin du premier siècle de notre ère..

\subsection{Évaluation sur des données hors domaine}
\label{subsec:lemmatisation:hors-domaine}

Si les tests du modèle nous présentent un outil performant, au-delà des 97\% de reconnaissance des lemmes, ils ne nous montrent qu'une face de son usage. En effet, la constitution du corpus de test est faite d'extraits non connus certes, mais d'extraits des mêmes textes que le corpus d'entraînement. Ils ont donc une très forte probabilité de posséder les mêmes caractéristiques en termes de syntaxe, de vocabulaire, de coupes opérées par l'éditeur. Par ailleurs, ils posent un second problème qui est celui de la période couverte par le corpus d'origine, à savoir que l'auteur le plus tardif dans le corpus est Juvénal (127\footcite[p. 320]{fredouille}), puis Sénèque (65\footcite[p. 231]{fredouille}). Ces données quand elles sont issues du même corpus, qu'elles sont des fractions des mêmes textes que les corpus d'entrainement et de \textit{dev} sont appelées \enquote{en domaine} (\textit{in domain}).

Au contraire, on parle en apprentissage machine de données hors domaine, des données dont les traits de définitions (auteur, genre, période, style, œuvre, etc.) diffèrent pour tout ou partie des données d'entraînement. Pour cette étude, nous proposons deux corpus hors domaines, l'un constitué de l'ensemble des priapées 1 à 78 d'après la numérotation de Baehrens, l'autre constitué de 10 textes d'auteurs tardifs, majoritairement chrétiens. Les \textit{Priapea} offrent un texte à l'extrême fin de notre corpus en termes de chronologie\footcite{citroni_les_2008} et sont parmi les œuvres non étiquetées les plus portées sur le sexe. Par ailleurs, le style court tout particulier de cet ouvrage offre aussi un grand nombre de sujets différents sur un corpus finalement assez réduit (environ 3~000 mots). Le second corpus\footcite{glaise_2020_corpus_tardif} est constitué d'échantillons de dix-neuf œuvres de dix-sept auteurs, dont les passages (hors ponctuation) comptent \textit{a minima} 500 mots et dont les auteurs s'étendent du deuxième siècle au neuvième de notre ère (Eginhard) avec une concentration plus forte autour du quatrième (cf. \ref{corpus:glaise:dates}). Si la date de fin de ce corpus est plus tardive que les bornes que nous nous sommes posées, elles permettent d'évaluer tout de même la capacité du modèle à s'étendre dans le temps malgré un entraînement spécifique sur un corpus allant de la république au Haut-Empire, où les marques de la chrétienté sont de fait absentes.

\begin{table}[h]
\begin{tabular}{l|rrrrrrr}
Siècle           & 2 & 3 & 5 & 6 & 7 & 7 & 9 \\
Nombre d'auteurs & 1 & 3 & 5 & 4 & 2 & 1 & 1 \\
Nombre de textes & 2 & 3 & 6 & 4 & 2 & 1 & 1
\end{tabular}
\caption{Répartition par siècle des auteurs et œuvres du corpus Glaise}
\label{corpus:glaise:dates}
\end{table}

\subsubsection{Résultats généraux}

Le modèle propose des résultats convaincants sur une très grande majorité des \textit{Priapea}. La particularité de cette œuvre tient, d'un point de vue statistique, à leur très petite taille: une médiane à 30 mots sur les soixante-seize premiers poèmes et une moyenne à 40 à cause d'individus aberrants tels que le poème 51 (150 mots) et 68 (237)\footnote{Au niveau des quartiles, 25\% des \textit{priapées} font moins de 24 mots, 75\% en font moins de 68. Toutes les tailles ne concernent que les mots, hors ponctuation donc, mais incluant les clitiques.}. L'\textit{accuracy} peut donc grandement varier d'un texte à l'autre, une erreur pouvant représenter au pire une chute de 9\% d'un score\footnote{C'est le cas potentiellement des priapées 13, 59 et 62 dans notre édition.} ou au mieux 0,4\% pour la plus grande priapée. Les résultats restent plus qu'honorables, avec une baisse d'uniquement trois points par rapport au corpus de test principal et une accuracy globale de 94,2\% sur la tâche de lemmatisation (\textit{cf.} table \ref{tab:out_of_domain_global_accuracy} et figure \ref{fig:priapea_varations_boxplot}). Certaines priapées ont des scores particulièrement bas qui ne semblent pas avoir de lien avec leur taille: on notera ainsi les priapées 75 (82 mots, 84\%) et 46 (45 mots, 84\%) qui se trouvent dans la moyenne haute des tailles de poème.

Comme le corpus des \textit{Priapea}, le corpus de latin tardif propose des résultats avec une faible perte, de l'ordre de trois points aussi. L'ensemble des autres annotations performent globalement mieux que sur les \textit{Priapea} (annotation \textit{cas} à 92,9\% contre 89,4\%) ce qui peut potentiellement s'expliquer par la forme poétique et courte des poèmes qui ne laisse pas une grande place à l'erreur. Dans ce corpus, deux textes font office de résultats aberrants, à savoir Jérôme, \textit{In Hieremiam} et Grégroire de Tours, \textit{Historia Francum} qui chutent respectivement à 87 et 88\%. Les textes ont majoritairement la même taille, aux alentours de 500 à 600 mots, avec Commodien, \textit{Instructiones} comme texte le plus court (412 mots) et Augustin, \textit{De Civitate Dei} ainsi que Hilaire de Poitiers, \textit{Tractatus super psalmos} comme textes extrêmement longs\footnote{L'export des textes à corriger ayant allongé les textes involontairement, et l'annotation ayant été corrigée, l'ensemble de ces textes a tout de même été conservé, produisant de fait un écart avec la moyenne de 500 mots.}.

\begin{table}[h]
    \centering
    \begin{tabular}{l|rr|rr}
    \toprule
         Catégorie &  \multicolumn{2}{c}{\textit{Accuracy}} & \multicolumn{2}{c}{\textit{Accuracy} quand applicable} \\
    \midrule    
                {} &  Priapées &    Tardif                  & Priapées &    Tardif                                   \\
    \midrule
             lemma &     94,2 &    94,5                   &   N/A    &    N/A                                      \\
               Deg &     96,8 &    97,5                   &   90,3   &    91,5                                     \\
              Numb &     94,7 &    96,5                   &   94,1   &    96,4                                     \\
            Person &     99,0 &    99,7                   &   96,0   &    99,1                                     \\
Mood\_Tense\_Voice &     96,1 &    97,7                   &   87,0   &    92,3                                     \\
              Case &     89,4 &    92,9                   &   84,1   &    88,0                                     \\
              Gend &     91,7 &    92,8                   &   77,8   &    79,2                                     \\
               pos &     95,0 &    67,6                   &   N/A    &    N/A                                      \\
    \bottomrule
    \end{tabular}
    \caption{Résultat du modèle sélectionné sur le corpus des priapées et de latin tardif. L'accuracy quand applicable désigne la précision de l'algorithme sur les termes qui nécessitent une annotation, en d'autres termes, pour la personne par exemple, seules les annotations sur les verbes sont concernées.}
    \label{tab:out_of_domain_global_accuracy}
\end{table}


\begin{figure}[ht]
    \centering
    \includegraphics[width=0.7\linewidth]{figures/chap3/extensibilite/PriapeaBoxPlot.png}
    \caption{Variation des taux d'erreur sur l'ensemble du corpus des \textit{Priapées}}
    \label{fig:priapea_varations_boxplot}
\end{figure}

\subsubsection{Étude des erreurs}

Si ces chiffres annoncent potentiellement une très bonne nouvelle pour la période tardive et les sujets qui nous intéressent, ne s'attarder que sur l'échelle quantitative macro serait une erreur de taille: 6\% de réponses fausses laissent possible un très grand nombre d'erreurs dont il faut avoir conscience. D'abord, parce que dans un contexte traditionnel de richesse lexicale, la loi de Zipf se retrouvant logiquement dans chacun des corpus, on sait que les termes les plus fréquents représentent une très grosse partie du corpus final. Sur le corpus latin tardif, sur 15~188 formes, 4~908 formes dépendent de 34 lemmes qui apparaissent au minimum plus de 100 fois\footnote{Il s'agit du quantile 0,99 du corpus.}. Parmi ces lemmes, seuls 3 sont des verbes (\textit{sum}, \textit{dico}, \textit{facio}) et 3 des noms communs (\textit{deus}, \textit{dominus}, \textit{filius}); seuls les noms présentent une particularité thématique, les trois verbes étant particulièrement fréquents dans la langue latine en général. Il est fort peu probable que des erreurs arrivent sur des termes comme \textit{de}, \textit{pro} ou encore \textit{ab}. Ensuite, parce que notre corpus tardif global représente au moins la moitié de notre corpus latin pris en compte dans nos recherches, et que l'on ne saurait alors se satisfaire d'une inconnue aussi importante.

Dans un premier temps, nous proposons de regarder quelles POS posent un problème au lemmatiseur. Dans un contexte de distribution normale, une POS représentant 40\% des POS du corpus devrait posséder 40\% des erreurs. Or, en figure \ref{fig:latin_tardif_error_pos}, on remarque nettement que les noms propres (NOMpro) ont une très nette surreprésentation parmi les erreurs, avec des cas extrêmes de plus de 50\% des points d'\textit{accuracy} perdus, mais moins de 5\% des mots du corpus. Si l'on prend un corpus type de 500~mots, et 94\% de succès, on dénombre pour ces cas extrêmes 25 noms propres (5\% du corpus), 30 erreurs (6\% d'erreurs) dont 16 ou 17 sont des noms propres. Pour mesurer cet écart, on propose une mesure $Impact$ définie telle que $\text{Impact}_{pos} = \frac{\text{Erreurs}_{pos}}{\left | \text{Erreurs} \right |} \div \frac{\text{Tokens}_{pos}}{\left | \text{Tokens} \right |}$. Ainsi, un impact médian de 6,84 pour la POS nom propre indique que la proportion d'erreurs sur les noms propres est presque sept fois supérieure à la proportion de tokens qu'elle représente. Bien que les verbes et les noms communs aient une forte responsabilité dans le nombre d'erreurs, les médianes de leur Impact sont respectivement de 1,37. Ces impacts très légèrement supérieurs à 1 ne sont pas anormaux, tant certaines catégories ne représentent aucune erreur malgré une importante fréquence: les prépositions représentent ainsi  7\% en moyenne des corpus contre 0\% des erreurs, il en va de même pour les adverbes négatifs (14\%) et les conjonctions de coordination (7\%).

\begin{figure}
    \centering
    \includegraphics[width=1\linewidth]{figures/chap3/extensibilite/LatinTardifPosErrorBoxPlot.png}
    \caption{Responsabilité des POS dans les erreurs du lemmatiseur sur le corpus tardif.}
    \label{fig:latin_tardif_error_pos}
\end{figure}

Désormais, nous pouvons étudier la catégorie des noms propres de plus près. On peut classer les erreurs dans quatre grandes catégories:
\begin{itemize}
    \item Le lemme n'est pas reconnu comme nom propre, car il est vraisemblablement une personnification d'un nom commun par exemple ou bien qu'il n'est pas reconnu comme tel. On rencontre le premier cas avec \textsc{terra} à la place de \textsc{Terra}, \textsc{mater}/\textsc{Mater}, etc. et le second avec *\textsc{uenus} en place de \textsc{Uenus}, *\textsc{iesus}/\textsc{Iesus}.
    \item Le lemme est pris au pluriel à la place du singulier: c'est un problème commun sur les noms de peuples (\textsc{Graecus} contre \textsc{Graeci}).
    \item Le lemme est issu de la langue grecque ou hébraïque: \textsc{Christus}, \textsc{Israel}, \textsc{Pascha}, \textsc{Dauid}. Sur le corpus entier du LASLA, les occurrences de termes dépendant de la déclinaison grecque ne représentent que 1,1\% des noms, soit 5~585 individus sur environs 487~000. Bien que des noms d'origine grecque puissent suivre des déclinaisons latines comme Christus, on peut s'attendre à de plus grandes difficultés sur les autres termes tant ils ont été peu vus à l'apprentissage.
    \item Le lemme possède une variation graphique latine forte, chose inconnue dans le corpus classique, ainsi on trouve les couples \textit{Bethleem}/\textit{Bethlehem}, \textit{Hierusalem}/\textit{Ierusalem}, \textit{Hiezechiel}/\textit{Ezechiel}, ou encore \textit{Hadrianus}/\textit{Adrianus}.
\end{itemize}

Une fois la catégorie des noms propres analysée, on trouve, en plus de ces cas d'erreurs, les types suivants:
\begin{itemize}
    \item Une erreur sur le choix de racine, à savoir le choix entre le verbe \textit{debeo} et le nom \textit{debitum} par exemple.
    \item Une situation inverse au premier cas des noms propres, à savoir un nom commun pris pour un nom propre et dont la première lettre est ainsi capitalisée.
    \item Une erreur sur une racine reconstruite en adjectif de la première classe (en -us donc) à la place d'un nom, comme pour \textit{prophetia}/\textit{prophetius}.
    \item Un problème de lemme à l'orthographe "réactionnaire": dans le corpus des Priapées, on rencontre plusieurs fois l'erreur \textit{pedico} contre l'orthographe "juste" \textit{paedico} du LASLA.
\end{itemize}

Ces catégories semblent couvrir une très grande partie des erreurs hors pronom et permettent d'éclairer les résultats obtenus, mais aussi de cibler les domaines où le corpus peut être amélioré ou bien là où un correcteur automatique pourrait intervenir, en ciblant par exemple les lemmes d'origine grecque ou hébraïque. Ils ne donnent cependant qu'un éclairage particulier sur le résultat du modèle, mais pas sur sa capacité d'apprentissage et de l'influence du corpus d'entraînement.

\subsection{Étude de l'impact de la taille du corpus sur l'efficacité du modèle}
\label{lemmatisation:extensibilite:tailles}

Pour la première expérience, nous voulons évaluer l'impact de la taille d'un corpus sur les différents éléments de mesure à notre disposition. Nous proposons donc des coupes du corpus LASLA à 1\%, 5\%, 7,5\%, 10\%, 20\%, 40\%, et enfin 80\%. Dans chacun des jeux d'entraînement, on réservera 10\% du résultat de la coupe pour un \textit{set} de développement. La découpe se fait au niveau de chacun des fichiers d'entraînement du LASLA, à savoir un découpage au niveau de l'œuvre (\textit{Carmina} de Catulle) ou à un sous-niveau de celle-ci (César, \textit{Bellum Gallicum}, 1). Chaque séquençage connait alors la même variété d'auteur, la même variété de genre et de thème. Les jeux formés d'extraits pris aléatoirement dans les phrases constituées par le LASLA. Nous mesurons trois tâches, à savoir:

\begin{itemize}
    \item la tâche \texttt{lemma}, qui indique une compréhension du vocabulaire, de la syntaxe et de la morphologie,
    \item la tâche \texttt{POS}, qui indique une compréhension morphosyntaxique
    \item la tâche \texttt{Gend}, qui indique une compréhension de la morphologie.
\end{itemize}

Durant l'entraînement, le modèle principal ayant obtenu les meilleurs résultats sur le corpus global a montré des faiblesses sur les corpus de petite taille. Pour chaque corpus, deux configurations ont donc été testées, une avec un module GRU simple couche (RNN=1) et un module GRU double couche (RNN=2) pour la partie de \textit{character embeddings}. Les résultats (\textit{cf.} Table \ref{tab:percent_corpus_comparaison}) montrent que l'augmentation de la taille du corpus avec une même variété de genres et d'auteurs a un impact fort (plus de 1 point de gagné) jusqu'à l'utilisation de 40 \% du corpus, quelle que soit la tâche, ou presque\footnote{Il existe en effet des valeurs aberrantes pour le genre et la POS au moment du passage de 7,5\% à 10\% du corpus.}. Au contraire, les passages de 40\% à 60\% du corpus et de 60\% à 80\% ne représentent des gains que de 0,23 et 0,46 point respectivement, et donc, un passage de 96,79\% d'\textit{accuracy} pour le corpus à 40\% à seulement 97,48\% pour celui à 80\%, pour un doublement du nombre de tokens annotés utilisés pour l'entraînement, soit une augmentation d'environ 600~000 tokens (passage de 622~238 tokens d'entraînement à 1~226~227).


\begin{table}[]
    \centering
    \resizebox{\textwidth}{!}{%
    \begin{tabular}{ll|rrrrrrrr}
    \toprule
     RNN &  \% Corpus &  Lemme &  $\Delta$(Lemme) &  $\theta$ Lemme &  $\Delta$($\theta$ Lemme) &   POS &  $\Delta$(POS) &  Genre &  $\Delta$(Genre) \\
    \midrule
       1 &     0,010 &  0,741 &     0,000 &    0,725 &       0,000 & 0,791 &   0,000 & 0,851 &    0,000 \\
       1 &     0,050 &  0,889 &     0,149 &    0,858 &       0,133 & 0,887 &   0,096 & 0,902 &    0,052 \\
       2 &     0,075 &  0,921 &     0,031 &    0,855 &      -0,002 & 0,921 &   0,034 & 0,930 &    0,028 \\
       2 &     0,100 &  0,936 &     0,015 &    0,894 &       0,039 & 0,929 &   0,008 & 0,933 &    0,003 \\
       2 &     0,200 &  0,949 &     0,013 &    0,882 &      -0,012 & 0,947 &   0,019 & 0,948 &    0,015 \\
       2 &     0,400 &  0,968 &     0,019 &    0,922 &       0,041 & 0,961 &   0,013 & 0,961 &    0,012 \\
       2 &     0,600 &  0,970 &     0,002 &    0,924 &       0,002 & 0,964 &   0,003 & 0,965 &    0,004 \\
       2 &     0,800 &  0,975 &     0,005 &    0,910 &      -0,014 & 0,966 &   0,002 & 0,970 &    0,005 \\
    \bottomrule
    \end{tabular}}
    \caption{Résultat des tests sur un corpus de test de 140~000 mots représentant 80\% du corpus de test original. RNN correspond au nombre de couches du module d'encodage des caractères. Les mesures de gain entre deux étapes, notées $\Delta$ correspondent à la différence avec le modèle précédent, il faut donc lire \enquote{Le modèle à 80\% du corpus performe 0,46 point de plus que le modèle à 60\% du corpus.}. On ajoute une mesure sur un corpus hors domaine, celui de latin tardif, que l'on désigne dans les colonnes par un $\theta$}
    \label{tab:percent_corpus_comparaison}
\end{table}

D'après ces résultats, on peut penser que le lemmatiseur apprend "assez" de la richesse de vocabulaire et de la syntaxe d'un auteur pour s'enrichir jusqu'à environ 40\% de ses textes. Une stratégie de création de corpus pourrait donc se focaliser sur l'annotation d'extraits d'auteurs, plutôt qu'une approche d'annotation complète. Cependant, on peut aussi voir les gains minimes comme de véritables bonds en valeur absolue: si, pour un corpus de 10 millions de tokens, les résultats passent de 96,09\% d'accuracy à 96,97\%, et que ces résultats se retransmettent de manière régulière, le nombre d'erreurs passe de 391~000 à 303~000, soit une chute de 88~000 erreurs. Les progressions en \texttt{POS} et en \texttt{Genre} sont toutes les deux limitées au même titre que celle de lemme. Par ailleurs, lors d'une évaluation sur le corpus de latin tardif, on se rend compte que l'extensibilité connait de grosses variations et y compris des reculs (-0,2 point, -1,2 point, -1,4 point sur les passages 5\%, 20\% et 80\%). Nous ne pouvons ici que faire des hypothèses, et deux semblent plus probables que d'autres. La première consiste à estimer qu'il y a un phénomène aléatoire sur la performance en hors domaine, et que, sur une dizaine d'entraînements, on trouverait majoritairement des modèles améliorant le score précédent. Si cette hypothèse est alléchante, elle n'en reste pas moins assez faible, surtout pour des baisses de 1,4 point en hors domaine, mais une augmentation même faible en test classique: elle serait plus crédible sur un écart de moins de 0,5 point, comme nous avons déjà pu le voir sur d'autres entraînements. La deuxième hypothèse consiste à un surapprentissage sur le corpus d'entraînement: le modèle se surspécialiserait sur le corpus classique, se rendant incapable de reconnaître le corpus tardif. Là encore cette hypothèse pose problème, car elle n'expliquerait pas pourquoi, sur un corpus à 100\%, nos scores rebondissent à plus de 94,5\% (+3 points). Il n'est cependant pas à exclure une savante combinaison des deux phénomènes qui créerait ainsi une chute des plus remarquables.



\begin{comment}
\subsection{Taille et diversité: corpus Perseus}
\label{lemmatisation:extensibilite:perseus}

Nous avons vu en \ref{subsec:treebank_corpora} que le corpus de Perseus pour le treebank, mis en forme pour le projet Universal Dependencies (\Gls{UD}), présente un corpus comprenant peu de textes et peu de mots comparés au corpus du LASLA (26 000 mots contre 1,7 million), mais cependant une variété de style et d'époques assez forte parmi les corpus de ce genre.

\newpara

Pour compléter l'analyse précédente \ref{lemmatisation:extensibilite:tailles}, on reproduit un corpus similaire à celui de Perseus, dans la limite où 4 œuvres présentes dans ce dernier ne le sont pas dans le corpus LASLA, à savoir les \textit{Fabulae} de Phèdres, les \textit{Res Gestae} d'Auguste, la \textit{Vie d'Auguste} de Suétone, la \textit{Vulgate} de Jérôme. On propose de remplacer pour le même nombre de phrases par deux autres œuvres \footnote{Ces œuvres sont mentionnées dans le corpus de Perseus original, mais n'ont pas été nettoyées pour le projet \Glspl{UD})} à savoir les \textit{Fastes} d'Ovides et le \textit{Satyricon} de Pétrone. Le corpus d'entraînement, de développement et de test sont constitués à partir du seul corpus d'entraînement en \ref{lemmatisation:extensibilite:tailles}: les séquences sont prises aléatoirement, sans prendre en compte la position de la séquence dans l'œuvre originale de Perseus, l'équivalent de 10\% et 20\% du nombres de phrases en entraînement sont utilisées pour le corpus spécifique de développement et de test (ci-après \texttt{perseus-test}). Le résultat de cette génération de corpus donne 1361 séquences contre 1334 originellement, mais surtout une forte augmentation du nombre de mots 26 081 contre 18 184 démontrant des tailles de séquences plus grandes dans le corpus LASLA (\textit{cf.} tables \ref{table:perseus-ud:chunks-and-tokens}, \ref{table:lasla:perseus-ud}).  Pour l'évaluation, nous proposons à la fois le résultat sur un corpus \texttt{perseus-test} qui représente la même diversité d'œuvres, mais aussi le corpus test utilisés pour nos mesures. Nous n'évaluons que les tâches \texttt{lemma}, \texttt{pos} et \texttt{Gend} pour les raisons mentionnées plus haut. 

\newpara

% Analyse des résultats

\begin{table}[h]
\centering
\begin{tabular}{lll}
\toprule
 Title                  & Chunks & Tokens \\ \midrule
 Auguste, Res Gestae    & 38     & 708    \\
 Caesar                 & 24     & 352    \\
 Cicero, In Catilinam   & 137    & 1897   \\
 Phèdre, Fables         & 233    & 2397   \\
 Properce               & 224    & 2776   \\
 Salluste, Catilina     & 336    & 4999   \\
 Suétone, Vie d'Auguste & 109    & 2046   \\
 Tacite, Histoires      & 64     & 866    \\
 Virgile, Énéide        & 15     & 142    \\
 Vulgate                & 154    & 2001   \\ \midrule
 Total                  & 1334   & 18184  \\ \bottomrule
\hline
\end{tabular}
\caption{Répartition par œuvres du nombres de séquences et de tokens dans le corpus Perseus UD 2.1}
\label{table:perseus-ud:chunks-and-tokens}
\end{table}

\begin{table}[h]
\centering
\resizebox{\textwidth}{!}{%
\begin{tabular}{l|llll|llll}
\toprule
                  & \multicolumn{4}{l}{Test Perseus}        & \multicolumn{4}{l}{Test LASLA}          \\ 
                  & Accuracy & Précision & Recall & Support & Accuracy & Précision & Recall & Support \\ \midrule
\textbf{Lemma}    &          &           &        &         &          &           &        &         \\
\textit{Tous}     & 0.6550   & 0.2864    & 0.2919 & 5754    & 0.6132   & 0.0990    & 0.0858 & 172968  \\
\textit{Inconnus} & 0.0306   & 0.0094    & 0.0155 & 1534    & 0.7480   & 0.2760    & 0.2438 & 12790   \\
\textit{Ambigus}  & 0.7558   & 0.4873    & 0.5027 & 434     & 0.019    & 0.0035    & 0.0061 & 53069   \\ \midrule
\textbf{POS}      &          &           &        &         &          &           &        &         \\
\textit{Tous}     & 0.8427   & 0.7991    & 0.7482 & 5754    & 0.8183   & 0.6831    & 0.6062 & 172968  \\
\textit{Inconnus} & 0.5913   & 0.0947    & 0.0790 & 1534    & 0.7835   & 0.5866    & 0.5358 & 12790   \\
\textit{Ambigus}  & 0.7672   & 0.5453    & 0.5678 & 799     & 0.5671   & 0.0755    & 0.0691 & 53069   \\ \midrule
\textbf{Gender}   &          &           &        &         &          &           &        &         \\ 
\textit{Tous}     & 0.8637   & 0.8448    & 0.6139 & 5754    & 0.8646   & 0.6315    & 0.4288 & 172968  \\
\textit{Inconnus} & 0.6780   & 0.2043    & 0.1463 & 1534    & 0.6604   & 0.6981    & 0.6698 & 12790   \\
\textit{Ambigus}  & 0.6506   & 0.7027    & 0.6620 & 538     & 0.7067   & 0.1615    & 0.1274 & 53069   \\ \bottomrule
\end{tabular}%
}
\caption{Évaluation d'un modèle linéaire entraîné sur le corpus \texttt{perseus-train} (26 081 tokens, 9 textes) dans Pie contre un corpus test de Perseus et un corpus LASLA plus générique (les modèles avec decodeur plafonnant à 0 pour la tâche \texttt{lemma})}
\label{table:lemmatisation:perseus-scores}
\end{table}

\end{comment}

\subsection{Variation de genre, variation de style}
\label{lemmatisation:extensibilite:prose-vers}

\subsubsection{Mise en place}

Afin d'aller encore plus loin dans l'analyse de l'effet du corpus, nous proposons de reproduire les principes de l'étude de C. Poudat et D. Longrée\footcite{poudat2009variations}. Dans cette étude, les auteurs analysent l'impact de l'unicité de genre, de style, chronologique sur l'entrainement de taguer automatiques, en particulier TreeTagger cité plus haut. L'expérience repose sur les comparaisons suivantes:
\begin{itemize}
    \item Style d'ouvrage contre style d'auteur (César, \textit{Bellum Gallicum}; César, \textit{De bello civili})
    \item Style d'auteur contre style de genre (César; Pseudo-César et Salluste)
    \item Style du genre à travers le temps (César, Pseudo-César et Salluste; Quinte-Curce et Tacite)
    \item Style de la prose à l'épreuve des variations génériques (histoire; traités, dialogues)
    \item Style de la prose contre le vers
\end{itemize}{}

Cependant, il nous semble important de dévier sur les corpus utilisés et constitués: les corpus constitués dans le cadre de cette évaluation présentent une très grande variation de taille. Or, dans le contexte d'une modélisation, et de l'impact de ces traits (à savoir genre, style, période), il ne semble pas souhaitable de comparer des entraînements effectués sur 52~000 tokens contre des modèles entraînés sur près de 400~000 tokens\footcite[par exemple, p.~135, 2.~2.~4]{poudat2009variations}. Nous reprenons donc les grandes lignes et constituons 4 sous-expériences.

\begin{itemize}
    \item Une première expérience, sur de petits corpus en vers (environ 90~000 tokens), permettra d'identifier l'impact d'un auteur dans un modèle, ou d'une époque: un corpus double Horace \& Lucrèce (91~555 tokens, 6 œuvres différentes), un Virgile (87488 tokens), un Ovide (95~409 tokens).
    \item Une seconde expérience, sur des corpus plus larges (environ 117~000 tokens), permettra d'étudier l'impact du genre avec deux corpus Sénèque (\textit{Ad Lucilium}, 118~801 tokens; Autres œuvres philosophiques, 117~550 tokens), un corpus mixte théâtre Sénèque et Plaute (115~571 tokens, 16 œuvres), un corpus Cicéron court (116~367 tokens, 2 œuvres), un corpus Tacite (117~552 tokens, 4 œuvres).
    \item Une troisième expérience cherchera à analyser l'impact du mode de rédaction (vers contre prose) sur l'extensibilité des résultats avec un corpus Prose (257 034 tokens, 7 auteurs, 9 œuvres) et un corpus Vers (259 717 tokens, 7 auteurs, 17 œuvres).
    \item Une quatrième expérience portant sur des corpus de 390~000 tokens profitera de la richesse du corpus de Cicéron, en évaluant l'impact de la diversité d'auteurs (corpus divers, 392~402 tokens, 13 auteurs, 24 textes) contre celle de la diversité d'œuvres d'un même auteur (corpus Cicéron Discours, 391~390 tokens, 1 auteur, 44 œuvres)\footnote{On peut retrouver le découpage dans la table en annexe \textbf{A FAIRE}}.
\end{itemize}{}

Les œuvres de test sont toujours prises en hors domaine, à savoir qu'elles ne peuvent pas faire partie des mêmes œuvres que les œuvres d'entraînement. Dans la mesure du possible, elles sont aussi issues de corpus d'auteurs différents (\textit{cf.} Table \ref{table:lemmatisation:extensibilite:test-corpus}) et uniquement des données du LASLA, afin de ne pas introduire de variations d'annotation ou d'édition.

\begin{table}
\resizebox{\linewidth}{!}{%
\begin{tabular}{lll|ll|r|l}
    \toprule
     Auteur & œuvre & Passage & Genre & Tokens & Auteur présent dans corpus \\ \midrule
     César & Guerre des Gaules & 3 & Histoire & 3 637 & Prose \\
     Catulle & Poésie & Complet & Poésie & 13 020 & \\
     Cicéron & De l'Amitié & Complet & Traité & 9 272 & Cicéron Petit \\
     &&&&& Cicéron Discours \\
     &&&&& Prose \\
     Cicéron & Catilinaires & 1 & Traité & 9 272 & \textit{idem} \\
     Quinte Curce & Histoires & 3 & Histoire & 7 175 & Prose \\
     &&&&& Divers \\
     Horace & Épodes & Complet & Poésie & 3 071 & Vers \\
     &&&&& Horace+Lucrèce \\
     &&&&& Corpus Divers \\
     Ovide & Ibis & Complet & Poésie &  4 196 & Vers \\
     &&&&& Ovide \\
     &&&&& Corpus Divers \\
     Salluste & Catilina & Complet & Histoire & 10 598 & \\
     Sénèque & De la brièveté de la vie & Complet & Dialogues & 6113 & Sénèque (Les 3) \\
     &&&&& Prose \\
     &&&&& Divers \\
     Sénèque & Médée & Complet & Tragédie & 5 685 & \textit{idem} \\
     Tacite & Germanie & Complet & Histoire & 5 648 & Prose \\
     &&&&& Tacite \\
     &&&&& Divers \\ \bottomrule
\end{tabular}%
}%
     \caption{œuvres et nombre de tokens pour le test }
     \label{table:lemmatisation:extensibilite:test-corpus}
\end{table}

\subsubsection{Outils de mesure}

Avant d'analyser les résultats, nous devons nous assurer que les variables sont indépendantes, à savoir que les résultats sur deux corpus de tests ne sont pas liés entre eux: le résultat sur un Cicéron n'est pas nécessairement lié au résultat sur Properce. Pour cela, on calcule un $\chi^{2}$ qui, ramené au degré de liberté de notre table, nous donne une valeur $p$ indiquant la probabilité que nos variables soient dépendantes. Pour cela, on calcule une fréquence attendue, ici le nombre de bonnes réponses attendues toutes choses égales par ailleurs, notée $e{ij}$ où $i$ correspond au modèle d'entraînement parmi $m$ et $j$ le corpus de test parmi $n$:

\begin{equation}
    e_{ij} = \left ( \sum_{i=1}^{m}O_{i} \times \sum_{j=1}^{n}O_{j}  \right ) / \sum_{i=1}^{m}\sum_{j=1}^{n}O_{ij}
\end{equation}

Cette probabilité nous permet ensuite de calculer un degré de magnitude entre la valeur observée et la valeur attendue appelé résidu de Pearson noté $r$ et calculé via la formule

\begin{equation}
    r_{ij} = \frac{O_{ij} - e_{ij}}{\sqrt{e_{ij}}}
\end{equation}

Par exemple, en figure \ref{fig:lemmatisation:longree:poetes}, le nombre de bonnes réponses $O_{Virgile/Eclogues}$ est de 5~224, mais sa probabilité attendue, prenant en compte les résultats du modèle Virgile sur les autres textes et des autres modèles sur le corpus \textit{Eclogiae}, est de 4~884. Son résidu de Pearson est donc de 4,86, ce qui indique un très fort écart à ce qui était attendu.

\subsubsection{Poètes contre poètes}

\begin{figure}[ht]
    \hspace*{-0.05\linewidth}
    \begin{minipage}[c]{0.55\linewidth}
        \includegraphics[width=1\linewidth]{figures/chap3/longreeVariante/LongreeVariante-AccuracyStyleDePoesie-Lemme.png}
    \end{minipage} \hfill
    \begin{minipage}[c]{0.55\linewidth}
        \includegraphics[width=1\linewidth]{figures/chap3/longreeVariante/LongreeVariante-AssocPlotStyleDePoesie-Lemme.png}
    \end{minipage}
    \caption{\textit{Accuracy} en lemme de modèles entraînés sur des poètes avec leurs résidus de Pearson}
    \label{fig:lemmatisation:longree:poetes}
\end{figure}

Les corpus d'entraînement des auteurs Virgile et Ovide ont un effet net (r=49 et r=1,6) sur les corpus de tests des auteurs respectifs. Si l'\textit{accuracy} de Virgile s'en ressent, celle d'Ovide reste globalement basse. Il faut rappeler ici la taille des corpus d'entraînement, seulement 90~000 tokens environ. Au contraire, le corpus mixte Horace et Lucrèce ne semble pas décoller, y compris sur les corpus d'Horace ou de Catulle. Les trois corpus n'ont pas une bonne reconnaissance du corpus prosaïque (César) ou semi-prosaïque (Pétrone). Bien que la \textit{Medea} de Sénèque soit mieux reconnue dans l'ensemble, le résidu de Pearson ne permet pas d'établir une forte corrélation avec un modèle en particulier.

\subsubsection{Style d'auteur, style de genre}

\begin{figure}[ht]
    \hspace*{-0.05\linewidth}
    \begin{minipage}[c]{0.55\linewidth}
        \includegraphics[width=1\linewidth]{figures/chap3/longreeVariante/LongreeVariante-AccuracyStyleDAuteurStyleDeGenre-Lemme.png}
    \end{minipage} \hfill
    \begin{minipage}[c]{0.55\linewidth}
        \includegraphics[width=1\linewidth]{figures/chap3/longreeVariante/LongreeVariante-AssocPlotStyleDAuteurStyleDeGenre-Lemme.png}
    \end{minipage}
    \caption{Accuracy en lemme de modèles avec leurs résidus de Pearson: les corpus d'entraînements ont été élaborés pour permettre d'identifier l'importance de l'auteur et du genre en termes de lemmatisation.}
    \label{fig:lemmatisation:longree:auteurVSforme}
\end{figure}

Dans cette étude, deux prosateurs font face à la prose de Sénèque sous deux formes (ses dialogues et traités d'une part, ses lettres d'autre part) et aux vers de Sénèque et de Plaute. On note tout de suite une nette relation entre accuracy et forme, puisque Sénèque a des R et des \textit{accuracies} particulièrement hauts sur Catulle et le théâtre de Sénèque. Au contraire, l'œuvre versifiée de Sénèque ne semble pas mieux comprendre le Sénèque prosaïque que d'autres prosateurs (résultats similaires entre Cicéron, \textit{De Amicitia}, \textit{In Catilinam} et le \textit{De Brevitate Vitae} de Sénèque). L'inverse est aussi remarqué puisqu'aucun des modèles de prose de Sénèque ne réagit nettement mieux que celui de Tacite, en \textit{accuracy} ou en résidu de Pearson. Les auteurs de proses fonctionnent bien sur eux-mêmes et Cicéron ainsi que Tacite fonctionnent particulièrement bien sur Salluste: on peut supposer que l'\textit{In Verrem} et les \textit{Philippicae} de Cicéron, formant le corpus d'entraînement ici, partagent avec Tacite et les \textit{In Catilinam} de Salluste des questions de pouvoir politique, de corruption et - par extension - un vocabulaire et un style assez proche. Au contraire, les œuvres philosophiques de Sénèque ne sont pas assez marquées, et en tout cas trop éloignées du \textit{De Amicitia} dans leur style pour influer de quelque manière. Pour finir, après deux tests, on se rend compte du caractère presque neutre du \textit{Satyricon} de Pétrone qui ne semble profiter ni de la prose ni du vers ni encore de la période, mais qui profite tout de même du passage de 90~000 à 115~000 tokens du corpus d'entraînement.

\subsubsection{Versificateur contre prosateur}

\begin{figure}[ht]
    \centering
    \includegraphics[width=0.7\linewidth]{figures/chap3/longreeVariante/LongreeVariante-AccuracyModeDExpression-Lemme.png}
    \includegraphics[width=0.7\linewidth]{figures/chap3/longreeVariante/LongreeVariante-AssocPlotModeDExpression-Lemme.png}
    \caption{\textit{Accuracy} en lemme de modèles avec leurs résidus de Pearson: chaque corpus est formé uniquement de prose ou de vers.}
    \label{fig:lemmatisation:longree:proseVSvers}
\end{figure}

Avec la troisième expérience, le corpus double en taille, mais présente désormais une unité de forme: celle d'un modèle en prose contre celle d'un modèle en vers. Les résultats sont nets, avec une très forte efficacité du modèle prose sur la prose, en $r$ ou en \textit{accuracy}, avec un écart allant jusqu'à 12~points pour le \textit{Bellum Africanum}. Ici, trois auteurs sont totalement absents  des corpus d'entraînement et doivent retenir notre attention, car, logiquement, le style d'auteur ne joue pas dans les scores: il s'agit de Pétrone, de Salluste et de Catulle (nous excluons le Pseudo-César du \textit{Bellum Africanum} par la proximité stylistique recherchée par l'auteur avec César). Nous notons que si la prose fonctionne bien mieux, les vers ne semblent pas nécessairement surperformer en \textit{accuracy} avec des écarts plus réduits, mais tout de même présents. Encore une fois, le \textit{Satyricon} fait office de frontière, avec l'écart le plus réduit de tout le corpus.

\subsubsection{Diversité des auteurs contre celle d'un auteur}

\begin{figure}[ht]
    \centering
    \includegraphics[width=0.7\linewidth]{figures/chap3/longreeVariante/LongreeVariante-AccuracyDiversiteDAuteursContreDiversiteDAuteur-Lemme.png}
    \includegraphics[width=0.7\linewidth]{figures/chap3/longreeVariante/LongreeVariante-AssocPlotDiversiteDAuteursContreDiversiteDAuteur-Lemme.png}
    \caption{Analyse de l'efficacité d'un grand corpus d'auteur contre une diversité de styles (\textit{Accuracy} en lemme et résidu de Pearson)}
    \label{fig:lemmatisation:longree:divAuteursVSTailleAuteur}
\end{figure}

Sur cette dernière expérience, nous posons la question suivante: dans le cadre d'un modèle entraîné sur un corpus déjà conséquent 390~000 tokens, la diversité des auteurs et des genres prime-t-elle sur celle d'un auteur ? Autrement dit, dans le cadre d'une création de corpus d'entraînement, vaut-il mieux annoter beaucoup d'auteurs, quitte à ne pas chercher à annoter les œuvres complètes, ou bien un seul ? Les résultats, en \textit{accuracy}, sont sans appel: la diversité des auteurs permet de surclasser un corpus uni-auteur sur tous les auteurs hors domaine (Salluste, Catulle, Pétrone), sur les auteurs partagés à l'exception de l'auteur utilisé dans le corpus unifié. Et nous noterons sur ce dernier corpus test que la différence est extrêmement limitée, de 0,01 à 0,02 points malgré une spécification du modèle sur le corpus cicéronien. Enfin, il semble que le \textit{Satyricon} fonctionne enfin mieux en \textit{accuracy} avec un mélange prose et vers, cependant, le $r$ ne nous permet pas de conclure en faveur de cette hypothèse.

\subsubsection{Conclusion}

Sur la base de ces résultats, on distingue les effets suivants:
\begin{itemize}
    \item Le style d'un auteur semble plus déterminant que l'époque de rédaction (figure \ref{fig:lemmatisation:longree:poetes}) ou la diversité en termes d'auteurs (figure \ref{fig:lemmatisation:longree:divAuteursVSTailleAuteur}), y compris quand ce style est copié par des imitateurs. Cet effet tend à se limiter sur des corpus très importants (figure \ref{fig:lemmatisation:longree:divAuteursVSTailleAuteur})
    \item La forme, en particulier poétique, prend le dessus sur le style d'un auteur (figures \ref{fig:lemmatisation:longree:auteurVSforme} et \ref{fig:lemmatisation:longree:proseVSvers}).
    \item La proximité de genre (philosophie, histoire en \ref{fig:lemmatisation:longree:auteurVSforme}) a un impact relativement faible. Mais, cet impact faible est à relativiser avec une possible proximité des thèmes des œuvres sélectionnées pour l'expérience..
\end{itemize}

\section{Modèle LASLA+ et étiquetage du corpus}

Après évaluation des limites du modèle sur le hors-domaine, puis des potentielles influences de composition des corpus d'entraînement, on entraîne un second modèle, \textit{LASLA+}, testé sur le même corpus de test, mais que l'on augmentera des corpus hors domaine ainsi que de la \textit{Vulgate} taguée par \textit{Proiel} alignée semi-automatiquement. Nous revenons alors sur les problèmes d'alignement de corpus, les probables erreurs qui se sont glissées et les difficultés que ce nouveau modèle rencontre. Nous aborderons ensuite la question du prétraitement (\textit{preprocessing}) et de son impact sur la lemmatisation pour finir sur une évaluation rapide du corpus annoté automatiquement, via la recherche de termes posant des problèmes dans le corpus qui nous intéresse.

\subsection{Modèle LASLA+ avec ajout des corpus et alignement de la \textit{Vulgate}}

Notre corpus LASLA ne présente aucun auteur tardif, et donc aucun auteur chrétien. Pour remédier à ce manque, un corpus latin tardif (mais non médiéval) existe et est libre de droits: celui de PROIEL\footcite{haug_creating_2008}. Proiel propose trois textes qui ne font pas partie du corpus LASLA, à savoir la \textit{Vulgate} de Jérôme, le \textit{Peregrinatio Aetheriae} et enfin l'\textit{Opus agriculturae} de Palladius. Pour des raisons de temps et d'optimisation de ce dernier, seule la Vulgate a été alignée avec le corpus du LASLA.

On pourrait penser qu'un alignement d'un modèle à l'autre est une tâche simple, d'autant qu'\textit{a priori} le LASLA découpe rarement les lemmes en fonction de leur sens, mais uniquement en fonction de leur POS et leur genre\footnote{On note tout de même \textit{populus1}, le peuple, et \textit{populus2} parmi les exceptions.}. Cependant, un alignement lemma/POS/morphologie requiert de s'assurer que les deux corpus ont le même usage de la morphologie et dans une moindre mesure des POS. Pour les lemmes, l'alignement se fait dans l'ordre suivant:
\begin{enumerate}
    \item Le lemme est présent tel quel dans le dictionnaire du LASLA. L'annotation lemme est donc conservée.
    \item Le lemme existe avec un seul indice de désambiguïsation: il est aligné
    \item Le lemme existe avec plusieurs indices de désambiguïsation, on compare alors les POS, par exemple pour \textit{ad} entre ADV et PRE.
    \item Dans le cas où le lemme ne rentre pas dans ces catégories, on vérifie qu'il n'existe pas sous une autre orthographe, par exemple \textit{adulescens} chez Proiel existe sous \textit{adolescens1} ou \textit{adolescens2} suivant la POS, \textit{excaedo} en \textit{excido1}.
\end{enumerate}

La morphologie nécessite par ailleurs un alignement via la suppression du genre pour les formes qui ne sont ni des adjectifs, ni des pronoms, ni des participes. Ensuite, les genres sont alignés via la déclinaison dont relèvent les adjectifs ou les formes. La morphologie est aussi complétée: les indéclinables ne sont pas analysés ainsi par Proiel (mais plutôt comme des absences de morphologie), ce qui nécessite un réalignement. On finit par normaliser l'orthographe des formes et lemmes pour les couples U/V et J/I puis une première conversion est faite. Si cette étape est itérative, l'objectif reste de réduire au maximum les lemmes à corriger manuellement. Enfin, sur les derniers lemmes, plusieurs centaines, on aligne à la main avec le référentiel du LASLA et on crée des entrées quand le lemme est absent, ce qui est souvent le cas pour les noms propres. Cette dernière catégorie est très importante et démontra une nouvelle difficulté pour l'apprentissage, avec des cas d'hapax comme cette généalogie de \textit{Luc}, 3.23:

\begin{quote}
\blockquote{\textit{Et ipse Jesus erat incipiens quasi annorum triginta, ut putabatur, filius Joseph, qui  \\fuit Heli, qui fuit Mathat, \\
qui fuit Levi, qui fuit Melchi, qui fuit Janne, qui fuit Joseph, \\
qui fuit Mathathiæ, qui fuit Amos, qui fuit Nahum, qui fuit Hesli, qui fuit Nagge, \\
qui fuit Mahath, qui fuit Mathathiæ, qui fuit Semei, qui fuit Joseph, qui fuit Juda, \\
qui fuit Joanna, qui fuit Resa, qui fuit Zorobabel, qui fuit Salatheil, qui fuit Neri, \\
\textit{...} \\
qui fuit Sarug, qui fuit Ragau, qui fuit Phaleg, qui fuit Heber, qui fuit Sale, \\
qui fuit Cainan, qui fuit Arphaxad, qui fuit Sem, qui fuit Noë, qui fuit Lamech, \\
qui fuit Methusale, qui fuit Henoch, qui fuit Jared, qui fuit Malaleel, qui fuit Cainan, \\
qui fuit Henos, qui fuit Seth, qui fuit Adam, qui fuit Dei.}}
\end{quote}

Cette présence d'hapax pose un double problème dans la mesure où l'orthographe hébraïque ne se fixe pas par usage, telle \textit{Sarug/Seruch}, ou des formes qui n'existent dans le \textit{Forcellini} ou son appendice \textit{Onomasticon} que dans le corps du texte, tel \textit{Azor}, présent dans l'entrée \textit{Sadoc}.

Après cet alignement, on ajoute les données des \textit{Priapea} et du corpus chrétien, nous permettant d'obtenir une plus grande diversité de textes et une plus grande couverture thématique et temporelle et donc lexicale. L'entraînement de ce nouveau modèle obtient un score très légèrement moins bon que le corpus d'origine (\textit{cf.} table \textbf{A AJOUTER}). Il est probable que l'ajout de données dont l'édition n'a pas été traitée de la même manière que les autres données du corpus du LASLA ajoute du bruit: c'est finalement une bonne chose, car cette variété permet aussi de s'éloigner d'un corpus "parfait" et de se rapprocher \textit{de facto} du corpus que l'on traitera ensuite sur lequel l'intervention humaine, au niveau de la segmentation syntaxique, est minimale.

\subsection{Importance du \textit{pre-processing}}

Nous avons parlé jusque là principalement des méthodes d'entraînement et de l'algorithme responsable de la lemmatisation. Il peut exister un certain nombre de problèmes lorsque l'on passe d'une application sur données éditées - tokenisées à la main au niveau phrase et mot, harmonisées en orthographe, etc. - à des données \textit{in situ}. Cette problématique a déjà été étudiée sur d'autres types de tâches comme la classification de textes\footcite{camacho-collados_role_2018} et montre en général des conséquences - au pire mineures - en termes de résultats. Or, nous avons vu que les données du corpus LASLA possèdent une très forte éditorialisation, allant des normalisations de couple u/v et j/i à l'absence de ponctuation, qui, dans le cas d'une lemmatisation \textit{in situ}, influencerait probablement la lemmatisation. En effet, dans la phase de transcodage de \textit{pie}, toute lettre inconnue - comme v par exemple - est codée \textit{de facto} en tant que lettre inconnue et non en tant que v, puisqu'elle n'a pas pu avoir de représentation apprise dans l'espace de projection des caractères. Il faut donc s'assurer que ces lettres, lorsqu'une alternative est possible, sont traduites, remplacées. Du côté de la ponctuation, le choix est tout autre: il faut pouvoir s'appuyer sur la ponctuation pour les tâches telles que la reconnaissance des phrases ou celles des abréviations, mais s'assurer qu'elles disparaissent pour la phase de lemmatisation pure puis réapparaissent pour les futures étapes de notre recherche. D'autres cas, comme la présence de formes grecques ou de nombres, sont indiqués et simplifiés afin d'éviter au lemmatiseur d'apprendre à lemmatiser le grec ainsi que les chiffres romains en chiffres arabes, chose faisable par un programme simple, car extrêmement régulé, mais plus difficile pour un réseau neuronal avec très peu d'exemples.

\begin{table}[ht]
    \centering
    \resizebox{\linewidth}{!}{%
    \begin{tabular}{l|rrrr}
    \hline
                                              & A                 & B                 & C              & D              \\ \hline
    Normalisation des lettres                 & Non               & Oui               & Non            & Oui            \\
    Suppression des formes inconnues (points) & Non               & Non               & Non            & Oui            \\
    Tokenisation Phrase                       & Ponctuation Forte & Ponctuation Forte & 35 Mots        & 35 Mots        \\ \hline
    lemma              & \textbf{-1,51} & -0,22          & \textbf{-1,62} & -0,23 \\
    Deg                & -0,30          & -0,16          & -0,26          & -0,06 \\
    Numb               & -0,41          & -0,31          & \textbf{-0,56}          & -0,12 \\
    Person             & -0,05          & 0,00           & -0,06          & -0,03 \\
    Mood\_Tense\_Voice & -0,11          & -0,02          & -0,12          & -0,04 \\
    Case               & \textbf{-0,76} & \textbf{-0,66} & \textbf{-1,41} & \textbf{-0,32} \\
    Gend               & -0,28          & -0,13          & -0,23          & -0,11 \\
    pos                & \textbf{-0,57} & -0,33          & \textbf{-0,66} & -0,18 \\ \hline
    \end{tabular}}
    \caption{Impact de la normalisation sur l'\textit{accuracy}, en \%.}
    \label{tab:lemmatisation:normalisation}
\end{table}

Dans ce cadre, on propose une expérience assez simple: à partir des données de latin tardif, qui contiennent une orthographe \textit{in situ} et une ponctuation, on fait varier le contenu à lemmatiser en fonction de trois variantes combinables:
\begin{itemize}
    \item Une normalisation des lettres et formes, principalement u/v et j/i,
    \item Une suppression pour l'étape de lemmatisation de la ponctuation et des formes inconnues comme le grec,
    \item Une tokenisation du texte sur la base de la ponctuation forte (points, guillemets, parenthèses, etc.) ou sur une fenêtre de 35 mots. Cette fenêtre est la fenêtre proposée par défaut par \textit{pie}. Plusieurs fenêtres raisonnables (multiples de 5 jusqu'à 35) ont été testées ici, mais elles présentaient des résultats dans des marges minimales.
\end{itemize}
Les résultats, visibles en table \ref{tab:lemmatisation:normalisation}, montrent un fort impact de la conservation des signes de ponctuation et des lettres inconnues avec des scores baissant de 1,51\% à 1,62\% quand les deux sont ignorés. Il faut noter que la conservation des lettres inconnues a - semble-t-il - un impact assez limité (-0,22\%). Au niveau des tâches auxiliaires, toutes les tâches sont touchées sauf la reconnaissance de personne, le cas et la POS étant particulièrement influencées avec des baisses allant jusqu'à -0,76\%. Dans le cadre où la tokenisation des phrases est faite tous les 35 mots, la lemmatisation, le cas et la POS sont légèrement influencées aux alentours de -0,20\%. On note ici que l'accumulation de bruit n'est pas de l'ordre de l'addition: plus le nombre de régularisations manquantes croît, plus les tâches sont influencées. Si les signes de ponctuations ont été supprimés des calculs d'\textit{accuracy} dans l'expérience précédente, nous pouvons retrouver aussi les bizarreries que crée la lemmatisation de signes totalement inconnus tels que dans l'\textit{incipit} de la \textit{Bellum Gallicum} (cf. \textit{infra}). Dans une telle situation, il est clair que le prétraitement est inséparable d'une bonne lemmatisation via des réseaux neuronaux.

\begin{quote}
    \textbf{Source}\\
    \blockquote{Gallia omnis divisa in partes tres , quarum unam incolunt Belgae , aliam Aquitani tertiam , qui ipsorum lingua Celtae nostra Galli appellantur.}\\
    \textbf{Lemmatisation brute}\\
    \blockquote{gallia omnis \textbf{dico} in pars tres \textbf{qvi} qvi vnvs incolo belgae \textbf{vnde} alivs aqvitani tertivs \textbf{vi} qvi ipse lingva celtae noster galli appello \textbf{nonvs}}\\
    \textbf{Lemmatisation avec normalisation}\\
    \blockquote{gallia omnis divido in pars tres , qvi vnvs incolo belgae , alivs aqvitani tertivs , qvi ipse lingva celtae noster galli appello .}\\
    
    Exemple de bruit causé dans la lemmatisation lorsqu’ aucun prétraitement n'est effectué. En gras les erreurs introduites par la lemmatisation: on remarque qu'en dehors de la lemmatisation des points qui est totalement erronée, \textit{divisa} est aussi mal analysé, certainement à cause du bruit de ponctuation et de la lettre v.
    \label{quote:lemmatisation:gallia-errors}
\end{quote}


%\subsection{Étiquetage du corpus: exploration et différences entre le modèle LASLA et le modèle LASLA+}
 
%Dans cette section, nous ne proposons pas une étude aussi poussée que celles en, \ref{subsec:lemmatisation:hors-domaine} mais une analyse rapide des différences de résultats sur quelques termes qui ont montré des limites pour nos modèles.

\section*{Conclusion}


%\chapter{Un modèle de détection du discours sexuel et sur la sexualité}
% A intégrer ici :
% quittant pour de bon les problématiques grammaticales du chapitre précédent

L'un des objectifs de notre recherche est d'aider à l'identification de passages dont la sexualité est un outil discursif ou un thème dans les textes latins classiques et tardifs. Comme J.N. Adams, nous ne distinguerons pas dans notre recherche les usages de la sexualité comme moyens d'appuyer une problématique autre (cf. usage de \textit{mollus} comme outil de dévalorisation) d'un discours sur les pratiques d'un personnage (comme ceux sur Philaenis chez Martial). Nous nous intéresserons dans un premier temps à la définition, dans le cadre du traitement automatique des langues, de cette tâche d'identification en l'intégrant dans le faible réseau des recherches similaires dans le domaine littéraire. Nous présenterons ensuite les différentes architectures de données et de modèles que nous avons testés. Nous nous intéresserons ensuite aux résultats, à ĺ'interprétation des échecs de certains modèles, à l'extensibilité des modèles, mais aussi aux valeurs constamment apprises par l'ensemble des modèles. Enfin, nous utiliserons les modèles comme outils de prédiction dans le cadre d'une étude de terrain sur le \textit{Centon Nuptial} d'Ausone et l'Énéide de Virgile, nous aidant à comprendre l'intérêt du modèle pour la recherche "traditionnelle", ses limites et les possibles améliorations à prévoir.

\section{La détection thématique, les sciences de l'antiquité et l'apprentissage profond}

% Aggarwal and Zhai (2012) defined the text classification task as given the training set D = {X1, ..., Xn}, where each data point Xi ∈ D is labeled with a value derived from the set of labels are numbered {1. . . k}

\begin{quote}[Cento Nuptialis, 115]{Ausone}
    \textit{huc iuvenis nota fertur regione \textbf{viarum}} \\
    \enquote{Là se porte le jeune chef, par des voies qu'il connaît}\footcite{ausone_d_2010}
\end{quote}

\begin{quote}[Epigrammata, II.33.4]{Martial}
    \textit{Haec qui basiat, o Philaeni, \textbf{fellat}. \\}
    \enquote{Qui embrasse ces choses, ô Philaenis, suce.}
\end{quote}

\begin{quote}[Priapea, 54]{Priapées}
    \textit{\textbf{CD} si scribas temonemque insuper addas, qui medium uult te scindere , pictus erit.} \\
    \enquote{Si tu écris CD et que tu ajoutes en plus un timon, Il y aura dessiné ce qui veut te déchirer le milieu}
\end{quote}

\begin{quote}[Gynaeciorum Sorani, I.21]{Caelius Aurelianus}
    \textit{Interior ergo pars collo \textbf{matricis} connectitur , exterius vero fibris adnexa est quas \textbf{pinnas} vocant \textbf{feminini sinus}} \\
    \enquote{Ainsi, la partie la plus à l'intérieure est connectée au col de la \textbf{matrice}, quant à l'extérieur, elle est rattachée aux plis que l'on appelle les \textbf{ailes} du \textbf{giron féminin}.}
\end{quote}

Ces quatre textes ne partagent aucun terme voire lexique en commun autre que des mots outils et pourtant tous partagent une chose: ils parlent de sexualité. Par des voix détournées (métaphore d'Ausone), par un lexique identifié (Martial), à travers un rébus et un langage figuré (Priapées) ou bien dans un discours médical (Caelius), chacun de ces textes en parle. Pour le premier, dans le contexte de la nuit de noces et de la chambre des mariés, l'actualisation de /mouvement/ de \textit{fertur} et de /chemin-empruntable/ de \textit{viarum} portent l'isotopie d'une pénétration débutante. De même, les sèmes d'/entrée/ et /diviser/ de \textit{scindere} et /centre/ de l'/humain/ pour \textit{medium te} émettent pour la Priapée 54 l'information suffisante pour comprendre une menace de \textit{fututio} ou de \textit{pedicatio}: l'auteur annonce vouloir élargir l'endroit pénétré avec violence. Enfin, pudiquement et via l'emprunt au grec, ce sont les sèmes /forme-élancée/ et /deux/ de \textit{pinnas} attachés aux /sillon/, /féminin/ et /producteur/ de \textit{sinus feminini}, représentant le vagin, qui évoquent les lèvres inférieures. Ces textes ont donc la particularité, via des actualisations diverses, de représenter des isotopies de la sexualité.

La détection isotopique\footnote{Par simplification, nous utiliserons aussi le terme de \textit{thématique}.} de la sexualité entre dans de multiples catégories de tâches du TAL. Ces tâches peuvent relever de problématiques purement commerciales, comme la classification automatique d'objets à vendre sur des \textit{market places}, mais sont aussi entrées dans le domaine des sciences de l'antiquité, et en particulier de l'étude des textes latins et grecs. Nous présenterons deux techniques de classification utilisées. Puis, nous nous intéresserons tour à tour à l'incursion du domaine computationnel dans le domaine de la philologie et de la littérature. Enfin, nous nous intéresserons aux tâches qui rejoignent la nôtre, soit par leurs méthodes, soit thématiquement.

\subsection{Entre classification de texte et analyse de similarité}

\subsubsection{Que classons-nous ?}


%En gros, ici:
%- quelle est l'unité que l'on cherche à détecter.
%- quelles sont les méthodes que l'on peut utiliser (classification vs. similarité)
% Faire aussi de la review rapide de littérature hors champs pour comparer

La détection isotopique revient à enseigner à une machine à reconnaître dans un échantillon de texte pré-séquencé $E$ l'existence d'un motif thématique $M$. Cette détection peut se relever facile dans quelques cas: la présence de termes d'un lexique spécifique, ici \textit{mentula}, \textit{futuo} ou \textit{cunnus}, ne laisse que peu de doute sur l'un des sèmes de l'échantillon. Cependant, elle devient beaucoup plus problématique pour la machine voire pour l'être humain lorsqu'un réseau d'abstraction, d'actualisations sémiques, permet de faire ressortir une isotopie, entre autres car \enquote{ce n'est pas la récurrence de sèmes déjà donnés qui constitue l'isotopie, mais, rétroactivement, la présomption d'isotopie qui permet d'actualiser des sèmes, voire les sèmes}\footcite[p. 34]{rastier_isotopie_1985}. Une approche d'apprentissage supervisé peut s'avérer intéressante: on apprend alors à la machine à passer de la présomption à la confirmation ou l'information de la présence d'isotopie.

Le développement de tels modèles pose un intérêt majeur pour les sciences humaines: il permettrait en effet de constituer rapidement des corpus thématiques. Ceux-ci donneraient à leur tour la capacité d'approfondir nos connaissances des modèles de pensées, anciens ou non, en forçant constamment cette présomption et en reposant sur d'autres biais que ceux des spécialistes. Un groupe de recherche pourrait alors, à partir de corpus d'exemples préparés à l'avance, commencer à entraîner un premier modèle, quitte à obtenir un haut taux d'erreurs, de corriger ces erreurs et d'agrandir ce corpus au fur et à mesure de ces corrections, dans un cercle vertueux apprentissage-annotation-correction. 

Ces modèles relèvent principalement de deux types d'architectures: une "classique", utilisant un apprentissage par étiquette (\textit{label} en anglais), et une plus récente et empruntée au domaine de la vision par ordinateur, qui s'effectue par mesure de similarité.

Les modèles de classification de textes par étiquette sont particulièrement ancrés dans le domaine du traitement automatique des langues. Il s'agit pour un modèle d'être entraîné à annoter une séquence de mots comme faisant partie d'une (classification simple) ou plusieurs catégories (classification \textit{multilabel}). Dans leur chapitre de 2012, C. Aggarwal et C.X. Zhai\footcite{aggarwal_survey_2012} la définissent comme l'entraînement d'un modèle visant à faire reconnaître dans un ensemble d'échantillons $D$ un nombre défini $k$ de catégories\footnote{Nous reprenons ici leurs signes.}. Ils réintroduisent la notion de classification douce (\textit{soft}) et dure (\textit{hard}): une épigramme de Martial serait ainsi attribuée à l'auteur Martial, tandis qu'elle pourrait être affiliée à l'isotopie de la sexualité et à celle des bains. Ils font aussi entrer la distinction entre une classification à probabilité -- un pourcentage est fourni pour chaque classe -- à une classification binaire où un seul label est fourni sans mesure: pour un texte de l'\textit{Anthologie Latine}, un texte pourrait ainsi soit présenter une probabilité d'être de Sénèque indépendamment de sa probabilité d'être de Virgile dans le cas du premier type de classification, tandis que le deuxième produirait une affirmation ou infirmation d'autorité.

Dans leur chapitre, C. Aggarwal et C. X. Zhai citent alors les modèles les plus communs en 2012, presque uniquement des modèles n'appartenant pas à la catégorie de l'apprentissage profond: classificateur bayésien, SVM, à base de règles ou d'arbres décisionnels. Ces modèles continuent d'être utilisés: ils représentent des solutions souvent peu coûteuses computationellement, efficaces dans de nombreux cas. Ces solutions ont par ailleurs la particularité de ne pas être nécessairement moins bonnes que les méthodes plus modernes, à apprentissage profond, dont le coût de calcul -- et donc énergétique -- est en général beaucoup plus grand\footcite{fell_comparing_2019}. Aujourd'hui, en \textit{deep learning}, ces modèles sont principalement représentés par des modèles dont la dernière couche est constituée de réseaux linéaires, bien qu'il existe une survivance pour cette couche des CRFs ou d'autres formes de couches de classification, notamment dans le cadre de classification au niveau mot\footcite{alkhwiter_part--speech_2021, shang_speaker-change_2020}.

% Les modèles de classification peuvent être utilisés à divers niveaux du texte, du document entier au mot, en passant par des séquences arbitraires produites par l'analyseur ou des séquences éditoriales. 
% Virer cette partie car reproduction de plus bas ?
% Parmi les classifications de texte, l'une des applications les plus connues dans le domaine des humanités est celle de l'analyse de sentiment, qui cherche à distinguer - au plus simple - des phrases positives ou négatives sur un sujet: R. Sprugnoli et ses collègues développaient ainsi en 2016 un outil pour étudier l'évolution de la perception de certains sujets dans un corpus historique\footcite{sprugnoli_towards_2016}. D'autres exemples de type de classification existent: des plus sensibles, tels que la détection d'appel à la haine ou d'apologie du terrorisme sur les réseaux sociaux, à des sujets au contraire beaucoup plus légers comme l'attribution d'auteur, de période d'écriture ou de dialecte. 

\begin{figure}
    \centering
    \includegraphics[width=\linewidth]{figures/chap4/Siamois.png}
    \caption{Architecture simplifiée d'un réseau siamois}
    \label{fig:chap4:structures:siamese-network}
\end{figure}

Malheureusement, les modèles de classification nécessitent en général un nombre conséquent de données d'entraînement pour être généralisables et donc dépasser le simple stade de preuve de concept. Si notre thématique est large et ses exemples répétés dans la littérature latine quelle que soit la période, l'objectif n'est pas ici de ne proposer une solution qu'à la question sexuelle: d'autres thématiques pourraient intéresser d'autres chercheurs (par exemple, la paternité, les phénomènes météorologiques en mer, etc.) et une étude des limites quantitatives s'impose. On peut aussi s'intéresser à un découpage de notre isotopie en plusieurs sous-ensembles, soit du point de vue de la stylistique -- usage métaphorique, explicite, comparatif, allusif --, des acteurs -- femmes, hommes, enfants, esclaves, etc. -- ou soit, encore, d'un point de vue moral -- condamnation, neutre, positif. Qu'il s'agisse d'un autre thème ou d'une section du nôtre, l'usage du cycle apprentissage-prédiction-correction comme outil de constitution de corpus dès la dizaine, la centaine ou le millier d'exemples change le temps de travail nécessaire à cette compilation. Or, pour ce qui est des architectures neuronales, la classification par étiquette est connue pour ses limites sur de petits corpus\footnote{Bien qu'il s'agisse d'un corpus d'images, on peut trouver une étude relativement bien fournie sur une comparaison entre ces deux types d'outils: \cite{pasupa_comparison_2016}}. % Exemple ?

On s'intéresse alors à un autre mode de classification, celui par mesure de similarité et en particulier des modèles de réseaux siamois. Ces derniers incorporent la particularité de comparer au moins deux échantillons, dont l'un a une classe connue: après avoir produit une représentation numérique des échantillons, comme les modèles plus classiques, le réseau siamois est entraîné à réduire ou augmenter les distances entre ces échantillons afin de les catégoriser. À travers ces comparaisons, il va détecter plus rapidement les traits saillants des petites catégories. Cette architecture a le bénéfice d'être nettement plus efficace dans des situations dites de \textit{few-shot learning}, c'est-à-dire à faible set d'entraînement. 

Largement utilisé dans le domaine de la \textit{computer vision}, on en trouve un usage en histoire du livre à travers l'intéressant travail du projet Filigrane\footcite{shen_large-scale_2021} qui permet, dans des scores raisonnables, de classer facilement des filigranes papiers à partir de très peu d'exemples d'entraînement, mais y compris de potentiellement détecter des filigranes inconnus. Il est important de prendre en compte, lorsque l'on compare les scores des deux types d'architecture dans des publications différentes, la quantité des données d'entraînement: ces réseaux apprennent rapidement, pour des situations avec parfois de très nombreuses catégories. Dans un tel contexte, les modèles de classification ne proposent que rarement d'aussi bons résultats sans tomber dans le surapprentissage.

Avec un peu plus de deux mille exemples issus d'Adams, dans quel cas de figure se situe notre problème ? Si un tel set de données peut paraître important du point de vue d'un romaniste, il est en fait relativement bien faible comparé à des sets de données d'analyse de sentiment avec la pauvreté morphologique de l'anglais: assez petit, l'\textit{IMDB Movie Reviews Dataset}\footcite{maas-EtAl:2011:ACL-HLT2011}, un ensemble de critiques de films associées à des notes, ne compte \enquote{que} 50~000 exemples, à l'autre bout du spectre, l'\textit{Amazon Review Data}\footcite{ni_justifying_2019} en compte 233.1 millions. En regard de ces chiffres, notre modèle se limiterait-il à une recherche par similarité ? Rien n'est moins sûr, car, au contraire de l'exemple des filigranes, non seulement le nombre d'exemples pour une même catégorie est beaucoup plus élevé (les 2000 exemples relèvent de l'isotopie sexuelle), mais la variété à l'intérieur de ce set est beaucoup plus large: entre une métaphore militaire du courtisan d'Ithaque qui bande son arme et une claire mention d'\textit{irrumo}, la ressemblance est mince, tandis que deux prises de vue du même filigrane ont plus de points communs. C'est entre autres pourquoi nous nous appliquerons à évaluer la résistance des deux types d'architecture en fonction de plusieurs paramètres, y compris celui de la taille du set d'entraînement, dans l'optique de favoriser la construction \textit{ex nihilo} de futurs corpus sur d'autres thématiques.

% C'est une question importante, mais je me demande si ça arrive au bon endroit.

Question quelque peu laissée pour compte jusque-là, la question de l'unité syntaxique ou sémantique dans laquelle on souhaite identifier l'isotopie sexuelle reste assez centrale. On peut distinguer plusieurs niveaux allant de l'oeuvre à celui du mot. Le premier niveau, le niveau \textit{oeuvre}, pose un problème d'élasticité des tailles -- entre une épigramme de deux vers de Martial et un livre de Tite-Live un monde existe -- et d'utilité de l'outil: il n'est probablement pas intéressant de savoir si on parle de sexualité dans un livre d'historien romain, mais plutôt d'où se trouveraient de tels usages. Le deuxième niveau consisterait donc à détecter l'information au sein de petites séquences éditoriales comme des  \textit{paragraphes}, mais ils posent le problème de l'équivalence en poésie classique ou en théâtre: si une épigramme ou un court poème peut facilement être comparé à un paragraphe épistolaire de Cicéron en termes de taille, la question de leur équivalence en théâtre classique (une scène ? Une réplique ? Quid des répliques extrêmement courtes ?) ou en poésie épique semble difficile à résoudre. Ils restent alors séquences syntaxiques, représentées par des \textit{phrases} de l'éditeur, ou les unités qui les composent, les \textit{mots}, qui peuvent être intéressantes: la première forme une unité assez longue pour pouvoir développer des thématiques et la plus fine, dans un second temps, fournit une granularité à l'analyse. Cependant, cette unité syntaxique n'est pas sans inconvénient: comme le dit F. Rastier, le phénomène d'isotopie \enquote{est indépendant par principe des structures syntaxiques et de la prétendue limite de la phrase. Une isotopie peut s'étendre sur deux mots, sur un paragraphe, sur tout un texte.}\footcite[p. 34]{rastier_isotopie_1985}. Par exemple, si la mention claire de \textit{futuo} dans une phrase ne laisse pas de doute sur la thématique sexuelle, celle de l'allusion construite autour de la prise d'un chemin connu chez Ausone nécessite peut-être le texte en entier: d'ailleurs, ce morceau emprunté à Virgile (\textit{Aen.} 11.530) n'en avait pas le sens dans le contexte original; le jeune chef était Metebus, en exil avec sa fille. 

% Reprendre ici
\subsection{L'analyse de phrases en sciences humaines}

\subsubsection{La détection de métaphore: une tâche similaire ?}

Tâche formalisée plus clairement dans les années 2010, la détection automatique de métaphore partage avec notre tâche son échelle (l'information est recherchée au niveau phrase) et sa difficulté pour la machine (détecter les faisceaux de preuves qui indiquent un usage non littéral d'un certain nombre de mots). Il faut cependant prendre garde au sens qui est prêté à \textit{métaphore} ici: dans leur ouvrage fondateur pour la partie traitement automatique du langage, \enquote{A Method for Linguistic Metaphor Identification}\footcite{steen_method_2010}, Steen et. al indiquent clairement s'insérer dans la tradition établie par Ortony, mais surtout Lakoff et Johnson\footcite{lakoff_metaphors_2003} dont les citations ponctuent le texte collectif. La définition anglo-saxonne utilisée relève ainsi plus de la détection d'usage figuratif du langage que d'un usage métaphorique au sens entendu en stylistique: "Sortir vainqueur d'un débat" serait ainsi une accumulation de métaphores, vainqueur empruntant au domaine du combat, et sortir du mouvement.

L'usage de cette définition de la métaphore conduit les auteurs à choisir le niveau du mot comme portant l'information quand ils établissent leur propre jeu de données, le \textit{VU Amsterdam Metaphor Corpus} (VUA\footcite{steen_method_2010}). Une compétition est ouverte pour la première fois en 2018 dans le cadre de la grande conférence NAACL sous le nom de \textit{Shared Task on Metaphor Detection}\footcite{leong_report_2018}. Ce système de compétition est assez commun dans le monde du TAL et de l'intelligence artificielle en général: il s'agit, quelques mois avant une conférence (ici cinq), de fournir un set de données d'entraînement, de demander aux équipes de fournir soit un script permettant d'annoter des données tests, soit d'annoter, sans connaître la vérité de terrain, ces données puis d'obtenir le classement au moment de la conférence. Cette compétition utilise alors le set de données cité précédemment, et retient le \textit{F1-Score} comme méthode de mesure: la présence d'un set particulièrement déséquilibré -- il y a plus de mots sans usage figuratif -- favorise l'usage d'un outil laissant plus de place aux erreurs dans le décompte final. Durant cette compétition, le seul outil n'utilisant pas de réseau neuronal profond (DNN) est classé dernier sur toutes les tâches. En tête de l'ensemble des tableaux, deux des trois outils utilisent en plus d'informations sémantiques portées par des \textit{embeddings} des informations morphosyntaxiques, l'annotation POS des mots. Le papier \textit{bot.zen}\footcite{stemle_using_2018} utilise par ailleurs plusieurs \textit{embeddings} différents, certains entraînés sur des corpus de personnes en cours d'apprentissage de la langue anglaise\footnote{Les auteurs s'appuient sur l'hypothèse d'une surreprésentation d'un langage figuratif chez les personnes apprenant une nouvelle langue, \textit{cf.} \cite{klebanov_argumentation-relevant_2013}}. 

Une seconde compétition a eu lieu en 2020\footcite{leong_report_2020} dans le cadre de la conférence ACL. Cette compétition voit logiquement l'arrivée massive des \textit{transformers} (RoBERTa, Bert, Albert, etc.) dans le domaine, et ces architectures prennent constamment au moins les 5 premières places sur les 12 participants retenus. Le meilleur des modèles, \textit{DeepMet}, propose encore une architecture utilisant des embeddings, cette fois issue de transformers, et des informations morphosyntaxiques (deux formes de POS). Le F1-Score reste l'élément principal pour évaluer les contributions et les meilleurs scores atteignent 76.9\% toutes catégories morphosyntaxiques confondues sur le dataset VUA. Pour aller plus loin dans la compréhension de ce que peuvent apprendre ces algorithmes, ces deux compétitions se sont intéressées au taux de succès en fonction des genres (Académique, Fiction, Presse, Conversations) et à l'écart de performance en fonction de ceux-ci. % Et ? Ajouter quelque chose ?

La relation entre cette tâche et les études littéraires ou la stylistique est cependant bien ténue: le seul lien que l'on puisse faire est celui de la présence d'une personne identifiée comme faisant partie du domaine des humanités numériques, J. B. Herrmann, parmi les auteurs du corpus VUA et de l'ouvrage collectif fondateur\footcite{steen_method_2010}. Peu de publications réutilisent la notion du côté SHS, J. B. Hermann ne proposant elle-même qu'une publication autour de ce thème, une analyse quantitative des métaphores dans le registre universitaire\footcite{herrmann_high_2015}. Et des outils issus de cette compétition, il semble qu'aucun n'ait vu un usage en SHS. Comme beaucoup de recherches en TAL, l'usage de ces modèles développés dans le cadre de ces \textit{workshops} est mince dans des recherches littéraires. Ainsi, si les auteurs de \textit{Metaphor Detection in a Poetry Corpus}\footcite{kesarwani_metaphor_2017} s'intéressent aux capacités de transfert d'apprentissage entre un corpus poétique et non poétique, il ne s'intéresse pas à la réutilisation de cette méthode pour classer les auteurs en fonction de leur usage figuré du langage. Cette absence d'intérêt pour des analyses littéraires quantitatives montre bien cette forme d'isolement qui peut se produire autour du TAL, focalisé sur la linguistique et l'obtention du meilleur score. Il est pourtant facile d'imaginer une application de ces modèles, dans le milieu francophone: nous pourrions, pour des auteurs du 20e siècle, étudier si un usage figuratif du langage sépare les auteurs s'opposant au surréalisme, comme l'annonce distinctement Jules Supervielle dans son \textit{En songeant à un art poétique}, à ces surréalistes; ou encore, nous pourrions tout autant analyser si le langage figuratif est un critère de style pour les auteurs ayant baigné dans plusieurs genres, comme Jean-Paul Sarte ou Albert Camus.

\subsubsection{L'attribution d'auteur}

Loin des questions métaphoriques, celle de l'attribution d'auteur est un domaine particulièrement présent en philologie classique: à travers la transmission des textes, l'autorité a parfois été perdue (textes anonymes), modifiée (textes remaniés), créée (faux de l'antiquité tardive et du haut moyen-âge) et enfin noyée (textes composites). De la question homérique à l'identité des auteurs préfixés habilement par \textit{Pseudo} en passant par les dizaines de sermons douteusement attribués à Augustin et Chrysostome, la recherche d'autorité est une tradition qui s'est longtemps basée sur une connaissance d'experts et une analyse fine de quelques passages, notamment en cherchant des informations permettant de postdater un texte à la mort de son auteur par exemple. 

Cette question de l'attribution d'auteur a par ailleurs la particularité d'avoir un vrai retentissement médiatique: si la question homérique ne passionne plus les médias de masse -- l'a-t-elle jamais fait ? --, celle de l'autorité shakespearienne ou moliériesque passionne encore aujourd'hui\footnote{En 2019, via une publication de J.B. Camps et F. Cafiero sur le sujet, la débat a été clôs -- ou relancé, au choix -- et a atteint les rédactions de grands journaux français (le Figaro, le Point), les radios (France Culture) et jusqu'au plateau de télévision (BFM-TV).}. L'anonymat d'un auteur -- comme celui de J. K. Rowling\footcite{juola_how_2013} -- ou l'hypothétique subterfuge de l'auteur connu qui aurait utilisé des plumes anonymes passionnent: l'enquête policière rentre ainsi dans le débat littéraire. Et c'est bien l'incursion de la statistique, et en particulier celle qui est calculée sur ordinateur, qui a complètement changé la donne. 

De premières publications dans les années 50 et 60, la stylométrie, c'est-à-dire la quantification du style, a connu dans le monde des humanités numériques un véritable pic à partir de 2015 (jusqu'à 10\% des conférences en humanités numériques, \textit{cf.} Figure \ref{fig:chap4:attribution-auteurs}). Mais si toute attribution d'auteur n'est pas stylométrie et inversement, il convient de séparer aussi la première des catégories en deux sous-catégories: la découverte d'auteur (\textit{authorship verification}), une méthode en général non supervisée , qui cherche à faire des regroupements d'autorité sans présumer des individus impliqués (par exemple, chercher à identifier les différents auteurs dans l'Anthologie Latine), et l'attribution d'auteur (\textit{authorship attribution}) à proprement parler, méthode supervisée avec l'intention d'entraîner la machine à reconnaître des auteurs en particulier.

Dans le domaine classique, quatre articles retiennent notre attention par la diversité de leurs approches. L'article de Kestemont et al.\footcite{kestemont_authenticating_2016} sur le corpus césarien approche la problématique comme une question ouverte, et conclut à l'existence possible de 5 auteurs: la méthode se base sur les propriétés textuelles classiques de la stylométrie, sélectionnant ainsi les unigrammes de mots et n-grams de caractères les plus fréquents du corpus, en normalisant leur présence et en projetant l'ensemble dans un espace contraint. Il s'inscrit dans un renouveau du domaine stylométrique après 2015 et intègre un plus gros réseau de publications et d'interventions où les principaux acteurs, Kestemont, Eder et Rybicki, sont aussi les auteurs de l'outil popularisant la méthode, stylo\footcite{stylo_r}. %

L'article de R. Gorman\footcite{gorman_author_2020} s'intéresse quant à lui à l'utilité de la syntaxe avançant l'hypothèse que, contrairement à l'information lexicale, cette information est susceptible de mieux résister aux différences génériques et thématiques. Il approche la question sous l'angle de la vérification: une fois les propriétés sélectionnées parmi les informations morphologiques et de dépendance, la représentation vectorielle de chaque texte est transmise à un réseau décisionnel (SVM ou régression linéaire) avec des résultats allant de 85\% à 100\% de réussite.

L'article de B. Nagy\footcite{nagy_metre_2021} s'intéresse de la même manière à des propriétés non lexicales, mais ne s'intéresse pas à l'information syntaxique pour autant: ici, c'est la construction des hexamètres, à travers les césures, les accents, les combinaisons de dactyles et de spondées, qui informe la décision finale, faite par quatre algorithmes différents, mais dont les deux plus performants sont SVM et une régression linéaire. Les scores sont particulièrement élevés, mais l'auteur lui-même émet des réserves: le mètre n'est qu'une facette du style, et il n'est pas improbable qu'une situation de surapprentissage ait lieu. Par ailleurs, sa méthode n'est pas robuste à un changement de style volontaire et stylistiquement plausible. Enfin se pose la question de l'impact des choix d'éditeurs qui -- parfois -- choisissent aussi la forme qui leur semble la plus cohérente stylistiquement et métriquement parlant, et qui favoriserait d'une certaine manière le choix d'auteur. 

Enfin, le dernier article de Vanni et al.\footcite{vanni_textual_2018} ne s'intéresse directement ni à une question de style, ni à une question philologique, mais plutôt à la possibilité de proposer un meilleur réseau d'encodage pour la vérification d'auteur. Comme les deux précédents, il s'inscrit dans la logique de vérification d'auteurs, mais est le seul à introduire une approche par réseau neuronal via projection, encodage et couche décisionnelle. C'est aussi le seul article qui ne filtre pas l'information initiale par des seuils de présence: dans le cadre de la vérification d'auteur, une telle absence de seuil peut introduire des variations thématiques comme outils de séparation. Ces variations thématiques fonctionnent bien sur un corpus fermé (César peut parler de Cicéron, mais pas de Domitien, Domitien sera donc une propriété négative pour la reconnaissance de César), mais sur un corpus plus rapproché dans le temps et uni du point de vue générique et thématique, le fonctionnement du modèle laisse peu de doute sur sa propension à se focaliser sur ces traits saillants du lexique et à ignorer ainsi les traits de style.

Bien que la vérification d'auteur soit une tâche de classification de texte, il est intéressant de voir qu'elle a la particularité de reposer souvent le masquage de l'information lexicale générique et thématique pour faire ressortir le stylome: cette suppression d'information est à l'opposé de la tâche qui nous intéresse. Cependant, l'intérêt des informations syntaxiques relevé par les deux articles de Gorman et Nagy doit être évidemment pris en compte, d'autant plus qu'il répète des besoins relevés par la recherche en détection de métaphores où la POS améliorait les scores.

\subsubsection{L'intertextualité: un domaine central de philologie computationnelle}

% Reprendre ici, premiers paragraphes gérés mais y a une duplication de vetus latina

L'intertextualité, ou l'étude des \enquote{transferts de matériaux textuels à l'intérieur de l'ensemble des discours\footcite{aron_intertextualite_2010}}, a trouvé dans les études quantitatives et numériques en général un nouveau souffle. Domaine formalisé dans les années 1960, il n'en est pas moins une tâche pluriséculaire de recherche de connexions entre plusieurs textes: on peut ainsi trouver dans le travail des scoliastes médiévaux et des éditeurs de la patrologie de Migne des chercheurs d'intertextualités lorsque, face à un texte de père de l'Église, ils s'efforcent de retrouver pour chaque passage, chaque allusion ou citation, la source biblique utilisée. Tant est que c'est à travers ce travail d'annotation que l'on finit par produire une -- puis des -- \textit{vetus latina}, bibles composites reprenant les citations de versions de la Bible supplantées par la \textit{Vulgate}. Bien que \enquote{l'intertextualité ne se [limite] pas à une série de citations ou d'allusions repérables}\footcite{aron_intertextualite_2010}, c'est plutôt cette branche qui intéresse dans un premier temps la recherche quantitative sur corpus -- et pour cause, la détection de \enquote{clichés et de stéréotypes} est aussi plus difficile pour le lecteur humain. Les manifestations d'intertextualité relevant de la citation ou de l'allusion sont très clairement impossibles à ignorer dans la tradition du commentaire, religieux ou non, présent dans la littérature latine à travers les premiers ouvrages chrétiens, mais aussi les grammairiens ou commentateurs tels que Donat ou Porphyrion: pour Donat, le texte dans les éditions transmises est clairement cité, pour les pères de l'Église, les citations ou allusions sont le coeur des ouvrages. 

Par ailleurs, ce travail de citation entre auteurs est lui-même un angle d'étude pour la littérature latine et grecque\footcite{darbo-peschanski_citation_2004}: on peut penser, à trois siècles d'écart, à la citation directe de Martial par Ausone (\enquote{\textit{Lasciva est nobis pagina, vita proba}}\footnote{\enquote{Notre page est lascive, notre vie probe.}}) dans ses propres épigrammes qui peut lui permettre d'introduire un avertissement au lecteur. Cette pratique gréco-latine de la citation est telle qu'elle permet d'ailleurs de reconstituer, même à l'état de fragment, les oeuvres perdues, de la \textit{Vetus Latina} aux oeuvres d'Accius citées par Priscien, Festus, Macrobe, etc. La période antique relève donc de la véritable mine d'or pour le développement de modèles automatiques de reconnaissance de citation, car elle propose des corpus qui ont déjà été maintes fois étudiés, de sorte que \enquote{petits} projets comme le \textit{Digital Fragmenta Historium Graecorum} atteignent 7256 fragments tandis que le plus gros d'entre eux, \textit{Biblindex}, représente 370.000 références vérifiées et inscrites. Historiquement, il est intéressant de remarquer que le même lieu qui accueillait le colloque de C.  Darbo-Peschanski en 2004 est celui qui réunit pour la première fois -- à notre connaissance -- ce domaine computationnel et celui des belles lettres à travers la journée d'étude \textit{International Workshop on Computer Aided Processing of Intertextuality in Ancient Languages} qui a eu lieu du 2 au 4 juin 2014 à Lyon, ville qui héberge aussi le projet \textit{Biblindex}.

L'\textit{intertextualité quantitative}, dont C. W. Forstall et W. J. Scheirer figent le nom en 2019\footcite{forstall_quantitative_2019}, ne prend cependant pas toutes ses sources dans l'étude de l'antiquité. Au contraire, un autre phénomène intéresse ce domaine: la question de la détection du plagiat. Bien sûr, et ils le citent, le centon est probablement l'une des premières oeuvres plagiaires, mais la notion de \enquote{d'invention n'est pas le critère discriminant de la valeur littéraire avant le XVIIIe siècle}\footcite{aron_plagiat_2010}: si le mot \textit{plagiarius} existe, il ne concerne pas à la réutilisation de morceaux de texte mais le vol d'autorité (Martial, I.53). Mais l'histoire du plagiat, de la circulation des textes et de leur remaniement ne s'arrête pas au 4e siècle de notre ère: au contraire, cette question peut se relever particulièrement intéressante lorsqu'elle commence à toucher les médias de masse telle que la presse. À travers le projet de D. Smith par exemple\footcite{smith_computational_2015, smith_infectious_2013}, dès 2013, on retrace la propagation aux États-Unis de textes -- tels que des recettes de cuisine, des dépêches, des petites histoires -- dans la presse. Cependant, sans en dénier la difficulté, surtout au début des années 2000, un monde existe entre la détection de textes entiers légèrement remaniés dans la presse américaine et la détection même discutable d'intertextualité telle que l'allusion virgilienne par Ovide que mentionnent Forstall et Scheirer en introduction (\textit{Arma gravi numero violentaque bella parabam Edere}, Ov. \textit{Am}. 1.1). 

% Dois-je définir allusion, citation. réécriture ? Pas envie...

Mais quel niveau de difficulté présente ce domaine ? D'un point de vue humain, la détection d'allusion reste toujours sujette à débat, et si les mentions virgiliennes par Ovide ou martialiennes par Ausone sont sans équivoque, restent de nombreux passages où la mention fait question, et où certains verront dans la simple présence d'un mot solitaire une intertextualité marquée avec un autre auteur. Mais du point de vue humain comme du point de vue de la machine, la difficulté réside aussi dans les dimensions de recherche que la détection d'allusion nécessite: de la recette de cuisine remaniée transmise de journal à journal à l'incipit des \textit{Amours} ovidiennes, le problème de l'intertextualité se trouve dans le fait que tout peut en être source, et que tout peut faire mention. Pour tout ensemble de mots, du simple mot au texte entier pour la presse du XIXe siècle américain, il existe potentiellement une unité\enquote{source} dans l'ensemble de la production littéraire de même langue -- voire de langue étrangère -- qui la précède. Même pour le corpus latin transmis du 2e siècle, cela représente plusieurs millions de mots et autant de séquences à la taille indéfinie. Cette tâche de minage de corpus s'apparente alors à la recherche d'information (\textit{information retrieval}) plus qu'à la simple classification.

Du point de vue technique donc, il ne s'agit pas des mêmes méthodes que la détection de métaphore vue précédemment. L'intertextualité quantitative est passée respectivement du \textit{fuzzy matching} -- une recherche de correspondance partielle de chaînes de caractères -- à la mesure de similarité via des réseaux neuronaux. Pour l'étude des textes anciens en latin, le projet \textit{Tesserae}\footcite{coffee_tesserae_2013} s'intéresse principalement à la présence dans un passage d'une taille définie (d'un à plusieurs vers) des mêmes lemmes, quelles que soient les différences de flexion, soit par l'usage de lemmatisation, soit par l'usage de stemmes, résultat de la découpe du lemme à la racine. Cette approche se limite donc à l'intertextualité \enquote{lexicale} telle que définie par C. W. Forstall\footcite{forstall_quantitative_2019} mais produit non seulement de nouvelles données (seule la moitié des intertextualités reconnues par Tesserae chez Lucain sont connues des commentateurs modernes), mais aussi de nouvelles pistes de réflexion sur l'objet étudié: on sort de l'exploit technique et on réintroduit des intérêts littéraires dans l'analyse quantitative. 

Pour passer de la détection d'intertextualité lexicale, qui partage des lemmes, à l'intertextualité sémantique, qui partage des sèmes, le passage à d'autres techniques de calcul est nécessaire. Dans leur manuel, Forstall et Schreier utilisent simplement la \textit{Latent Dirichlet Allocation} (LDA) -- une technique liée au \textit{topic modeling} et des calculs de similarité à base de distance cosinus pour relier les passages entre eux. D'un point de vue technologique, c'est d'ailleurs étonnant que, malgré une publication en 2019, ces derniers parlent des projections Word2Vec ou GloVe comme de futures applications: les techniques d'embeddings citées datent du tout début des années 2010, 2019 voyant l'apparition des technologies de \textit{transformers}. 

Il semble en effet que la prochaine étape pour l'intertextualité quantitative soit de briser la barrière de l'allusion fine, où le partage sémantique est faible entre le passage cité et le citant. Dans leur article \textit{On the Feasibility of Automated Detection of Allusion\footcite{manjavacas_feasibility_2019}}, les auteurs s'intéressent à la situation propre de cette catégorie, et montrent une forte différence de difficulté avec la détection de citation complète ou d'intertextualité lexicale en général: dans le set de données utilisé, des sermons annotés de Clairvaux, pour les allusions, seuls 6\% des formes et 12\% de lemmes sont partagés avec le texte référence, soit respectivement 4 et 3 fois moins que les autres catégories d'intertextualité que catégorise L. Mellerin à Biblindex. Architecturalement, le modèle final le plus performant est un modèle hybride, mêlant informations lexicales et sémantiques, utilisant la projection de mot \textit{FastText} sur un texte lemmatisé. Cet usage de \textit{FastText}, et sa primeur face à \textit{Word2Vec}, sont intéressants, car \textit{FastText} est connu pour capturer moins bien la proximité sémantique que d'autres formes d'\textit{embeddings} tout en capturant mieux les groupes de mots (par exemple \textsc{lascivus} et \textsc{lascivo}) et les relations syntaxiques\footcite{hartmann_portuguese_nodate}. Mais les résultats sont loin d'être satisfaisants, les auteurs admettant eux-mêmes que la marche reste longue: au mieux, pour moins de la moitié des exemples, la bonne réponse se trouvait dans le top 20 des résultats de la recherche. 


\subsection{Des tâches sans rapports ? De la classification d'article sur eBay à la détection de harcèlement dans le monde du jeu vidéo}

% Reprendre ici

Si certaines applications de la recherche en traitement automatique des langues croisent régulièrement les intérêts des sciences humaines, d'autres peuvent sembler bien étrangers. Et pourtant, qu'il s'agisse de moyen (classification), de difficulté (information implicite ou explicite) ou de thème (sexuel), on peut trouver dans les recherches actuelles quelques problématiques qui peuvent aiguiller notre propre recherche du point de vue architecture comme de celui de la mesure des résultats. 

Sur les réseaux sociaux comme dans le monde du jeu vidéo, la question de la modération quasi instantanée des contenus, en particulier ceux impliquant une forme de harcèlement ou des contenus sexuellement explicites, est devenue un véritable enjeu. Pour les deux types d'espaces d'échanges, et en particulier pour ceux des jeux massivement multijoueurs, la gestion de la toxicité des plates-formes et sa réduction sont des conditions \textit{sine qua non} d'un maintien d'une communauté d'utilisateurs actifs. Si chaque partie d'un jeu était un torrent d'insultes, la base de joueurs réguliers tendrait logiquement à se réduire, de même que les bénéfices de l'éditeur dudit jeu, particulièrement pour les jeux gratuits à l'installation qui tirent leurs bénéfices d'achats de bonus (éléments esthétiques, fonctionnalités supplémentaires, etc.) sur une longue durée. 

Ces problématiques se lient rapidement à la question de l'apprentissage machine pour deux raisons: d'une part, il permet une gestion en flux des \enquote{délits} de contenus; d'autre part, il réduit les coûts humains de la modération\footnote{À laquelle ils sont tenus légalement dans de nombreux pays.} manuelle des contenus, potentiellement extrêmement importants pour des plates-formes aussi grosses que Twitter par exemple. Pour \textit{League of Legends}, un jeu paru fin des années 2000, des chercheurs établissent en 2014\footcite{blackburn_stfu_2014} une méthode de capture de mots à partir d'un dictionnaire de valeurs, l'\textit{Affective Norms for English Words}\footcite{bradley_affective_1999}, et d'informations liées à la partie indiquée comme problématique. Parmi les informations utilisées par l'algorithme, un simple \textit{Random Tree Forest}, celles des messages et de la polarité (\textit{valence} en anglais) des termes employés, représentent trois des cinq informations contribuant le plus à la décision finale de punir ou d'innocenter une personne indiquée comme toxique par ses partenaires temporaires de jeu. Cependant, les dictionnaires de polarité posent le problème du contexte: "Toi au moins, tu n'es pas [TermeInsultant]" ne serait pas un abus de langue, tandis que "Toi au moins, tu n'es pas [TermeInsultant] contrairement à [NomDePersonne]". Sept ans plus tard, la recherche a intégré d'autres modèles d'information tentant de réduire cette limite: dans leur article \textit{Automatically Detecting Cyberbullying Comments on Online Game Forum}\footcite{vo_automatically_2021}, les auteurs comparent les performances de modèles à propriétés (\textit{features} soumises à SVM et aux régressions linéaires), à encodage après projection (projections via [Glove, fastText] puis encodage avec [CNN,GRU]) et à projection par \textit{transformer} (Toxic-BERT). Les résultats sont mesurés par F-Score à cause du déséquilibre des sets de données (beaucoup moins de données positives que négatives) et donnent lieu à de véritables bonds entre les différentes technologies: de 50\% de F-Score pour les premiers modèles, les seconds atteignent 75 à 80,7\% tandis que le modèle à base de \textit{transformer} atteint 82,7\%. On note un très faible écart entre les modèles Text-CNN+Glove et le modèle à base de Toxic-Bert, tout juste 2\% sur les données issues du jeu League of Legends, tandis que seuls 0,76\% les séparent sur celles du jeu World of Warcraft.

Dans la catégorie de classification de textes, la question de la détection de l'isotopie sexuelle peut se manifester sous d'autres traits. De la détection automatique de prédateurs sexuels à la classification de chansons pour leur commercialisation, le contenu explicitement ou implicitement sexuel pose plusieurs problèmes en fonction du contexte de recherche. Pour la prédation sexuelle, la problématique glisse vers l'implicite et la détection de progression discursive. Dans ce contexte, il semble que la compétition du PAN\footnote{\textit{Plagiarism Analysis, Authorship Identification, and Near-Duplicate Detection} - http://pan.webis.de/} de 2012 \textit{Sexual Predator Identification}\footcite{inches_overview_2012} constitue encore le plus gros dataset existant. Comme pour notre dataset, les données sont -- heureusement -- déséquilibrées en faveur de contenus non problématiques, avec seulement 4\% de conversations marquées comme contenant une prédation. On distingue pour ce domaine les mêmes évolutions que pour celui de la toxicité, à savoir un passage d'un modèle \textit{Bag-Of-Words} avec des dictionnaires de valeurs vers des modèles d'apprentissage profond, avec une évaluation via le F1-Score. Parmi les modèles les plus récents, celui de Muñoz et al.\footcite{munoz_smartsec4cop_2020} utilise une projection via \textit{Word2Vec} puis un encodage via des réseaux convolutionnels de filtres de taille 2 à 5, mais ses résultats en F1 sont particulièrement bas (environ 40\%). Au contraire, Vogt et al.\footcite{vogt_early_2021} atteint entre 89 et 80\% en utilisant une projection via Bert et en prenant optionnellement en compte la dimension temporelle d'une discussion, et donc, sa progression discursive. 

Dans une dimension plus légère, la classification des contenus explicites des paroles de chanson pose un problème pour leur commercialisation et leur mise en écoute libre sur des plates-formes comme Amazon, Youtube ou Spotify chez les mineurs, particulièrement aux États-Unis avec la mise en place du \textit{Parental Advisory Label} qui requiert l'annotation de ces messages. Les membres du projet WASABI proposent un récapitulatif des méthodes, mises au banc d'essais sur un dataset fraichement formalisé, via les mêmes techniques que les applications précédentes, à savoir une attaque par dictionnaire, une attaque par \textit{Bag-Of-Word}, un réseau utilisant des CNN et une projection via des \textit{embeddings}, un modèle inédit pour la littérature étudiée ici à savoir un réseau récurrent avec attention (HAN) et enfin un modèle BERT. De cet article, plusieurs conclusions des auteurs sont intéressantes:
\begin{itemize}
    \item Le modèle convolutionnel brille face à l'ensemble des autres modèles pour la catégorie explicite avec un score de 63\%.
    \item Le modèle Bert surpasse les autres sur la catégorie non explicite, majoritaire dans le corpus (90\% des exemples).
    \item Les différences de score restent particulièrement minimes (quelques points d'écart) suivant les catégories, quelle que soit l'architecture.
    \item Plus important encore, le modèle semble se spécialiser dans la détection du contenu explicite en Hip-Hop, et les auteurs posent justement la question du biais du set de données ou de la possibilité de surapprentissage du modèle sur ces traits facilement reconnaissables.
\end{itemize}
De manière assez intéressante, contrairement aux travaux en détection de métaphore, aucune information grammaticale ne semble avoir été intégrée au réseau, alors même que les auteurs indiquent le problème de la bivalence grammaticale verbe/nom du mot \textit{bitch} (fr. salope) qui peut comprendre aussi le simple sens \enquote{se plaindre} de manière vulgaire. Par ailleurs, ils notent aussi que le simple usage d'une vulgarité explicite comme \textit{bitch} dans une chanson telle que celle de Meredith Brook est alors prise à contre-courant, pour se réapproprier le sens et le questionner: il devient alors un sujet de la chanson et se défait en partie de sa connotation vulgaire.

Pour finir, la classification de texte peut devenir un problème particulièrement complexe lorsque le nombre d'exemples est particulièrement bas, moins d'une dizaine par exemple, et le nombre de catégories élevé. Cette problématique se retrouve particulièrement dans le domaine du marketing web où les catalogues sont crowdsourcés, tel eBay, ou pour les agrégateurs de contenu, tels Google News. Historiquement, en 2007, la génération de ces rapprochements pour les flux tels que Google News a principalement été opérée dans un contexte de classification non supervisé, via des méthodes de clustering ou d'indexation latente\footcite{das_google_2007}: il s'agit alors à la fois de regrouper des événements, mais aussi d'intégrer à cette classification une forme de personnalisation des résultats. Il existe bien sûr des outils internes, tels que ceux développés par Reuters\footcite{nugent_comparison_2017}, pour détecter des catégories précises, mais la classification instantanée d'informations nouvelles reste majoritairement une question de mesure de proximité entre des projections de document. 

Parallèlement, on retrouve ces problématiques dans le catalogage de masse de boutiques crowdsourcées telles qu'eBay. Dans ce cadre, la classification par réseau siamois semble apporter un avantage considérable\footcite{shah_neural_2018}. En effet, non seulement le nombre d'exemples est extrêmement faible  -- parfois, un nouveau produit émerge, donnant ainsi lieu à une classification à 0 exemple -- mais le nombre de classes extrêmement élevé (de l'ordre du million). Les modèles proposés utilisent alors une projection via FastText\footcite{fasttext}, un encodage via un LSTM bidirectionnel de 3 couches et les décisions prises sur la base d'une mesure de distance euclidienne. Les résultats varient alors de 82 à 97\% d'\textit{accuracy} en fonction des catégories, la catégorie "Vêtements, chaussures et accessoires", probablement la plus variée, obtenant le score le plus bas.

\subsection{Résumé des architectures rencontrées}

Après cette revue des différentes tâches qui s'apparentent à la classification de texte, il est intéressant de remarquer une véritable constante: presque tous les modèles reposent sur une structure Projection-Encodage-Décision où la couche de décision peut-être au choix une régression linéaire ou bien une mesure de similarité. Évidemment, chaque tâche intègre ça et là des variations, par exemple la prise en compte de la progression de discussion, mais la structure générale des éléments d'encodage reste la même. Il est par ailleurs intéressant de voir largement utilisée la mesure du F1-Score pour l'ensemble des tâches à labels déséquilibrés. On remarque aussi un véritable changement du domaine sur la décennie 2010-2020, avec  d'abord l'avènement des projections de mots aux alentours de 2013 grâce à \textit{Word2Vec} et \textit{GloVe}, à l'apparition de nouveaux systèmes de classification dans le domaine du traitement textuel (réseaux siamois) vers 2017/2018, la diversification des couches d'encodage, du LSTM au CNN, y compris modifiés, et enfin à la fin de la décennie, le changement profond qu'apporte Bert et les transformers pour les langues modernes. Cependant, il reste intéressant de voir la résistance de certains modèles dans quelques cas de figure (stylométrie par exemple ou classification de chanson) qui résistent plus que correctement aux changements technologiques et aux différences de coût qu'ils impliquent.

% Rajouter un cas d'étude ici en Siamese ?


\section{Développement des architectures et mise en place des modèles}

\subsection{Du corpus aux données d'entraînement et d'évaluation}

Avant de pouvoir produire des modèles, la transformation du corpus de recherche en jeu de données d'entraînement est nécessaire. Celle-ci passe par plusieurs étapes: il faut passer d'un format de conservation à un format d'exploitation, il faut contrebalancer les données \enquote{unicatégorielles} (classe \texttt{isotopie-sexuelle}) par des données ne présentant pas cette isotopie.


\subsubsection{Formats des fichiers, simplification des données}

\begin{figure}
    \centering
    \begin{adjustbox}{width=0.9\textwidth,keepaspectratio}
    \lstset{language=XML}
    \begin{lstlisting}[language=XML]
<div type="fragment" ana="#acte #fututio">
  <bibl ref="#adams">
      <author>J. N. Adams</author>,
      <title xml:lang="lat">The Latin Sexual Vocabulary</title>,
      <biblScope unit="page">119</biblScope>
    </bibl>
    <bibl type="source">
        <author>
            <persName xml:lang="fr">Catulle</persName> [<persName xml:lang="eng">Catullus, Gaius Valerius</persName>]
            <idno type="VIAF">https://viaf.org/viaf/100218993/</idno><idno type="LC">n79-6943</idno>
        </author>,
        <title xml:lang="lat">Carmina</title>,
        <biblScope unit="ref">6.13-6.14</biblScope>
        <idno type="CTS_URN">urn:cts:latinLit:phi0472.phi001.perseus-lat2</idno>
    </bibl>
    <quote xml:lang="lat" source="urn:cts:latinLit:phi0472.phi001.perseus-lat2:6.13-6.14" type="line">
    <w ref="6.13" lemma="non" pos="ADVneg" 
      msd="MORPH=empty">non</w>
    <w ref="6.13" lemma="tam" pos="ADV" 
      msd="Deg=Pos">tam</w>
    <w ref="6.13" lemma="latus1" pos="NOMcom" 
      msd="Case=Acc|Numb=Plur">latera</w>
    <w ana="#acte #fututio" ref="6.13" lemma="effutuo" pos="VER" 
      msd="Case=Acc|Numb=Plur|Gend=Neut|Mood=Par|Tense=Perf|Voice=Pass">ecfututa</w>
    <w ref="6.13" lemma="pando2" pos="VER" 
      msd="Numb=Sing|Mood=Sub|Tense=Pres|Voice=Act|Person=2">pandas</w>
    <w ref="6.13" lemma="," pos="PUNC" 
      msd="MORPH=empty">,</w>
    <w ref="6.14" lemma="ni2" pos="CONsub" 
      msd="MORPH=empty">ni</w>
    <w ref="6.14" lemma="tu" pos="PROper" 
      msd="Case=Nom|Numb=Sing">tu</w>
    <w ref="6.14" lemma="quis1" pos="PROint" 
      msd="Case=Acc|Numb=Sing|Gend=Neut">quid</w>
    <w ref="6.14" lemma="facio" pos="VER" 
      msd="Numb=Sing|Mood=Sub|Tense=Pres|Voice=Act|Person=2">facias</w>
    <w ref="6.14" lemma="ineptiae" pos="NOMcom" 
      msd="Case=Gen|Numb=Plur">ineptiarum</w>
    <w ref="6.14" lemma="." pos="PUNC" 
      msd="MORPH=empty">.</w>
  </quote>
</div>
    \end{lstlisting}
    \end{adjustbox}
    \caption{Exemple de ressource en XML pour l'entraînement}
    \label{fig:xml-model-example}
\end{figure}

En premier lieu, afin d'assurer la réutilisabilité des données, il est nécessaire d'expliquer les diverses transformations qu'elles ont subies pour devenir jeu de données pour l'entraînement de modèles. Le format du corpus pour leur conservation et exploitation \enquote{manuelle} consiste en un ensemble de fichiers XML (\textit{cf.} Figure \ref{fig:xml-model-example}) contenant à la fois les informations grammaticales sur chaque mot (lemme, POS, annotations morphosyntaxiques), l'analyse des termes centraux portant les sèmes actualisés de l'isotopie, les métadonnées sur l'extrait (auteur, oeuvre, identifiant de passage) et les diverses annotations issues de la lecture d'Adams (champs lexicaux, figure de style employée, etc.). Si les données XML sont particulièrement adaptées à la réexploitation de corpus, nous faisons le choix de convertir ce corpus vers un nouveau format permettant une lecture plus rapide des données. Le format de sortie, dont un extrait est présenté en figure \ref{code:chap4:training-data-format}, est une forme de TSV personnalisé et propose quatre types de lignes:
\begin{itemize}
    \item Le premier type, qui commence par \texttt{[header]}, est tout simplement la ligne d'en-tête, où chaque colonne qui suivra dans le TSV est nommée. Elle inclut ainsi les noms des colonnes \textit{token}, \textit{lemma}, etc. Pour une exploitation plus fine de la morphologie (genre, nombre, cas, mode, temps…), l'information rassemblée dans un seul élément\texttt{@msd} du fichier XML est désormais réorganisée dans le tsv avec une colonne pour chaque trait grammatical.
    \item Le second type de ligne est celui de division des échantillons: chaque phrase échantillon est séparée par une à plusieurs lignes vides.
    \item Le troisième type de ligne concerne le contenu des échantillons: chacun de leurs \textit{tokens} est représenté par une ligne en suivant le format et l'ordre de l'en-tête. Les enclitiques et les phénomènes d'élision du type \textit{est → -st} sont alors représentés par une duplication de la forme entourée des caractères \texttt{\{\}} ou bien une forme indépendante (pour \texttt{-ne, -que, -cum et -ve}) avec son analyse grammaticale propre. Les formes les plus communes ont pu être capturées de cette manière, les autres sont donc dédoublées.
    \item Le quatrième type concerne les métadonnées: qu'elles soient activables (intéressantes pour la classification) ou descriptives (afin de pouvoir mieux analyser les résultats), ces lignes de métadonnées commencent nécessairement par un \texttt{\#} ou par un \texttt{[}. Un échantillon commence nécessairement par une ligne indiquant sa catégorie (\texttt{\#[TAG]positive} ou \texttt{negative}). 
\end{itemize}

\begin{figure}
\centering
\begin{adjustbox}{width=0.9\textwidth,keepaspectratio}
\begin{lstlisting}

[header]	token	lemma	pos	Case	Numb	Gend	Mood	Tense	Voice	Person	Deg
#[TAG]negative
[GENERIC-METADATA]urn=urn:cts:latinLit:phi1318.phi001.perseus-lat1
[GENERIC-METADATA]fp=./dataset/negative-examples/latinLit_phi1318.phi001.perseus-lat1--12.xml
[TAGS-METADATA]TAG=#negative-example
[TOKEN-METADATA]WrittenType=prose
[TOKEN-METADATA]Century=0
[TOKEN-METADATA]Century=1
[TOKEN-METADATA]CitationTypes=book
[TOKEN-METADATA]CitationTypes=letter
[TOKEN-METADATA]CitationTypes=section
[TOKEN-METADATA]Textgroup=urn:cts:latinLit:phi1318
Dixi	dico2	VER	-	Sing	-	Ind	Perf	Act	1	-
proxime	prope1	ADV	-	-	-	-	-	-	-	Sup
pro	pro1	PRE	-	-	-	-	-	-	-	-
Vareno	Uarenus	NOMpro	Abl	Sing	-	-	-	-	-	-
postulante	postulo	VER	Abl	Sing	Com	Par	Pres	Act	-	-
,	,	PUNC	-	-	-	-	-	-	-	-
ut	ut4	CONsub	-	-	-	-	-	-	-	-
sibi	sui1	PROref	Dat	Sing	-	-	-	-	-	-
invicem	inuicem	ADV	-	-	-	-	-	-	-	Pos
evocare	euoco	VER	-	-	-	Inf	Pres	Act	-	-
testes	testis1	NOMcom	Acc	Plur	-	-	-	-	-	-
liceret	licet1	VER	-	Sing	-	Sub	Impa	Act	3	-
;	;	PUNC	-	-	-	-	-	-	-	-

#[TAG]positive
[GENERIC-METADATA]urn=urn:cts:latinLit:phi0119.phi016.perseus-lat2
[TAGS-METADATA]TAG=#armes
[TAGS-METADATA]TAG=#mentula
[TOKEN-METADATA]WrittenType=versified
[TOKEN-METADATA]Century=-3
[TOKEN-METADATA]Century=-2
[TOKEN-METADATA]CitationTypes=line
[TOKEN-METADATA]Textgroup=urn:cts:latinLit:phi0119
noctu	noctu	ADV	-	-	-	-	-	-	-	Pos
in	in	PRE	-	-	-	-	-	-	-	-
vigiliam	uigilia	NOMcom	Acc	Sing	-	-	-	-	-	-
quando	quando2	ADVint	-	-	-	-	-	-	-	-
ibat	eo1	VER	-	Sing	-	Ind	Impa	Act	3	-
miles	miles	NOMcom	Nom	Sing	-	-	-	-	-	-
,	,	PUNC	-	-	-	-	-	-	-	-
quom	cum3	CONsub	-	-	-	-	-	-	-	-
tu	tu	PROper	Nom	Sing	-	-	-	-	-	-
ibas	eo1	VER	-	Sing	-	Ind	Impa	Act	2	-
simul	simul1	ADV	-	-	-	-	-	-	-	Pos
,	,	PUNC	-	-	-	-	-	-	-	-
conveniebatne	conuenio	VER	-	Sing	-	Ind	Impa	Act	3	-
-ne	ne2	ADV	-	-	-	-	-	-	-	-
in	in	PRE	-	-	-	-	-	-	-	-
vaginam	uagina	NOMcom	Acc	Sing	-	-	-	-	-	-
tuam	tuus	PROpos	Acc	Sing	Fem	-	-	-	-	-
machaera	machaera	NOMcom	Nom	Sing	-	-	-	-	-	-
militis	miles	NOMcom	Gen	Sing	-	-	-	-	-	-
?	?	PUNC	-	-	-	-	-	-	-	-
\end{lstlisting}%
\end{adjustbox}
\caption{Exemple de données pour l'entraînement}
\label{code:chap4:training-data-format}
\end{figure}

En plus de cette nouvelle formalisation, dans le cadre de l'incorporation des métadonnées, celles-ci subissent une simplification des dates de naissance et mort de leurs auteurs: seuls les siècles sont conservés. Cette réduction-simplification a pour objectif de réduire la richesse des données et de simplifier leur interprétation pour la machine. De plus, les métadonnées portant sur les structures logiques de citation sont aussi à la source d'une nouvelle métadonnée \enquote{simplifiée}: chaque échantillon est qualifié par sa forme, à savoir en vers ou en prose (\texttt{versified} ou \texttt{prose} dans les fichiers). Cette simplification, comme celle des siècles, a la capacité de rendre compte des informations induites pour le lecteur humain par les structures en chapitre$\longrightarrow$paragraphes et les structures en poème$\longrightarrow$vers.


\subsubsection{Équilibrage des données: les contre-exemples}

Pour proposer un entraînement, quelle que soit son architecture, le réseau neuronal a besoin d'exemples n'appartenant pas à l'isotopie de la sexualité, que l'on appellera simplement \enquote{négatifs} ou \enquote{contre-exemples}. Cette introduction de nouvelles données pose la question de leur sélection et des biais que cette sélection peut intégrer. Elle est produite à partir du corpus complet lemmatisé et annoté, avec les mêmes informations (hors annotations stylistiques et champs lexicaux) que les exemples positifs.

Un échantillonnage manuel présente le risque d'intégration de biais \enquote{inconscients} dans la sélection des contre-exemples, puisqu'aucune propriété ne définit ceux-ci à part l'absence de l'isotopie sexuelle. Dans ce cadre, un biais aurait pu s'exprimer par la sélection d'isotopies volontairement éloignées de celle de la sexualité, au moins dans nos catégories modernes: des discussions agraires, légales, des lettres philosophiques traitant de la mort, etc. Mais d'autres formes de biais peuvent apparaître: les biais d'emplacements par exemple, avec des échantillons pris plus particulièrement en début ou fin de textes, peuvent introduire une répétition lexicale ou structurelle sur lesquels les modèles pourraient s'appuyer; après tout, un modèle apprend à reconnaître par tous les moyens.

Pour éviter cette possible introduction, une sélection semi-automatisée est effectuée. Pour essayer de produire un corpus varié, on extrait de chaque oeuvre du corpus trente phrases prises au hasard dans le texte, on s'assure qu'elles ne sont pas présentes dans les exemples positifs et qu'elles présentent une taille assez importante pour faire l'objet d'une analyse, arbitrairement choisie d'au moins cinq mots, seuil permettant d'assurer une forme de diversité lexicale. Cette sélection présente encore des biais, qui sont inhérents au corpus de texte: Cicéron est plus représenté que Martial par exemple. Mais elle a l'avantage d'introduire une variété de lexiques, de situations discursives, etc.

Pour autant, d'autres méthodes de sélection sont possibles, mais n'ont pas été appliquées -- par manque de temps. Une sélection prenant en compte la surreprésentation de certains auteurs dans le jeu positif par exemple -- tout en prenant soin d'exclure certains corpus tels que les \textit{Priapées} ou de faire une sélection manuelle pour ce dernier -- permettrait de contre-balancer les exemples positifs présents chez les auteurs tels que Martial. Enfin, pour contrer des possibles surreprésentations de lexiques dues à des domaines de métaphore répétés comme ceux de la guerre, il est aussi possible d'augmenter les données peu à peu afin de s'assurer d'un jeu de données aussi représentatif de la réalité que possible, à savoir que tous les usages de \textit{bellum gerere} ne soient pas nécessairement des allusions sexuelles.

\subsubsection{Répartitions des données et propriétés des différents jeux d'exemples}

Une fois la création du jeu de contre-exemples produits, deux étapes restent à venir: d'une part, il est intéressant d'analyser les possibles différences statistiques des deux sous-jeux ainsi produits; d'autre part, il faut assurer un mélange de ces données afin de pouvoir entraîner au mieux le modèle.

Pour mieux comprendre les différences entre ces deux sets, on propose de s'intéresser aux propriétés lexicales et quantitatives de ces deux versants du jeu complet d'entraînement. Il est ainsi intéressant de comparer la diversité lexicale de chacune des catégories d'échantillons, les mesures de diversités lexicales étant \enquote{construites pour capturer la proportion de mots dans un échantillon [textuel] qui ne sont pas des répétitions de mots déjà rencontrés}\footcite[p. 44]{jarvis_capturing_2019}, elles permettent de savoir ce que chacun des corpus réussit à capter en termes de représentativité du corpus latin. Parmi les outils de mesure, nous utilisons la \textit{Measure of Textual Lexical Diversity (MTLD)}\footcite{mccarthy_assessment_2005}. Nous comparons aussi, en suivant les conseils de S. Jarvis, quelques propriétés globales des échantillons: leur taille, leur richesse (\enquote{le nombre de types de mot}), leur distribution (à travers la déviation standard des fréquences). On ajoute à ces mesures un calcul de partage lexical $L$, à savoir le pourcentage de l'ensemble de formes ou lemmes uniques $F$ qu'une catégorie $K$ possède en commun avec l'autre catégorie $J$ précédente tel que $L(K, J) = \frac{|F_{K} \cap F_{J}|}{|F_{K}|}$. 

Le résultat des analyses n'est pas sans surprise (\textit{cf.} table \ref{tab:chap4:mesures-corpora}): si le corpus total Négatif est dix fois plus important que le Positif, une nécessité pour assurer un semblant de représentativité du réel à travers les données, la diversité lexicale est plus importante dans le corpus positif. Malgré un rapprochement thématique, on peut faire l'hypothèse que, d'une part, à travers la faible présence de chaque oeuvre en dehors de quelques grandes exceptions (\textit{Priapées}, Martial, etc.), le vocabulaire spécifique à chacune de ces oeuvres ne se voit pas répété, et que d'autre part, à travers la présence de nombreuses catégories de métaphores et de sous-thèmes, une forme de diversité se met en place). Fait plus attendu, ni le corpus positif ni le corpus négatif ne présentent une couverture totale des formes ou des lemmes dans son \textit{alter ego}, atteignant au maximum 78\% de lemmes communs au pour le corpus positif dans le corpus négatif, et 10\% des formes au minimum pour le corpus négatif dans le corpus positif. L'avantage du corpus Négatif sur le Positif tient principalement à la différence de richesses (Négatif a six fois plus de formes uniques que Positif). Enfin, l'écart-type de fréquence du corpus Positif est beaucoup plus faible, et son écart avec celle de la fréquence des lemmes est moins importante que dans le corpus Négatif: la plus petite taille du corpus alliée à une plus petite richesse lexicale baisse l'impact des termes extrêmement communs comme celui des termes uniques ou rares.

\begin{table}[ht]
    \centering
    \begin{tabular}{l|rr}
    \toprule
    {} &  Négatifs &  Positifs \\
    \midrule
    Taille               & 525220    &  44964    \\
    Richesse             &  80153    &  13305    \\
    MTLD(Lemmes)         &    114.70 &    153.90 \\
    MTLD(Formes)         &    286.56 &    384.53 \\
    Distribution(Lemmes) &    393.70 &     62.46 \\
    Distribution(Formes) &    191.22 &     40.69 \\
    Partage(Lemmes)      &   22.39\% &   78.12\% \\
    Partage(Formes)      &   10.72\% &   64.55\% \\
    \bottomrule
    \end{tabular}
    \caption{Mesures appliquées aux deux parties du corpus d'entraînement.}
    \label{tab:chap4:mesures-corpora}
\end{table}

Une fois cette sélection produite et analysée, il reste à joindre ces échantillons à leur label (positif ou négatif), et à les réunir en trois sets pour la production de modèle: chaque jeu d'exemples ou de contre-exemples est également réparti en section de 80\% pour l'entraînement, 10\% pour l'évaluation et 10\% pour le test. Ces données sont ensuite mélangées aléatoirement et sauvegardées dans des fichiers séparés. % à compléter.

\subsection{Encodage de la donnée et le texte comme information: enrichir le texte pour l’ordinateur}


\subsubsection{Forme, lemmes, caractères et \textit{sous-tokens}}

% Reprendre relecture ici

Pour fournir une représentation de la phrase à classer, la première information utilisée dans le cadre du traitement automatique des langues est le \textit{token}, que nous appelons aussi \enquote{forme}. Cette forme porte donc de l'information syntaxique en latin, à travers sa flexion, et l'information sémantique, à travers le lemme sous-jacent et la syntaxe. Son ordre dans la phrase -- même en latin -- % TC: Inclure ici un rapide graphe des distances dans le treebank de Perseus ?
est un indicateur de sa relation sémantique et syntaxique avec les termes du voisinage direct: cette information est souvent perdue par les modèles \textit{Bag-Of-Words} qui se focalisent sur les phénomènes d'occurrence et qui ignorent les co-occurrences en ne prenant pas en compte le contexte. Il est donc important non seulement de pouvoir conserver la forme mais de la conserver dans le contexte des termes qui l'entoure. 

Si la forme tend à suffire dans un très grand nombre d'articles de TAL, cette autosuffisance peut-être liée aux corpus utilisés dans la plupart des grandes conférences: les corpus en anglais, à la morphologie extrêmement pauvre (une dizaine de flexions verbales hors formes composées, peu ou pas de flexions liées au cas, deux nombres qui n'affectent que les pronoms, noms et verbes, pas de flexion ou presque liée au genre), sont très loin de la complexité morphologique du latin ou du grec ancien. Dans un contexte de richesse morphologique, mais aussi d'un corpus réduit (pour rappel, le corpus total latin utilisé ici fait 20 millions de mots), l'usage du lemme comme remplaçant ou addition à celui de la forme permet d'intégrer une simplification de l'information et une réduction du bruit ambiant. Dans leur étude de la langue basque\footcite{zipitria_observing_2006}, une langue agglutinante non indo-européenne, rarement utilisée dans les laboratoires de TAL développant les outils les plus connus\footnote{Appelés \textit{low-resource languages}, ces langues posent des problèmes de corpus accessibles et de données annotées. Le TAL se focalisant malheureusement aujourd'hui plus sur les modèles que sur la production de données, plus chère, ces langues sont très peu présentes dans les plus grosses conférences. \textit{Cf.} \cite{magueresse_low-resource_2020}}, I. Zipitria et ses collègues proposent de comparer les résultats d'un outil de mesure à partir de l'algorithme LSA en passant d'une part l'information lemmatisée et et d'autre part l'information brute: malgré une réduction de la quantité d'unités lexicales prises en compte de 56\% via la lemmatisation, la classification obtient un score en-dessous de celui des formes brutes, bien que l'écart soit assez faible. 

Les lemmes, pris seuls, seraient la source d'une perte d'information ayant tout de même un impact négatif, là où nous attendrions un possible impact positif via la normalisation de l'information. Dans son article de 2008, B.~Lemaire remet en contexte cette course à la lemmatisation qui précède la fin des années 2000: il ne s'agissait pas tant d'éliminer un bruit produit par les formes que de réduire la taille de l'information modélisée pour des raisons de limites matérielles\footcite[p. 1]{lemaire_limites_2008}. L'auteur avertit que la lemmatisation peut être utilisée dans le cas de corpus spécialisés -- c'est notre cas -- mais qu'elle fait perdre de l'information, notamment car le contexte des formes fléchies n'est pas le même: en prenant l'exemple de \textit{soleil} et \textit{soleils}, il indique que \textit{rayon} est un co-occurrent important du premier tandis qu'il est rare pour le second et montre ainsi l'importance des morphèmes de flexions. Mais, cette description des problèmes de co-occurrents traduit bien le contexte de ces articles de la fin des années 2000, qui précèdent l'avènement des réseaux neuronaux récurrents ou convolutionnels: les algorithmes considérés sont alors ceux qui utilisent le mode du \textit{Bag-Of-Words}, ici LSA encore et qui ne conservent pas le contexte. Dans un cadre de richesse de mémoire informatique (en RAM et en ROM), il est donc possible non seulement de conserver les formes mais en plus d'y ajouter les lemmes, afin de conserver les informations morphologiques ou syntaxiques en général (forme avec enclitique, majuscule, élidée, etc.) propre aux premières et une information sémantique plus générale portée par le lemme.

Mais les lemmes ne sont pas suffisants, ni les formes par ailleurs, car ils posent un dernier problème: celles de leur représentation numérique par l'ordinateur. Derrière deux lemmes à la même racine, le lecteur humain fait une connexion: \textsc{lascivus} et \textsc{lascivo} semblent reliés sémantiquement. Et pourtant, pour l'ordinateur, ils seront aléatoirement indexés, par exemple par les identifiants 7 et 18~500. Si \textit{Word2Vec} peut faire des merveilles, il suffit que la lemme \textsc{lascivo} soit très rare et ne partage que peu de contexte avec \textsc{lascivus} pour que ces derniers ne soient pas regroupés dans leur espace de projection. L'algorithme \textit{FastText} tente de résoudre ceci, mais montre de fait une focalisation importante sur la morphologie pour certains termes peu fréquents: ainsi, pour \textsc{lasciuus}, sur 10 des mots les plus proches pour \textit{FastText}~(Table~\ref{tab:fasttext:lemmes}), si 5 partagent la même racine et qu'au moins deux peuvent être considérés comme potentiellement proches sémantiquement (\textsc{luxuus*\footnote{Erreur de lemmatisation par le lemmatiseur de \textsc{Luxus+Ve} de \textit{luxuue}.}, mollitiuus}\footnote{Mot rare, présent chez Caelius Aurelianus dans notre corpus, enregistré dans le \textit{Du Cange} et le \textit{Dictionary of Medieval Latin from British Sources}}), 3 sont clairement présents à cause de leur suffixe \textit{-iuus} et de leur faible compte d'occurrences (29 au total).


\begin{table}[ht]
    \centering
    \begin{tabular}{c|c}
        Word2Vec    &  FastText      \\ \hline
        lasciuio    &  Lasciuus*      \\
        procax      &  lasciuibundus \\
        libidinosus &  lasciuiosus   \\
        lusus       &  lasciue       \\
        blandus     &  lasciuio      \\
        garrulus    &  nesciuus*      \\
        proteruus   &  uaciuus       \\ % Lié à l'otium
        salto       &  nociuus       \\
        obscenus    &  luxuus*        \\
        petulans    &  mollitiuus    \\ % Peu attesté, mais présent chez Caelius Aurelianus (5 fois) et Du Cange
    \end{tabular}
    \caption{10 lemmes les plus proches de \textsc{lasciuus} d'après FastText et Word2Vec dans le corpus via une annotation automatique. Les formes mal lemmatisées sont accompagnés d'un \texttt{*}.}
    \label{tab:fasttext:lemmes}
\end{table}


Partant du constat de l'incapacité pour l'ordinateur à connecter entre elles deux formes partageant des traits communs, la focalisation sur des n-grams de caractères a parfois été une bonne solution alternative au mot: les n-grams en début de mot permettent de capturer préfixes et racines, tandis que les n-grams de fin de mots capturent des informations aussi bien morphologiques que syntaxiques à travers le système de désinences et de clitiques, du moins pour les langues qui nous intéressent: ces informations sont particulièrement utilisées en stylométrie\footcite{kestemont_authenticating_2016, camps_stylometry_2020} où elles suffisent à intégrer assez d'informations pour distinguer des idiolectes et des stylômes. Malheureusement, cette approche se limite bien souvent à une approche de type \textit{Bag-Of-Words} et perd donc aussi les phénomènes de séquences: si les poèmes qui contiennent \enquote{\textit{Lasciva est nobis pagina, vita proba}} son découpés en n-grams, \textit{lasc} et \textit{prob} ne seront que deux informations parmi tant d'autres, sans information sur la proximité qui les relie.

% Simon: Ajouter un exemple ?

Dans le cadre de l'apprentissage profond, cette approche par n-gram connait deux héritiers: les \textit{sous-tokens} et l'encodage de caractère. L'encodage par \textit{sous-token}, technique employée par les \textit{transformers}, a pour principal objectif de réduire l'espace de vocabulaire (il existe moins de sous-tokens que de tokens: on peut représenter l'ensemble d'une langue uniquement à travers des sous-mots de taille 1, les lettres) tout en pouvant couvrir la quasi-intégralité des mots que l'encodeur pourrait rencontrer sur le terrain. En n'atteignant pas la complexité de l'encodage par caractère, il propose un vocabulaire de plusieurs milliers de sous-token pour représenter le texte: ainsi, en latin, on peut imaginer des préfixes (\textit{ab-}) ou flexions (\textit{-us}) présentes comme sous-tokens permettant de représenter facilement le mot final \textit{ab-solut-us, ab-solut-um}, etc.

Le second héritier des n-grams est un traitement des mots ou de la phrase au niveau caractère, à travers des couches d'encodage supplémentaires. Dans leur article de 2016, X. Zhang et Y. LeCun comparent différentes techniques d'encodage pour la classification de texte. Dans leur article, ils passent en revue l'usage de réseaux convolutionnels au niveau caractère, au niveau mot, mais aussi d'autres algorithmes, y compris des techniques \textit{Bag-Of-Words}. Ils concluent que, si les modèles convolutionnels appliqués aux caractères fonctionnent plutôt bien, ils ont tendance à uniquement dépasser les autres modèles dans le contexte de dataset de plusieurs millions de mots\footcite[p. 7]{zhang_text_2016}, mais ils seraient aussi moins sensibles à des données dont la qualité n'est pas très bonne. Fait rare, ils mentionnent la question de la normalisation, en particulier celle de la casse, un réflexe constant étant d'utiliser tous les textes en minuscules, et concluent que, sur des datasets importants, conserver la casse se fait au bénéfice des résultats. 



\subsubsection{Intégration des informations morphosyntaxiques}

Comme nous l'avons vu dans le cadre de la revue des différentes tâches de TAL qui se rapprochent de nos intérêts, certaines informations morphosyntaxiques, principalement les informations de \textit{Part-Of-Speech} (POS) ont prouvé leur intérêt dans le cadre général de  la classification, qu'il s'agisse de méthodes en \textit{deep learning} ou bien de méthodes plus anciennes. Alliée à l'information \textit{token}, elle permet de désambiguïser (l'exemple cité par Vogt et al., \textit{bitch} nom ou verbe), de contextualiser un lemme si la forme n'est pas conservée et permet de rendre compte de phénomèmes de syntaxes (ruptures volontaires pour mettre en valeur un phénomène): c'est d'ailleurs dans ce cadre de la syntaxe qu'elle est souvent utilisée comme information pour la détection d'entités nommées (\textit{cf.} par exemple  Bai et al. \footcite{bai_adversarial_2020}). 

Dans leur article de 2020, U. Naseem et ses collègues\footcite{naseem_towards_2020} proposent d'utiliser diverses représentations concaténées en un seul vecteur, dont une projection de la POS. Dans l'objectif de détecter ironie et sarcasme, les auteurs utilisent pour chaque mot une projection par POS, par \textit{embedding} de forme via GloVe, par embedding contextuel (ALBERT, ELMo), par embedding de dictionnaire de \textit{sentiment} et enfin par un encodage au niveau caractère via un réseau BiLSTM. Nous avions déjà vu, pour la détection de métaphore, que la concaténation d'embeddings pouvait proposer de meilleurs résultats (étude de \textit{bot.zen}\footcite{stemle_using_2018} par exemple), et encore une fois, il semble que le mélange des biais d'algorithme, en particulier des plongements de mots qu'ils produisent, et des informations qui les accompagnent (POS): dans son article, U. Naseem montre un gain de 2.5 points en score F1 sur les modèles précédents mais surtout que, quelle que soit l'information retirée, le modèle perd en performance.

% Reprendre ici

Si la projection en POS des embeddings est assez largement traitée dans la littérature scientifique\footnote{\textit{Cf.} entre autres \cite{fell_comparing_2019,leong_report_2018}}, la question de l'introduction d'informations morphologiques est un peu plus rare, voire inexistante, notamment dans les modèles de classification (nous n'en avons pas trouvé d'exemples). Il existe plusieurs articles qui recouvrent la question de l'entraînement d'\textit{embeddings} morphologiques\footcite{cotterell_morphological_2015} ou de l'ajout des informations morphologiques pendant l'entraînement\footcite{cui_knet_2015}. Parmi ces derniers, l'article de R. A. Salama, A. Youssef et A. Fahmy sur les embeddings de l'arabe retrace les options disponibles pour les embeddings à information morphologique\footcite{salama_morphological_2018}, à savoir ceux utilisant des tags morphosyntaxiques (POS) et ceux utilisant des sous-unités (racines, préfixes, suffixes), avec, à l'intérieur de ces catégories, des variantes du point de vue des objectifs d'entraînement des différents modèles. Leur modèle est un modèle du premier type, de sorte que, avec ces plongements, ont puisse obtenir: $V(Homme,NOM) - V(_,NOM) + V(_,ADJ) = V(Masculin,ADJ)$, soit le vecteur Homme-Nom dont on retranche la POS Nom et auquel on ajoute le vecteur Adjectif doit produire un vecteur équivalent à Masculin-Adjectif.

En conclusion, si les POS \textit{simples} (catégories grammaticales, \enquote{Nom} par exemple) connaissent un intérêt en TAL, les informations liées à la flexion nominale ou verbale y sont rarement étudiées comme propriétés à introduire dans des plongements sémantiques. Et pourtant, il semble que l'annotation \textit{Impératif} ou que l'annotation \textit{Genre+Cas} puissent introduire des détails plus subtils: n'y-a-t-il pas de différence sémantique entre \textit{Prends-moi !} (\texttt{Mode=Impératif}) et \textit{Tu me prends ?} (\texttt{Mode=Indicatif}) ? La raison derrière cette absence d'intérêt peut émaner, à notre avis, de ces multiples facteurs:
\begin{itemize}
    \item L'information morphosyntaxique, du moins en latin et en grec, possède une telle richesse et une telle ambiguïté que le développement des modèles permettant leur traitement a pris beaucoup de temps.
    \item Dans le cadre d'autres langues, au corpus non fermé, il est possible d'entraîner ces langues sur de très vastes corpus de formes, quitte à introduire des informations de sous-mots (qu'il s'agisse des méthodes de \textit{FastText} ou d'identifier préfixe, racine et suffixe): ces modèles, même sans cette spécialisation sur les morphèmes de la forme, ont montré une capacité de représentation de l'information morphosyntaxique\footcite{qian_investigating_2016}.%, au point où ils introduisent de larges biais. La question du biais de genre par exemple, surtout pour les langues romanes avec un accord plus marqué des adjectifs (heureuse / heureux), est devenu une question importante du domaine. % pas beau ? Cheveux sur la soupe ?
    
\end{itemize}


\begin{figure}
    \centering
    \includegraphics[width=\linewidth]{figures/chap4/Projection.drawio.png}
    \caption{Projection maximaliste réutilisant l'ensemble des informations à disposition du modèle (forme, lemme, caractère, traits morphologiques, POS): chaque information est toujours insérée dans le même ordre.}
    \label{fig:chap4:projection:morphosyntax}
\end{figure}
% ~  < 1 page ? (un gros paragraphe)

% POS et héritage des différentes problèmes cités plus hauts sur lemme / forme et perte d'information
% Problème des embeddings avec le mot

\subsubsection{Intégration des métadonnées}
\label{chap4:encodage:metadonnees}

\enquote{[Quand] la métaphore [est] \textit{in absentia}, elle instaure une connexion symbolique qui doit être identifiée par des conjectures concordantes sur le discours, le type de l'œuvre, le genre du texte, la hiérarchisation idiolectale des isotopies.\footcite[p. 98]{rastier_tropes_1994}}. Ces seuls mots de F.~Rastier indiquent combien la question du contexte est importante dans notre cas. Non content d'avoir dans nos textes de très nombreux passages du lexique de la sexualité, sur lesquels la machine ne devrait pas buter, la question de la métaphore et en général des tropes risque de se heurter à un mur: la machine ne connait pas ce que Martial est à la littérature latine, mais ne sait pas non plus, si on ne l'informe pas, que \enquote{\textit{Lasciva est nobis pagina}} est de Martial. La formalisation de ce contexte, de ces métadonnées, est alors répartie sur deux niveaux: un premier niveau qui concerne le document, l'œuvre (auteur, siècle, etc.) mais nous proposons de distinguer un second niveau, celui des \enquote{métadonnées de mots} (propriétés extra-linguistiques entre autres).

La problématique du contexte et de sa formalisation est encore assez récente dans les travaux de classification et de traitement séquentiel du langage en général. Parmi les rares publications existantes sur le sujet, on note celle de J. Kim et al.\footcite{kim_categorical_2019}. Les auteurs de cet article ajoutent l'information en parallèle au texte, en testant plusieurs moments d'insertion: les métadonnées sont d'abord projetées puis intégrées via des modifications de réseaux ou des concaténations. Le principe qui sous-tend cette insertion de la métadonnée par J. Kim est celui des goûts propres à un locuteur: cherchant à évaluer des critiques (culinaires en autres), les particularités de chaque locuteur peuvent s'exprimer à travers une relation lexique-évaluation notée. Ainsi, si une personne donne des critiques négatives en utilisant le terme \textit{épicé}, il est possible que ce terme ne relève pas des informations positives et aide à classer de futures des remarques. En ajoutant l'information liée à l'identité du locuteur, les auteurs montrent une augmentation du score atteignant jusqu'à +4.45 points de pourcent de la classification: la machine se met à prendre en compte l'idiolecte dans sa classification\footnote{L'exemple donné relève principalement de la partie lexicale d'un idiolecte, mais il est possible aussi que des faits de syntaxes puissent trahir les particularités propres de chaque critique.}.

% HERE 
Au-delà de l'intégration de l'idiolecte, la question du sociolecte peut-être tout aussi importante: \textit{virgo} n'a pas la même valeur pour un Romain chrétien et un romain païen, pour un romain du IVe siècle et du IIe d'avant notre ère. On retrouve d'ailleurs un usage des particularités d'un sociolecte dans l'une des études précédemment citées, celle de \textit{bot.zen}, où les chercheurs ont utilisé des plongements de mots dérivés de textes écrits par des apprenants de la langue anglaise\footcite{stemle_using_2018}. L'usage pour les plongements de mots d'une contextualisation temporelle\footcite{carlo_training_2019} ou géographique liées au contexte\footcite{gong_enriching_2020} n'est pas rare, contrairement à son utilisation dans des tâches de classification. Parmi les études sur l'usage de ces données, celle de Huang et Paul\footcite{huang_neural_2019} a la particularité de mettre au banc d'essais l'influence de l'injection de métadonnées temporelles dans les données pour des tâches de classification sur les sets habituels d'Amazon, Yelp, etc. Sur l'espace de trois à plus de dix ans, ils montrent et quantifient l'existence de changements de contexte et de déplacements sémantiques dans les datasets. L'usage d'informations temporelles montre une amélioration dans tous les cas, avec des augmentation de l'\textit{accuracy} allant d'un négligeable +0.4 à un impressionnant +3.8 points.

Enfin, il existe un dernier type d'information, en partie sociolectale, qui ne semble pas avoir intéressé outre mesure: celui de la prise en compte des \textit{sèmes} liés aux entités nommées, à savoir les lieux, les personnes, les divinités, les organisations, etc. Quand un romain parle d'Athènes, ou de Troie, un ensemble de sèmes peuvent apparaître: /Grèce/, /Épopée/, /Philosophie/, etc. Ces informations ne sont pas toujours déductibles des co-occurrences -- tous les locuteurs ne font pas comme les Américains en indiquant l'état dans lequel une ville se trouve (\enquote{Austin, Texas}), et encore, ce phénomène est réservé aux villes nord-américaines -- et peuvent relever d'une projection géographique, historique, littéraire -- les prostituées ont des noms grecs à Rome: qu'est-ce qu'un nom grec pour la machine ? Cette connaissance des propriétés des personnes (Martial et Sénèque viennent des régions espagnoles, Sénèque est un \enquote{conseiller politique} via sa position auprès de Néron), des lieux (localisation géopolitique) et des auteurs (Martial et Sénèque sont des auteurs dans des genres différents) ne peut transpirer entièrement et assez régulièrement pour que ces sèmes soient présents associé à leur nom. Les \textit{embeddings} étant une représentation de l'espace sémantique, des études pour fusionner des graphes de connaissances \enquote{encyclopédiques} avec ces derniers existent, qu'il s'agisse d'\textit{embeddings} classiques\footcite{wang_knowledge_2014} ou contextuels\footcite{zhang_ernie_2019}, mais leur application au latin semble encore assez lointaine. Dans \enquote{\textit{Semantic structure-based word embedding by incorporating concept convergence and word divergence}}\footcite{liu_semantic_2018}, Liu et ses collègues utilisent dans leur entraînement d'\textit{embeddings} un second objectif, celui de réduire la distance entre synonymes, hyponymes et hyperonymes issus d'un \textit{WordNet}: les gains en classification sur un corpus de presse dépassent leur meilleure \textit{baseline} de +0.4\% de score F1. Si un tel gain peut paraître minime, la limitation à un graphe sémantique tel que WordNet nous paraît dommage, surtout dans le cadre d'une classification d'articles de journaux, leur terrain d'expérimentation, où la relation entre entités nommées et leur classification ne peuvent être extraites du dictionnaire.


\subsection{Modèles et variations de modèles}

La construction du modèle pour cette recherche repose sur une architecture modulaire, permettant de tester de nombreuses hypothèses, tant du point de vue des modules d'encodage que des informations conservées. Du point de vue technique, l'ensemble de ces modèles sont construits sur la même architecture, avec des variations afin d'évaluer les apports des différentes informations et divers réseaux. Le plan du modèle est découpé en trois blocs principaux: 
\begin{itemize}
    \item un bloc de projection, qui vise à représenter chacune des unités d'information (mot, information morphologique, POS, métadonnée, etc.) dans l'espace;
    \item un bloc d'encodage, qui cherche à représenter le texte;
    \item une couche décisionnelle, qui produit la classification.
\end{itemize}
En fin de modèle, suivant qu'il s'agisse d'un entraînement ou d'un moment de prédiction, on trouvera une transformation de la décision en perte (\textit{loss}) ou en classification. Cette architecture est commune à l'ensemble des différents modèles étudiés jusqu'ici et nous permet d'y étudier des variations (\textit{cf.} Figure \ref{fig:chap4:Architecture}).

\begin{figure}
    \centering
    \includegraphics[width=\linewidth]{figures/chap4/architecture.png}
    \caption{Découpage de l'architecture générale. On distingue trois étapes: le passage des mots à des projections au niveau token, leur encodage au niveau phrase puis la partie de prise de décision. Selon que l'on entraîne ou que l'on utilise l'outil pour des prédictions, on obtient un score de perte ou de classification. Plusieurs modèles sont utilisables pour les deux premières couches de l'architecture.}
    \label{fig:chap4:Architecture}
\end{figure}

\subsubsection{Le bloc de projection}

La première forme de transformation s'effectue au niveau des tokens produits par \textit{Pie-Extended}: la séquence de tokens est fournie au modèle, qui utilise ou non l'ensemble des informations disponibles, à savoir les informations morphosyntaxiques, de formes ou de lemmes. Les formes et lemmes sont projetés par des embeddings pré-entraînés via Gensim\footcite{gensim} et non-entraînables basés sur \textit{FastText} ou \textit{Word2Vec} et de taille 200. Les catégories individuelles de morphosyntaxe (\texttt{Cas=X}, \texttt{Genre=X}, etc.) sont projetées sur une dimension assez faible de 3, tandis que la concaténation de ces tags (\texttt{Cas=X|Genre=Y}) est projetée sur une dimension de 20. En cas de projection au niveau caractère, disponible pour les lemmes et les formes, afin de prendre en compte la morphologie flexionnelle pour les caractères et les morphèmes non-flexionnels pour les deux, on utilise un encodage au niveau caractère utilisant un réseau LSTM avec une \textit{hidden size} de 150, deux couches et un \textit{dropout} de 30\%. L'ensemble des projections est concaténée en un seul vecteur pour chaque token (\textit{cf.} figure \ref{fig:chap4:projection:morphosyntax}).

On peut ajouter à cette transformation des formes, lemmes et informations morphosyntaxique un encodage via le module Bert latin, développé par D. Bamman et P. J. Burns. Deux problèmes se sont posés avec cette utilisation, dont un n'a pu être résolu. D'une part, le module pose un problème d'ingénierie: l'outillage entourant le modèle Bert est peu documenté, lent et repose sur des versions plutôt anciennes de certains outils qui ont parfois été un véritable puzzle de dépendances\footnote{CLTK posant notamment de gros problèmes sur ce point.}. Ensuite, il y a un véritable problème dans le pré-traitement de l'information: les deux auteurs du projet ont utilisé la tokenisation de CLTK en pré-traitement, que nous avons voulu à tout prix éviter, car, basée sur des règles, elle a une forte propension à interpréter chaque morphème final \textit{-ne} comme un clitique, de même que les \textit{-ve} ou les \textit{-que} (par exemple, \textit{observatio-ne} et \textit{observatio-ve} sont systèmatiquement tokenizés en deux morceaux, bien que le premier est souvent un simple ablatif). L'usage de ce \textit{pre-tokenizer} est par ailleurs une bizarrerie algorithmique du point de vue de Bert: l'objectif même de ce dernier, avec ses systèmes de tokenization reposant sur des \textit{sous-tokens}, est de pouvoir gérer ce type de phénomènes. Par exemple, dans le cadre du modèle Bert allemand de Chan et al.\footcite{chan_german_2019}, on retrouve ainsi logiquement traités par le \textit{tokenizer} les sous-tokens grammaticaux \texttt{\#\#ge}, les préfixes \texttt{\#\#be} et \texttt{\#\#ver}, sans pré-traitement manuel par les chercheurs: \textit{vergenommen} est ainsi transcrit en \texttt{\#\#ver + \#\#genommen}\footnote{\textit{nehmen} ayant de nombreux dérivés et étant relativement fréquent dans la langue allemande, il n'est pas étonnant que le composant \texttt{\#\#genommen} existe permettant ainsi une construction économique en bi-gram pour \textit{abnehmen, vernehmen, etc.}}. Nous nous sommes ainsi refusé à utiliser la prétokenization faussée de CLTK, espérant que le composant de tokenization de Bert soit suffisant pour pallier à cette situation\footnote{Mais la différence de tokenization pousse tout de même au final à l'incapacité d'aligner -- simplement -- les tokens.}. Enfin, en cas d'usage de projections via Bert et de projection classiques par \textit{embeddings} non contextuels, on utilise soit une concaténation des vecteurs Bert avec les vecteurs issus des plongements non contextuels, soit une réduction via une réduction en réseau linéaire.

\subsubsection{Le bloc d'encodage}
\label{part:chap4:architectures:variations-architectures:encodage}

Le bloc d'encodage est probablement le plus simple des trois blocs. Il prend comme entrée les projections unitaires précédentes -- au niveau token -- pour produire une projection séquentielle -- au niveau texte. On utilise majoritairement dans ce contexte des réseaux récurrents, LSTM et GRU, en mode bi-directionnel, avec un \textit{dropout} de 30\% et et une taille de 50 par direction. Alternativement aux réseaux récurrents, on retrouve aussi une option pour un réseau convolutionnel, avec des tailles de filtres de 2 à 5 (n-grams) et 5 filtres, un filtrage par MaxPool et une activation par ReLU.

Si ces réseaux sont communs, la recherche en classification a fait émerger d'autres moyens d'obtenir une représentation en augmentant ces modules. Le plus important de ceux-ci est sans aucun doute celui décrit dans \textit{Hierarchical Attention Networks for Document Classification}(\textit{HAN})\footcite{yang_hierarchical_2016}, présenté en 2016, basé sur des réseaux récurrents et qui sera décliné avec un modèle convoluttionnel (CHAN) en 2019\footcite{gao_hierarchical_2018}. Le principe d'\textit{attention} qu'il intègre et qui essaye de tirer d'une séquence les unités les plus importantes dans le cadre d'une classification, permet de pondérer ces unités dans le cadre de l'encodage de cette séquence. Par exemple, dans \textit{Lasciva est nobis pagina, vita proba}, dans le cadre de notre recherche, les mots présentant l'isotopie de la sexualité pourraient être en premier lieu \textit{lasciva} puis, peut-être, \textit{proba}, dans son jeu d'opposition /lascif/ contre Non(/lascif/). Les termes \textit{vita} et \textit{pagina} n'ont pas ou peu d'importance directe pour notre sujet, tandis que \textit{est}, \textit{nobis} et les signes de ponctuation n'en ont absolument aucune à priori. Cette importance des termes se traduit par une attention forte, tandis que le caractère négligeable des autres unités se traduit par une attention quasi-nulle.

Ce complément pour l'encodage séquentiel a aussi l'avantage de fournir au lecteur un outil d'analyse de ce qu'apprend le modèle. En ajoutant aux informations classiquement produites par le modèle, telles la classification les pondérations de l'attention, il met en exergue les traits saillants détectés. Cet outil a deux intérêts majeurs. D'une part, à travers les jeux de répétition qui auraient échappé à l'œil humain, il peut permettre de questionner une analyse plus manuelle des sèmes actualisés par une isotopie: les mots portants ces sèmes sont identifiés par le phénomène d'attention et amènent à interroger leur rôle. D'autre part, il peut permettre de détecter des phénomènes que nous qualifierions de \enquote{triche} mais qui relève bien du sur-apprentissage: dans notre cas, il nous est arrivé de trouver que pour les textes d'une édition en particulier, l'obèle de l'édition était un token spécifique au corpus positif, que le modèle a ainsi interprété comme trait saillant. Ce phénomène peut se retrouver avec d'autres mots, dont les actualisations /sexuel/ ne sont pas les plus courantes: c'est à cet endroit que l'équilibrage du corpus est important.

Dans le cadre de notre usage de Bert, il existe deux situations: une première où Bert est la seule projection utilisée, une second où Bert est employé avec d'autres projections.

Pour la première des deux, il existe plusieurs stratégies, dont la plus commune consiste à exploiter les tokens de délimitation des phrases \texttt{[CLS]} et \texttt{[SEP]} qui représentent respectivement le début de la séquence et la fin de la séquence. Bert proposant des plongements de token contextualisés, ces deux éléments ont la particularité de contenir une information contextualisée sur la phrase sans porter eux-mêmes de signification propre. Parmi les autres stratégies, on distingue les fonctions de réduction par moyenne (\textit{Reduce Mean}) et par valeur maximum (\textit{Reduce Max}), ainsi qu'une combinaison des deux via concaténation. Bert étant son propre réseau neuronal avec plusieurs couches, certains chercheurs et services recommandent d'éviter d'utiliser la dernière couche si il n'y a pas de ré-entraînement\footnote{\textit{Cf.} \url{https://github.com/hanxiao/bert-as-service/blob/master/docs/section/faq.rst\#bert-has-12-24-layers-so-which-layer-are-you-talking-about}}: elle serait ainsi trop liée à l'objectif initial de Bert (prédiction de mot ou de phrase suivante) et servirait donc de mauvaise ressource pour des tâches différentes. Afin d'ajouter une couche d'entraînement supplémentaire, on peut passer cette représentation via Bert dans un réseau ayant pour objectif de réduire l'information (couche linéaire): ainsi, d'un vecteur de 768 ou 1536 (\textit{MeanMax}) on passe alors à 256 avant d'entrer dans le bloc décisionnel. Il est aussi possible d'utiliser la projection contextualisée dans des réseaux plus traditionnels, tels que GRU, HAN ou LSTM.

\begin{figure}[ht]
    \centering
    \includegraphics[width=\linewidth]{figures/chap4/BertZoom.png}
    \caption{Zoom sur les possibilités et relations entre projection et encodage.}
    \label{fig:chap4:zoom-projection}
\end{figure}

Pour le second cas, où projections par Bert et projections par embeddings non-contextualisés se complètent, on procède ainsi: la projection de Bert est effectuée comme dans le cas précédent via une méthode de \textit{pooling} et une optionnelle réduction linéaire, tandis que la projection traditionnelle est encodée via l'un des réseaux décrits plus hauts, récurrent ou convolutionnel. Les deux entrées sont ensuite ou concaténées, ou réduite via une couche linéaire, afin de fournir les deux informations (\textit{cf.} figure \ref{fig:chap4:zoom-projection}).

\subsubsection{Classification, similarité: la couche décisionnelle et les méthodes d'entraînement}

Après avoir projeté les unités et encodé la séquence, le modèle passe à la dernière phase de la classification, celle de la couche décisionnelle. Cette couche décisionnelle se décline en deux architectures principales: l'une, classique, orientée classification via une projection linéaire; l'autre, plus particulière, orientée comparaison.

La première architecture est assez simple: elle prend en entrée l'information encodée par le bloc précédent et, via une couche linéaire, réduit celle-ci en un vecteur de taille 2, où chacune des cordonnées du vecteur représente une classe, ici \texttt{isotopie sexuelle} ou \texttt{absence de cette isotopie}. Pour produire la classification, on applique au vecteur une fonction \textit{softmax}, qui a la particularité d'équilibrer les valeurs telles que la somme des valeurs du vecteur soit égale à 1 en conservant la hiérarchisation des valeurs (le score le plus haut reste le plus haut). Pour la perte, on applique une traditionnelle \textit{Cross Entropy Loss}.

La seconde architecture est plus complexe, car elle recoupe plusieurs variations. Son principe fondamental est de comparer la sortie du bloc d'encodage -- donc l'échantillon encodé -- avec un ou plusieurs échantillons encodés dont la classe est connue. Cette méthode, dite de réseau siamois, consiste donc à faire passer dans le même réseau les données et d'inférer une classe en fonction de la proximité d'un texte avec un ou plusieurs autres textes. 


\begin{figure}[ht]
    \centering
    \includegraphics[width=\linewidth]{figures/chap4/contrastive.png}
    \caption{Schéma de fonctionnement d'un réseau siamois à double échantillons.}
    \label{fig:chap4:reseau:ContrastiveLoss}
\end{figure}

La première variation des réseaux siamois utilise deux échantillons, un que l'on souhaite classer, et un dont on connait la classe, ci-après le comparant. Une fois encodés, on les compare à l'aide d'une fonction de similarité cosinus: si la similarité entre les deux dépasse un seuil, fixé arbitrairement pour nous à 0.6, l'échantillon est considéré comme appartenant à la classe du comparant. Au contraire, si le seuil n'est pas atteint, cette appartenance est rejetée. Pour la perte (\textit{loss}), qui permet de diriger l'entraînement, on utilise une \textit{contrastive loss}\footcite[Entre autres, ]{khosla_supervised_2021} qui prend directement comme paramètres les encodages d'échantillons pour calculer un écart entre les deux (\textit{cf.} image \ref{fig:chap4:reseau:ContrastiveLoss}). La seconde variation des réseaus siamois utilise trois échantillons, dont deux comparants. Ces deux échantillons de comparaison ne doivent pas représenter la même classe, et l'un d'eux doit, au possible, avoir la même classe que celui avec lequel on entraîne le modèle. On calcule la distance entre les comparants encodés et l'échantillon via une distance euclidienne, et l'on attribue la classe du comparant le plus proche à l'échantillon évalué. Pour la perte, on utilise une perte appelée \textit{Triplet Margin Loss}\footcite{hermans_defense_2017} qui est alors calculée directement à partir des données issus de la phase d'encodage.

\begin{figure}[ht]
    \centering
    \includegraphics[width=\linewidth]{figures/chap4/triplet.png}
    \caption{Schéma de fonctionnement d'un réseau siamois à triple échantillons.}
    \label{fig:chap4:reseau:Triplet}
\end{figure}

Pour ces deux variations, les stratégies d'entraînement ou de prédiction peuvent varier. La méthode la plus simple, à la fois algorithmiquement et méthodologiquement, est la sélection manuelle d'un ou plusieurs représentant(s) par classes: à chaque fois que le modèle entraîne un échantillon, il tire au hasard un représentant positif ou négatif avec lequel les distances seront calculées. Ces représentants sont ensuite sauvegardés avec le modèle afin de pouvoir continuer à effectuer des prédictions. Une autre méthode d'échantillonnage consiste à prendre tous les binômes ou triplets possibles d'un même ensemble d'échantillons et de les utiliser comme données à évaluer. Il existe cependant des méthodes d'échantillonages qui cherchent à mathématiquement maximiser les erreurs afin de rendre l'apprentissage le plus difficile possible, nous en testons deux: le \textit{BatchEasyHard} de Xuan et al.\footcite{xuan_improved_2020}, qui cherche à trouver des triplets dont le comparant positif est simple mais le comparant négatif est complexe, et le \textit{BatchHard} de Hermans et al.\footcite{hermans_defense_2017} qui cherche par défaut les triplets les plus complexes à classer pour chaque échantillon\footnote{Nous utilisons les implémentations de ces échantillonneurs issu es de la librairie \texttt{pytorch-metric-learning}, \cite{musgrave2020pytorch}}. 

\subsubsection{Le cas particulier de l'inclusion des métadonnées}
\label{chap4:part2:metadata}

Nous parlions en \ref{chap4:encodage:metadonnees} de la possibilité et de la potentielle importance d'incorporer les métadonnées en tant qu'elles apportent des identificateurs extra-linguistiques pour les idiolectes et sociolectes. Nous approchons le problème de deux manières différentes: d'une part, à travers l'intégration de tokens de métadonnées dans la séquence, d'autre part via l'utilisation de modules modifiés en suivant les travaux de \footcite{kim_categorical_2019}. Quatre domaines de métadonnées peuvent être injectés, à savoir: la période d'écriture (le siècle), l'auteur, la structure logique de citation (par exemple: chapitre,section) et la forme d'écriture (prose ou vers).

Dans le cas de token de métadonnées, les vecteurs de mots sont nécessairement entraînables et non figés par le pré-entraînement, et on ajoute au vocabulaire des ces différents modèles de de nouveaux éléments suivants une syntaxe spécifique, telle \texttt{$[$Date:1$]$}. Ces tokens sont traités sans POS et sans morphologie, leur lemme étant identique à leur forme. Ils sont ensuite traités comme le reste des tokens par les réseaux d'encodage.

Dans le second cas, les métadonnées sont projetées dans des espaces propres de taille 64 comme des embeddings. Ces données sont ensuite injectées dans des modules modifiés: nous proposons ainsi, grâce aux travail de Kim et al. des modifications des réseaux LSTM, HAN basé sur LSTM et de la couche linéaire pour la partie décisionnelle. De même que dans l'article original, cette information ne peut être injectée qu'une seule fois dans le réseau par échantillon.

\section{Les modèles obtenus}

L'entraînement de modèles et la construction d'architectures est un processus itératif et exploratoire: il s'agit de tester des hypothèses de modèles, des hypothèses de paramétrages de modèles, puis d'aller dans l'une ou l'autre direction qui semble apporter les meilleurs réponses. Rendre compte de ce processus, et de ses résultats, n'est pas une mince affaire: la chronologie a un rôle dominant dans la sélection de modèles qui survivront aux différents tests.

\subsection{Méthodes et résultats principaux}

\subsubsection{Méthode de sélection globale}

Pour définir les meilleurs modèles, nous avons procédé par test d'architecture: les réflexions qui suivront, leurs analyses, ont enrichi cette méthode continuellement pendant nos expériences. Le résultat produit une grande variété d'architectures: vingt-huit architectures avec couches linéaires et vingt-huit en réseau siamois, que l'on multiplie par deux -- en faisant varier la taille de la couche d'encodage entre 128 et 256 -- pour atteindre les cent~douze configurations testées au total. À partir de ce premier set d'entraînement, nous récupérons les cinq meilleures configurations toute couche décisionnelle confondue. Cette sélection d'architecture fait l'objet d'une seconde salve d'entraînements où chacune d'entre elles fait l'objet de dix entraînements: cette multiplication des entraînements permet prendre en compte les possibles variations de ces entraînements (\textit{cf.} figure~\ref{fig:chap4:50configurations}). Chaque architecture est testée avec les mêmes hyperparamètres pour l'entraînement: un \textit{learning rate} de $1e^{-5}$, une taille de \textit{batch} de 4 et 20 \textit{epochs} au maximum. 

\begin{figure}[ht]
    \centering
    \includegraphics[width=\linewidth]{figures/chap4/50config.png}
    \caption{Résumé de la méthode de sélection des configurations}
    \label{fig:chap4:50configurations}
\end{figure}

% En cas de résultats plus bas que ce que promet la littérature scientifique, d'autres tests d'architectures ou d'outillages peuvent s'ajouter à ces méthodes. Si de nouvelles configurations ont dû être comparées, elles sont entraînées elles aussi dix fois.
% Commencer par les métadonnées en fait
% Ensuite, séparer Bert et Non Bert

% Expliquer itération: un run par archi. d'abord, pour évacuer les mauvaises archis.

\subsubsection{Modèles enrichis par métadonnées}

La publication de J. Kim et al.\footcite{kim_categorical_2019} proposait une méthode simple d'intégration de métadonnées dans l'entraînement de modèles de classification de texte permettant de prendre en compte les identités des locuteurs. Cette méthode, reposant donc sur une identification même partielles de traits idiolectaux ou sociolectaux, a été testée sur notre corpus avec quatre métadonnées dont l'identité de l'auteur, le siècle de naissance de l'auteur, la structure logique de citation de l'oeuvre et la forme de cette dernière\footnote{\textit{Cf.} section \ref{chap4:part2:metadata}}. 

\begin{table}[ht]
\centering
\resizebox{\linewidth}{!}{%
\begin{tabular}{@{}lllllllll@{}}
\toprule
Rang & Score F1 (Positif) & Rappel (Positif) & Précision (Positif) & Couche enrichie & Réseau d’encodage & Métadonnées utilisées              \\ \midrule
1    & 86.05\%            & 87.25\%          & 84.88\%             & Linéaire        & HAN               & Toutes                             \\
2    & 84.45\%            & 80.08\%          & 89.33\%             & Linéaire        & HAN               & Toutes                             \\
3    & 84.15\%            & 88.84\%          & 79.93\%             & Linéaire        & HAN               & Toutes                             \\
4    & 84.12\%            & 81.27\%          & 87.18\%             & Linéaire        & HAN               & Forme,Siècle,Structure de Citation \\
5    & 83.79\%            & 79.28\%          & 88.84\%             & Linéaire        & HAN               & Forme,Siècle,Structure de Citation \\
6    & 83.30\%            & 78.49\%          & 88.74\%             & LSTM            & LSTM              & Toutes                             \\
7    & 83.17\%            & 83.67\%          & 82.68\%             & Linéaire        & HAN               & Forme,Siècle,Structure de Citation \\
8    & 82.08\%            & 78.49\%          & 86.03\%             & Linéaire        & HAN               & Forme,Siècle,Structure de Citation \\
9    & 81.82\%            & 78.88\%          & 84.98\%             & Linéaire        & HAN               & Toutes                             \\
10   & 81.65\%            & 78.88\%          & 84.62\%             & LSTM            & LSTM              & Forme,Siècle,Structure de Citation \\
11   & 81.24\%            & 78.49\%          & 84.19\%             & LSTM            & LSTM              & Forme,Siècle,Structure de Citation \\
12   & 81.08\%            & 77.69\%          & 84.78\%             & LSTM            & LSTM              & Forme,Siècle                       \\
     & …                  &                  &                     &                 &                   &                                    \\
23   & 78.54\%            & 72.91\%          & 85.12\%             & Aucune          &                   & Aucune                             \\ \bottomrule
\end{tabular}%
}
\caption{Résultats ordonnés par le score F1 de la catégorie positive des meilleurs modèles sur la recherche de meilleure architecture. Les onze premiers modèles utilisent toutes les métadonnées ou toutes sauf la métadonnée auteur. Les auteurs paramètres n'ont pas d'effet aussi marquant sur le classement.}
\label{tab:chap4:resultats-metadata}
\end{table}

\begin{table}[ht]
\centering
\begin{tabular}{l|r|r}
                                   & \multicolumn{2}{r}{Entraînements} \\ \hline
                                   & Tous & Top 25 (Score F1 Positive) \\ \hline
Toutes                             & 32   & 9                          \\
Forme,Siècle,Structure de Citation & 16   & 8                          \\
Forme,Siècle                       & 20   & 7                          \\
Token de Métadonnées               & 4    & 0                          \\
Aucune                             & 40   & 2                          \\ \hline
\end{tabular}
\caption{Répartitions des configurations par usage des métadonnées. $p=3.54^{-14}$, on peut rejeter l'hypothèse nulle: la variation de distribution est significative, l'usage des toutes les métadonnées ou de trois d'entre elles ont un impact sur les résultats finaux.}
\label{tab:chap4:metadata-p-value}
\end{table}

\begin{figure}[ht]
    \centering
    \includegraphics[height=6cm]{figures/chap4/scoreMetadata.png}
    \caption{Dispersion des résultats en fonction des usages de métadonnées. Les populations ne sont pas de taille équivalentes. On voit une précision équivalente pour les meilleurs modèles de chaque catégorie, tandis que le rappel est extrêmement variant d'un usage à un autre, avec un net bénéfice pour l'usage des métadonnées (modèle le plus performant au niveau de la médiane des autres modèles).}
    \label{fig:chap4:metadata-boxplot}
\end{figure}

L'hypothèse de l'efficacité qu'amènerait cette information est confirmée (\textit{cf.} figure \ref{fig:chap4:metadata-boxplot}): sur les cinquante entraînements, en utilisant le score F1 sur la catégorie positive comme mesure de classement, les onze meilleurs utilisent un réseau enrichis de métadonnées, soit au niveau de la couche linéaire soit au niveau du réseau d'encodage LSTM. Avec 86,05\% en score F1, 87,25\% en rappel et 84,88\% en précision sur la catégorie positive, le meilleur des modèles se situe à 4.87 points du score F1 du premier modèle n'utilisant pas les métadonnées identifiant les auteurs et structures logiques\footnote{Et à 9,56 points en rappel et à 0,01 en précision, ce qui est négligeable pour ce dernier} (\textit{cf.} table \ref{tab:chap4:resultats-metadata}) . Si on le compare au premier modèle qui n'utilise aucune métadonnée, l'écart se creuse à 8 points de score F1 et 17,13 de rappel! Sur les 25 premiers modèles, 9 utilisent toutes les métadonnées (36\%), 8 utilisent toutes les métadonnées sauf celle identifiant l'auteur (32\%), 7 utilisent les métadonnées \texttt{Forme} et \texttt{Siècle} (28\%), 2 n'en utilisent aucune (8\%, \textit{cf.} table \ref{tab:chap4:metadata-p-value}). Les modèles à métadonnées sur-performent donc face aux modèles sans, et l'inclusion de la métadonnée Auteur ou Structure de Citation est plus importante que celles du siècle ou de la forme.
% Dans une première salve d'entraînement, ceux-ci performent le mieux. Quelques scores

\begin{table}[ht]
    \centering
    \resizebox{\textwidth}{!}{%
    \begin{tabular}{llrr}
    \toprule
                   Texte & Métadonnées utilisées        & En tant que Cicéron & En tant que Martial \\
    \midrule
    \textit{De Finibus}, Cicéron & Toutes                       &               4.25\% &              72.60\% \\
    \textit{Epigrammata}, Martial &                              &               7.89\% &              82.05\% \\ \midrule
    \textit{De Finibus}, Cicéron & Toutes sauf Auteur           &               4.69\% &              47.54\% \\
    \textit{Epigrammata}, Martial &                              &               7.33\% &              59.76\% \\ \midrule
    \textit{De Finibus}, Cicéron & Forme et Siècle              &               0.91\% &              36.03\% \\
    \textit{Epigrammata}, Martial &                              &               6.41\% &              48.10\% \\ \midrule
    \textit{De Finibus}, Cicéron & Aucune                       &               2.91\% &               2.91\% \\
    \textit{Epigrammata}, Martial &                              &              15.29\% &              15.29\% \\
    \bottomrule
    \end{tabular}%
    }
    \caption{Nombre de phrases annotées automatiquement comme positive en fonction des modèles avec les meilleurs scores pour chaque type d'usage des métadonnées. L'absence d'usage de métadonnées n'a évidemment aucun impact sur les scores, tandis que l'impact des métadonnées sur le taux de positif est corrélé aux nombres de catégories de métadonnées utilisées.}
    \label{tab:chap4:martial-ciceron}
\end{table}




Cependant, l'application du modèle sur des données hors domaine, sur des données de terrain, réserve une surprise: si les modèles de métadonnées avec l'information Auteur et Structure logique de citation ont d'aussi bons scores, c'est parce qu'ils ont été victime d'un sur-apprentissage dûs aux biais inhérents de notre corpus. Pour prouver ce problème de sur-apprentissage, on propose l'expérience suivante:
\begin{enumerate}
    \item Le modèle est appliqué sur deux oeuvres: les Épigrammes de Martial, connus pour la sexualité qui y est présente mais pas omniprésente (de nombreux épigrammes si ce n'est la majorité parlent de sujets tout autres); le \textit{De Finibus} de Cicéron, texte philosophique qui ne devrait pas contenir beaucoup de passages présentant l'isotopie sexuelle.
    \item On modifie intentionnellement les métadonnées fournies au modèle: pour chaque texte, on passe les métadonnées de l'autre texte une fois. Ainsi, \textit{Epigrammata} est analysé une fois en tant que Martial, et une autre fois en tant que \textit{De Finibus}.
    \item Chaque texte est annoté avec le meilleur modèle avec l'ensemble des métadonnées, l'ensemble sauf l'information auteur, les métadonnées Forme et Siècle, et enfin sans aucune métadonnée (\textit{cf.} table \ref{tab:chap4:resultats-metadata} pour les scores de ces modèles). Les modèles n'utilisant pas de métadonnées ne sont pas utilisés qu'une fois.
    \item On récupère le pourcentage de phrases dont la catégorie prédite est \texttt{Positive}.
\end{enumerate}
Le résultat de cette expérience montre un sur-apprentissage évident sur les métadonnées, décroissant avec le nombre de catégories de ces dernières. Au plus bas, annoter le \textit{De Finibus} avec les informations \texttt{Versifié} et \texttt{1er siècle} multiplie par quatre-vingt-dix le nombre de phrases annotées comme positives.

L'usage de métadonnées est-il à proscrire ? Si l'on en s'arrête aux résultats, en particulier l'escalade importante qui s'opère sur Martial et Cicéron, la réponse est oui, l'inclusion de métadonnées provoque une bien trop haute variation pour être prise sans risque. Même redressé à l'aide de la précision (84.78\% pour le modèle \texttt{Siècle} et \texttt{Forme}), le modèle trouverait 40\% de phrases présentant des traits d'une isotopie sexuelle détectées par le modèle de manière juste. Bien que le corpus présente cette sur-représentation, il sera difficile de faire autrement; les \textit{Epigrammata} seront toujours composés d'une très large partie de poèmes à isotopie sexuelle, tandis que Cicéron le sera très rarement, aucune recomposition de corpus ne peut changer cela ! Il est dans un sens normal que le modèle et soit trop attentif à des tournures particulières chez Martial que chez Cicéron, et le lecteur humain sera sûrement aussi en ce sens \enquote{surentraîné} par des \enquote{biais} accumulés via une connaissance fine du corpus. Même en ne prenant en compte que la métadonnée \texttt{Forme}, vu le corpus latin, il est attendu que l'on trouve plus facilement en vers une isotopie de la sexualité qu'en prose\footnote{Est-ce un biais lié à la sélection opérée sur les oeuvres par le filtre des siècles cependant ?}. En ce sens, le modèle fonctionne: il prend bien en compte les particularités de chaque idiolecte ou sociolecte, bien qu'un peu trop.

Cependant, en dehors des modèles à métadonnées, le meilleur modèle sans métadonnée, même s'il est moins efficace sur les données de test, donnerait des résultats plus crédibles pour Martial comme pour Cicéron avec 2,91\% de phases identifiées dans le \textit{De Finibus} et 15,29\% dans les \textit{Epigrammata} (redressées en vraies positives avec l'hypothèse d'avoir un test représentatif: 2,47\% et 13,01\%). Il reste à choisir ce que l'on préférera: doit-on avoir plus de faux positifs, y compris à cause d'un \enquote{léger} surentraînement, ou doit-on réduire ce bruit au risque de perdre de l'information (le modèle sans métadonnée tombant à 72\% de rappel) ? Nous faisons le pari des faux positifs, toujours dans l'objectif de questionner le texte, de laisser la place au philologue dans la compréhension du texte, tout en lui fournissant un corpus à explorer plus facilement que s'il ne fallait chercher soi-même les centaines de phrases qui pourraient rentrer dans cette catégorie.

\subsubsection{Embeddings contextuels}

Les \textit{embeddings} traditionnels (\textit{Word2Vec, GloVe, FastText} entre autres) posent le problème de l'absence de désambiguisation. Si un travail important a lieu pour essayer d'introduire cette information dans le dictionnaire de vecteurs qu'ils produisent, il ne porte pour l'instant pas ses fruits et s'est fait dépasser par une autre approche: les \textit{embeddings} contextuels, tels \textit{Bert}. En effet, pour reprendre l'exemple de PEDRO ET AL., pour les outils traditionnels comme \textit{Word2Vec}, \texttt{Washington} représente à la fois l'état, la ville, le président (mais aussi quelques acteurs, îles, etc.). Bert, en produisant des projections en contexte, permet d'outrepasser -- en partie -- ce problème. En utilisant \textit{Latin Bert}\footcite{bamman2020latin} comme entrée seule ou comme complément à la phrase encodée, nous proposions d'évaluer les capacités de Bert dans un objectif de classification.

\begin{figure}[t]
    \centering
    \includegraphics[height=6cm]{figures/chap4/scoreDispersionEmbeddings.png}
    \caption{Dispersion des résultats en fonction du système d'\textit{embeddings} utilisé. Bert est utilisé ici comme source  d'information supplémentaire, et non comme source unique, via un \textit{pool} du token \texttt{[CLS]} en dernière couche d'encodage.}
    \label{fig:chap4:bert-dispersion-fusion}
\end{figure}

Le résultat est clair: \textit{Latin Bert} n'apporte pas de bénéfices importants en tant que complément d'information, voire enregistre des sous-performance face aux autres modèles (\textit{c.f.} figure \ref{fig:chap4:bert-dispersion-fusion}). Le premier modèle mixte \texttt{HAN(Lemme)+Bert->Linéaire} se place en 36e position du classement original avec un score F1 de 76.02\%, un rappel de 70.11\% et 83.01\% de précision, à 2 points du premier modèle n'utilisant pas Bert. La puissance de calcul demandée part Bert étant importante, si ses performances ne sont pas compétitives avec un modèle "traditionnel", il n'y a pas d'intérêt à le conserver. Non affiché dans les résultats, une tentative de ré-entraînement de Bert pour le spécialiser (anglais \textit{fine tune}) sur le dataset bloquait les résultats sous la barre des 20\%, probablement à cause de la très petite taille des données.

\afterpage{%

\begin{sidewaysfigure}
    \begin{subfigure}{0.48\hsize}
        \centering
        \includegraphics[width=0.9\hsize]{figures/chap4/BertPoolingGeneric.png}
        \caption{Dispersion des résultats en fonction des catégories de méthodes d'exploitation des encodages Bert: \textit{Pooling} contient les méthodes \texttt{MeanMax}, \texttt{Max}, \texttt{Mean} et l'usage du token \texttt{[CLS]}}
        \label{fig:chap4:bert-pooling-generic-eval}
    \end{subfigure}%
    \hfill%
    \begin{subfigure}{0.48\hsize}
        \centering
        \includegraphics[width=0.9\hsize]{figures/chap4/BertPoolingDetailed.png}
        \caption{Dispersion des résultats en fonction de la méthode d'exploitation des encodages Bert et de la couche (entre parenthèse) utilisée (-1 étant la dernière couche, -2 l'avant-dernière). Les méthodes de ré-encodages sont les plus performantes, la couche utilisée ayant peu d'effet sur les résultats. Chaque architecture est entraînée 10 fois, avec une couche linéaire comme couche décisionnelle.}
        \label{fig:chap4:bert-pooling-detailed-eval-fscore}
    \end{subfigure}%
    \caption{Étude des dispersions des résultats des modèles utilisant Bert dans leur architecture.}
\end{sidewaysfigure}
\clearpage
}
Au contraire, une architecture reposant sur l'usage de l'encodage de Bert, sans spécialisation, par une nouvelle couche d'encodage de type GRU ou HAN fournit des résultats plus convaincants (\textit{c.f.} figure \ref{fig:chap4:bert-pooling-generic-eval}). Les méthodes mentionnées plus tôt en \ref{part:chap4:architectures:variations-architectures:encodage} n'ont pas nécessairement amélioré les résultats face à un usage classique du token \texttt{[CLS]}, mais induisent cependant bien des variations. Là où le \textit{pooling} du token \texttt{[CLS]} est nécessairement dans les 4 plus mauvais modèles en score F1, précision et rappel, la sélection par la valeur maximum (\texttt{Max}) ou par concaténation de la valeur maximum et de la moyenne (\texttt{MeanMax}) sont régulièrement dans le top 4 des modèles à \textit{pooling}. Mais ces architectures n'en restent pas moins très très loin des modèles à ré-encodage, presque 20 points entre les médianes de score F1 des deux catégories d'usage de Bert, avec un maximum pour le \textit{pooling} qui touche à peine le bas de la moustache inférieure des méthodes GRU et HAN.

En général, les encodages des projections Bert par HAN et GRU, potentiellement suivis d'une couche linéaire enrichie, touchent les résultats en Score F1 du milleur modèle sans métadonnées. L'enrichissement par métadonnées influent clairement les résultats de plusieurs points et inversent même l'intérêt des modèles (HAN est meilleur que GRU sans métadonnées, mais cède sa place de premier avec). GRU montre de très bon résultat dans son premier quartile, à égalité avec les modèles enrichis. Mais, bien que les scores soient largement meilleurs comparés aux modèles \texttt{[CLS]} présentés plus haut, ils restent au niveau d'architectures plus traditionnelles, pour un coût plus élevé en entraînement et calcul d'inférence.

\subsubsection{L’échec des modèles de similarité}

\begin{figure}[p]
    \centering
    \includegraphics{figures/chap4/SiameseVsLinear.png}
    \caption{Dispersion des résultats, à population quasi-égale, entre les modèles à couche linéaire et à couche de similarité. En dehors du rappel (proportion de réponse sûres parmi les réponses fournies), les modèles à couche de similarité sont largement dépassés par les modèles linéaires sur les autres éléments de mesure.}
    \label{fig:chap4:modeles-siamois-vs-lineaires}
\end{figure}

Nous avions proposé en type d'architecture de faire varier les couches décisionnelles entre couche en réseau siamois et couche en réseau linéaire, les réseaux siamois étant vantés par la littérature scientifique comme particulièrement adaptés pour les stocks de données. Les résultats présentés en \ref{fig:chap4:modeles-siamois-vs-lineaires} ne reprennent que les modèles utilisant une couche en \textit{constrastive loss} et montrent qu'ils ne sont pas à la hauteur de nos attentes.

Seuls les modèles en \textit{constrastive loss} sont présentés dans le schéma mais toutes les variations suivantes ont été utilisées, sans provoquer de différence nette avec la configuration inclue dans les schémas:
\begin{itemize}
    \item les configurations en \textit{triplet loss} n'apportaient aucun bénéfice. La comparaison avec un exemple positif et un exemple négatif ou la comparaison avec un seul exemple positif ne change pas les résultats du modèle.
    \item Les techniques d'échantillonage mentionnées plus haut -- \textit{BatchEasyHard} \footcite{xuan_improved_2020} et \textit{BatchHard}\footcite{hermans_defense_2017} -- n'ont pas modifié les résultats non plus. Ces deux méthodes cherchent à sélectionner les exemples les plus difficiles à différencier ou à classer ensemble dans un batch.
    \item Une variation de \textit{BatchEasyHard} a été utilisée où tous les documents sont comparés à l'ensemble des autres documents d'un même batch. Cette méthode n'a pas fournit de modifications non plus.
\end{itemize}

Les résultats sont donc peu probants et posent légitimement la question de la source de cette \enquote{déception}. En dehors d'une possible -- mais peu probable au vue du nombre de tests -- erreur d'implémentation, il reste l'hypothèse de l'inadéquation des modèles siamois pour ce type de tâche, reposant sur des textes extrêmement courts avec une très forte disparité lexicale entre les différents éléments. En effet, la plupart des cas utilisant les réseaux en comparaison cherchent à sélectionner des éléments qui présentent de fortes similarités, graphiques ou textuelles: peu de variations existent entre un filigrane et un autre, une annonce de vente pour un téléphone portable est proche de celle d'un autre de la même marque et utilisent probablement des informations et lexiques similaires. À la surface, nos textes sont différents, et le modèle n'est pas capable de comprendre suffisamment bien leur similarité sous-jacente. Une autre hypothèse: les modèles ne restent pas catastrophiques (50\% de précision environ, soit 50\% de faux positifs parmi ceux trouvés) et pourraient très bien être plus résistants sur des données nouvelles: si l'on regarde les résultats des modèles à métadonnées et leur 36\% de contenus \enquote{sexuels} dans les \textit{Epigrammata}, il n'est pas improbable que de nombreux résultats parmi eux soit de faux positifs, plus que le taux de 75\% le promet en tout cas.

\subsubsection{Architectures des meilleurs modèles}

D'après les résultats de la recherche originale, les modèles présentant les meilleurs résultats sont les modèles:
\begin{itemize}
    \item incluant les métadonnées dans leur couche d'encodage (LSTM) ou de décision (Linéaire);
    \item incluant ou non l'information morphologique;
    \item avec une couche de décision linéaire
    \item utilisant un encodeur LSTM ou un réseau avec attention (HAN).
\end{itemize}

Ces résultats sont confirmés par la seconde phase d'entraînements qui montre un net avantage au 4 modèles principaux sélectionnés (en premiers sur la table \ref{tab:chap4:general-results}). L'enrichissement en métadonnées, même limitées à l'information formelle et celle du siècle d'appartenance de l'auteur, permet au modèle de mieux analyser les résultats. Mais l'on peut apercevoir aussi que certains modèles, bien qu'éloignés par leur médiane du top 4, peuvent rivaliser: le modèle utilisant HAN et Word2Vec sans enrichissement vient ainsi battre 3 des tops modèles en précision, un modèle en GRU et Word2Vec vient battre en rappel les valeurs maximales du top 4 aussi. Il est donc possible de se reposer sur des modèles moins sur-entraînés, potentiellement aussi capables que des modèles enrichis, avec une perte en évaluation de 1,80 à 4 points suivant les mesures.

Parmi les types de modèles, seuls sont à proscrire les modèles Bert, dont l'entraînement et l'usage en terrain réel sont coûteux et dont les scores sont bien éloignés du top 4, et les réseaux utilisant CNN comme encodeurs, du moins avec des filtres $[2,3,4,5]$ (\textit{cf.} figure \ref{fig:chap4:main-results-dispersion-encoder}). Au contraire, les réseaux utilisant des encodeurs LSTM, GRU ou HAN ont tendance à avoir des résultats équivalents entre eux, avec une dispersion équivalente des résultats. Parmi les paramètres, la taille de la dimension d'encodage n'a pas ou peu d'impact sur les résultats (figure \ref{fig:chap4:main-results-dispersion-dimension}). Il reste que chaque modèle ou type d'architecture présente une forte dispersion de leurs résultats, hors Score F1 (5points sur le F1 pour les meilleures architectures, mais de 10 à 15 points sur les deux autres mesures): une modification du \textit{learning rate} n'affecte pas cette disparité en précision et rappel. La taille du corpus de test peut être à la source de cette variance.

\begin{table}[ht]
\centering
\resizebox{\textwidth}{!}{%
\begin{tabular}{lllll|rrr} \hline
Morphologie & Encodeur & Enrichissement & Embeddings & Taille encodée & Précision      & Rappel         & Score F1       \\ \hline
Aucune      & HAN      & Linéaire       & W2V        & 128            & \textbf{86.86} & \textbf{75.50} & \textbf{80.07} \\
Aucune      & LSTM     & LSTM           & W2V        & 256            & 84.28          & 75.10          & 79.53          \\
Agglomérée  & HAN      & Linéaire       & W2V        & 256            & 83.38          & 75.30          & 79.51          \\
Agglomérée  & LSTM     & LSTM           & W2V        & 128            & 85.72          & 72.71          & 78.51          \\
Agglomérée  & GRU      & -              & W2V        & 128            & \textbf{85.07} & 71.12          & \textbf{77.22} \\
Agglomérée  & HAN      & -              & W2V        & 128            & 82.38          & \textbf{71.51} & 76.54          \\
Agglomérée  & HAN      & -              & W2V        & 256            & 84.51          & 69.52          & 76.50          \\
Agglomérée  & GRU      & -              & W2V        & 256            & 84.26          & 69.72          & 76.47          \\
Aucune      & GRU      & Linéaire       & Bert       & 256            & 80.47          & 70.72          & 76.11          \\
Aucune      & HAN      & Linéaire       & Bert       & 256            & 82.34          & 69.52          & 74.94          \\
Agglomérée  & LSTM     & -              & W2V        & 128            & 81.80          & 70.72          & 74.57          \\
Agglomérée  & LSTM     & -              & W2V        & 256            & 79.55          & 70.12          & 74.31          \\
Aucune      & LSTM     & -              & Bert       & 256            & 77.97          & 70.12          & 73.15          \\
Aucune      & GRU      & -              & Bert       & 256            & 79.41          & 67.73          & 72.50          \\
Agglomérée  & CNN      & -              & W2V        & 128            & 82.20          & 55.78          & 66.50          \\
Aucune      & MeanMax  & -              & Bert       & 256            & 71.63          & 54.58          & 62.07          \\
Agglomérée  & CNN      & -              & W2V        & 256            & 84.65          & 45.62          & 59.90          \\ \hline
\end{tabular}%
}
\caption{Score médian des meilleurs modèles, des modèles BERT et de modèles simples (CNN, GRU, HAN, LSTM sans enrichissements) sur 10 entraînements. En gras, les meilleurs scores des modèles avec enrichissement et sans enrichissement. L'enrichissement apporte un réel bonus (+2,8 points en Score F1, +4,40 en rappel, +1,80 en précision), contrairement à l'usage de modèles \enquote{riches} et \enquote{lourds} comme Bert, à la traîne, 1 points environ des meilleurs modèles sans enrichissements.}
\label{tab:chap4:general-results}
\end{table}

\afterpage{%
    \begin{figure}[h]
            \centering
            \includegraphics[width=.9\linewidth]{figures/chap4/main-dispersionscores.png}
            \caption{Dispersion des score des meilleurs modèles, des modèles BERT et de modèles simples (CNN, GRU, HAN, LSTM sans enrichissements) sur 10 entraînements.}
            \label{fig:chap4:main-results-dispersion}
    \end{figure}%
    \begin{figure}[h]
            \begin{subfigure}{0.65\textwidth}
                \centering
                \includegraphics[width=\linewidth]{figures/chap4/main-lineartop5vstheworld.png}
                \caption{Dispersion des score en fonction des types de modèles (Enrichis, Bert, Classiques) (à gauche) et en fonction des tailles d'encodage (à droite)}
                \label{fig:chap4:main-results-dispersion-encoder}
            \end{subfigure} \hfill %
            \begin{subfigure}{0.30\linewidth}%
                \centering
                \includegraphics[width=\linewidth]{figures/chap4/main-encodingdimension.png}
                \caption{Dispersion des score en fonction de la dimension d'encodage, toutes architectures confondues.}
                \label{fig:chap4:main-results-dispersion-dimension}
            \end{subfigure}
            \caption{Dispersion des résultats au niveau macro et non au niveau architecture.}
    \end{figure}%
    \clearpage
}

\subsection{Extensibilité des modèles}

Pour notre recherche, nous avons bénéficié de l'immense travail de J. N. Adams qui nous a permis de compiler une base de plus d'un millier d'exemples. Cependant, le travail de ce dernier reste rare et unique par sa dimension. L'usage de l'intelligence artificielle pour produire des corpus est donc une chance de produire d'autres recherches équivalentes à celle du britannique. Mais pour cela, évaluer l'efficacité du modèle à d'autres échelles est important. Nous proposons donc trois terrains d'explorations différentes, permettant d'évaluer l'apport des différentes architectures:

\begin{enumerate}
    \item Explorer, avec un corpus diversifié en source, l'importance de sa taille. Avec un corpus beaucoup plus petit mais comportant une variété d'exemples (avec des métaphores, des termes \enquote{explicitement} lié à l'isotopie, etc.) plus importante.
    \item Explorer l'efficacité d'un modèle constitué d'exemples présentant uniquement des termes particulièrement liés à l'isotopie considérées: par exemple, \textsc{culus}, \textsc{mentula}, \textsc{pedico}, etc.
    \item Dans un objectif un peu contraire, étudier la capacité du modèle à reconnaître des termes explicites à partir du simple apprentissage d'exemples \enquote{figurés}. Cela permet d'évaluer la confiance à donner au modèle d'un point de vue qualitatif.
\end{enumerate}

\begin{table}[]
    \centering
    \begin{tabular}{ll|rrr}
    \toprule
            &         &  train &   dev &  test \\
    Corpus & Set &        &       &       \\
    \midrule
    Général & Négatif &  19940 &  2493 &  2491 \\
    Litéral & Négatif &  15701 &  1745 &  7478 \\
    Métaphores & Négatif &  15701 &  1745 &  7478 \\
    Partiel & Négatif &   3970 &  2493 &  2491 \\
    Général & Positif &   2013 &   252 &   251 \\
    Litéral & Positif &   1439 &   618 &   459 \\
    Métaphores & Positif &    413 &    46 &  2057 \\
    Partiel & Positif &    420 &   252 &   251 \\
    \bottomrule
    \end{tabular}
    \caption{Taille des corpus d'entraînement en fonction de leur composition}
    \label{tab:chap4:dataset-sizes}
\end{table}
\subsubsection{Impact de la taille du dataset: random petit}

\subsubsection{Entraîner à partir de termes explicites uniquement}

%mentula, cunnus, pedico, futuo, culus ?

\subsubsection{Entraîner à partir de métaphores uniquement}

\subsection{Ce qu’apprend le modèle?}

\subsubsection{Comparer “l’attention” en fonction des modèles}

%Note pour moi-même: Max(attention) similaire par modèle est-elle toujours la même ?

\subsubsection{Analyse des erreurs des X meilleurs modèles}

%Erreurs les plus fréquentes: Quels tags ? Quels mots ?

%Classification des types d’erreurs

%	Sur-apprentissage (Martial == Sexe)

%	Sous-apprentissage (Termes de la guerre =? Sexe)



% Indiquer les annexes disponibles en ligne.
% Description des fichiers
% Déposer l'application aussi

\chapter*{Conclusion}
% pour faire apparaitre l'introduction dans le sommaire
\addcontentsline{toc}{chapter}{Conclusion}

% Ce qu'ouvre le chapitre 5 ici
% Ce qu'on peut faire avec ça
% Comment je suis arrivé grâce à la technique à faire apprendre à la machine de la stylistique.

% Ré-entraînement,
% Ça marche 

% Pour que l'entete soit correcte car chapter* ne redefinit pas l'entete.
\markboth{CONCLUSION}{}




\chapter{Une annexe}

\section{Résultats de lemmatiseurs}


\begin{table}[]

\resizebox{\textwidth}{!}{%
\begin{tabular}{@{}l|llll||l|llll@{}}
 \toprule
 target            & precision & recall & f1-score & support &  target            & precision & recall & f1-score & support\\ \midrule
 Adj-N/A-Dep         & 0.90      & 0.87   & 0.88     & 61      & Par-Fut-Act       & 0.89      & 0.89   & 0.89     & 214     \\ 
 Adj-N/A-Pass        & 0.96      & 0.96   & 0.96     & 736     & Par-Fut-Dep       & 0.80      & 0.86   & 0.83     & 14      \\ 
 Adj-N/A-SemDep      & 1.00      & 0.67   & 0.80     & 3       & Par-Fut-SemDep    & 0.00      & 0.00   & 0.00     & 1       \\ 
 Ger-N/A-Act         & 0.91      & 0.90   & 0.91     & 273     & Par-Perf-Act      & 0.00      & 0.00   & 0.00     & 1       \\ 
 Ger-N/A-Dep         & 0.89      & 0.94   & 0.91     & 67      & Par-Perf-Dep      & 0.69      & 0.76   & 0.72     & 363     \\ 
 Ger-N/A-SemDep      & 1.00      & 1.00   & 1.00     & 1       & Par-Perf-Pass     & 0.74      & 0.84   & 0.79     & 2927    \\ 
 Imp-Fut-Act       & 0.97      & 0.96   & 0.96     & 142     & Par-Perf-SemDep   & 0.69      & 0.48   & 0.56     & 23      \\ 
 Imp-Fut-Dep       & 1.00      & 0.50   & 0.67     & 4       & Par-Pres-Act      & 0.94      & 0.96   & 0.95     & 1210    \\ 
 Imp-Pres-Act      & 0.97      & 0.95   & 0.96     & 805     & Par-Pres-Dep      & 0.96      & 0.96   & 0.96     & 188     \\ 
 Imp-Pres-Dep      & 0.91      & 0.91   & 0.91     & 54      & Par-Pres-SemDep   & 0.80      & 0.80   & 0.80     & 5       \\ 
 Imp-Pres-Pass     & 0.00      & 0.00   & 0.00     & 1       & Sub-Fut-Act       & 1.00      & 1.00   & 1.00     & 28      \\ 
 Imp-Pres-SemDep   & 1.00      & 1.00   & 1.00     & 5       & Sub-Fut-Dep       & 0.00      & 0.00   & 0.00     & 1       \\ 
 Ind-Fut-Act       & 0.95      & 0.90   & 0.92     & 1581    & Sub-Fut-Pass      & 0.00      & 0.00   & 0.00     & 10      \\ 
 Ind-Fut-Dep       & 0.90      & 0.90   & 0.90     & 108     & Sub-Impa-Act      & 1.00      & 1.00   & 1.00     & 1541    \\ 
 Ind-Fut-Pass      & 0.95      & 0.75   & 0.84     & 144     & Sub-Impa-Dep      & 0.99      & 0.99   & 0.99     & 136     \\ 
 Ind-Fut-SemDep    & 1.00      & 1.00   & 1.00     & 5       & Sub-Impa-Pass     & 0.99      & 0.99   & 0.99     & 308     \\ 
 Ind-Impa-Act      & 1.00      & 1.00   & 1.00     & 1336    & Sub-Impa-SemDep   & 1.00      & 1.00   & 1.00     & 17      \\ 
 Ind-Impa-Dep      & 0.97      & 0.98   & 0.98     & 113     & Sub-Perf-Act      & 0.80      & 0.95   & 0.87     & 459     \\ 
 Ind-Impa-Pass     & 0.99      & 0.99   & 0.99     & 235     & Sub-Perf-Dep      & 0.53      & 0.44   & 0.48     & 18      \\ 
 Ind-Impa-SemDep   & 1.00      & 1.00   & 1.00     & 40      & Sub-Perf-Pass     & 0.25      & 0.15   & 0.19     & 92      \\ 
 Ind-Perf-Act      & 0.97      & 0.97   & 0.97     & 4056    & Sub-Perf-SemDep   & 0.00      & 0.00   & 0.00     & 2       \\ 
 Ind-Perf-Dep      & 0.63      & 0.59   & 0.61     & 208     & Sub-PeriPerf-Pass & 0.00      & 0.00   & 0.00     & 3       \\ 
 Ind-Perf-Pass     & 0.58      & 0.55   & 0.57     & 827     & Sub-PeriPqp-Dep   & 0.00      & 0.00   & 0.00     & 1       \\ 
 Ind-Perf-SemDep   & 0.56      & 0.74   & 0.64     & 19      & Sub-PeriPqp-Pass  & 0.00      & 0.00   & 0.00     & 4       \\ 
 Ind-PeriFut-Pass  & 0.00      & 0.00   & 0.00     & 2       & Sub-Pqp-Act       & 1.00      & 1.00   & 1.00     & 569     \\ 
 Ind-PeriPerf-Pass & 0.00      & 0.00   & 0.00     & 5       & Sub-Pqp-Dep       & 0.29      & 0.26   & 0.28     & 19      \\ 
 Ind-PeriPqp-Dep   & 0.00      & 0.00   & 0.00     & 2       & Sub-Pqp-Pass      & 0.42      & 0.22   & 0.29     & 68      \\ 
 Ind-PeriPqp-Pass  & 0.00      & 0.00   & 0.00     & 4       & Sub-Pqp-SemDep    & 0.00      & 0.00   & 0.00     & 5       \\ 
 Ind-Pqp-Act       & 1.00      & 1.00   & 1.00     & 639     & Sub-Pres-Act      & 0.97      & 0.97   & 0.97     & 2617    \\ 
 Ind-Pqp-Dep       & 0.00      & 0.00   & 0.00     & 10      & Sub-Pres-Dep      & 0.95      & 0.93   & 0.94     & 197     \\ 
 Ind-Pqp-Pass      & 0.42      & 0.11   & 0.18     & 90      & Sub-Pres-Pass     & 0.98      & 0.94   & 0.96     & 332     \\ 
 Ind-Pqp-SemDep    & 0.33      & 0.40   & 0.36     & 5       & Sub-Pres-SemDep   & 0.96      & 0.96   & 0.96     & 25      \\ 
 Ind-Pres-Act      & 0.98      & 0.98   & 0.98     & 8453    & SupUm-N/A-Act       & 0.92      & 0.65   & 0.76     & 17      \\ 
 Ind-Pres-Dep      & 0.97      & 0.98   & 0.97     & 689     & SupU-N/A-Act        & 0.77      & 0.67   & 0.71     & 15      \\ 
 Ind-Pres-Pass     & 0.99      & 0.98   & 0.98     & 963     & SupU-N/A-Dep        & 0.83      & 0.71   & 0.77     & 7       \\ 
 Ind-Pres-SemDep   & 1.00      & 1.00   & 1.00     & 109     & N/A               & 1.00      & 1.00   & 1.00     & 133503  \\ 
 Inf-Fut-Act       & 0.86      & 0.95   & 0.90     & 194     \\ 
 Inf-Fut-Dep       & 0.92      & 0.75   & 0.83     & 16      \\ 
 Inf-Fut-Pass      & 0.00      & 0.00   & 0.00     & 1       \\ 
 Inf-Fut-SemDep    & 0.67      & 1.00   & 0.80     & 2       \\ 
 Inf-Perf-Act      & 1.00      & 1.00   & 1.00     & 586     \\ 
 Inf-Perf-Dep      & 0.49      & 0.66   & 0.56     & 32      \\ 
 Inf-Perf-Pass     & 0.59      & 0.48   & 0.53     & 368     \\ 
 Inf-Perf-SemDep   & 0.57      & 1.00   & 0.73     & 4       \\ 
 Inf-PeriFut-Act   & 0.00      & 0.00   & 0.00     & 9       \\ 
 Inf-PeriPqp-Pass  & 0.00      & 0.00   & 0.00     & 3       \\ 
 Inf-Pres-Act      & 0.99      & 1.00   & 0.99     & 3720    \\ 
 Inf-Pres-Dep      & 0.98      & 0.96   & 0.97     & 428     \\ 
 Inf-Pres-Pass     & 0.96      & 0.97   & 0.97     & 854     \\ 
 Inf-Pres-SemDep   & 1.00      & 1.00   & 1.00     & 16      \\ 
 avg / total       & 0.67      & 0.66   & 0.66     & 172968  \\ \bottomrule
\end{tabular}}
\label{table:lasla:verb-scores}
\caption{Résultat sur les données brutes du LASLA pour les annotations verbales hors personne}
\end{table}


\begin{table}[]
\begin{tabular}{lrrrrrrrr}
\hline
               &     0.01 &     0.05 &      0.1 &      0.2 &      0.25 &      0.3 &      0.4 &      0.5 \\
 Caesar        &   536    &  3224    &   6784   &  13525   &  17071    &  20843   &  28460   &  35607   \\
 Cato          &   172    &   977    &   1795   &   4107   &   4930    &   5649   &   7098   &   8601   \\
 Catullus      &   160    &   611    &   1084   &   2225   &   2825    &   3271   &   4254   &   4899   \\
 Cicero        &  5067    & 25296    &  47651   &  90787   & 111373    & 130462   & 170087   & 209291   \\
 Curtius       &   404    &  2901    &   5977   &  12330   &  15589    &  18233   &  24140   &  30352   \\
 Hirtius       &    59    &   313    &    597   &   1251   &   1525    &   1851   &   2369   &   2960   \\
 Horatius      &   257    &  2046    &   4084   &   8016   &  10016    &  12078   &  15874   &  19855   \\
 Juvenalis     &   304    &   989    &   1921   &   4126   &   5175    &   6267   &   8100   &  10586   \\
 Lucretius     &   614    &  2664    &   4947   &   9255   &  11506    &  13828   &  18419   &  23012   \\
 Ovidius       &   799    &  4525    &   9158   &  17711   &  22023    &  26503   &  35741   &  44389   \\
 Persius       &    41    &   113    &    331   &    751   &    949    &   1128   &   1522   &   2038   \\
 Petronius     &   411    &  1595    &   3203   &   5667   &   6513    &   7632   &  10073   &  12598   \\
 Plautus       &   827    &  3469    &   6426   &  11636   &  13760    &  16390   &  21760   &  26340   \\
 Plinius       &   789    &  3813    &   7506   &  15135   &  19232    &  23315   &  30850   &  38437   \\
 Propertius    &   230    &  1291    &   2660   &   4962   &   6202    &   7304   &   9526   &  11912   \\
 PseudoCaesar1 &    72    &   460    &   1028   &   2371   &   2908    &   3590   &   4822   &   5954   \\
 PseudoCaesar2 &   129    &   485    &    930   &   1704   &   2146    &   2663   &   3605   &   4613   \\
 PseudoCaesar3 &     5    &   106    &    317   &    588   &    720    &    863   &   1257   &   1859   \\
 Sallustius    &   347    &  1765    &   3204   &   6407   &   7980    &   9609   &  12966   &  16563   \\
 Seneca        &  3374    & 16028    &  30275   &  58539   &  72262    &  84821   & 113123   & 140158   \\
 Tacitus       &  1413    &  7702    &  14826   &  29697   &  36707    &  44223   &  58931   &  73242   \\
 Tibullus      &    87    &   564    &   1123   &   2277   &   2813    &   3358   &   4294   &   5577   \\
 Vergilius     &   714    &  3745    &   7567   &  15307   &  19267    &  23359   &  31348   &  39075   \\
 Total         & 16811    & 84682    & 163394   & 318374   & 393492    & 467240   & 618619   & 767918   \\
\hline
\end{tabular}
\label{table:lasla:extensibilite}
\caption{Nombre de tokens par auteur en fonction du pourcentage de coupe }
\end{table}

\printglossary

% bibliographie
\printbibliography

\end{document}

