\begin{figure}[h]
    \centering
    \noindent\begin{minipage}{.32\linewidth}
        \begin{equation*}
            \begin{matrix}
            Fortissimi \\ 
            sunt \\ 
            Belgae \\ 
            timendi \\ 
            ... \\ 
            ... \\ 
            ... \\ 
            \end{matrix}
            \rightarrow 
            \begin{bmatrix}
            0.97 & 0.85 \\ 
            0.12 & 0.85 \\ 
            0.54 & 0.28 \\ 
            0.92 & 0.90 \\ 
            ... & ... \\ 
            ... & ... \\ 
            n & ... \\ 
            \end{bmatrix}
        \end{equation*}
    \end{minipage}%
    \begin{minipage}{.32\linewidth}
        \begin{equation*}
            \begin{matrix}
                \textrm{Fortissimi sunt} \\ 
                \textrm{Belgae} \\ \\
                \textrm{Timendi sunt} \\
                \textrm{Belgae}
            \end{matrix}
            \rightarrow
            \begin{matrix}
            
            \begin{bmatrix}
            0.97 & 0.85 \\ 
            0.12 & 0.85 \\ 
            0.54 & 0.28 \\ 
            \end{bmatrix} \\ \\
            
            \begin{bmatrix}
            0.92 & 0.90 \\
            0.12 & 0.85 \\ 
            0.54 & 0.28 \\ 
            \end{bmatrix}
            
            \end{matrix}
        \end{equation*}
    \end{minipage}
    \caption{Remplacement de l'encodage \textit{one-hot} par une couche \textit{embeddings}. La proximité entre les vecteurs de \textit{Fortissimi} et \textit{Timendi} et la répétition des deux autres mots produisent deux phrases très proches mathématiquement.}
    \label{figure:deep-learning:embeddings-encoding}
\end{figure}