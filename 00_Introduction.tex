\chapter*{Introduction}
% pour faire apparaitre l'introduction dans le sommaire
\addcontentsline{toc}{chapter}{Introduction}
% Pour que l'entete soit correcte car chapter* ne redefinit pas l'entete.
\markboth{Introduction}{}
\label{intro}

\begin{quote}[\enquote{Le Poète Martial}]{G.~Boissier}
    \enquote{Martial n’est pas du nombre des poètes dont on entretienne volontiers les écoliers : ses ouvrages, si pleins d’esprit et d’agrément, contiennent des obscénités dégoûtantes, et l’on n’ose rien dire aux jeunes gens des jolies choses qui s’y trouvent, de peur de leur donner l’envie de lire le reste.\footcite{boissier_poete_1900}}
\end{quote}

\begin{quote}[\enquote{Réalisme et poésie chez Martial}]{É.~Wolf}
    \enquote{On ne saurait donc trop féliciter le jury de l'agrégation externe de lettres classiques d'avoir pour la première fois, sauf erreur, dans l'histoire de ce concours, mis ses \textit{Épigrammes} au programme, en choisissant deux livres parmi ceux qui sont les moins susceptibles d'effaroucher.\footcite{wolff_realisme_1997}}
\end{quote}

Que contiennent donc les textes de Martial pour qu'ils ne puissent pas être mis entre les mains des jeunes élèves, au risque de les \enquote{effaroucher} -- voire même de les troubler ? Si on n'a pas étudié le latin en licence ou en master, ou si on n'a pas ou peu étudié l'histoire romaine, on peut facilement rester interloqué, peut-être même être choqué, par la liberté d'expression concernant la sexualité dans les textes latins. Pire encore, cet écart entre nos deux sociétés se fait encore plus sentir quand celui qui ne connaît pas, découvre -- avec effroi ? -- qu'il était tout à fait accepté d'avoir des peintures suggestives dans les chambres à coucher, que des lampes figuraient sans honte des scènes explicites dont le rôle reproductif n'est pas manifeste, et qu'un sexe en érection fait de pierre, de bois ou de fer pouvait trôner devant la porte d'une maison pour éloigner le mauvais oeil\footcite{parker_bells_2018}. Si l'on ne connaît pas la poésie complète de Catulle, les \textit{Priapées} ou les \textit{Épigrammes} de Martial\footnote{Dont Dominique Noguez a donné une brillante traduction littéraire en plus d'une traduction littérale à même de donner un aperçu de la richesse d'un tel auteur.}, on n'a probablement pas eu l'occasion de découvrir ce pan de la littérature latine et de la culture romaine. Et pourtant, même Caton, connu pour sa rigueur morale extrême, aurait conseillé aux jeunes hommes d'aller au bordel, dans les limites du raisonnable: cela évitait de s'approcher des femmes mariées\footnote{Caton chez Horace, \textit{Satires}, 1, 2, 31--35: \enquote{lorsqu'un hideux désir a gonflé leurs veines, c'est qu'il convient aux jeunes de descendre [au bordel], plutôt que de broyer les épouses des autres}, \textcite[p.~30]{puccini_delbey_vie_2010}.}.

La place et le statut de la sexualité romaine sont devenus des enjeux importants des études latines, et ce champ de recherche se renouvelle depuis une cinquantaine d'années. Elle n'est pas juste affaire d'obscénités: si Martial est le plus grand utilisateur de \textit{pedico} (\enquote{enculer}), \textit{irrumo} (\enquote{violer la bouche}) ou \textit{futuo} (\enquote{baiser}), le reste de la littérature latine est elle aussi truffée d'allusions, plus ou moins explicites, au sexe. L'obscénité est chez Martial au service d'un style, tout comme des tournures figurées le sont dans d'autres genres, plus ou moins \enquote{littéraires}, plus ou moins \enquote{techniques}. Dans la poésie latine, Catulle comme Tibulle parlent de sexualité, mais n'empruntent pas nécessairement les mêmes chemins que celui de l'épigramme satirique. Quand Tibulle écrit en \textit{Élégie}, I, 10, 53--54 \enquote{Mais alors les guerres de Vénus s'enflamment, et la femme déplore ses cheveux arrachés et les portes qui ont été forcées}\footnote{\enquote{\textit{Sed Veneris tune bella calent; scissosque capillos Femina perfractas conqueriturque fores.}}}, il ne fait aucun doute que ni les \enquote{guerres}, ni les \enquote{portes}, doivent être comprises littéralement, et que les cheveux sont probablement plus tirés qu'arrachés. La métaphore filée permet ici de parler de sexe sans qu'aucune vulgarité lexicale n'apparaisse, et sans pour autant réduire la portée érotique de la scène. Mais la littérature latine n'est pas qu'une affaire de poètes, qu'ils soient vulgaires ou non, d'orateurs et d'historiens, qui sont généralement les trois grands genres pratiqués par les auteurs lus et connus dans le secondaire: elle est aussi la langue de scientifiques (médecins, astrologues, personnes s'intéressant au fonctionnement du règne animal), de l'exégèse biblique, de grammairiens, etc. Or, ces genres à la marge de l'enseignement traditionnel français sont aussi un espace de créations lexicales, d'emprunts, d'usages de figures de style, permettant d'éviter de \enquote{dire les choses}.

Si l'on cherche ainsi à entrer dans l'histoire de la sexualité romaine à travers le prisme de l'écrit et donc du vocabulaire latin, \enquote{dire les choses} n'est donc pas qu'affaire de lexique, mais aussi de style. Il n'est alors pas question de lister uniquement les termes obscènes et non obscènes qui, sans contexte ou dans un dictionnaire, feraient référence sans aucun doute à la sexualité. Il s'agit de recenser l'ensemble des formulations qui, dans un texte, montrent ou dénotent une activité sexuelle. Seulement avec cette partie du \enquote{dire}, il est possible pour la recherche de mieux comprendre ce que recouvre la notion de sexualité à Rome -- notion anachronique pour le peuple du Latium. James Noel Adams\footnote{Dont nous avons appris le décès dans les derniers mois de rédaction de cette thèse de doctorat.}, lexicographe et spécialiste du latin vulgaire, a marqué définitivement ce domaine en publiant en 1982 un ouvrage, \textit{The Latin Sexual Vocabulary}\footcite{adams}, où il s'efforce justement d'embrasser l'expression de la sexualité dans toutes ses variations, et d'établir un catalogue raisonné des formulations permettant cette expression.


\paragraph{Le latiniste et l'ordinateur}

J.~N.~Adams a produit ce travail à un âge où l'ordinateur n'était pas encore un outil indispensable à la recherche. En 1982, ni le \textit{Packhard Humanities Institute} et son corpus latin, ni le \textit{Perseus Projet}, ni même la \textit{Library of Latin Texts} de Brepols et la \textit{Patrologia Latina Database} n'existaient. Seul le \textit{Thesaurus Languae Latinae} (TLL), dictionnaire unilingue, constitué depuis plus de cent ans à Munich, pouvait alors être vu comme un premier avatar d'une base de données: pour chaque entrée du dictionnaire, une liste de sens puis de références pouvait nourrir suffisamment une recherche, en y ajoutant des publications sur le sujet. En 2022, les études latines s'appuient toujours sur le TLL et des éditions scientifiques sous forme papier, mais l'apparition de corpus numériques massifs, d'outils de recherche appropriés, de base de données bibliographiques a permis d'atteindre bien plus facilement une masse critique d'informations auparavant difficilement accessible.

Les années 1980 sont le début d'une massification de l'accès à l'informatique, d'abord dans les lieux de vie commune (bibliothèques, travail, écoles) puis dans les foyers, avec un transfert technologique progressif depuis l'Amérique du Nord vers le reste du Nord Économique (dont Europe de l'Ouest) et enfin le Sud Économique. L'arrivée du \textit{Web} et sa propre popularisation dans les années 2000 ouvrent, dans le monde de la recherche, la porte à des changements épistémologiques. Dans le milieu des lettres classiques et des sciences humaines en général, l'un de ces changements est indéniablement celui de l'explosion des méthodes dites des \enquote{\textit{Humanities Computing}}, des \enquote{\textit{Digital Humanities}} puis des \enquote{Humanités Numériques}. Nous ne cherchons pas ici à redéfinir entièrement ces dernières, et nous ne prétendons pas en proposer une définition universelle : nous proposerons donc humblement nôtre. Selon nous, une manière de concevoir les \enquote{Humanités Numériques} consiste à les penser comme des \enquote{sciences auxiliaires des sciences humaines}. Nous reprenons ici à notre compte la définition des \enquote{sciences auxiliaires de l'histoire} (parfois appelées \enquote{sciences fondamentales de l'histoire}): celles-ci rassemblent les disciplines indispensables à l'étude de l'histoire (comme, par exemple, la paléographie ou l'ecdotique), car seuls leurs résultats et leurs méthodes peuvent garantir une étude critique des sources et la compréhension de ces dernières. Il en va de même pour les Humanités Numériques qui rassemblent les méthodes et outils indispensables pour produire, analyser et publier des données numériques dans le champ des sciences humaines, et pour garantir la qualité, la visibilité et la réutilisabilité de ces données. Parfois définies comme \enquote{une transdiscipline\footcite{mounier_manifeste_2010}}, les Humanités Numériques intéressent en effet toutes les sciences humaines et sans doute au-delà, jusqu'aux sciences sociales. Comme toutes les disciplines, même auxiliaires, les HN produisent leurs propres recherches, cultivent leurs propres débats et développent des ambitions qui leur sont propres.

% Note pour moi-même à partir des modifications: Rajouter que comme les sciences auxiliaires de l'histoire, elle a ses propres débats.
% Ancienne version : C'est-à-dire qu'il s'agit d'un domaine qui développe ses propres recherches, ses propres débats et ses propres ambitions, mais qui a vocation à s'appliquer \textit{in fine} à des sujets de sciences humaines et à être mis en pratique. 

Aborder l'expression écrite de la sexualité latine sans avoir recours aux humanités numériques n'aurait pas eu de sens ; car sans cela, il n'aurait pas été possible de véritablement enrichir, voire dépasser, le travail d'Adams et de ses prédécesseurs. Mais tout comme Adams n'a pas fait ignorer le savoir accumulé jusqu'à sa publication\footnote{Il reste assez décrié pour son attitude envers ses prédécesseurs et ses pairs. \textit{Cf.} \textcite{richlin_sexual_1978}.}, il nous faut bâtir notre travail sur celui de notre prédécesseur. Notre recherche se construit donc autour de la question suivante : est-il possible, à partir du travail préexistant d'Adams, de venir compléter celui-ci avec des méthodes computationnelles ? La détection automatique des passages où la littérature latine parle de sexualité ouvre une nouvelle voie pour améliorer la connaissance générale du domaine latin -- que ce soit sur la sexualité ou sur autre chose, comme la notion de parenté. Même si les latins classique et tardif constituent un corpus fermé\footnote{Il nous est permis de rêver et de croire que d'autres textes seront découverts, mais le corpus latin a bien une finitude.}, de nombreuses questions (lexicographiques comme anthropologiques) se heurtent inévitablement face à l'ampleur de celui-ci. Grâce aux dernières évolutions informatiques, il est aujourd'hui possible de développer une méthode autorisant les chercheurs à construire des relevés qui dépassent la recherche d'occurrences de forme. C'est aussi un pas de plus vers la constitution de bases de données secondaires -- nous parlerons d'\enquote{exempliers numériques} --, permettant elles-mêmes de faire avancer notre connaissance du domaine.
%SG: cette affaire n'est pas claire du tout: elle prend quelle forme cette base secondaire, et en quoi elle se différencie de la base primaire, qui n'est d'ailleurs pas définie non plus. Faut reprendre


\paragraph{L'isotopie et le dépassement du mot}
%l'occurrence de quoi? c'est l'apparition d'un fait linguistique. là il me semble que tu parle d'occurrence de tokens ou de mots. C'est pas clair du tout

Dans son article de 1985, François Rastier définit le concept d'isotopie comme \enquote{la récurrence d'un même trait sémantique\footcite{rastier_isotopie_1985}}. Le sème est chez Rastier un \enquote{élément de signification {[qui peut être]} commun à différent mots {[}\textit{sème générique}{]}\footcite{pincemin1999semantique}}. L'étude des isotopies se distingue d'une simple analyse du champ lexical, car on ne retient pas uniquement les sèmes dénotatifs, mais aussi les sèmes connotatifs qui dépendent autant du locuteur que de la situation du discours. En d'autres termes, on ajoute aux sens premiers les sens seconds, figurés ou même implicites, le tout ne trouvant une cohérence que dans le contexte d'apparition.

Par exemple, quand Tibulle parle de \enquote{guerres de Vénus}, c'est l'ajout des sèmes /confrontation/, /corps à corps/, /deux camps/, /soumission/ et /amour/ (de déesse) qui mènent à une image claire pour le lecteur, celle de deux corps nus qui cherchent à \enquote{soumettre} ou à \enquote{vaincre} l'autre dans un \enquote{combat} bien plus érotique que guerrier\footnote{La métaphore guerrière pour l'amour (\enquote{\textit{Love is war}}) est fréquente et produit bon nombre d'activations sémantiques. \textcite{lakoff_metaphors_2003}.}. La notion d'isotopie pousse à dépasser le simple relevé de formes pour devenir un relevé d'extraits et de contextes, sans lesquels il n'est pas possible de faire sens.

Les notions d'extraits et de contextes sont elles-mêmes développées par F.~Rastier et son \enquote{école}. Où se situe la limite d'un contexte ? Pour Bénédicte Pincemin, un \enquote{extrait} est opposé à \enquote{une totalité et [à une] unité contextualisante\footcite{pincemin1999semantique}}, mais c'est bien à partir d'extraits que se fait la recherche de co-occurrences, et pour \enquote{forcer le trait} comme le fait Damon Mayaffre\footcite{mayaffre2008occurrence}, tout est probablement un extrait d'autre chose, tant les combinaisons sont possibles et les niveaux variés: mot, mots, phrases, paragraphes, section, livre, volume, série, genre, période, etc. Évidemment, B.~Pincemin ne s'oppose pas à l'utilisation de l'extrait, mais elle en définit les limites à prendre en compte: pour qu'un extrait \enquote{fasse sens}, il faille qu'il soit, bon an, mal an, contextualisant. L'isotopie reposant sur la répétition de sèmes, il faut s'assurer que cette dernière soit contenue dans les passages retenus. Mais F.~Rastier pousse la notion de contexte beaucoup plus loin, en insistant sur l'importance de la \enquote{présomption d'isotopie\footcite{rastier_isotopie_1985}}, responsable de l'actualisation des sèmes. Cette présomption signifie que le lecteur s'attend à trouver des thématiques et/ou des répétitions thématiques, en fonction du texte qu'il lit. Pour lui, le \enquote{langage neutre, purement dénotatif} n'existe pas. Les textes relèvent donc de \enquote{pratiques sociales}, partagent des traits génériques, voire thématiques, et font partie d'un \enquote{discours} qui le régit. Dans ce contexte, il est non seulement important de relever les unités textuelles qui, ensemble, permettent de déceler l'isotopie, mais d'aider à \enquote{présumer} son existence en fournissant les éléments suffisants à l'identification des traits extralinguistiques.  L'isotopie de la sexualité, ou isotopie sexuelle chez certains auteurs\footcite{leon_semes_1976}, a d'ailleurs la chance de s'exprimer dans un nombre de registres particulièrement variés et nombreux, auxquels on a parfois donné des noms: quand on parle de texte \enquote{pornographique} ou de texte \enquote{érotique}, les adjectifs distinguent l'intensité et la crudité de ce qui est montré et décrit dans ces écrits.

Certaines expressions et certaines isotopies ne peuvent être comprises qu'à travers le prisme du sociolecte et du genre littéraire. Prenons deux exemples issus des œuvres de Cicéron. En \textit{Fam.} 9.22.2, l'orateur propose d'éviter les successions des mots \enquote{\textit{cum nos}} ou encore, \enquote{\textit{illam dicam}} car elles provoquent des associations de syllabes qui produisent des \enquote{obscénités} (ici \textit{cunnus} (chatte) et \textit{landicam} (clitoris)\footnote{C'est ce que les grammairiens appellent un \textit{cacemphaton}. \textit{Cf.} \textcite{nicolas2007gros}}). On comprend d'autant mieux ce qu'il ne faut pas lire ou entendre : d'une part, parce que Cicéron dit que c'est vulgaire (\enquote{\textit{Nam obscenus est. [...] potuit obscenius?}}: \enquote{En effet, c'est obscène [...] Puisse-t-il exister plus obscène ?}), et d'autre part, parce que la notion \enquote{d'obscénité} chez l'auteur Cicéron nous prépare à attendre tout mot (quelque soit son registre) qui dépasse le sien retenu d'orateur. Si \textit{cunnus} est vulgaire par exemple, \textit{landica}, dans les attestations que nous avons, est généralement réservé aux textes médicaux\footnote{Caelius Aurelianus, \textit{Gynaeciorum}, 1.13, 2.1, 2.112; Muscio, \textit{Gynaecia}, 8; une attestation dans les \textit{Priapées}, 79. Peut-être que le terme était plus vulgaire à l'époque de Cicéron et qu'il a perdu sa force au fil des siècles. \textit{Cf.} \enquote{con} en français.}. Le deuxième exemple chez Cicéron est celui de \textit{lascivus} dans la qualification de Pompée, un ennemi politique qui vient à peine de perdre un jugement (\textit{Att.}, 2.3): \enquote{\textit{Epicratem suspicor, ut scribis, lascivum fuisse}} (\enquote{Je suspecte qu'Épicrate (Pompée) avait l'air, comme tu l'écris, \textit{lascivus}}. \textit{Lascivus} est un mot généralement utilisé par les poètes, pour parler de jeunes enfants ou esclaves (\textit{pueri}), et de jeunes femmes dans des contextes de jeux, en particulier sexuels. Plus tard, il intervient chez les auteurs chrétiens presque uniquement pour critiquer une attitude licencieuse. En prose, \textit{Lascivus} n'est utilisé dans les genres historiques et dans la bouche des orateurs que pour traduire une féminité décriée\footnote{Chez Ausone, pour Othon (3, 24); dans les \textit{Histoires Augustes}, à propos d'Hadrien (14, 11).}. Ici, le sème /sexuel/ de \textit{lascivus}, celui /grec/ du surnom /Epicrate/, celui encore de /pouvoir/ dans l'étymologie de cette appellation grecque -- particulièrement dans un contexte de défaite juridique --, le caractère inédit de l'usage du terme \textit{lascivus} chez Cicéron et ses pairs, la situation (que vient faire une \textit{lascivité} dans une histoire de sortie de procès ?), et enfin l'importance portée à la description de ses habits dans la phrase qui suit (souliers et bandelettes blanches), participent au renforcement du mot et posent même un doute sur sa \enquote{vulgarité} chez Cicéron. D'autant plus qu'il s'agit ici d'une lettre, et non d'un discours: peut-être est-ce simplement le signe que Cicéron fait usage d'un langage plus relâché ? On pourrait presque traduire, en trahissant un peu et en poussant la vulgarité, \enquote{qu'Épicrate avait l'air de s'être fait baiser}.

Nous ne sommes pas les premiers à se poser la question d'un traitement automatique de la détection isotopique. F.~Rastier lui-même note l'importance des corpus, leur utilité pour le linguiste et surtout les traitements qui peuvent déjà leur être appliqués dans les années 1990\footcite{rastier_semantique_1996}. Les premières approches pour la détection d'isotopie ou leur collecte ont été celles de l'analyse lexicale, qui avait pour but de rechercher des co-occurrences dans de vastes corpus: c'est par exemple ce que fait Damon Mayaffre dans sa recherche sur les discours présidentiels\footcite{mayaffre2008occurrence}. Dans une autre mesure, les travaux sur les discours des Premiers ministres chez Sjöblom et Leblanc reprennent cette ambition en comparant plusieurs approches quantitatives des textes. Mais ces deux exemples proposent une méthode particulière: c'est à partir de termes et de leurs relevés, ainsi que de leurs co-occurrents, qu'ils vont essayer de repérer les isotopies, en distinguant également les auteurs ou encore les périodes de production. Rastier insiste toutefois sur la nécessité de \enquote{mettre  au  point  un  système  d’aide à l’analyse  sémantique  qui dépasse les méthodes fondées sur les co-occurrences de mots clé, et qui permette de sélectionner les sous-corpus pertinents en fonction des tâches à accomplir\footcite[p.~31]{rastier_semantique_1996}}. S'il semble penser à la détection de multiples isotopies dans ce contexte, nous reprenons dans notre travail la notion de \enquote{présomption d'isotopie}: nous chercherons avant tout à savoir si un extrait de texte contient une isotopie sexuelle ; et nous n'aurons donc pas pour but de déterminer si, parmi les isotopies présentes, l'une d'entre elles est sexuelle.

\paragraph{Vers une méthode reproductible}

% \enquote{Pour le linguiste, le corpus est un outil de travail essentiel\footcite{pincemin1999semantique}}, mais il est devenu important dans bien d'autres domaines. Si un véritable intérêt pour

% Je ne trouve pas ce passage utile : "Nous avons défini les sujets de la thèse, dans une certaine largeur, et nous reviendrons au besoin sur certains des points mentionnés pour approfondir les concepts, en ayant toujours le souci de réintroduire ces derniers dans une histoire longue." Je le supprimerai

Alors que nous engagions nos recherches vers une thèse venant compléter la lexicographie de la sexualité avec des approches numériques, les opportunités offertes par les développements en apprentissage profond, et le souhait de dépasser les limites de la lexicographie assistée par la recherche automatisée d'occurrences -- originellement sous la forme d'un chapitre d'ouverture --, nous ont menés vers un terrain qui nous a passionné: celui de la détection automatique d'isotopie. Dans ce cadre, le temps accordé à la lexicographie latine a été réduit, et la sexualité romaine, aussi passionnante soit-elle, est devenue un terrain d'essai. Une fois cette situation admise, cette thèse est aussi devenue autre chose:
une exploration et une tentative d'établir une méthode, rigoureuse et reproductible, documentée et compréhensible, à même de produire de nouvelles recherches dans nos domaines d'origine.

Cette thèse est donc une expérience, et dans ce sens, elle doit être reproductible. Mais dans la mesure où elle produit des résultats exploitables, elle doit aussi s'assurer d'être réutilisable. Nous avons essayé de renseigner, dans la mesure du possible, l'ensemble des tentatives que nous avons réalisées, y compris les échecs. Or, si nos domaines ont l'habitude de parler des succès, la rédaction de comptes-rendus des \enquote{insuccès} en sciences humaines reste encore assez rare. Pour autant, même si cette thèse avait été un échec, elle nous aurait renseignés sur la faisabilité, avec les méthodes envisagées, de la détection d'isotopie en latin. Chaque année, nous avons pris l'habitude de faire lire à nos étudiants un article fantastique de Quinn Dombrowski qui ne parle que d'une chose: un échec incroyable, coûteux par ailleurs, d'un des plus gros projets en humanités numériques, et une tentative d'analyse des raisons de cet échec\footcite{dombrowski2014ever}. À la suite de Q.~Dombrowski, nous avons mentionné et explicité toutes les limites que nous avons rencontrées, tous les problèmes d'ingénierie qui ont bloqué notre avancée : ils indiquent quels leviers pourraient être actionnés pour transformer les outils présentés et en améliorer la performance.

Cette thèse est également une thèse d'ingénieur, car elle nécessite la production d'outils et de données exploitables et réutilisables. Nous invitons donc le lecteur à ne pas hésiter à parcourir les annexes numériques, qui font tout autant partie de la thèse que le texte de ces pages\footnote{Nous étions même tentés d'écrire \enquote{le texte qui les accompagne}}. La production technique en humanités numériques doit aussi permettre d'éviter de refaire ce qui a déjà été fait: elle doit donc être documentée et ouverte. Mais il faut aussi qu'elle soit citable, et qu'elle soit citée: il ne faut pas que soit invisibilisé -- au prétexte d'un manque d'habitude au mieux, par mépris au pire -- le travail scientifique indispensable à la production de données réutilisables et à la création d'outils généralisables.

Cette thèse est aussi une thèse de romaniste, de \textit{digital humanist} et d'ingénieur, et elle a été écrite en ce sens. La question que nous avons fini par poser est celle-ci: est-il possible de détecter l'implicite ? Et dans ce cadre, nous avons posé une question qui se trouve inévitablement au carrefour de la stylistique, de la lexicographie, du traitement automatique des langues, de l'ingénierie de corpus, de l'anthropologie de l'antiquité et des études latines\footnote{C'est un gros carrefour.}.

% Dans mon introduction générale, un engagement à la fois méthodologique et expérimentale fort: je travaille pour bâtir un modèle, à réfléchir à partir d'outils existants, ma thèse ne doit pas être lu comme une thèse sur la sexualité mais comme une approche stylistique guidée par le TAL ? Peut-être différent des prédateurs sexuels etc. mais en fait détection de l'implicite.


\paragraph{Structure de l'étude}

Après cinq années de recherche, nous présentons un travail qui se veut pluridisciplinaire, résolument ré-exploitable, pédagogique (si nous avons réussi), et parfois indécent - par (malin) plaisir, mais aussi par nécessité. Il a été important pour nous d'inscrire notre démarche \enquote{d'humanités numériques} dans une histoire qui est propre à celles-ci, en produisant quelques fois un rassemblement inédit d'une documentation sur l'histoire de leurs applications dans le champ des études latines, mais aussi des confrontations qui ont pu exister entre tenants d'une philologie classique et défenseurs d'une philologie numérique. En qualité d'ancien ingénieur ayant participé à plusieurs projets, et en qualité d'ami ou de collègues d'éditeurs, nous avons tenté de redonner leur place, aux côtés des chefs de projet, à ces \enquote{petites mains} (quand nous les avons retrouvées et dans la lignée de travaux comme ceux de Julianne Nyhan\footcite{nyhan2017uncovering}) dont l'activité fut fondamentale pour la réussite de ces initiatives. Bien trop de personnes sont ignorées dans l'histoire de ces corpus: bien que nous ayons été employé par Perseus, il nous aura fallu rédiger une thèse pour enfin découvrir qu'Elli Mylonas a été non seulement une membre fondatrice du projet, mais qu'elle porte probablement le plus de responsabilité dans le succès de la pérennisation des données d'origine. Les données, comme les logiciels, sont des produits scientifiques qui aujourd'hui sont indissociables de la recherche.
% Introduction

Nous traiterons notre sujet en quatre grands chapitres. Le premier chapitre présente le corpus ; le second chapitre, le plus court, sert davantage d'appui pédagogique pour comprendre les systèmes d'apprentissage machine à l'ère de l'apprentissage profond ; le troisième s'intéresse à la question de la lemmatisation et à l'annotation linguistique du latin ; et le quatrième présente enfin notre expérience finale, à savoir la détection automatique d'isotopie.

Dans notre premier chapitre, nous présentons deux corpus, dans leur production comme dans leurs caractéristiques: le premier est celui des sources latines numériques que nous pouvons exploiter (le meta-corpus), le second est celui des extraits latins collectés à partir du travail de James Noel Adams, le \textit{Latin Vocabulary of Sexuality}\footcite{adams}. Dans ce cadre, nous commençons par aborder la question des corpus sous l'angle de leur histoire, particulièrement riche depuis les années 1980, tant du point de vue des changements techniques que des acteurs ou des orientations thématiques. Ensuite, nous développons la question de la production du meta-corpus. Constitué de sources inédites et issues de projets historiques, il pose la question des besoins opérationnels pour mener à bien une recherche quantitative ; il pose aussi la question des formats à utiliser, avec pour objectif une pérennisation de notre travail. Enfin, nous traitons la problématique spécifique de l'isotopie sexuelle en latin, sous le prisme de la constitution d'une compilation d'extraits présentant cette particularité sémantique. Nous réintroduisons à cette occasion la discussion autour de l'histoire de la sexualité, mais aussi de sa lexicographie en latin.


Notre second chapitre porte sur le vocabulaire et les méthodes du traitement automatique des langues. Il est beaucoup plus court que l'ensemble des autres chapitres, et a pour vocation de permettre de comprendre les deux suivants. À la limite de l'annexe, il nous a semblé nécessaire de le laisser dans le corps de texte, afin de s'assurer que les lecteurs néophytes en la matière puissent comprendre la suite, quitte à y revenir. Il revient d'abord sur des considérations générales (qu'est-ce qu'un texte pour un algorithme ?) et sur les méthodes d'entraînement et d'évaluation en traitement automatique des langues. Le reste du chapitre présente des types de réseaux neuronaux et espère permettre au lecteur d'avoir ainsi une compréhension minimale de ses spécificités.

Notre troisième chapitre est entièrement dédié à la lemmatisation du latin. Nous profitons de la question de la lemmatisation, sous-entendue numérique, pour la recontextualiser dans une histoire qui dépasse le cadre de l'ordinateur: celle des concordanciers et des index. Nous étudions ensuite la riche histoire et les révolutions qu'ont connues les outils de lemmatisation en fonction de l'évolution des puissances de calcul. Les dernières méthodes reposant sur un entraînement supervisé requièrent également des corpus, que nous présentons donc à cette occasion. Nous mettons ensuite en place une expérience, visant non seulement à faire émerger les configurations de réseaux neuronaux les plus efficaces pour notre recherche et pour le latin, mais aussi à évaluer les limites de tels outils. Nous présentons enfin le modèle de lemmatisation final, que nous appelons LASLA+ en hommage au corpus fourni par le laboratoire liégeois.

Notre quatrième chapitre aborde enfin la question des modèles de détection d'isotopie. Nous établissons d'abord un large panorama des méthodes qui peuvent se rapprocher de notre sujet, et des tentatives d'incursions du traitement statistique dans le domaine des études antiques. Nous souhaitons ainsi en retirer des éléments à expérimenter en termes de technologies ou d'utilisations des données. À partir de ce panorama, nous proposons un ensemble d'architectures à évaluer en vue de détecter les isotopies sexuelles, et une explication du passage de notre corpus d'extraits à un corpus d'entraînement et de test permettant l'exploitation des informations qu'il contient. Ensuite, nous comparons les modèles obtenus et essayons de proposer une méthode pour évaluer l'extensibilité de cette approche à d'autres thématiques. Nous analysons alors les résultats, en prenant soin de revenir au texte, afin de comprendre ce qui se \enquote{cache} derrière les pourcentages de réussite. Enfin, nous établissons des recommandations d'architectures neuronales pour quiconque voudrait ré-appliquer les méthodes présentées ici.

% Comment orienter la lecture du jury via l'exposé et l'introduction: de quoi parle la thèse ?

% 

