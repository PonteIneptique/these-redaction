\chapter*{Introduction}
% pour faire apparaitre l'introduction dans le sommaire
\addcontentsline{toc}{chapter}{Introduction}
% Pour que l'entete soit correcte car chapter* ne redefinit pas l'entete.
\markboth{Introduction}{}
\label{intro}

\begin{quote}[\enquote{Le Poète Martial}]{G.~Boissier}
    \enquote{Martial n’est pas du nombre des poètes dont on entretienne volontiers les écoliers : ses ouvrages, si pleins d’esprit et d’agrément, contiennent des obscénités dégoûtantes, et l’on n’ose rien dire aux jeunes gens des jolies choses qui s’y trouvent, de peur de leur donner l’envie de lire le reste.\textcite{boissier_poete_1900}}
\end{quote}

\begin{quote}[\enquote{Réalisme et poésie chez Martial}]{É.~Wolf}
    \enquote{On ne saurait donc trop féliciter le jury de l'agrégation externe de lettres classiques d'avoir pour la première fois, sauf erreur, dans l'histoire de ce concours, mis ses \textit{Épigrammes} au programme, en choisissant deux livres parmi ceux qui sont les moins susceptibles d'effaroucher.\textcite{wolff_realisme_1997}}
\end{quote}

Que se passe-t-il donc hors des classes de latin du secondaire pour que le latin de certains auteurs obtienne une telle réputation ? En effet, qui n'a pas étudié le latin en licence ou en master, ou n'a pas étudié l'histoire romaine sera rapidement interloqué, peut-être même choqué, par la liberté d'expression autour de ce que nous appelons la sexualité dans les textes latins. Pire encore, cet écart entre nos deux sociétés se fait sentir encore plus quand celui qui ne connaît pas découvre -- avec effroi ? -- qu'il était tout à fait accepté d'avoir des peintures suggestives dans les chambres à coucher, que des lampes figuraient avec joie des scènes d'accouplement explicites et parfois clairement sans objectif reproductif, et qu'un sexe en érection fait de pierre, de bois ou de fer pouvait trôner devant la porte d'une maison, accompagnés de clochettes qui rythmaient probablement au gré des vents les rues romaines, avec pour seul but connu d'éloigner le mauvais oeil. Si l'on ne connaît pas, du côté littéraire, la poésie complète de Catulle, les \textit{Priapées} ou les \textit{Épigrammes} de Martial\footnote{Dont Dominique Noguez a donné une brillante traduction littéraire en plus d'une traduction littérale et qui peut, dans ce sens, donner un aperçu de la richesse d'un tel auteur.}, on n'a probablement pas découvert cette partie des textes latins. Et pourtant, même Caton, connu pour sa rigueur extrême, aurait conseillé aux jeunes hommes d'aller au bordel, dans les limites du raisonnable: cela évitait de s'approcher des femmes mariées\footnote{Horace, \textit{Satires}, 1, 2, 31--35: \enquote{lorsqu'un hideux désir a gonflé leurs veines, c'est qu'il convient aux jeunes de descendre [au bordel], plutôt que de broyer les épouses des autres}, \textcite[p.~30]{puccini_delbey_vie_2010}.}.

La question du statut de la sexualité romaine se pose donc, et se renouvelle même depuis cinquante ans. Mais elle n'est pas juste affaire d'obscénités: si Martial est le plus grand utilisateur de \textit{pedico} (enculer), \textit{irrumo} (violer la bouche) ou \textit{futuo} (baiser), le reste de la littérature latine n'est pas sans parler de sexe. L'obscénité est chez Martial au service d'un style, tout comme des tournures figurées le sont dans d'autres genres, plus ou moins \enquote{littéraires}, plus ou moins \enquote{techniques}. Dans la poésie latine, Catulle comme Tibulle parle de sexualité, mais n'empruntent pas nécessairement les mêmes chemins que celui de l'épigramme satirique. Quand Tibulle écrit en Élégie, I, 10, 53--54 \enquote{Mais alors les guerres de Vénus s'enflamment, et la femme déplore ses cheveux arrachés et les portes qui ont été forcées}\footnote{\enquote{\textit{Sed Veneris tune bella calent; scissosque capillos Femina perfractas conqueriturque fores.}}}, il ne fait aucun doute que les guerres n'en sont pas vraiment, ni les portes d'ailleurs, et que les cheveux sont probablement plus tirés qu'arrachés. La métaphore filée permet ici de parler de sexe sans qu'aucune vulgarité lexicale n'apparaissent, sans réduire la portée érotique de la scène. Mais la littérature latine n'est pas qu'une affaire de poètes, qu'ils soient vulgaires ou non, d'orateurs et d'historiens, qui sont généralement les trois grands genres d'auteurs lus et connus: elle est aussi la langue de scientifiques (médecins, astrologues, personnes intéressées dans le fonctionnement du règne animal), de l'exégèse biblique, de grammairiens, etc. Or, ces genres à la marge de l'enseignement classique sont aussi l'espace de création lexicales, d'emprunts, d'usages de figure de style afin d'éviter de \enquote{dire les choses}.

Si l'on cherche ainsi à entrer dans l'histoire de la sexualité romaine à travers le prisme de l'écrit et donc du vocabulaire latin, \enquote{dire les choses} n'est donc pas qu'affaire de lexique mais aussi de style. Il n'est alors pas question de lister uniquement les termes obscènes et non-obscènes qui, sans contexte, dans un dictionnaire, feraient référence sans aucun doute à la sexualité. Il s'agit de recenser l'ensemble des formulations qui, dans un texte, montre ou dénote une activité sexuelle, et seulement avec cet partie du \enquote{dire}, il est possible pour la recherche de mieux comprendre la notion romaine de cette \enquote{sexualité}, notion qui elle-même n'existe pas à Rome. James Noel Adams\footnote{Dont nous avons appris le décès dans les derniers mois de rédaction de cette recherche.}, lexicographe et spécialiste du latin vulgaire, a marqué définitvement ce domaine en publiant en 1982 un ouvrage, \textit{The Latin Sexual Vocabulary}\footcite{adams}, où il s'efforce justement d'embrasser l'expression de la sexualité dans toutes ses variations, et d'en établir une catalogue raisonné.

\paragraph{Le latiniste et l'ordinateur}

Mais J.~N.~Adams a produit ce travail à un âge où l'ordinateur, probable medium par lequel ce texte est lu, n'était pas encore un outil indispensable à la recherche. En 1982, ni le Packhard Humanities Institute et son corpus latin, ni le \textit{Perseus Projet}, ni même pour les périodes tardive la \textit{Library of Latin Texts} de Brepols et la \textit{Patrologia Latina Database} n'existaient. Le \textit{Thesaurus Languae Latinae} (TLL), dictionnaire unilingue constitué depuis plus de cent ans à Munich, régnait alors presque seul comme forme de base de données: pour chaque entrée du dictionnaire, une liste de sens puis de références pouvait nourrir suffisamment une recherche en prenant en compte les publications sur le sujet. En 2022, les études latines s'appuient bien sûr toujours sur le TLL et des éditions scientifiques papiers, mais l'apparition de corpus numériques massifs, d'outils de recherche appropriés, de base de données bibliographiques ont permis d'atteindre bien plus facilement une masse d'information auparavant difficilement accessible.

Les années 1980 sont le début d'une massification de l'accès à l'informatique, d'abord dans les lieux de vie commune (bibliothèques, travail, écoles) puis dans les foyers, avec une translation Amérique du Nord -- Reste du Nord Économique (dont Europe de l'Ouest) -- Sud Économique. L'arrivée du \textit{web} et sa propre massification dans les années 2000 ouvrent, dans le monde de la recherche, la porte à d'autres changements épistémologiques. Dans le milieu des lettres classiques et des sciences humaines en général, l'un de ces changements est indéniablement celui de l'explosion des méthodes dites des \enquote{\textit{Humanities Computing}}, des \enquote{\textit{Digital Humanities}} puis des \enquote{Humanités Numériques}. Sans chercher nécessairement à redéfinir ces dernières sur de nombreuses pages et prétendre à une définition universelle, nous proposerons donc la nôtre. Au même titre que la paléographie ou l'ecdotique, les humanités numériques sont pour nous des \enquote{sciences auxiliaires des sciences humaines}. C'est-à-dire qu'il s'agit d'un domaine qui connaît sa propre recherche, ses propres débats et ses propres ambitions, mais qui a vocation à s'appliquer \textit{in fine} à des sujets de sciences humaines et à être mis en pratique. 

Aborder l'expression écrite de la sexualité latine sans avoir recours aux humanités numériques et aux derniers développements informatiques n'aurait pas eu de sens, car il n'aurait pas été possible, probablement dépasser le travail d'Adams et de ses prédécesseurs. Mais tout comme Adams n'a pas fait totalement table rase du passé\footnote{Il reste assez décrié pour son attitude envers ses prédécesseurs et ses pairs. \textit{Cf.} \textcite{richlin_sexual_1978}.}, il nous faut bâtir ce travail sur celui d'Adams. Notre recherche se construit autour d'une simple question: est-il possible, à partir du travail pré-existant d'Adams, de venir compléter celui-ci avec des méthodes computationnelles ? Si détecter les moments où la littérature latine parle de sexualité est une option ouverte par les méthodes quantitatives, alors une nouvelle voix s'ouvre aussi pour notre connaissance du domaine latin en dehors même de la sexualité. Le latin étant un corpus fermé, auquel de nombreuses questions (lexicographiques comme anthropologiques) se heurtent face à son ampleur, produire une méthode autorisant les chercheurs à construire des relevés qui dépassent la recherche d'occurrences de forme est un pas vers la constitution de bases de données secondaires -- nous parlerons d'\enquote{exempliers numériques} -- permettant elles-mêmes de faire avancer le savoir.
% Définition Adams et séxualités

\paragraph{L'isotopie, ou le dépassement d'une approche purement lexicale}

Dans son article de 1985, François Rastier définit le concept d'isotopie comme \enquote{la récurrence d'un même trait sémantique\footcite{rastier_isotopie_1985}}. Cette définition est primordiale pour aborder le travail d'Adams, car elle permet de dépasser le simple lexique en accordant aux sèmes une place d'envergure dans le traitement d'un domaine particulier, ici celui de la sexualité. Quand Tibulle parle de \enquote{guerres de Vénus}, c'est l'ajout des sèmes /confrontation/, /corps (des soldats) en action/, /deux camps/, /soumission/ même peut-être et /déesse de l'amour/ qui mènent à une image claire pour le lecteur, celle de deux corps nus qui cherchent à soumettre ou à \enquote{vaincre} l'autre dans un désir érotique.
% Définition Isotopie
% Au moment de la définition d'isotopie, parler d'isotopie de la sexualité ou d'isotopie sexuelle

\paragraph{Vers une méthode reproductible}

% Généricisation de l'objet
% Reproductibilité et réutilisabilité
% Manière de transmettre les outils
% Comment on peut en servir ?

% Dans mon introduction générale, un engagement à la fois méthodologique et expérimentale fort: je travaille pour bâtir un modèle, à réfléchir à partir d'outils existants, ma thèse ne doit pas être lu comme une thèse sur la sexualité mais comme une approche stylistique guidée par le TAL ? Peut-être différent des prédateurs sexuels etc. mais en fait détection de l'implicite.
% BTW, limites floues de la zone de compréhension ?
% Défricher et construire une méthodo, un système d'enquête

\paragraph{Plan}

% Introduction


% Comment orienter la lecture du jury via l'exposé et l'introduction: de quoi parle la thèse ?

% 

