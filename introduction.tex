\chapter*{Introduction}
\label{intro}



\begin{quote}{\textcite{boissier_poete_1900}}
    \enquote{Martial n’est pas du nombre des poètes dont on entretienne volontiers les écoliers : ses ouvrages, si pleins d’esprit et d’agrément, contiennent des obscénités dégoûtantes, et l’on n’ose rien dire aux jeunes gens des jolies choses qui s’y trouvent, de peur de leur donner l’envie de lire le reste.}
\end{quote}

\begin{quote}{\textcite{wolff_realisme_1997}}
    \enquote{On ne saurait donc trop féliciter le jury de l'agrégation externe de lettres classiques d'avoir pour la première fois, sauf erreur, dans l'histoire de ce concours, mis ses \textit{Épigrammes} au programme, en choisissant deux livres parmi ceux qui sont les moins susceptibles d'effaroucher.}
\end{quote}

Que se passe-t-il donc hors des classes de latin du secondaire pour que le latin de certains auteurs obtienne une telle réputation ?

% Définition Isotopie
% Définition Adams et séxualités

% Introduction
% Généricisation de l'objet
% Reproductibilité et réutilisabilité
% Manière de transmettre les outils
% Comment on peut en servir ?


% Comment orienter la lecture du jury via l'exposé et l'introduction: de quoi parle la thèse ?

% Dans mon introduction générale, un engagement à la fois méthodologique et expérimentale fort: je travaille pour bâtir un modèle, à réfléchir à partir d'outils existants, ma thèse ne doit pas être lu comme une thèse sur la sexualité mais comme une approche stylistique guidée par le TAL ? Peut-être différent des prédateurs sexuels etc. mais en fait détection de l'implicite.
% BTW, limites floues de la zone de compréhension ?
% Défricher et construire une méthodo, un système d'enquête
% 


% Au moment de la définition d'isotopie, parler d'isotopie de la sexualité ou d'isotopie sexuelle