\chapter*{Introduction}
\label{intro}

% Définition Isotopie
% Définition Adams et séxualités

% Introduction
% Généricisation de l'objet
% Reproductibilité et réutilisabilité
% Manière de transmettre les outils
% Comment on peut en servir ?


% Comment orienter la lecture du jury via l'exposé et l'introduction: de quoi parle la thèse ?

% Dans mon introduction générale, un engagement à la fois méthodologique et expérimentale fort: je travaille pour bâtir un modèle, à réfléchir à partir d'outils existants, ma thèse ne doit pas être lu comme une thèse sur la sexualité mais comme une approche stylistique guidée par le TAL ? Peut-être différent des prédateurs sexuels etc. mais en fait détection de l'implicite.
% BTW, limites floues de la zone de compréhension ?
% Défricher et construire une méthodo, un système d'enquête
% 
